
In Chapters \ref{ch:feens} - \ref{ch:fereeds}, developments of free-energy methodology and its application for binding free energy calculations were presented. Chapter \ref{ch:cycPep} described a study of the conformational behavior of semi-peptidic macrocycles  to connect the change of a single stereocenter with their lipid-membrane permeability. Finally, Chapters \ref{ch:feens} and \ref{ch:pyGrom} illuminated aspects of software development in science.

\section{Development of Scientific Software}
Software development is and will be an essential part of computational chemistry for different reasons.  Software in this area becomes steadily more complex in order to increase computational efficiency and the amount of available functionality. Further, many published studies are difficult to reproduce or methods cannot be further developed because the used source code is unavailable or not transferable to different platforms.\cite{Walters2013, Walters2020} The open-source movement, which has become an important driving force of academic sciences, can be considered a role model for improving this situation.\cite{Walters2020, Oliphant2007} In this line, many journals have started to request non-commercial software of a publication to be open source. \cite{Jcconduct2021, Scienceconduct2021, Natureconduct2021}  Overall, these developments will help to increase the readability and transferability of code.\cite{Walters2020} The latter issue can be solved by using programming environment tools such as pip or anaconda for Python. \cite{Anaconda2020, Pypi2021}

All software packages developed in this thesis are open source and can be accessed via the rinikerlab GitHub repository. %(except for the produced Gromos code)

%PyGromos Tools
Next, an outlook for the PyGromosTools package is provided. In our opinion, PyGromosTools combines scripting and programming languages in a productive way, making the package easy to use and efficient, thus fulfilling key conditions of modern scientific codes of conduct of scientific journals. 
A long-term vision is to build from PyGromosTools a PyGROMOS package. This package should integrate GROMOS++ \cite{Eichenberger2011} into the Python layer to make it easier and faster to extend its functionality. Efficiency issues could be solved by using Numba or other efficiency-improving tools. In addition, the GROMOS MD engine \cite{Schmid2012} should be integrated tighter with the use of binding tools like pyBinds or SWIG. \cite{Wenzel2011, Beazley1996} These changes are expected to lead to a more ``future-ready'' GROMOS environment that provides easier access to its functionality. The package could be compiled by the Python package managing tools. 


%---------------------------------------------------------------------------------------

\section{Perspectives for RE-EDS}
%global trends and needs
In recent years, an increasing amount of publications on path-free multi-state methods has appeared. \cite{Donnini2011, Knight2011, Bieler2015, Sidler2016, Perthold2018} An attractive aspect of such methods is their computational efficiency. This effect can be attributed to (i) the phase-space overlap that allows simultaneous sampling of multiple end-states, and (ii) no predefined paths for the sampling of the end-state transitions is required, thus allowing the system to find an optimal spanning tree of the state graph dynamically. 
Insufficient sampling resulting from the choice of difficult state transitions was reported in the literature as a reason for the efficiency loss of pairwise path-dependent methods.\cite{Jespers2019}
To assess how well each end-state is sampled, robust metrics are needed for path-free multi-state methods. With such metrics in hand, the simulation parameters can be refined to ensure equal sampling of all end-states, as described for RE-EDS in Chapter \ref{ch:fereeds}. 

%RE-EDS
\subsection{Method Development}
A pre-processing pipeline was established to optimize all RE-EDS parameters based on the defined general metrics. 
In order to reduce the optimization time of the pipeline, multiple options can be explored. 
First, the information about the replica-exchange gap region included in the initial short simulation to obtain energy offsets could be used to refine the initial $s$-distribution (instead of a logarithmic distribution). 

% Non-Equilibrium Eoff estimation
A second idea is to investigate whether the initial state optimization process could be used to estimate the energy offsets. This could be done by using the Jarzynski equality and the work that is required to change the initial maximally contributing end-state of the system to the desired end-state. \cite{Jarzynski1997,  Xiong2006} 
\begin{equation}
	\Delta F_{BA} = -\frac{1}{\beta} \ln\langle e^{-\beta W_{BA}} \rangle_R ,
\end{equation}
where the work $W$ is defined as, \cite{Xiong2006}
\begin{equation}
	W_{BA}(t)= \int_{0}^{t} \frac{\partial H(t)}{\partial t} dt .
\end{equation}
%
% Use all Data to Estimate dF
Another significant improvement could be the integration of the information from all replicas in the final free-energy estimation, not only from replica $s=1.0$. 
For this, a free-energy estimator like M-BAR\cite{Shirts2008} or any other multi-state (here in the sense of replicas) reweighting scheme may be used. Note that BAR\cite{Bennett1976} was already applied to $\lambda$-EDS in Ref.~\citenum{Konig2021}.

%2D-REEDS
Finally, the sampling of the implemented 2D-RE-EDS approach, exchanging both $s$-parameters and energy offsets, needs to be tested. For conformational studies, it may be of interest to develop a 2D - RE-EDS variant that exchanges both $s$-parameters and temperature to enhance sampling further.

%Software
\subsection{RE-EDS Software Development}
%%Dynamic Pipelining
From an implementation point of view, the RE-EDS pipeline \cite{Ries2021B} could be made more dynamic such that it is decided on-the-fly which modules of the pipeline are needed to be applied during optimization. Such a dynamic modular approach could improve the robustness and efficiency of the pipeline.

%%PygromosTools update
%On this note, it should be added that a PyGromosTools\cite{Lehner2021} version update should be considered, such that the approach can be made more platform-independent.

\subsection{RE-EDS Applications}
%%Towards high-throughput
In the future, aspects such as the complexity of transformation and the number of end-states in RE-EDS simulations will be further investigated. 
This experience gained from these studies could be used to develop a robust high-throughput approach with RE-EDS, where feasible subsets of ligands are selected from databases, e.g. by clustering. The subsets will share one or multiple reference ligands such that all relative binding free energies can be calculated. The clustering metric could be based on simple topological and 3D-structure-based criteria, or employ molecular dynamic fingerprints (MDFP),\cite{Riniker2017} thus including the dynamic ligand behavior into the clustering.


%% Docking position validation.
Another possible RE-EDS application could be to validate docking results. Docking is a commonly utilized method to generate ligand-protein complex. \cite{Zhao2015, Eberhardt2021, Morris2009} However, the validation of such results is often non-trivial, especially as the docking scoring functions are relatively simplistic.\cite{Chen2015} Therefore, MD simulations are usually employed to check the robustness of docking poses.\cite{Zhao2015, Feng2015, Sokkar2011, Chavda2019} With RE-EDS, a performant approach could be established that evaluates the docking results. For this, a separated dual topology approach \cite{Rocklin2013} with weak position restraints could be employed. A challenge might thereby be the undersampling region of the $s$-distribution when the end-states are clearly separated in the coordinate space. 


%---------------------------------------------------------------------------------------

\section{Membrane Permeability Beyond Rule of 5}
Cyclic peptides belong to the so-called ``beyond the Lipinski rule-of-5'' (bRo5) class of compounds, and have a complex conformational behavior. Some cyclic peptides are able to passively cross membranes despite their size, which is one of their fascinating aspects.\cite{Rezai2006, Rezai2006A, Matsson2017,Whitty2016} 
A reason for this is hypothesized to be a chameleonic character in terms of their conformational behavior, which allows them to adapt to apolar and polar environments.\cite{Witek2016, Witek2017, Witek2019, Wang2021} This interesting property might give rise to new concepts in rational drug design for bRo5 molecules.
Important factors appear to be the shielding of polar atoms and the rigidification of the cyclic structure in the permeable conformation.\cite{Witek2019, Wang2021} 
The ongoing modeling of how cyclic peptides pass through cell membranes could further increase our understanding of the mechanism of membrane permeation, and help to identify essential structural features of bioavailable cyclic peptides.



