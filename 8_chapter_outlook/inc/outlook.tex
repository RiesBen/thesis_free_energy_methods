
This thesis presented in chapter \ref{ch:feens} - \ref{ch:fereeds} developments of methodology and applications for free energy calculations. In chapter \ref{ch:cycpep} a study of the conformational behavior of macrocyclic peptoids  based on the change of a single stereo-center in context of their lipid-membrane permeability. Additionally did Chapter \ref{ch:feens} and \ref{ch:pyGrom} illuminate aspects of software development in science.

\section{Development of Scientific Software}
Software development will be and is an essential concern in computational chemistry.  One reason is that computational chemistry software is getting more complex to boost calculation efficiency or the amount of functionality present. Another reason is that many published studies are hard to reproduce or can not be further developed, as the used source code is unavailable or can not be used on different platforms.\cite{Walters2013, Walters2020} A role model improving this situation could be seen in the open-source movement, which is a great driver of academic sciences.\cite{Walters2020, Oliphant2007} In this sense, many journals started to request non-commercial software of a publication to be open source. \cite{Jcconduct2021, Scienceconduct2021, Natureconduct2021}  Overall, this might also increase the readability and transferability from one hardware setup to another.\cite{Walters2020} The latter problem can be solved by using programming environment tools such as pip or anaconda for Python. \cite{Anaconda2020, Pypi2021}

All software produced in this thesis is open-source and can be accessed via the rinikerlab GitHub. (except for the produced Gromos code)

%PyGromos Tools
Next, a possible outlook for the PyGromosTools package will be provided. We believe that PyGromosTools shows the potential of combining scripting and programming languages in terms of ease of use and project efficiency and fulfills crucial keystones to modern scientific codes of conduct of scientific journals. 
A long-term vision is to build from PyGromosTools a PyGROMOS package. On the one hand, it should integrate GROMOS++ \cite{Eichenberger2011} into the Python layer to make it easier and faster to develop its functionality. Efficiency issues could be solved by using Numba or other efficiency-improving tools. On the other hand, the integration of GROMOSXX \cite{Schmid2012} should be integrated tighter with the use of binding tools like pyBinds or SWIG. \cite{Wenzel2011, Beazley1996} These changes would lead to a more future-ready GROMOS environment that provides easier access to GROMOS and could even be shipped in one package and be compiled by the Python package managing tools. 


%---------------------------------------------------------------------------------------

\section{Free Energy Calculations with RE-EDS}
%global trends and needs
In recent years, a trend can be identified for the free energy method development, a substantial amount of publications on path-free multi-state methods appeared. \cite{Donnini2011, Knight2011, Bieler2015, Sidler2016, Perthold2018} One of the attractive aspects of such methods is the simulation time efficiency. This effect results from reasons, first the phase space overlap that allows simultaneous valuable sampling of multiple end-states.  Second, the method does not require predefined paths for the sampling of the end-state transitions, thus enabling the systems to find an optimal spanning tree of the state graph dynamically. Finding a good spanning tree can generate a performance loss for many other methods that rely for efficiency reasons on a minimal set of defined paths.\cite{Jespers2019}
Nevertheless, robust and accurate metrics are needed to classify each end-state sampling amount and quality as generalized quality control for path-free multi-state methods. Subsequently, the simulation parameters can be adapted according to these metrics to counter sampling deficiencies of the end-state space, as described for RE-EDS in chapter \ref{ch:fereeds}. 

%RE-EDS
\subsection{Method Development}
%global trends and needs
In recent years, a trend can be identified for the free energy method development, a substantial amount of publications on path-free multi-state methods appeared. \cite{Donnini2011, Knight2011, Bieler2015, Sidler2016, Perthold2018} One of the attractive aspects of such methods is the simulation time efficiency. This effect results from reasons, first the phase space overlap that allows simultaneous valuable sampling of multiple end-states.  Second, the method does not require predefined paths for the sampling of the end-state transitions, thus enabling the systems to find an optimal spanning tree of the state graph dynamically. Finding a good spanning tree can generate a performance loss for many other methods that rely for efficiency reasons on a minimal set of defined paths.\cite{Jespers2019}
Nevertheless, robust and accurate metrics are needed to classify each end-state sampling amount and quality as generalized quality control for path-free multi-state methods. Subsequently, the simulation parameters can be adapted according to these metrics to counter sampling deficiencies of the end-state space, as described for RE-EDS in chapter \ref{ch:fereeds}. 

%RE-EDS
A RE-EDS methodology was established to adapt all sampling defining parameters based on criteria retrieved from the simulation by the defined general metrics. 
In order to reduce the required optimization time of the pipeline, multiple developments could be tested. 
First, the provided information from the energy offset information about the RE exchange probability of the replica positions could be used as an initial s-optimization step. 

% Non-Equilibrium Eoff estimation
Second, it is an interesting question if the initial state optimization process could estimate the free energy offset as the work of the state transformation with the general starting state as a reference and the Jarzynski equality. \cite{Jarzynski1997,  Xiong2006} 
\begin{equation}
	\Delta F_{AB} = -\frac{1}{\beta} \ln\langle e^{-\beta W_{AB}} \rangle_R
\end{equation}
Wit the work W as: \cite{Xiong2006}
\begin{equation}
	W_{A \rightarrow B}(t)= \int_{0}^{t} \frac{\partial H(t)}{\partial t} dt .
\end{equation}
s
% Use all Data to Estimate dF
Another significant improvement would be the integration of the information from all replicas in the final free energy estimation, not only from replica $s=1$. 
Therefore a free energy estimator like M-BAR\cite{Shirts2008} or any other multi-state (here in the sense of replicas) reweighting scheme could be an essential asset to the RE-EDS pipeline in order to further cut the production time. Note that BAR\cite{Bennett1976} was already applied to $\lambda$-EDS in Ref. \citenum{Konig2021}.

%2D-REEDS
Finally, the sampling of the implemented 2D-RE-EDS approach, exchanging s-parameters and energy offsets, needs to be tested. For conformational studies, it could be of interest to implement a 2D - RE-EDS variant that exchanges s-parameters and temperature to generate a hybrid that explores with enhanced sampling the conformational space.

%Software
\subsection{Software Development}
%%Dynamic Pipelining
From an implementation point of view, it could be of interest to support the RE-EDS pipeline \cite{Ries2021B} with a dynamic approach that decides by sampling behavior which module of the pipeline is required to be applied in the optimization phase. Such a dynamic module approach could improve the robustness and efficiency of the pipeline to build up a more project-independent methodology.

%%PygromosTools update
On this note, it should be added that a PyGromosTools\cite{Lehner2021} version update should be considered, such that the approach can be made more platform-independent.

\subsection{Applications}
%%Towards high-throughput
Finally, perspectives on possible future applications will be given. 
In the future, aspects such as the complexity of transformation and the number of end-states will be further investigated. 
This knowledge could be used to develop a robust high-throughput approach for RE-EDS by deriving metrics that help construct feasible reeds systems from subselections of a ligand database. Such a clustering approach (e.g., k-means) could facilitate the selections, which groups the database of ligands into large simulation systems with one or multiple reference ligands shared by the simulations.  The clustering metric could be based on simple topological and 3D-structure-based criteria or employing molecular dynamic fingerprints (MDFP), including the dynamic ligand behavior into the clustering.\cite{Riniker2017}


%% Docking position validation.
 Another perspective for a novel RE-EDS application could be validating docking results. Docking is a commonly utilized method to generate ligand-protein complex. \cite{Zhao2015, Eberhardt2021, Morris2009} However, often, the validation of such results is tricky, especially as the docking scoring functions are relatively simplistic.\cite{Chen2015} Therefore, MD simulations are usually employed to perform an improved validation with conformational ensembler of the docking results.\cite{Zhao2015, Feng2015, Sokkar2011, Chavda2019} With RE-EDS, a performant approach could be established that investigates the docking results. In this approach, a separated topology approach \cite{Rocklin2013} with weak position restraints could be employed. However, an open question remains: how the undersampling space behaves, containing multiple end-states that are clearly separated in the coordinate space. 


%---------------------------------------------------------------------------------------

\section{Membrane Permeability beyond RO5}
Cyclic peptides are well experimentally studied molecules beyond the Lipinski rule of 5 (BRO5) and have a complex conformational space. The molecules undergo passive membrane transport, which is one of their fascinating aspects.\cite{Rezai2006, Rezai2006A, Matsson2017,Whitty2016} 
The chameleonic character of cyclic peptides is based on their conformational behavior allowing them to adapt to apolar and polar environments.\cite{Witek2016, Witek2017, Witek2019, Wang2021} This interesting property might give rise to new concepts in rational drug design for BRO5 molecules.
Important factors seem to be hiding polar atoms and rigidifying the cycle structure in the permeable conformation.\cite{Witek2019, Wang2021} 
The ongoing modeling of how cyclic peptides pass through cell membranes could further increase our knowledge on the mechanism of membrane permeation and help identify essential features of the cyclic peptides.



