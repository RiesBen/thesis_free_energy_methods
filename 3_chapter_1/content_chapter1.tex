\makeatletter
\def\input@path{{../}}
\makeatother
\documentclass[../main.tex]{subfiles}
\begin{document}
\renewcommand{\path}{3_chapter_1/}
\chapter[Toy Modelling for teaching and developing Methodology]{Using simple Models to understand and develop Methodology - Ensembler
    \footnote{\label{footnoteChapter1CopyRight} 
    Reprinted (adapted) with permission from Benjamin Ries, Stephanie M. Linker, David F. Hahn, Gerhard König and Sereina Riniker ,
    J. Chem. Inf. Model., \textbf{61}, 560-564 (2021). Copyright 2021 American Chemical Society.}
}
\chaptermark{Free Energy Calculation}
\label{ch:feens}

\aquote{
"It all works because Avogadro's number is closer to infinity than to 10."
}{R. Baierlein; Gromacs quote collection \cite{Abraham2015}}




\begin{abstract}
Ensembler is a python package that allows for fast and easy access to the simulation of one and two-dimensional model systems.
It enables method development using small test systems and to deepen the understanding of a broad spectrum of molecular dynamics (MD) methods, starting from basic techniques to enhanced sampling and free-energy calculations.
The ease of installing and using the package increases shareability, comparability, and reproducibility of scientific code developments.
Here, we provide a description of the implementation and usage of the package as well as an application example for free-energy calculation.
The code of Ensembler is freely available on GitHub \textit{https://github.com/rinikerlab/Ensembler}.
%

\end{abstract}

\clearpage
\pagebreak

%%%%%%%%%%%%%%%%%%%%%%%%%%%%%%%%%%%%%%%%%%%%%%%%%%%%%%%%%%%%%%%%%%%%%
%% Start the main part of the manuscript here.
%%%%%%%%%%%%%%%%%%%%%%%%%%%%%%%%%%%%%%%%%%%%%%%%%%%%%%%%%%%%%%%%%%%%%
%================================================================================
\section{Introduction}
%================================================================================

%The Problem and the Solution
Newly developed advanced simulation methods are routinely tested on simple one- and two-dimensional model systems. They provide valuable insights into the theory, conceptual advantages and limitations (for examples see e.g. Refs. \citenum{Huber1994, Laio2002, Christ2007, Konig2012, Koenig2020, Donnini2016, Weiss2016, Lemke2018}).
While the results of new methods are published, the implementation details may not always be available or difficult to use with different computer infrastructure.
As a result, sharing, reproducing, understanding, and comparing simulation methodologies is often cumbersome.\cite{Peng2011}
To address this issue, we have developed the Ensembler package, an easy-to-use, yet powerful platform that enables fast prototyping of new methods and comparison against existing techniques using one or two-dimensional test systems.

%Global ethical goal
Ensembler is designed following the recommendations of Stodden \textit{et al.}\cite{Stodden2016} for the enhanced reproducibility of computational methods, which includes making code publicly accessible, providing documentation, and using open licensing.\cite{Stodden2016} 
Furthermore, Ensembler uses state-of-the-art software engineering tools (i.e. git,\cite{Chacon2014} MolSSI cookie-cutter,\cite{Naden2018} and Binder\cite{Binder2018}/Colab\cite{Bisong2019}) to fulfill these recommendations and enable features like continuous integration and the transparent versioning of the code. 

%-------------------------------------------------------------------------------------------------------
\subsection{Method Development}
%-------------------------------------------------------------------------------------------------------

%Why not using normal MD-Packages
The lean Python3 code\cite{Vanrossum2009} of Ensembler allows for easy prototyping of new methods and comparison against a wide range of already implemented techniques. 
In contrast, the C/C++\cite{Stroustrup1995} code of traditional high-performance molecular dynamics (MD) packages (e.g. Refs. \citenum{Berendsen1995,Lindahl2001,Vanderspoel2005,Eastman2017,Brooks2009}) is more efficient but also much more complex. 
%
%What we got
The methods currently available in Ensembler are:
\begin{itemize}
	\item \textit{Model systems}: Harmonic oscillators as well as dihedral-angle, double-well, and Lennard-Jones potential-energy functions\cite{Jones1924}
	\item \textit{Sampling algorithms}: Conjugated gradient\cite{Hestenes1952} for energy minimization, Metropolis Monte Carlo (MC),\cite{Hastings1970} leap-frog integration\cite{Vangunsteren1988} for MD, and Langevin integration\cite{Brunger1984} for stochastic dynamics (SD)
	\item \textit{Enhanced sampling techniques}: Umbrella sampling,\cite{Valleau1977} simulated tempering/temperature replica-exchange simulations,\cite{Sugita1999} local elevation/metadynamics\cite{Huber1994, Laio2002}
	\item \textit{Free-energy methods}: Free-energy perturbation (FEP),\cite{Zwanzig1954} Bennett's acceptance ratio (BAR),\cite{Bennett1976} thermodynamic integration (TI),\cite{Kirkwood1935} enveloping distribution sampling (EDS),\cite{Christ2007, Christ2008, Christ2009} $\lambda$-EDS,\cite{Koenig2020} replica-exchange EDS (RE-EDS),\cite{Sidler2016} and conveyor-belt TI\cite{Hahn2019}
\end{itemize}

%Teaching
%-------------------------------------------------------------------------------------------------------
\subsection{Teaching}
%-------------------------------------------------------------------------------------------------------

Simple model systems can also be used for teaching MD concepts to students, as they allow to intuitively understand fundamental concepts. \cite{Pohorille2010} 
Ensembler is well suited for didactic purposes because it is not only easy to use, but supports also a range of visualizations, i.e. interactive widgets, animations, and plots, which can be embedded in Jupyter notebooks.\cite{Kluyver2016}
Example Jupyter notebooks\cite{Kluyver2016} are provided in the Ensembler GitHub repository.

%================================================================================
\section{Theory}
%================================================================================



Ensembler is implemented in Python3\cite{VanRossum2009} and available on GitHub\cite{Github2020}  (\textit{\hyperlink{https://github.com/rinikerlab/Ensembler}{rinikerlab/Ensembler}}). 
The repository is based on the template of the MolSSI cookie-cutter\cite{Naden2018} and comprises a code folder, an example folder for tutorials, example models contained in the provided Jupyter notebooks,\cite{Kluyver2016} an automatic pytest suite,\cite{Krekel2004} and the automatically generated sphinx \cite{Brandl2008} documentation. 
The code is continuously integrated via GitHub Actions,\cite{githhubAction20} providing information about code quality, test correctness, test coverage, and generation of an up-to-date documentation. 
Ensembler uses only open-source packages like the SciPy library\cite{Virtanen2020, VanDerWalt2011, Meurer2017, Mckinney2010, Hunter2007} and Jupyter notebooks.\cite{Kluyver2016} 
In the following, a user and a developer perspective are provided for the code structure. 

%-------------------------------------------------------
\subsection{User level}
%-------------------------------------------------------
 A simulation model in Ensembler consists of a potential class, a sampler class, and a system class wrapping the potential and the sampler (Figure~\ref{fig:UML-Diagramm}), and provides control over the simulation approach. 
Additionally, multiple condition classes can be added that directly influence the simulation (e.g. periodic boundary condition\cite{Cheatham1995, Leach2001} or  thermostatting\cite{Andersen1980}). 
After the construction of the system, the simulation can be started directly with the \textit{simulate} function. 
The resulting trajectory is in the form of a Pandas data frame.\cite{Mckinney2010} The trajectory is thus easily compatible with other packages like NumPy\cite{VanDerWalt2011} or scikit-learn\cite{scikit-learn} and can be stored in different formats, e.g. as .csv or .hf5 file. The system itself can be stored directly via the save function using serialization of the object with the Python package pickle.
In most cases, only a few additional lines are needed to go from simple simulation technique to more advanced one, as shown below. 

%-------------------------------------------------------
\subsection{Developer level}
%-------------------------------------------------------
The code of Ensembler is built on five interface-like base classes that allow extensive use of the inheritance concept and polymorphism \cite{Stroustrup1995} throughout the package.
These fundamental classes are \textit{potential}, \textit{sampler}, \textit{condition}, \textit{system}, and \textit{ensemble} (Figure \ref{fig:UML-Diagramm}), which can be grouped into three layers.
\textit{Potential}, \textit{sampler}, and \textit{condition classes} form the primary layer, providing different techniques to be used as components in a simulation. 
\textit{Potential classes} provide the potential-energy functions in a symbolic form using SymPy,\cite{Meurer2017} enabling automatic on-the-fly derivation and simplification of the potential-energy function. 
\textit{Sampler classes} are used to explore the potential-energy function (e.g. conjugate gradient,\cite{Hestenes1952} Metropolis MC,\cite{Hastings1970} or leap-frog\cite{VanGunsteren1988} integration). A new method can easily be implemented by inheriting from the \textit{sampler class} and overwriting a single function called \textit{step}. 
Finally, \textit{condition classes} provide additional functionalities such as thermostatting\cite{Andersen1980} and periodic boundary conditions\cite{Cheatham1995, Leach2001}). New techniques can be implemented by inheriting the base \textit{condition class} and overwriting the function \textit{apply}.
In the second layer, the first-layer components are wrapped into one \textit{system class} that executes the simulation(s) and manages the input and output. 
An optional higher-order layer is available in form of the \textit{ensemble class}, which allows the user to perform simulations with replica exchange.\cite{Sugita1999, Sugita2000, Yamauchi2017, Sidler2016a}
If additional parameters are needed in a newly designed class, the constructor of the new child class can be adapted but must call the parent constructor.


%================================================================================
\section{Results and Discussion}
%================================================================================

\labsec{results}
%----------
\subsection{Reference TI Calculations}
%----------


\clearpage
\newpage

%================================================================================
\section{Conclusion}
%================================================================================

In this work, we introduced the Ensembler package as a tool to support teaching of MD simulations and free-energy techniques, and to enable rapid prototyping of new methods using 1D or 2D model systems. The package provides a large set of implemented methods for comparison. The open-source basis, the lean structure, and the simplicity of Python3 form a convenient and efficient framework. The code examples and application example for free-energy calculation highlight the ease of using Ensembler. With this, Ensembler can contribute to improving the shareability, comparability, and reproducibility for method development in our field.

\clearpage
\pagebreak


\bibliography{MendeleyMerge.bib}

\end{document}
