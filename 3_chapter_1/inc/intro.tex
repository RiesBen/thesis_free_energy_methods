%The Problem and the Solution
Newly developed advanced simulation methods are routinely tested on simple one- and two-dimensional model systems. They provide valuable insights into the theory, conceptual advantages and limitations (for examples see e.g. Refs. \cite{Huber1994, Laio2002, Christ2007, Konig2012a, Koenig2020, Donnini2016, Weiß2016, Lemke2018}).
While the results of new methods are published, the implementation details may not always be available or difficult to use with different computer infrastructure.
As a result, sharing, reproducing, understanding, and comparing simulation methodologies is often cumbersome.\cite{Peng2011}
To address this issue, we have developed the Ensembler package, an easy-to-use, yet powerful platform that enables fast prototyping of new methods and comparison against existing techniques using one or two-dimensional test systems.

%Global ethical goal
Ensembler is designed following the recommendations of Stodden \textit{et al.}\cite{Stodden2016} for the enhanced reproducibility of computational methods, which includes making code publicly accessible, providing documentation, and using open licensing.\cite{Stodden2016} 
Furthermore, Ensembler uses state-of-the-art software engineering tools (i.e. git,\cite{Chacon2014} MolSSI cookie-cutter,\cite{Naden2018} and binder\cite{Jupyter2018}) to fulfill these recommendations and enable features like continuous integration and the transparent versioning of the code. 

%-------------------------------------------------------------------------------------------------------
\section{Method Development}
%-------------------------------------------------------------------------------------------------------

%Why not using normal MD-Packages
The lean Python3 code\cite{VanRossum2009} of Ensembler allows for easy prototyping of new methods and comparison against a wide range of already implemented techniques. 
In contrast, the C/C++\cite{Stroustrup1995} code of traditional high-performance molecular dynamics (MD) packages (e.g. Refs. \cite{Berendsen1995,Lindahl2001a,VanDerSpoel2005,Eastman2017,Brooks2009}) is more efficient but also much more complex. 
%
%What we got
The methods currently available in Ensembler are:
\begin{itemize}
	\item \textit{Model systems}: Harmonic oscillators as well as dihedral-angle, double-well, and Lennard-Jones potential-energy functions\cite{Jones1924}
	\item \textit{Sampling algorithms}: Conjugated gradient\cite{Hestenes1952} for energy minimization, Metropolis Monte Carlo (MC),\cite{Hastings1970} leap-frog integration\cite{VanGunsteren1988} for MD, and Langevin integration\cite{Brunger1984} for stochastic dynamics (SD)
	\item \textit{Enhanced sampling techniques}: Umbrella sampling,\cite{Valleau1977} simulated tempering/temperature replica-exchange simulations,\cite{Sugita1999} local elevation/metadynamics,\cite{Huber1994, Laio2002}
	\item \textit{Free-energy methods}: Free-energy perturbation (FEP),\cite{Zwanzig1954} Bennett's acceptance ratio (BAR),\cite{Bennett1976} thermodynamic integration (TI),\cite{Kirkwood1935} enveloping distribution sampling (EDS),\cite{Christ2007, Christ2008a, Christ2009} $\lambda$-EDS,\cite{Koenig2020} replica-exchange EDS (RE-EDS),\cite{Sidler2016a} and conveyor-belt TI\cite{Hahn2019}
\end{itemize}

%Teaching
%-------------------------------------------------------------------------------------------------------
\section{Teaching}
%-------------------------------------------------------------------------------------------------------

Simple model systems can also be used for teaching MD concepts to students, as they allow to intuitively understand fundamental concepts. \cite{Pohorille2010} 
Ensembler is well suited for didactic purposes because it is not only easy to use, but supports also a range of visualizations, i.e. interactive widgets, animations, and plots, which can be embedded in Jupyter notebooks.\cite{Kluyver2016}
Example Jupyter notebooks\cite{Kluyver2016} are provided in the Ensembler GitHub repository.