%%%%%%%%%%%%%%%%%%%%%%%%%%%%%%%%%%%%%%%%%%%%%%%%%%%%%%%%%%%%%%%%%%%%%
%% This is a (brief) model paper using the achemso class
%% The document class accepts keyval options, which should include
%% the target journal and optionally the manuscript type.
%%%%%%%%%%%%%%%%%%%%%%%%%%%%%%%%%%%%%%%%%%%%%%%%%%%%%%%%%%%%%%%%%%%%%
\documentclass[journal=jctcce,manuscript=article,12pt]{achemso}
\batchmode

%%%%%%%%%%%%%%%%%%%%%%%%%%%%%%%%%%%%%%%%%%%%%%%%%%%%%%%%%%%%%%%%%%%%%
%% Place any additional packages needed here.  Only include packages
%% which are essential, to avoid problems later. Do NOT use any
%% packages which require e-TeX (for example etoolbox): the e-TeX
%% extensions are not currently available on the ACS conversion
%% servers.
%%%%%%%%%%%%%%%%%%%%%%%%%%%%%%%%%%%%%%%%%%%%%%%%%%%%%%%%%%%%%%%%%%%%%
\usepackage[version=4]{mhchem} % Formula subscripts using \ce{}
\usepackage[T1]{fontenc}       % Use modern font encodings
\usepackage{amssymb}
\usepackage{geometry}
\usepackage[dvipsnames]{xcolor}
\usepackage{colortbl}
%\usepackage{citesort}
\usepackage[export]{adjustbox}
\usepackage{epstopdf}
\usepackage{pdflscape}
\usepackage{subcaption}
\usepackage[ddmmyy,hhmmss]{datetime}
\usepackage[nomessages]{fp}
\usepackage{tikz} 
\usepackage{pgfplots}
\usepackage{csquotes}
\pgfplotsset{compat=1.13}
\usepackage{xifthen}
\usepackage{tcolorbox}
\usepackage{subfiles}
\usepackage{longtable}
\usepackage{xr}
\usepackage{etoolbox}
\usepackage{booktabs}
\usepackage{setspace}
\usepackage{xstring}
\usepackage{fancyhdr}
\usepackage{multirow}
\RequirePackage{epigraph}
\renewcommand{\headrulewidth}{0pt} 


%%%%%%%%%%%%%%%%%%%%%%%%%%%%%%%%%%%%%%%%%%%%%%%%%%%%%%%%%%%%%%%%%%%%%
%% If issues arise when submitting your manuscript, you may want to
%% un-comment the next line.  This provides information on the
%% version of every file you have used.
%%%%%%%%%%%%%%%%%%%%%%%%%%%%%%%%%%%%%%%%%%%%%%%%%%%%%%%%%%%%%%%%%%%%%
%%\listfiles

%%%%%%%%%%%%%%%%%%%%%%%%%%%%%%%%%%%%%%%%%%%%%%%%%%%%%%%%%%%%%%%%%%%%%
%% Place any additional macros here.  Please use \newcommand* where
%% possible, and avoid layout-changing macros (which are not used
%% when typesetting).
%%%%%%%%%%%%%%%%%%%%%%%%%%%%%%%%%%%%%%%%%%%%%%%%%%%%%%%%%%%%%%%%%%%%%
\newcommand*\mycommand[1]{\texttt{\emph{#1}}}
\newcommand{\nc}{\newcommand}

\nc{\captitital}[1]{\textit{#1}}

\nc{\ie}{\textit{i.e.}}
\nc{\via}{\textit{via}}
\nc{\eg}{\textit{e.g.}}
\nc{\vs}{\textit{vs.}}
\newcommand{\cf}{\textit{cf.}}
\nc{\suppmat}{}
\nc{\suppmatv}{}


%==============================================================================
% COLORS
%==============================================================================



\definecolor{m1}{HTML}{FF00FF}%{Magenta}%HTML}{FFB719}
\definecolor{m2}{HTML}{264bCF}
\definecolor{rec}{HTML}{83401B}
\definecolor{ts}{HTML}{0069B4}
\definecolor{hist}{HTML}{C86F3A}


\definecolor{flap2}{HTML}{C8176B}
\definecolor{flap2d1}{HTML}{F360A6}
\definecolor{flap2d4}{HTML}{EB1278}
\definecolor{flap2d2}{HTML}{AF0053}
\definecolor{flap2d3}{HTML}{780039}
\definecolor{flap4}{HTML}{EF7E1C}
\definecolor{flap4d1}{HTML}{FFAC65}
\definecolor{flap4d4}{HTML}{FF8013}
\definecolor{flap4d2}{HTML}{D16100}
\definecolor{flap4d3}{HTML}{904200}
\definecolor{flap1}{HTML}{128B91}
\definecolor{flap1d1}{HTML}{5BDCE2}
\definecolor{flap1d4}{HTML}{11C7CF}
\definecolor{flap1d2}{HTML}{01797F}
\definecolor{flap1d3}{HTML}{005357}
\definecolor{flap3}{HTML}{90DD1A}
\definecolor{flap3d1}{HTML}{BEF963}
\definecolor{flap3d4}{HTML}{9CF513}
\definecolor{flap3d2}{HTML}{75C100}
\definecolor{flap3d3}{HTML}{508400}

\definecolor{colsdlo}{HTML}{A038DA}
\definecolor{colsdmi}{HTML}{5F088F}
\definecolor{colsdhi}{HTML}{440367}
\definecolor{colchcl3lo}{HTML}{3BEA32}
\definecolor{colchcl3mi}{HTML}{09AD00}
\definecolor{colchcl3hi}{HTML}{067D00}
\definecolor{coltollo}{HTML}{FFBC36}
\definecolor{coltolmi}{HTML}{D68F00}
\definecolor{coltolhi}{HTML}{9B6700}


%==============================================================================
% MATH
%==============================================================================
%%%%%%%%%%%%%%%%%%%%%%%%%%%%%%%%%%%%%%%%%%%%%%%%%%%%%%%%
% Math
\nc{\dd}{\mathrm{d}}
\nc{\ee}{\mathrm{e}}
\nc{\fvector}[1]{\mathbf{#1}}
\nc{\fmatrix}[1]{\mathbf{\underline{#1}}}

\nc{\vecThreeD}[3] {
   \begin{pmatrix} #1 \\ #2 \end{pmatrix}
}

\nc{\lam}{\lambda}
\nc{\Lamb}{\Lambda}

\nc{\dhdl}{\frac{\partial \mathcal{H}}{\partial\lam}}
\nc{\dhdlav}{\left\langle\frac{\partial \mathcal{H}}{\partial\lam}\right\rangle}

\nc{\ham}{\mathcal{H}}

\nc{\bia}{\mathcal{B}}
\nc{\bio}{\mathcal{P}}

\nc{\pot}{\mathcal{U}}
\nc{\kin}{\mathcal{K}}
\nc{\frc}{\mathcal{F}}

\nc{\kb}{k_{\mathrm{B}}}
\nc{\rv}{\mathbf{r}}
\nc{\pv}{\mathbf{p}}
\nc{\xv}{\mathbf{x}}
\nc{\qv}{\mathbf{q}}
\nc{\Fv}{\mathbf{F}}
\nc{\xiv}{\boldsymbol{\xi}}

\nc{\Xv}{\mathbf{X}}
\nc{\lamv}{\boldsymbol{\lambda}}

\nc{\Dv}{\mathbf{D}}
\nc{\Cmat}{\underline{\mathbf{C}}}
\nc{\unit}[1]{\ensuremath{\,\mathrm{#1}}}


\nc{\std}{\standardstate}


%==============================================================================
% EQUATION MACROS
%==============================================================================

\nc{\beq}[1]{\begin{equation}\label{eq:#1}}
\nc{\eeq}{\end{equation}}
\nc{\refeq}[1]{Eq.~\ref{eq:#1}}
\nc{\refeqs}[1]{Eqs.~\ref{eq:#1}}
\nc{\refeqn}[1]{~\ref{eq:#1}}

%==============================================================================
% FIGURE MACROS
%==============================================================================

\nc{\reffig}[1]{Fig.~\ref{fig:#1}}
\nc{\reffigs}[1]{Figs.~\ref{fig:#1}}
\nc{\reff}[1]{\ref{fig:#1}}
\nc{\reffign}[1]{\ref{fig:#1}}
\nc{\reffignn}[1]{\ref{fig:#1}}

\nc{\refsifig}[1]{Fig.~\ref{fig:#1}}
\nc{\refsifigs}[1]{Figs.~\ref{fig:#1}}
\nc{\refsifign}[1]{~\ref{fig:#1}}
\nc{\refsifignn}[1]{\ref{fig:#1}}

\nc{\refmainfig}[1]{Fig.~\ref{fig:#1}}
\nc{\refmainfigs}[1]{Figs.\ref{fig:#1}}
\nc{\refmainfign}[1]{\ref{fig:#1}}
\nc{\refmainfignn}[1]{\ref{fig:#1}}

%==============================================================================
% TABLE MACROS
%==============================================================================

\nc{\reftab}[1]{Tab.~\ref{tab:#1}}
\nc{\reftabs}[1]{Tabs.~\ref{tab:#1}}
\nc{\reftabn}[1]{~\ref{tab:#1}}
\nc{\reftabnn}[1]{\ref{tab:#1}}

\nc{\refsitab}[1]{Tab.~\ref{tab:#1}}
\nc{\refsitabs}[1]{Tabs.~\ref{tab:#1}}
\nc{\refsitabn}[1]{~\ref{tab:#1}}
\nc{\refsitabnn}[1]{\ref{tab:#1}}

\nc{\refmaintab}[1]{Tab.~\ref{tab:#1}}
\nc{\refmaintabs}[1]{Tabs.~\ref{tab:#1}}
\nc{\refmaintabn}[1]{\ref{tab:#1}}
\nc{\refmaintabnn}[1]{\ref{tab:#1}}


%==============================================================================
% SECTION MACROS
%==============================================================================

\nc{\refch}[1]{Chapter~\ref{ch:#1}}
\nc{\refchs}[1]{Chapters~\ref{ch:#1}}
\nc{\refchn}[1]{~\ref{ch:#1}}
\nc{\refchnn}[1]{\ref{ch:#1}}

\nc{\refsec}[1]{Sect.~\ref{sec:#1}}
\nc{\refsecs}[1]{Sects.~\ref{sec:#1}}
\nc{\refsecn}[1]{~\ref{sec:#1}}
\nc{\refsecnn}[1]{\ref{sec:#1}}



\nc{\refsisec}[1]{Appendix~\ref{sec:#1}}
\nc{\refsisecs}[1]{Appendices\ref{sec:#1}}
\nc{\refsisecn}[1]{~\ref{sec:#1}}
\nc{\refsisecnn}[1]{\ref{sec:#1}}

\nc{\labsec}[1]{\label{sec:#1}}


\externaldocument[SI-]{suppmat}
%\externaldocument[SIB-]{inc/suppmatB}

%%%%%%%%%%%%%%%%%%%%%%%%%%%%%%%%%%%%%%%%%%%%%%%%%%%%%%%%%%%%%%%%%%%%%
%% Meta-data block
%% ---------------
%% Each author should be given as a separate \author command.
%%
%% Corresponding authors should have an e-mail given after the author
%% name as an \email command. Phone and fax numbers can be given
%% using \phone and \fax, respectively; this information is optional.
%%
%% The affiliation of authors is given after the authors; each
%% \affiliation command applies to all preceding authors not already
%% assigned an affiliation.
%%
%% The affiliation takes an option argument for the short name.  This
%% will typically be something like "University of Somewhere".
%%
%% The \altaffiliation macro should be used for new address, etc.
%% On the other hand, \alsoaffiliation is used on a per author basis
%% when authors are associated with multiple institutions.
%%%%%%%%%%%%%%%%%%%%%%%%%%%%%%%%%%%%%%%%%%%%%%%%%%%%%%%%%%%%%%%%%%%%%
%
%
\author{David F. Hahn}
\affiliation[ETH]
{Laboratory of Physical Chemistry, Department of Chemistry and Applied Biosciences, 
ETH Z\"urich, Vladimir-Prelog-Weg 2, 8093 Z\"urich, Switzerland}
%
%

%
\author{Philippe H. H\"unenberger}
\affiliation[ETH]
{Laboratory of Physical Chemistry, Department of Chemistry and Applied Biosciences, 
ETH Z\"urich, Vladimir-Prelog-Weg 2, 8093 Z\"urich, Switzerland}
\email{phil@igc.phys.chem.ethz.ch}
%\phone{+123 (0)123 4445556}
%\fax{+123 (0)123 4445557}

%%%%%%%%%%%%%%%%%%%%%%%%%%%%%%%%%%%%%%%%%%%%%%%%%%%%%%%%%%%%%%%%%%%%%
%% The document title should be given as usual. Some journals require
%% a running title from the author: this should be supplied as an
%% optional argument to \title.
%%%%%%%%%%%%%%%%%%%%%%%%%%%%%%%%%%%%%%%%%%%%%%%%%%%%%%%%%%%%%%%%%%%%%
\title[CBUS]
      {Intro\\
        \large{\normalfont Document date: \today}}

%Closing-opening (\vase{}-\kite{}) equilibrium of resorcin[4]arene cavitands
%investigated using molecular dynamics simulations
%with ball-and-stick local elevation umbrella sampling

%%%%%%%%%%%%%%%%%%%%%%%%%%%%%%%%%%%%%%%%%%%%%%%%%%%%%%%%%%%%%%%%%%%%%
%% Some journals require a list of abbreviations or keywords to be
%% supplied. These should be set up here, and will be printed after
%% the title and author information, if needed.
%%%%%%%%%%%%%%%%%%%%%%%%%%%%%%%%%%%%%%%%%%%%%%%%%%%%%%%%%%%%%%%%%%%%%
\abbreviations{MD, MC, CBTI, OSP, TI, HRE, MC, $\lam$D, EXTI, BAR, MBAR, $\lam$MC, LEUS, LE, US, B\&S-LEUS, FEP, SC, EDS, CB}
%\keywords{American Chemical Society, \LaTeX}

\renewcommand\path{./}
%%%%%%%%%%%%%%%%%%%%%%%%%%%%%%%%%%%%%%%%%%%%%%%%%%%%%%%%%%%%%%%%%%%%%
%% The manuscript does not need to include \maketitle, which is
%% executed automatically.
%%%%%%%%%%%%%%%%%%%%%%%%%%%%%%%%%%%%%%%%%%%%%%%%%%%%%%%%%%%%%%%%%%%%%
\begin{document}

%%%%%%%%%%%%%%%%%%%%%%%%%%%%%%%%%%%%%%%%%%%%%%%%%%%%%%%%%%%%%%%%%%%%%
%% The "tocentry" environment can be used to create an entry for the
%% graphical table of contents. It is given here as some journals
%% require that it is printed as part of the abstract page. It will
%% be automatically moved as appropriate.
%%%%%%%%%%%%%%%%%%%%%%%%%%%%%%%%%%%%%%%%%%%%%%%%%%%%%%%%%%%%%%%%%%%%%
%% \begin{tocentry}

%% Some journals require a graphical entry for the Table of Contents.
%% This should be laid out ``print ready'' so that the sizing of the
%% text is correct.

%% Inside the \texttt{tocentry} environment, the font used is Helvetica
%% 8\,pt, as required by \emph{Journal of the American Chemical}
%% The surrounding frame is 9\,cm by 3.5\,cm, which is the maximum
%% permitted for  \emph{Journal of the American Chemical Society}
%% graphical table of content entries. The box will not resize if the
%% content is too big: instead it will overflow the edge of the box.

%% This box and the associated title will always be printed on a
%% separate page at the end of the document.

%% \end{tocentry}

%%%%%%%%%%%%%%%%%%%%%%%%%%%%%%%%%%%%%%%%%%%%%%%%%%%%%%%%%%%%%%%%%%%%%
%% The abstract environment will automatically gobble the contents
%% if an abstract is not used by the target journal.
%%%%%%%%%%%%%%%%%%%%%%%%%%%%%%%%%%%%%%%%%%%%%%%%%%%%%%%%%%%%%%%%%%%%%
\begin{abstract}
\fancyhead[L]{To be submitted to: XXX}
\thispagestyle{fancy}
%  \tbref{TO BE WRITTEN LAST}
%\chapter{Summary}

Molecular dynamics (MD) simulations offer nowadays a valuable 
tool for studying phenomena in chemistry and biology.
%
The history of models in chemistry, as well as the methodological 
and theoretical background of MD is presented in \refch{intro}.

\refch{resa} deals with the application of explicit-solvent molecular dynamics (MD)
simulations to resorcin[4]arene cavitands, which
can adopt a close/contracted (\vase{})
and an open/expanded (\kite{}) conformation.
%
The \vase-\kite{} equilibria of a quinoxaline- and a dinitrobenzene-based 
resorcin[4]arene are investigated in three solvent environments (vacuum, chloroform and toluene)
and at three temperatures (198.15, 248.15 and 298.15 K).
%
The challenge of sampling the millisecond-timescale \vase-\kite{} transition
%, which occurs experimentally
%on the milisecond timescale, represents a challenge in terms of sampling.
%
%It 
is addressed by calculating relative free energies using ball-and-stick local 
elevation umbrella sampling (B\&S-LEUS) to promote interconversion transitions.
%
The calculated \vase-to-\kite{} free-energy changes $\Delta G$
% between \vase{} and \kite{} calculated
are in qualitative agreement with the 
experimental magnitudes and trends.
%


\refchs{cbti}-\refchnn{cbus} present the development of the conveyor belt scheme,
a method to calculate free-energy differences.
%
The working principle relies on $K$ coupled replicas of the system that are 
simulated at different values of a coupling parameter $\lambda$.
%
The number $K$ is taken to be even and the replicas are equally spaced
on a forward-turn-backward-turn path, akin to a conveyor belt (CB)
between the two end-states.
%
As in $\lambda$-dynamics ($\lam$D), the $\lambda$-values associated with the 
individual systems evolve in time along the simulation.
%
However, they do so in a concerted fashion, determined by the evolution
of a single dynamical variable $\Lambda$ of period $2\pi$ controlling 
the advance of the entire CB.
%
Thus, a change of $\Lambda$ is always
associated with one half of the replicas moving forward and the other half moving backward along $\lambda$.
%
As a result, the effective free-energy profile of the 
replica system along $\Lambda$ is characterized with 
decrasing barriers upon increasing $K$, at least as $K^{-1}$ \radd{in the limit of large $K$}.
%
When \radd{a sufficient number of replicas is used}, 
these variations become small, 
which enables a complete and quasi-homogeneous coverage of the $\lambda$-range
by the replica system.
%


\refch{cbti} introduces this scheme with respect to
alchemical free-energy calculations. Therefore it is termed \textit{conveyor belt thermodynamic integration}.
It provides the mathematical/physical formulation of the scheme, along with an initial 
application of the method to the calculation of the hydration free energy of methanol.

In \refch{ortho}, the conveyor belt thermodynamic integration is applied to 
a Lys-X-Lys tripeptide, involving a side-chain mutation in the central residue,
and guanosine triphosphate, involving a hydrogen-to-bromine mutation.
With both systems, sampling issues have been encountered, due to 
the large orthogonal barriers either along the backbone 
dihedral angles $\phi$ and $\psi$ (tripeptide) or along 
ribose-base dihedral angle $\chi$ (guanosine). 
This relative merits of different sampling schemes, orthogonal biasing and estimators to improve the convergence 
are investigated.
%
%
The thermodynamic integration scheme is shown to suffer from the 
constraint of simulations at fixed $\lambda$.
%
There is no significant improvement upon changing 
from the Simpson's quadrature to the MBAR estimator.
%
Both Hamiltonian replica exchange and conveyor belt 
thermodynamic integration improve the results for
the tripeptide.
%
For the guanosine triphosphate, improvement
is only achieved upon application of an orthogonal biasing potential, 
most efficiently in combination with Hamiltonian replica exchange or
conveyor belt thermodynamic integration.
%

The conveyor belt scheme is extended to the calculation of
conformational free-energy differences in \refch{cbus},
resulting in a so-called conveyor belt umbrella sampling scheme (\CBUS).
%
CBUS is here initally applied to the calculation of 45 standard absolute binding free energies 
of five alkali cations to three crown ethers in three different solvents.
%
Besides introducing and testing the new scheme, it is compared to 
other methods.
%
Expectedly, the direct counting approach has convergence issues
on the accessible simulation timescale, and the corresponding results
are rather unreliable. On the other hand, the results obtained 
using traditional umbrella sampling and CBUS are
very consistent. 
%
Additionally, comparison of the results to
those of previous alchemical calculations {\em via} an
alchemical pathway reveals excellent 
consistency while the trends of available experimental data
are qualitatively reproduced.

Conclusions and an outlook into future developments
are given in \refch{outlook}.

\end{abstract}

%\phicom{
%TO LOOK AT:
%- are there previous calculations (QM,MD) on resorcinarenes -
%  to be mentioned in the intro
%}

\pagebreak
%================================================================================
\section{Introduction}
\labsec{intro}
%================================================================================

Macrocycles have recently gathered increased interest in medicinal chemistry as beyond rule-of-5 (bRO5) molecules. \cite{Driggers2008, Mallinson2012, Doak2014, Dougherty2017, Marsault2011, Abdalla2018, Marsault2017, Caron2021}
A key feature of these molecules is their conformational complexity that can be leveraged in drug design to target protein-protein interactions. \cite{ Chene2006, Janin2008, Jones13, Scott2016, Modell2016}
Such protein-protein interactions are typically characterized by large flat binding sites that are difficult to target with small molecules. \cite{Doak2016}
If the macrocycles are peptidic, their toxicity is often relatively low. \cite{Zorzi2017}
Most Food and Drug Administration (FDA)-approved macrocyclic drugs belong to natural products (e.g., erythromycin, tacrolimus) or peptides (e.g., sandostatin, eptifibatide). \cite{Giordanetto2014}
Peptidic or semipeptidic scaffolds bridge the gap between small molecules and biologics. An advantage of this molecule class is that they are relatively easy to synthesis and allow a broad choice of natural and non-natural amino acids required for rapid and thorough pharmacophoric exploration. 
The main challenge with peptides resides in their physicochemical and pharmacokinetics-ADME (absorption, distribution, metabolism, and excretion) properties. 
While cyclic peptides are typically more stable to proteases compared to their linear counterparts, their high polarity often translates into low bioavailability.\cite{Naylor2017, Fosgerau2015}
However, some cyclic peptides were found to cross cell membranes.\cite{Naylor2017, Wang2014, Nielsen2014} 
Developing tools and knowledge to optimize and better predict their structure–permeability relationship is, therefore, a requirement for the field. Such quest found inspiration in studies of the cyclic undecamer cyclosporine A, which is administered orally. 
One prominent structural feature of this natural macrocycle is the high number of N-methylated residues (7 out of 11) and its dynamic structural adaptation to its environment described as chameleonic behavior. \cite{Whitty2016, Danelius2020, Witek2017}
The effect of N-methylation on the permeability of cyclic hexa- and heptapeptides has been systematically investigated since the number and position of N-methylations may be beneficial or detrimental for permeability. \cite{Nielsen2014, Raeder2018, White2011, Beck2012, Biron2008, White2011} 
%
Less explored are the N-alkylated glycines -- aka peptoids -- in which the side chain has been moved from the $\alpha$-carbon to the amide nitrogen. \cite{Schwochert2015} 
Similar to N-methylation, this modification removes one H-bond donor and removes one stereogenic center, and induces glycine-like secondary structures.
The peptoid amide also facilitates cis-trans isomerization compared to the corresponding N-methylation.\cite{Sui2007} 


More recently, the impact of the dynamics of macrocycles in response to their environment, which can range from polar in water, nonhomogeneous in the presence of its target, to lipophilic in the membrane, has been appreciated.\cite{Danelius2020, Witek2017, Riniker2019, Witek2019, Wang2021}
Studying macrocycles with computational methods leads to multiple criteria identified as being possibly essential for chameleonic behavior. Examples of these criteria are the presence of intramolecular H-bonds, 3D polar surface areas (3D-PSA), or kinetic Markov models as metrics for how macrocyclic structures yield polar atoms and rigidification of the backbone cycle into certain polar/apolar states. \cite{Witek2016, Witek2017, Tyagi2018, Witek2019, Wang2021}
A powerful tool to modulate the properties of peptidic macrocycles is the inclusion of a nonpeptidic tether unit.\cite{Marsault2007, Hoveyda2011, Roux2020} 
This tether can serve multiple purposes: in the context of a target interacting with a specific sequence, various tethers can be screened without modifying the peptide recognition sequence, while providing a simple handle for modulating affinity and pharmacokinetic properties. 
Small modifications in size, shape, or functional groups on the tether can dramatically influence on this kind of constrained system.\cite{Appavoo2019}
%
The relationship between structure and permeability is known to be elusive for this class of compounds, with small structural modifications often yielding permeability cliffs. \cite{Wang2014, Raeder2018, Beck2012, White2011, Roux2020, Bockus2015, Hewitt2015, Rezai2006, Over2016}


To investigate the structural effects of a tether with a length of five atoms and the peptide-peptoid change on the compound permeability, our collaborators synthesis a collection of 36 semipeptidic macrocycles. \cite{Comeau2021, Roux2020}
The structure of the compounds was composed of a tripeptide tethered head-to-tail with a nonpeptidic linker (Figure \ref{fig:MolDes}). 
Two classes of modifications were explored: single peptoid replacement and regio- and stereocontrolled linker C-methylation. 
%
\begin{figure}[h!]
    \centering
    \includegraphics[width=\textwidth]{7_chapter_5/fig/intro/MoleculeDesign.jpeg}
    \caption{Synthesis strategy of our collaborators for model compound (\textbf{A}) and two types of modifications: Nala, Nleu, and Nphe peptoids (\textbf{B} showing Nleu) and regio/stereocontrolled C-methylation (\textbf{C} showing 2R methylation).\cite{Comeau2021}}
    \label{fig:MolDes}
\end{figure}

\begin{figure}[h!]
    \centering
    \includegraphics[width=\textwidth]{7_chapter_5/fig/intro/pampa.jpeg}
    \caption{Permeability results in the form of heatmaps. For heatmaps 1–3, the values are expressed as $−log(P_{\text{app}})$, so lower values mean higher permeability (in order of increasing permeability: blue, white, red, and black). Heatmap 4 shows the BA/AB ratio, which represents a measure of efflux.}
    \label{fig:permAssays}
\end{figure}
The passive permeability of the resulting macrocycles was measured by our collaborators in the parallel artificial membrane permeability assay (PAMPA)\cite{Ottaviani2006,Di2015} and their cellular permeability in the Caco-2 assay\cite{Fogh1977,Di2015} (Figure \ref{fig:permAssays}).\cite{Comeau2021}

%
\begin{figure}
    \centering
    \includegraphics[width=\textwidth]{7_chapter_5/fig/intro/permCliffMols.jpeg}
    \caption{Four semipeptidic macrocycles were selected from the collection. In contrast to the pair Nleu-2R/S (bottom), the pair Nleu-5R/S (top) behave significantly  different in the permeability assays. All molecules were  studied with experimental NMR analysis and molecular dynamics (MD) simulations in a polar and apolar environment.}
    \label{fig:permCMols}
\end{figure}
Based on the permeability data, we selected two pairs of diastereomers that differ only by their stereochemistry of the tether methyl group (Figure \ref{fig:permCMols}). While one pair (Nleu-5R/S) differs greatly in their passive permeability behavior, the second one (Nleu-2R/S) does not. 
Prior studies on cyclosporine A showed that the conformational behavior of cyclic peptides in the context of membrane permeability can be studied by performing extensive molecular dynamics (MD) simulations in apolar and polar environments (e.g., chloroform and water) to mimic the behavior outside and inside a membrane. \cite{Witek2016,Witek2017, Witek2019, Wang2021}
Therefore, we carried out MD simulations of each of the four selected macrocycles in water and chloroform. The simulations results were validated by comparing to solution NMR measurements of the compounds. \cite{Balazs2019,Stadelmann2020}
Finally, we used different metrics such as torsional angles, hydrogen-bond formation, and 3D-PSA\cite{Sebastiano2018} to assess and compare the conformational behavior of the compounds. 



%================================================================================
\begin{thebibliography}{74}
 \input{inc/paper.lrs}
\end{thebibliography}
%================================================================================


\end{document}




