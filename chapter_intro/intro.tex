%%%%%%%%%%%%%%%%%%%%%%%%%%%%%%%%%%%%%%%%%%%%%%%%%%%%%%%%%%%%%%%%%%%%%
%% This is a (brief) model paper using the achemso class
%% The document class accepts keyval options, which should include
%% the target journal and optionally the manuscript type.
%%%%%%%%%%%%%%%%%%%%%%%%%%%%%%%%%%%%%%%%%%%%%%%%%%%%%%%%%%%%%%%%%%%%%
\documentclass[journal=jctcce,manuscript=article,12pt]{achemso}
\batchmode

%%%%%%%%%%%%%%%%%%%%%%%%%%%%%%%%%%%%%%%%%%%%%%%%%%%%%%%%%%%%%%%%%%%%%
%% Place any additional packages needed here.  Only include packages
%% which are essential, to avoid problems later. Do NOT use any
%% packages which require e-TeX (for example etoolbox): the e-TeX
%% extensions are not currently available on the ACS conversion
%% servers.
%%%%%%%%%%%%%%%%%%%%%%%%%%%%%%%%%%%%%%%%%%%%%%%%%%%%%%%%%%%%%%%%%%%%%
\usepackage[version=4]{mhchem} % Formula subscripts using \ce{}
\usepackage[T1]{fontenc}       % Use modern font encodings
\usepackage{amssymb}
\usepackage{geometry}
\usepackage[dvipsnames]{xcolor}
\usepackage{colortbl}
%\usepackage{citesort}
\usepackage[export]{adjustbox}
\usepackage{epstopdf}
\usepackage{pdflscape}
\usepackage{subcaption}
\usepackage[ddmmyy,hhmmss]{datetime}
\usepackage[nomessages]{fp}
\usepackage{tikz} 
\usepackage{pgfplots}
\usepackage{csquotes}
\pgfplotsset{compat=1.13}
\usepackage{xifthen}
\usepackage{tcolorbox}
\usepackage{subfiles}
\usepackage{longtable}
\usepackage{xr}
\usepackage{etoolbox}
\usepackage{booktabs}
\usepackage{setspace}
\usepackage{xstring}
\usepackage{fancyhdr}
\usepackage{multirow}
\RequirePackage{epigraph}
\renewcommand{\headrulewidth}{0pt} 


%%%%%%%%%%%%%%%%%%%%%%%%%%%%%%%%%%%%%%%%%%%%%%%%%%%%%%%%%%%%%%%%%%%%%
%% If issues arise when submitting your manuscript, you may want to
%% un-comment the next line.  This provides information on the
%% version of every file you have used.
%%%%%%%%%%%%%%%%%%%%%%%%%%%%%%%%%%%%%%%%%%%%%%%%%%%%%%%%%%%%%%%%%%%%%
%%\listfiles

%%%%%%%%%%%%%%%%%%%%%%%%%%%%%%%%%%%%%%%%%%%%%%%%%%%%%%%%%%%%%%%%%%%%%
%% Place any additional macros here.  Please use \newcommand* where
%% possible, and avoid layout-changing macros (which are not used
%% when typesetting).
%%%%%%%%%%%%%%%%%%%%%%%%%%%%%%%%%%%%%%%%%%%%%%%%%%%%%%%%%%%%%%%%%%%%%
\newcommand*\mycommand[1]{\texttt{\emph{#1}}}
\newcommand{\nc}{\newcommand}

\nc{\captitital}[1]{\textit{#1}}

\nc{\ie}{\textit{i.e.}}
\nc{\via}{\textit{via}}
\nc{\eg}{\textit{e.g.}}
\nc{\vs}{\textit{vs.}}
\newcommand{\cf}{\textit{cf.}}
\nc{\suppmat}{}
\nc{\suppmatv}{}

%==============================================================================
%==============================================================================
% SPECIFIC TO CH 1
%==============================================================================
%==============================================================================

\nc{\nitro}{\textsc{Nitro}}
\nc{\quino}{\textsc{Quino}}

\nc{\vase}{\textsc{Vase}}
\nc{\oneopen}{\textsc{Oneopen}}
%
\nc{\twoopen}{\textsc{Twoopen}}
%\nc{\twoopencone}{\textsc{Twoopen-cis-o}}
%\nc{\twoopenctwo}{\textsc{Twoopen-cis-p}}
%\nc{\twoopencco}{\textsc{Twoopen-cis-o}}
%\nc{\twoopenccp}{\textsc{Twoopen-cis-p}}
%\nc{\twoopent}{\textsc{Twoopen-trans}}

\nc{\twoopencone}{\textsc{Twoopen}-{\em cis}-$o$}
\nc{\twoopenctwo}{\textsc{Twoopen}-{\em cis}-$p$}
\nc{\twoopencco}{\textsc{Twoopen}-{\em cis}-$o$}
\nc{\twoopenccp}{\textsc{Twoopen}-{\em cis}-$p$}
\nc{\twoopent}{\textsc{Twoopen}-{\em trans}}

% David's
%\nc{\ciso}{\textsc{cis-o}}
%\nc{\cisp}{\textsc{cis-p}}
%\nc{\trans}{\textsc{trans}}
%
% Phil's
\nc{\ciso}{{\em cis-o}}
\nc{\cisp}{{\em cis-p}}
\nc{\trans}{{\em trans}}
%
\nc{\threeopencone}{\textsc{Threeopen} (\via{} \ciso{})}
\nc{\threeopenctwo}{\textsc{Threeopen} (\via{} \cisp{})}
\nc{\threeopent}{\textsc{Threeopen} (\via{} \trans{})}
\nc{\threeopen}{\textsc{Threeopen}}
\nc{\threeopenave}{\textsc{Threeopen} (Ave)}
\nc{\kitecone}{\textsc{Kite} (\via{} \ciso{})}
\nc{\kitectwo}{\textsc{Kite} (\via{} \cisp{})}
\nc{\kitet}{\textsc{Kite} (\via{} \trans{})}
\nc{\kite}{\textsc{Kite}}
\nc{\kiteave}{\textsc{Kite} (Ave)}
\nc{\dih}[3]{\IfEq{#3}{bar}{$\phi_{\mathrm{#1}\overline{#2}}$}{$\phi_{\mathrm{#1}#2}$}}

\nc{\dihnobar}[2]{$\phi_{\mathrm{#1}#2}$}
\nc{\dihbar}[2]{$\phi_{\mathrm{#1}\overline{#2}}$}

% {N,Q}{V,C,T}{l,m,h}{Nsph}-{0,1,X}{0,1,X}{0,1,X}{0,1,X}
\nc{\simu}[5]{\textsc{#1#2#3-#4}\-\textsc{-#5}}
\nc{\simup}[4]{\textsc{#1#2#3}\-\textsc{-#4}}
\nc{\simum}[5]{#1\_\textsc{#2#3#4}\-\textsc{-#5}}

\nc{\refsimov}[4]{Movie \textsc{#1#2#3}\-\textsc{-#4}}
\nc{\refsimovm}[5]{Movie #1\_\textsc{#2#3#4}\-\textsc{-#5}}

%==============================================================================
%==============================================================================
% SPECIFIC TO CH 3
%==============================================================================
%==============================================================================

\nc{\KXK}{\textsc{KXK}}
\nc{\KEK}{\textsc{KEK}}
\nc{\KGK}{\textsc{KG$_{\mathrm E}$K}}

\nc{\XTP}{\textsc{XTP}}
\nc{\GTP}{\textsc{GTP}}
\nc{\BTP}{\textsc{BTP}}

\nc{\CNFA}{\textsc{C$_{\mathrm A}$}}
\nc{\CNFB}{\textsc{C$_{\mathrm B}$}}

\nc{\TI}{\textsc{TI}}
\nc{\HR}{\textsc{HR}}
\nc{\CB}{\textsc{CB}}

\nc{\FWD}{\textsc{FWD}}
\nc{\BWD}{\textsc{BWD}}
\nc{\MIX}{\textsc{MIX}}

\nc{\SIMP}{\textsc{SIMP}}
\nc{\MBAR}{\textsc{MBAR}}
\nc{\EXTI}{\textsc{EXTI}}
\nc{\CFIT}{\textsc{CFIT}}

\nc{\BIAS}{\textsc{BIAS}}

\nc{\REF}{\textsc{REF}}
\nc{\REP}{\textsc{REP}}

\nc{\AAD}{\textsc{AAD}}
\nc{\MAD}{\textsc{AAD}}
\nc{\EOM}{\textsc{EOM}}


%==============================================================================
%==============================================================================
% SPECIFIC TO CH 4
%==============================================================================
%==============================================================================

\nc{\xim}{\xi_-}
\nc{\xip}{\xi_+}
\nc{\ximt}{\xi^{\mathrm{CB}}_-}
\nc{\xipt}{\xi^{\mathrm{CB}}_+}
\nc{\xic}{\xi_*}
\nc{\xio}{\xi_o}
\nc{\xiclose}{\xi_m}

% PHIL
% -----
%
\nc{\DCNT}{\textsc{DCNT}}
\nc{\TRUS}{\textsc{TRUS}}
\nc{\REUS}{\textsc{REUS}}
\nc{\CBUS}{\textsc{CBUS}}
%
\nc{\DCAN}{\textsc{DCAN}} % Direct counting analysis
\nc{\PMAT}{\textsc{PMAT}}
\nc{\WHAM}{\textsc{WHAM}}
\nc{\VFEP}{\textsc{VFEP}}
\nc{\UINT}{\textsc{UINT}} % umbrella integration
\nc{\TINT}{\textsc{TINT}} % true integration = force integration
%


\definecolor{colli}{HTML}{783632}
\definecolor{colna}{HTML}{69BC4E}
\definecolor{colk}{HTML}{7B49CA}
\definecolor{colrb}{HTML}{BBAB4E}
\definecolor{colcs}{HTML}{D052AE}
\definecolor{colh2o}{HTML}{4D6433}
\definecolor{coldmso}{HTML}{D44442}
\definecolor{colch3oh}{HTML}{63B09D}
\definecolor{colchcl3}{HTML}{58366F}
\definecolor{col12c4}{HTML}{C77C3D}
\definecolor{col15c5}{HTML}{798FC9}
\definecolor{col18c6}{HTML}{CD8597}

%% \newcommand{\ws}[1]{\IfEq{#1}{10_H2O}{{\color{colh2o} H$_2$O}}{\IfEq{#1}{11_DMSO}{{\color{coldmso} DMSO}}{\IfEq{#1}{12_CHCL3}{{\color{colchcl3} CHCl$_3$}}{\IfEq{#1}{13_CH3OH}{{\color{colch3oh} CH$_3$OH}}{%
%% \IfEq{#1}{H2O}{{\color{colh2o} H$_2$O}}{\IfEq{#1}{DMSO}{{\color{coldmso} DMSO}}{\IfEq{#1}{CHCL3}{{\color{colchcl3} CHCl$_3$}}{\IfEq{#1}{CH3OH}{{\color{colch3oh} CH$_3$OH}}}}}}}}}}

%% \newcommand{\wc}[1]{\IfEq{#1}{01_12C4}{{\color{col12c4} 12C4}}{\IfEq{#1}{02_15C5}{{\color{col15c5} 15C5}}{\IfEq{#1}{03_18C6}{{\color{col18c6} 18C6}}{%
%% \IfEq{#1}{12C4}{{\color{col12c4} 12C4}}{\IfEq{#1}{15C5}{{\color{col15c5} 15C5}}{\IfEq{#1}{18C6}{{\color{col18c6} 18C6}}{%
%% }}}}}}}
%% \newcommand{\wi}[1]{\IfEq{#1}{01_LI+}{{\color{colli} Li$^+$}}{\IfEq{#1}{02_NA+}{{\color{colna} Na$^+$}}{\IfEq{#1}{03_K+}{{\color{colk} K$^+$}}{\IfEq{#1}{04_RB+}{{\color{colrb} Rb$^+$}}{\IfEq{#1}{05_CS+}{{\color{colcs} Cs$^+$}}{%
%% \IfEq{#1}{LI+}{{\color{colli} Li$^+$}}{\IfEq{#1}{NA+}{{\color{colna} Na$^+$}}{\IfEq{#1}{K+}{{\color{colk} K$^+$}}{\IfEq{#1}{RB+}{{\color{colrb} Rb$^+$}}{\IfEq{#1}{CS+}{{\color{colcs} Cs$^+$}}{}}}}}}}}}}}


\newcommand{\ws}[1]{\IfEq{#1}{10_H2O}{{ H$_2$O}}{\IfEq{#1}{11_DMSO}{{ DMSO}}{\IfEq{#1}{12_CHCL3}{{ CHCl$_3$}}{\IfEq{#1}{13_CH3OH}{{ CH$_3$OH}}{%
\IfEq{#1}{H2O}{{ H$_2$O}}{\IfEq{#1}{DMSO}{{ DMSO}}{\IfEq{#1}{CHCL3}{{ CHCl$_3$}}{\IfEq{#1}{CH3OH}{{ CH$_3$OH}}}}}}}}}}

\newcommand{\wc}[1]{\IfEq{#1}{01_12C4}{{12C4}}{\IfEq{#1}{02_15C5}{{15C5}}{\IfEq{#1}{03_18C6}{{18C6}}{%
\IfEq{#1}{12C4}{{12C4}}{\IfEq{#1}{15C5}{{15C5}}{\IfEq{#1}{18C6}{{18C6}}{%
}}}}}}}
\newcommand{\wi}[1]{\IfEq{#1}{01_LI+}{{Li$^+$}}{\IfEq{#1}{02_NA+}{{Na$^+$}}{\IfEq{#1}{03_K+}{{K$^+$}}{\IfEq{#1}{04_RB+}{{Rb$^+$}}{\IfEq{#1}{05_CS+}{{Cs$^+$}}{%
\IfEq{#1}{LI+}{{Li$^+$}}{\IfEq{#1}{NA+}{{Na$^+$}}{\IfEq{#1}{K+}{{K$^+$}}{\IfEq{#1}{RB+}{{Rb$^+$}}{\IfEq{#1}{CS+}{{Cs$^+$}}{}}}}}}}}}}}

% phil prefers Ref. with superscripts...
\nc{\squarit}[1]{#1}
% David's way with square brackets...
%\nc{\squarit}[1]{~\square{#1}}

%==============================================================================
% COLORS
%==============================================================================

\definecolor{m1}{HTML}{FF00FF}%{Magenta}%HTML}{FFB719}
\definecolor{m2}{HTML}{264bCF}
\definecolor{rec}{HTML}{83401B}
\definecolor{ts}{HTML}{0069B4}
\definecolor{hist}{HTML}{C86F3A}


\definecolor{flap2}{HTML}{C8176B}
\definecolor{flap2d1}{HTML}{F360A6}
\definecolor{flap2d4}{HTML}{EB1278}
\definecolor{flap2d2}{HTML}{AF0053}
\definecolor{flap2d3}{HTML}{780039}
\definecolor{flap4}{HTML}{EF7E1C}
\definecolor{flap4d1}{HTML}{FFAC65}
\definecolor{flap4d4}{HTML}{FF8013}
\definecolor{flap4d2}{HTML}{D16100}
\definecolor{flap4d3}{HTML}{904200}
\definecolor{flap1}{HTML}{128B91}
\definecolor{flap1d1}{HTML}{5BDCE2}
\definecolor{flap1d4}{HTML}{11C7CF}
\definecolor{flap1d2}{HTML}{01797F}
\definecolor{flap1d3}{HTML}{005357}
\definecolor{flap3}{HTML}{90DD1A}
\definecolor{flap3d1}{HTML}{BEF963}
\definecolor{flap3d4}{HTML}{9CF513}
\definecolor{flap3d2}{HTML}{75C100}
\definecolor{flap3d3}{HTML}{508400}

\definecolor{colsdlo}{HTML}{A038DA}
\definecolor{colsdmi}{HTML}{5F088F}
\definecolor{colsdhi}{HTML}{440367}
\definecolor{colchcl3lo}{HTML}{3BEA32}
\definecolor{colchcl3mi}{HTML}{09AD00}
\definecolor{colchcl3hi}{HTML}{067D00}
\definecolor{coltollo}{HTML}{FFBC36}
\definecolor{coltolmi}{HTML}{D68F00}
\definecolor{coltolhi}{HTML}{9B6700}


%==============================================================================
% COMMENTS
%==============================================================================

\nc{\rem}[1]{}
\nc{\remrem}[1]{}
\nc{\phires}[1]{}
\nc{\phicom}[1]{}%{\color{Green}[PHIL: #1 ]}}
\nc{\davcom}[1]{}%{\color{blue}[DAVID: #1 ]}}
\nc{\tbdel}[1]{}%{\color{red}[TO BE DELETED: #1 ]}}
\nc{\tbref}[1]{}%{\color{Gray}[TO BE REFINED: #1 ]}}

\nc{\jovcom}[1]{}
\nc{\jovadd}[1]{}
\nc{\radd}[1]{#1}
\nc{\revphil}[1]{#1}
\nc{\revphildel}[1]{}
%==============================================================================
% MATH
%==============================================================================
%%%%%%%%%%%%%%%%%%%%%%%%%%%%%%%%%%%%%%%%%%%%%%%%%%%%%%%%
% Math
\nc{\dd}{\mathrm{d}}
\nc{\ee}{\mathrm{e}}
\nc{\fvector}[1]{\mathbf{#1}}
\nc{\fmatrix}[1]{\mathbf{\underline{#1}}}

\nc{\vecThreeD}[3] {
   \begin{pmatrix} #1 \\ #2 \end{pmatrix}
}

\nc{\lam}{\lambda}
\nc{\Lamb}{\Lambda}

\nc{\dhdl}{\frac{\partial \mathcal{H}}{\partial\lam}}
\nc{\dhdlav}{\left\langle\frac{\partial \mathcal{H}}{\partial\lam}\right\rangle}

\nc{\ham}{\mathcal{H}}

\nc{\bia}{\mathcal{B}}
\nc{\bio}{\mathcal{P}}

\nc{\pot}{\mathcal{U}}
\nc{\kin}{\mathcal{K}}
\nc{\frc}{\mathcal{F}}

\nc{\kb}{k_{\mathrm{B}}}
\nc{\rv}{\mathbf{r}}
\nc{\pv}{\mathbf{p}}
\nc{\xv}{\mathbf{x}}
\nc{\qv}{\mathbf{q}}
\nc{\Fv}{\mathbf{F}}
\nc{\xiv}{\boldsymbol{\xi}}

\nc{\Xv}{\mathbf{X}}
\nc{\lamv}{\boldsymbol{\lambda}}

\nc{\Dv}{\mathbf{D}}
\nc{\Cmat}{\underline{\mathbf{C}}}
\nc{\unit}[1]{\ensuremath{\,\mathrm{#1}}}


\nc{\std}{\standardstate}


%==============================================================================
% EQUATION MACROS
%==============================================================================

\nc{\beq}[1]{\begin{equation}\label{eq:#1}}
\nc{\eeq}{\end{equation}}
\nc{\refeq}[1]{Eq.~\ref{eq:#1}}
\nc{\refeqs}[1]{Eqs.~\ref{eq:#1}}
\nc{\refeqn}[1]{~\ref{eq:#1}}

%==============================================================================
% FIGURE MACROS
%==============================================================================

\nc{\reffig}[1]{Fig.~\ref{fig:#1}}
\nc{\reffigs}[1]{Figs.~\ref{fig:#1}}
\nc{\reff}[1]{\ref{fig:#1}}
\nc{\reffign}[1]{\ref{fig:#1}}
\nc{\reffignn}[1]{\ref{fig:#1}}

\nc{\refsifig}[1]{Fig.~\ref{fig:#1}}
\nc{\refsifigs}[1]{Figs.~\ref{fig:#1}}
\nc{\refsifign}[1]{~\ref{fig:#1}}
\nc{\refsifignn}[1]{\ref{fig:#1}}

\nc{\refmainfig}[1]{Fig.~\ref{fig:#1}}
\nc{\refmainfigs}[1]{Figs.\ref{fig:#1}}
\nc{\refmainfign}[1]{\ref{fig:#1}}
\nc{\refmainfignn}[1]{\ref{fig:#1}}

%==============================================================================
% TABLE MACROS
%==============================================================================

\nc{\reftab}[1]{Tab.~\ref{tab:#1}}
\nc{\reftabs}[1]{Tabs.~\ref{tab:#1}}
\nc{\reftabn}[1]{~\ref{tab:#1}}
\nc{\reftabnn}[1]{\ref{tab:#1}}

\nc{\refsitab}[1]{Tab.~\ref{tab:#1}}
\nc{\refsitabs}[1]{Tabs.~\ref{tab:#1}}
\nc{\refsitabn}[1]{~\ref{tab:#1}}
\nc{\refsitabnn}[1]{\ref{tab:#1}}

\nc{\refmaintab}[1]{Tab.~\ref{tab:#1}}
\nc{\refmaintabs}[1]{Tabs.~\ref{tab:#1}}
\nc{\refmaintabn}[1]{\ref{tab:#1}}
\nc{\refmaintabnn}[1]{\ref{tab:#1}}


%==============================================================================
% SECTION MACROS
%==============================================================================

\nc{\refch}[1]{Chapter~\ref{ch:#1}}
\nc{\refchs}[1]{Chapters~\ref{ch:#1}}
\nc{\refchn}[1]{~\ref{ch:#1}}
\nc{\refchnn}[1]{\ref{ch:#1}}

\nc{\refsec}[1]{Sect.~\ref{sec:#1}}
\nc{\refsecs}[1]{Sects.~\ref{sec:#1}}
\nc{\refsecn}[1]{~\ref{sec:#1}}
\nc{\refsecnn}[1]{\ref{sec:#1}}



\nc{\refsisec}[1]{Appendix~\ref{sec:#1}}
\nc{\refsisecs}[1]{Appendices\ref{sec:#1}}
\nc{\refsisecn}[1]{~\ref{sec:#1}}
\nc{\refsisecnn}[1]{\ref{sec:#1}}

\nc{\labsec}[1]{\label{sec:#1}}

%==============================================================================
% PLOTS
%==============================================================================
\providecommand{\path}{./}
\newcommand{\subgraph}[3]{
  \begin{minipage}[t]{1em}\subcaption{}\label{#1}\end{minipage}%
  \hspace{-0.5em}%
  \begin{minipage}[t]{\dimexpr#3\linewidth-0.5em\relax}
  \centering
%  \vspace{-.5\baselineskip}
  \hfill \\[.5em]
  \resizebox{\textwidth}{!}{\input{\path/plt/#2}}%
  \end{minipage}%
}

\newcommand{\subgraphpng}[3]{
  \begin{minipage}[t]{2em}\subcaption{}\label{#1}\end{minipage}%
  \hspace{-2.1em}
  \begin{minipage}[t]{\dimexpr#3\linewidth\relax}
  \centering
%  \vspace{-.5\baselineskip}
  \hfill \\
  \includegraphics[width=\textwidth]{\path/fig/#2}
  \end{minipage}%
}

\newcommand{\subgrapheps}[4]{
  \begin{minipage}[t]{2em}\subcaption{}\label{#1}\end{minipage}%
  \begin{minipage}[t]{\dimexpr#3\linewidth\relax}
  \centering
  \hfill \\
  \includegraphics[scale=#4]{\path/fig/#2}\hfill\\
  \end{minipage}%
}

\nc\appsection[3][]{%
  \setcounter{equation}{0}%
  %\setcounter{section}{0}
  \setcounter{figure}{0}%
  \setcounter{table}{0}%
  \section[#1]{#3}%
  \labsec{#2}
  
}


\externaldocument[SI-]{suppmat}
%\externaldocument[SIB-]{inc/suppmatB}

%%%%%%%%%%%%%%%%%%%%%%%%%%%%%%%%%%%%%%%%%%%%%%%%%%%%%%%%%%%%%%%%%%%%%
%% Meta-data block
%% ---------------
%% Each author should be given as a separate \author command.
%%
%% Corresponding authors should have an e-mail given after the author
%% name as an \email command. Phone and fax numbers can be given
%% using \phone and \fax, respectively; this information is optional.
%%
%% The affiliation of authors is given after the authors; each
%% \affiliation command applies to all preceding authors not already
%% assigned an affiliation.
%%
%% The affiliation takes an option argument for the short name.  This
%% will typically be something like "University of Somewhere".
%%
%% The \altaffiliation macro should be used for new address, etc.
%% On the other hand, \alsoaffiliation is used on a per author basis
%% when authors are associated with multiple institutions.
%%%%%%%%%%%%%%%%%%%%%%%%%%%%%%%%%%%%%%%%%%%%%%%%%%%%%%%%%%%%%%%%%%%%%
%
%
\author{David F. Hahn}
\affiliation[ETH]
{Laboratory of Physical Chemistry, Department of Chemistry and Applied Biosciences, 
ETH Z\"urich, Vladimir-Prelog-Weg 2, 8093 Z\"urich, Switzerland}
%
%

%
\author{Philippe H. H\"unenberger}
\affiliation[ETH]
{Laboratory of Physical Chemistry, Department of Chemistry and Applied Biosciences, 
ETH Z\"urich, Vladimir-Prelog-Weg 2, 8093 Z\"urich, Switzerland}
\email{phil@igc.phys.chem.ethz.ch}
%\phone{+123 (0)123 4445556}
%\fax{+123 (0)123 4445557}

%%%%%%%%%%%%%%%%%%%%%%%%%%%%%%%%%%%%%%%%%%%%%%%%%%%%%%%%%%%%%%%%%%%%%
%% The document title should be given as usual. Some journals require
%% a running title from the author: this should be supplied as an
%% optional argument to \title.
%%%%%%%%%%%%%%%%%%%%%%%%%%%%%%%%%%%%%%%%%%%%%%%%%%%%%%%%%%%%%%%%%%%%%
\title[CBUS]
      {Intro\\
        \large{\normalfont Document date: \today}}

%Closing-opening (\vase{}-\kite{}) equilibrium of resorcin[4]arene cavitands
%investigated using molecular dynamics simulations
%with ball-and-stick local elevation umbrella sampling

%%%%%%%%%%%%%%%%%%%%%%%%%%%%%%%%%%%%%%%%%%%%%%%%%%%%%%%%%%%%%%%%%%%%%
%% Some journals require a list of abbreviations or keywords to be
%% supplied. These should be set up here, and will be printed after
%% the title and author information, if needed.
%%%%%%%%%%%%%%%%%%%%%%%%%%%%%%%%%%%%%%%%%%%%%%%%%%%%%%%%%%%%%%%%%%%%%
\abbreviations{MD, MC, CBTI, OSP, TI, HRE, MC, $\lam$D, EXTI, BAR, MBAR, $\lam$MC, LEUS, LE, US, B\&S-LEUS, FEP, SC, EDS, CB}
%\keywords{American Chemical Society, \LaTeX}

\renewcommand\path{./}
%%%%%%%%%%%%%%%%%%%%%%%%%%%%%%%%%%%%%%%%%%%%%%%%%%%%%%%%%%%%%%%%%%%%%
%% The manuscript does not need to include \maketitle, which is
%% executed automatically.
%%%%%%%%%%%%%%%%%%%%%%%%%%%%%%%%%%%%%%%%%%%%%%%%%%%%%%%%%%%%%%%%%%%%%
\begin{document}

%%%%%%%%%%%%%%%%%%%%%%%%%%%%%%%%%%%%%%%%%%%%%%%%%%%%%%%%%%%%%%%%%%%%%
%% The "tocentry" environment can be used to create an entry for the
%% graphical table of contents. It is given here as some journals
%% require that it is printed as part of the abstract page. It will
%% be automatically moved as appropriate.
%%%%%%%%%%%%%%%%%%%%%%%%%%%%%%%%%%%%%%%%%%%%%%%%%%%%%%%%%%%%%%%%%%%%%
%% \begin{tocentry}

%% Some journals require a graphical entry for the Table of Contents.
%% This should be laid out ``print ready'' so that the sizing of the
%% text is correct.

%% Inside the \texttt{tocentry} environment, the font used is Helvetica
%% 8\,pt, as required by \emph{Journal of the American Chemical}
%% The surrounding frame is 9\,cm by 3.5\,cm, which is the maximum
%% permitted for  \emph{Journal of the American Chemical Society}
%% graphical table of content entries. The box will not resize if the
%% content is too big: instead it will overflow the edge of the box.

%% This box and the associated title will always be printed on a
%% separate page at the end of the document.

%% \end{tocentry}

%%%%%%%%%%%%%%%%%%%%%%%%%%%%%%%%%%%%%%%%%%%%%%%%%%%%%%%%%%%%%%%%%%%%%
%% The abstract environment will automatically gobble the contents
%% if an abstract is not used by the target journal.
%%%%%%%%%%%%%%%%%%%%%%%%%%%%%%%%%%%%%%%%%%%%%%%%%%%%%%%%%%%%%%%%%%%%%
\begin{abstract}
\fancyhead[L]{To be submitted to: XXX}
\thispagestyle{fancy}
%  \tbref{TO BE WRITTEN LAST}
%\chapter{Summary}

Molecular dynamics (MD) simulations offer nowadays a valuable 
tool for studying phenomena in chemistry and biology.
%
The history of models in chemistry, as well as the methodological 
and theoretical background of MD is presented in \refch{intro}.

\refch{resa} deals with the application of explicit-solvent molecular dynamics (MD)
simulations to resorcin[4]arene cavitands, which
can adopt a close/contracted (\vase{})
and an open/expanded (\kite{}) conformation.
%
The \vase-\kite{} equilibria of a quinoxaline- and a dinitrobenzene-based 
resorcin[4]arene are investigated in three solvent environments (vacuum, chloroform and toluene)
and at three temperatures (198.15, 248.15 and 298.15 K).
%
The challenge of sampling the millisecond-timescale \vase-\kite{} transition
%, which occurs experimentally
%on the milisecond timescale, represents a challenge in terms of sampling.
%
%It 
is addressed by calculating relative free energies using ball-and-stick local 
elevation umbrella sampling (B\&S-LEUS) to promote interconversion transitions.
%
The calculated \vase-to-\kite{} free-energy changes $\Delta G$
% between \vase{} and \kite{} calculated
are in qualitative agreement with the 
experimental magnitudes and trends.
%


\refchs{cbti}-\refchnn{cbus} present the development of the conveyor belt scheme,
a method to calculate free-energy differences.
%
The working principle relies on $K$ coupled replicas of the system that are 
simulated at different values of a coupling parameter $\lambda$.
%
The number $K$ is taken to be even and the replicas are equally spaced
on a forward-turn-backward-turn path, akin to a conveyor belt (CB)
between the two end-states.
%
As in $\lambda$-dynamics ($\lam$D), the $\lambda$-values associated with the 
individual systems evolve in time along the simulation.
%
However, they do so in a concerted fashion, determined by the evolution
of a single dynamical variable $\Lambda$ of period $2\pi$ controlling 
the advance of the entire CB.
%
Thus, a change of $\Lambda$ is always
associated with one half of the replicas moving forward and the other half moving backward along $\lambda$.
%
As a result, the effective free-energy profile of the 
replica system along $\Lambda$ is characterized with 
decrasing barriers upon increasing $K$, at least as $K^{-1}$ \radd{in the limit of large $K$}.
%
When \radd{a sufficient number of replicas is used}, 
these variations become small, 
which enables a complete and quasi-homogeneous coverage of the $\lambda$-range
by the replica system.
%


\refch{cbti} introduces this scheme with respect to
alchemical free-energy calculations. Therefore it is termed \textit{conveyor belt thermodynamic integration}.
It provides the mathematical/physical formulation of the scheme, along with an initial 
application of the method to the calculation of the hydration free energy of methanol.

In \refch{ortho}, the conveyor belt thermodynamic integration is applied to 
a Lys-X-Lys tripeptide, involving a side-chain mutation in the central residue,
and guanosine triphosphate, involving a hydrogen-to-bromine mutation.
With both systems, sampling issues have been encountered, due to 
the large orthogonal barriers either along the backbone 
dihedral angles $\phi$ and $\psi$ (tripeptide) or along 
ribose-base dihedral angle $\chi$ (guanosine). 
This relative merits of different sampling schemes, orthogonal biasing and estimators to improve the convergence 
are investigated.
%
%
The thermodynamic integration scheme is shown to suffer from the 
constraint of simulations at fixed $\lambda$.
%
There is no significant improvement upon changing 
from the Simpson's quadrature to the MBAR estimator.
%
Both Hamiltonian replica exchange and conveyor belt 
thermodynamic integration improve the results for
the tripeptide.
%
For the guanosine triphosphate, improvement
is only achieved upon application of an orthogonal biasing potential, 
most efficiently in combination with Hamiltonian replica exchange or
conveyor belt thermodynamic integration.
%

The conveyor belt scheme is extended to the calculation of
conformational free-energy differences in \refch{cbus},
resulting in a so-called conveyor belt umbrella sampling scheme (\CBUS).
%
CBUS is here initally applied to the calculation of 45 standard absolute binding free energies 
of five alkali cations to three crown ethers in three different solvents.
%
Besides introducing and testing the new scheme, it is compared to 
other methods.
%
Expectedly, the direct counting approach has convergence issues
on the accessible simulation timescale, and the corresponding results
are rather unreliable. On the other hand, the results obtained 
using traditional umbrella sampling and CBUS are
very consistent. 
%
Additionally, comparison of the results to
those of previous alchemical calculations {\em via} an
alchemical pathway reveals excellent 
consistency while the trends of available experimental data
are qualitatively reproduced.

Conclusions and an outlook into future developments
are given in \refch{outlook}.

\end{abstract}

%\phicom{
%TO LOOK AT:
%- are there previous calculations (QM,MD) on resorcinarenes -
%  to be mentioned in the intro
%}

\pagebreak
%================================================================================
\section{Introduction}
\labsec{intro}
%================================================================================

%The Problem and the Solution
Newly developed advanced simulation methods are routinely tested on simple one- and two-dimensional model systems. They provide valuable insights into the theory, conceptual advantages and limitations (for examples see e.g. Refs. \cite{Huber1994, Laio2002, Christ2007, Konig2012, Koenig2020, Donnini2016, Weiss2016, Lemke2018}).
While the results of new methods are published, the implementation details may not always be available or difficult to use with different computer infrastructure.
As a result, sharing, reproducing, understanding, and comparing simulation methodologies is often cumbersome.\cite{Peng2011}
To address this issue, we have developed the Ensembler package, an easy-to-use, yet powerful platform that enables fast prototyping of new methods and comparison against existing techniques using one or two-dimensional test systems.

%Global ethical goal
Ensembler is designed following the recommendations of Stodden \textit{et al.}\cite{Stodden2016} for the enhanced reproducibility of computational methods, which includes making code publicly accessible, providing documentation, and using open licensing.\cite{Stodden2016} 
Furthermore, Ensembler uses state-of-the-art software engineering tools (i.e. git,\cite{Chacon2014} MolSSI cookie-cutter,\cite{Naden2018} and binder\cite{Binder2018}/Colab\cite{Bisong2019}) to fulfill these recommendations and enable features like continuous integration and the transparent versioning of the code. 

%-------------------------------------------------------------------------------------------------------
\subsection{Method Development}
%-------------------------------------------------------------------------------------------------------

%Why not using normal MD-Packages
The lean Python3 code\cite{Vanrossum2009} of Ensembler allows for easy prototyping of new methods and comparison against a wide range of already implemented techniques. 
In contrast, the C/C++\cite{Stroustrup1995} code of traditional high-performance molecular dynamics (MD) packages (e.g. Refs. \citenum{Berendsen1995,Lindahl2001,Vanderspoel2005,Eastman2017,Brooks2009}) is more efficient but also much more complex. 
%
%What we got
The methods currently available in Ensembler are:
\begin{itemize}
	\item \textit{Model systems}: Harmonic oscillators as well as dihedral-angle, double-well, and Lennard-Jones potential-energy functions\cite{Jones1924}
	\item \textit{Sampling algorithms}: Conjugated gradient\cite{Hestenes1952} for energy minimization, Metropolis Monte Carlo (MC),\cite{Hastings1970} leap-frog integration\cite{Vangunsteren1988} for MD, and Langevin integration\cite{Brunger1984} for stochastic dynamics (SD)
	\item \textit{Enhanced sampling techniques}: Umbrella sampling,\cite{Valleau1977} simulated tempering/temperature replica-exchange simulations,\cite{Sugita1999} local elevation/metadynamics,\cite{Huber1994, Laio2002}
	\item \textit{Free-energy methods}: Free-energy perturbation (FEP),\cite{Zwanzig1954} Bennett's acceptance ratio (BAR),\cite{Bennett1976} thermodynamic integration (TI),\cite{Kirkwood1935} enveloping distribution sampling (EDS),\cite{Christ2007, Christ2008, Christ2009} $\lambda$-EDS,\cite{Koenig2020} replica-exchange EDS (RE-EDS),\cite{Sidler2016} and conveyor-belt TI\cite{Hahn2019}
\end{itemize}

%Teaching
%-------------------------------------------------------------------------------------------------------
\subsection{Teaching}
%-------------------------------------------------------------------------------------------------------

Simple model systems can also be used for teaching MD concepts to students, as they allow to intuitively understand fundamental concepts. \cite{Pohorille2010} 
Ensembler is well suited for didactic purposes because it is not only easy to use, but supports also a range of visualizations, i.e. interactive widgets, animations, and plots, which can be embedded in Jupyter notebooks.\cite{Kluyver2016}
Example Jupyter notebooks\cite{Kluyver2016} are provided in the Ensembler GitHub repository.



%================================================================================
\begin{thebibliography}{74}
 \input{inc/paper.lrs}
\end{thebibliography}
%================================================================================


\end{document}




