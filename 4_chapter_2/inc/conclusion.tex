In the present chapter, we proposed a new method called 
conveyor belt thermodynamic integration (CBTI)
to calculate alchemical free-energy differences based on 
MD simulations.
%
This approach borrows and combines ideas from
thermodynamic integration\cite{KI33.1,KI34.2,KI35.1} (TI),
\radd{replica exchange\cite{SU99.1,FU02.2,ZH16.2} (HRE) or permutation\cite{IT13.1,IT13.2,YA17.2} (HRP),}
and $\lambda$-dynamics\cite{KO96.1,DA01.7,GU03.1,KN09.1,KN11.2,DO11.2,AR15.2,HA17.1} ($\lambda$D),
along with the real-life working principle
of the funicular.
%

%
In CBTI, one simulates in parallel a set of $K$ 
\radd{equally spaced} replicas
(with $K$ even) on a forward-turn-backward-turn 
path along the alchemical coupling variable $\lambda$, 
akin to a conveyor belt (CB) between the two physical end 
states.
%
Because the $\lambda$-forces (Hamiltonian $\lambda$-derivative)
exerted by the individual replicas on the CB largely 
compensate each other, the overall $\Lambda$-force on 
the CB advance variable $\Lambda$ becomes increasingly 
small when $K$ is made increasingly large (residual free-energy
barriers decreasing at least as $K^{-1}$ \radd{in the limit of large $K$}, as shown in \refsec{quad}).
%
As a result, \radd{for a sufficient number $K$}, quasi-homogeneous
sampling of the $\lambda$-range can be achieved
without application of any biasing potential.
%
If a \radd{smaller $K$} is employed, a memory-based biasing potential 
can still be added to further homogenize the sampling,
the preoptimization of which is computationally inexpensive.
%
The results of a CBTI simulation (whether biased or not) can be analyzed similarly to TI,
by binning of the \radd{average} Hamiltonian $\lambda$-derivative as a function 
of $\lambda$ considering all replicas jointly, followed 
by quadrature integration. 
%
In this case, the continuous and quasi-homogeneous sampling of the $\lam$-range permits to use a large number of bins,
thereby essentially eliminating quadrature errors.

As a first application, 
the CBTI scheme was employed here to calculate the 
hydration free energy of methanol.
%
It was shown that the method is rather robust with respect 
to the choice of its parameters ($K$ as well as the mass-parameter $m_{\Lamb}$ and thermostat coupling time $\tau_\Lamb$ of the CB), the 
most important sensitivity being relative to $K$.
%
Upon increasing $K$, the distribution/dynamics of $\Lambda$ evolves 
from regularly spaced preferential values with a hopping dynamics
to quasi-homogeneous coverage with a diffusive dynamics.
%
For the smallest number of replicas considered ($K=8$), application 
of a biasing potential is recommended. For larger numbers of 
replicas ($K\ge 16$), it becomes unnecessary.
%
The calculated $\Delta G$ values compare well with those obtained using other methods.




The convergence is accelerated relative to TI with Simpson quadrature (smaller error bar at identical
total single-system sampling time), owing to improved orthogonal sampling and reduced quadrature errors.
It is comparable to HRE, which shares the same orthogonal-sampling advantage.
\revphil{
It is also similar to TI with EXTI or MBAR as free-energy estimator,
which achieve a similar improvement {\em via} an orthogonal-statistics advantage,
{\em i.e.} by effectively mixing information concerning distinct configurational 
wells across $\lambda$-points.
}
%One might refer to these two types of effects
%orthogonal-sampling {\em vs.} orthogonal-statistics
%advantages, respectively.
%
%. However, the TI-like
%\philrev{The free-energy estimator employed here for CBTI may nevertheless
%still be sub-optimal in terms of statistical efficiency relative to \eg{} EXTI and MBAR.}
%
%
\revphil{It should be stressed, however, 
that the present mutation
is rather non-challenging in terms of orthogonal sampling.
%, {\em i.e.} we expect a 
%simple unimodal conformational behavior in the orthogonal space at each $\lambda$-value.
%
Work is in progress to investigate other types of systems with more complicated
%difficult
%"challenging" 
orthogonal spaces:
($i$) the side-chain mutation in the central residue of a tripeptide considered in Refs.~\citenum{BI15.1,GR16.4};
($ii$) the hydrogen-to-bromine mutation in the base of a 
%flexible 
nucleotide considered in Refs.~\citenum{HR08.1,HR09.1,GA13.4}.
%
Here, it is expected that CBTI alone (just like HRE)
will help overcoming 
%high 
barriers in the orthogonal space
when these barriers are low at some $\lambda$-value
(as in the first system mentioned),
but may be insufficient on its own when these barriers
are high at all $\lambda$-values (as in the second system mentioned),
in which case additional modifications must be applied to create artificially 
an orthogonal tunnel at least over a limited $\lambda$-range.
}


Compared to existing MD-based alchemical free-energy calculation methods, the CBTI scheme can be viewed in at least three different ways:
%
($i$) as a \radd{continuous/deterministic/dynamical}
%dynamical 
(instead of discrete/stochastic) analog 
of the HRE scheme\cite{SU99.1,FU02.2,ZH16.2} or the HRP scheme\cite{IT13.1,IT13.2,YA17.2};
($ii$) as a correlated multiple-replica analog (reminiscent of other swarm\cite{HU98.6,BU15.5,KA18.6,AL18.2}, multiple-walker\cite{RA06.2,CO14.5} or flying-Gaussian\cite{SU16.3,KR17.1} approaches)
       of the $\lambda$-local elevation umbrella sampling ($\lambda$-LEUS) scheme\cite{BI14.1,BI14.2,BI15.1,BI15.2} (or the conceptually similar flat-histogram\cite{WA01.5,LA04.6}
       $\lambda$-metadynamics\cite{LA02.1,BA08.2,WU11.1}, adaptive integration\cite{FA04.3}, adaptive biasing force\cite{DA08.2}, \radd{adaptively biased\cite{BA08.1} and} expanded-ensemble\cite{LY92.1,LY94.1,LY96.2,ES07.1,ES07.2,PA11.7,RA18.2} methods);
       ($iii$) as an equilibrium multiple-replica variant of the slow-growth\cite{BE85.3,ST86.1} (SG) method (bypassing the associated hysteresis issues\cite{PE89.1,HE91.1,MA94.12} or the requirement for
       exponential averaging over multiple repeats\cite{JA97.3,CR00.2,HE01.4,HU02.2}).
       %
       
%
Compared to plain TI, it shares the advantage of HRE/HRP and $\lambda$-LEUS
in terms of enhanced orthogonal sampling\cite{WO03.1,WO03.2,KH10.1,KH11.2}.
%
Compared to HRE/HRP, it permits a deterministic and continuous sampling
of the $\lambda$-range, and bypasses the need for a careful preselection\cite{KO05.8,RA05.8,TR06.5,SI08.3,NA08.6,VO15.2,VO15.3,ZH16.2,SU17.3,MA18.8}
of the $\lambda$-ladder and exchange-attempt interval.
%
Compared to both TI and HRE/HRP, the quasi-homogeneous $\lambda$-sampling 
also essentially removes quadrature errors.
%
Finally, compared to $\lambda$-LEUS, it eliminates (or drastically reduces) 
the dead time associated with the preoptimization of a biasing potential\cite{BI14.1}
or, alternatively, the use of this formally non-equilibrium statistics\cite{HA10.1} 
in the production calculation\cite{BA08.2}.
%
%%%
For the above reasons, the CBTI scheme certainly represents a
%very 
useful addition to the alchemical free-energy calculation toolkit.


\revphil{Like TI and HRE/HRP, the CBTI method is also intrinsically parallel.
%
However, assuming that the replicas are assigned to separate processors 
(including possible GPU implementations or/and cloud-computing applications),
the requirement of an all-to-all information exchange between processors
at every timestep might represent a drawback of the method relative to the no-exchange
and infrequent exchange situations of TI and HRE/HRP, respectively.
%
Although the communication is lightweight (Hamiltonian $\lambda$-derivative,
{\em i.e.} a single real number), the synchronization requirement may cause 
a performance loss (reduced scalability and fault tolerance).
%
Unless asynchronous variants\cite{GA15.12,XI15.7} or multiple-timestep schemes\cite{MO11.3,MA16.24} can be developed,
this performance loss may represent a problem for parallel applications
in situations involving many replicas of a small system, as these will
involve more data exchange at a more frequent rate.
%
}


%
%
%
This scheme opens the way to at least two types of generalizations and extensions.
%
%
%
%
First, a number of components of the scheme can be modified/generalized and in particular the following:
%
($i$) different functions $\zeta$ may be used \radd{in \refeq{cb_lam_of_big_lam}}
%\refeq{zigzag_fct}
      to modulate the replica density along $\lambda$ ({\em e.g.} smoothing the tips
      or adding plateaus\cite{BI14.1} for a denser sampling
      close to or at the physical end-states);
%
($ii$) different matrices $\Cmat$ may be used in \refeq{cb_def} corresponding to
       a to non-uniform weighting of the $\lam$-forces into the $\Lamb$-force,
       ({\em e.g.} canceling the effect of higher-order
       derivatives by alternating non-integer weights in analogy to standard quadrature methods);
($iii$) the coupling between replicas may be generalized
        from sequential pairwise constraints to possibly non-pairwise potentials
        (\eg{} collective or harmonic);
($iv$) the TI-like free-energy estimator of \refeq{cbti_formula} may be replaced by
          a statistically more powerful one of the MBAR
%/UWHAM/RBE 
type\cite{LU04.3,SH05.6,SH08.7,FA09.4,TA12.1,DI17.5,ZH17.6};
($v$) CBTI would benefit from the use of an alchemical coupling path
          presenting a vanishing free-energy derivative at the physical end-states (residual free-energy 
          barriers along $\Lamb$ decaying at least in $K^{-2}$ instead of $K^{-1}$ \radd{for large $K$}, as shown in \refsec{quad}).



Second, the application range of CBTI, restricted here to alchemical processes,
can be extended to encompass either thermodynamic or conformational free-energy 
changes.
%
In the former case, extension to a CB \radd{variant of} parallel \radd{tempering\cite{SU99.1} ({\em i.e.} a CBPT scheme)}
appears relatively straightforward considering that scaling the temperature
is equivalent to scaling the system potential energy. 
Such a 
%\revdavid{[DELETE?: CBTI] 
CBPT scheme would represent a form of
\revphil{multicanonical} \radd{sampling\cite{BA87.2,BE91.6,BE92.1}}.
%
%          *** BA87.2  : [Baumann] Noncanonical path and surface simulation.
%          > early ref, has the basic idea of non-canonical weight factors (but BE92.6 and BE92.1 are probably more to the point)
%          *** BE91.6  : [Berg/Neuhaus] Multicanonical algorithms for first order phase transitions.
%          > I think they use an approximate parametrized (one-param) fct to represent the density of states
%          > Phase transition in a lattice (Potts) model
%          *** BE92.1  : [Berg/Neuhaus] Multicanonical ensemble: A new approach to simulate first-order phase transitions.
%          > Similar
%
In the latter case, extension to a CB \radd{variant of} umbrella \radd{sampling\cite{TO74.1,TO77.1} ({\em i.e.} a CBUS scheme)}
could be designed {\em e.g.} by anchoring harmonic potentials to the CB
and propagating the corresponding restraining forces onto the CB.
%
Finally the extension of the CB approach to multistate problems\cite{KN11.2,BI15.2,HA17.1,VI18.2} 
({\em e.g.} path or network of CBs connecting the different states of a system)
as well as multidimensional problems (sub-CBs anchored to a main CB) can also
be envisioned.
