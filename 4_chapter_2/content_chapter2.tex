\makeatletter
\def\input@path{{../}}
\makeatother
\documentclass[../main.tex]{subfiles}
\begin{document}
\renewcommand{\path}{3_chapter_1/}
\chapter[An Algorithmic Graph-Based Approach for Setting up Dual Topology Methods]{Dual Topology and the Challenge how to pick the restraints - Restraintmaker
    \footnote{\label{footnoteChapter2CopyRight} 
    Reprinted (adapted) from  Benjamin Ries$^{**}$, Salom\'e Rieder$^{**}$, Clemens Rhiner, Philippe H. H\"unenberger, and Sereina Riniker, 
    JCTC, \textbf{X}, XX-XX (202X)
    $^{**}$ equal contribution}
 }
\chaptermark{FE: System Generation}
\label{ch:feres}

\aquote{
    "Allzu straff gespannt, zerspringt der Bogen."\\ \color{OliveGreen} \footnotesize{``And much too tightly stretched the bow will split.''}
}{Rudenz in Wilhelm Tell\\ Friedrich Schiller\cite{Schiller1898}}



\begin{summary}
The calculation of relative binding free energies involves the choice of the end-state/system representation, of a sampling approach, and of a free-energy estimator. System representations are usually termed ``single topology'' or ``dual topology''. As the terminology is often used ambiguously in the literature, a systematic categorization of the system representations is suggested here. In the dual topology approach, the molecules are simulated as separate molecules. Such an approach is relatively easy to automate for high-throughput RBFE calculations compared to the ``single topology'' approach. Distance restraints are commonly applied to prevent the molecules from drifting apart, thereby improving the sampling efficiency.
In this study, we introduce the program RestraintMaker, which relies on a greedy algorithm to find optimal distance restraints between pairs of atoms based on geometric measures.
The algorithm is further extended for multi-state methods such as enveloping distribution sampling (EDS) or multi-site $\lambda$-dynamics. The performance of RestraintMaker is demonstrated for toy models and for the calculation of relative hydration free energies.
The Python program can be used in script form or through an interactive GUI within PyMol. The selected distance restraints can be written out in the GROMOS or GROMACS file formats. 
Additionally, the program provides a human-readable JSON format that can be easily parsed and processed further.
The code of RestraintMaker is freely available on GitHub \textit{https://github.com/rinikerlab/restraintmaker}.
\end{summary}

\clearpage
\pagebreak

%%%%%%%%%%%%%%%%%%%%%%%%%%%%%%%%%%%%%%%%%%%%%%%%%%%%%%%%%%%%%%%%%%%%%
%% Start the main part of the manuscript here.
%%%%%%%%%%%%%%%%%%%%%%%%%%%%%%%%%%%%%%%%%%%%%%%%%%%%%%%%%%%%%%%%%%%%%
%================================================================================
\section{Introduction}
%================================================================================


Recent methodological developments have improved the statistical robustness and the degree of automation of relative binding free-energy (RBFE) calculations, which are now routinely applied in drug discovery projects in industry. 
\cite{Cournia2017,Cournia2020, Meier2021, Armacost2020c, Barros2022,
       Heinzelmann2021, Gapsys2020, Jespers2019, Raman2020,
       Christ2014, Gao2018, Tielker2021, Loeffler2018}
       %reviews - FE methods/ usage in industry
       %Methods - automation, accuracy
       %FE benchmark - challenges
%
A free-energy calculation provides information about the relative populations of multiple end-states in equilibrium. Examples are drug design, where the end-states represent the different ligands that bind to a protein, \cite{Christ2009, Riniker2011, Wang2015, Wang2017, Aldeghi2016, Sidler2016,Yu2017, Jespers2019,Jiang2019, Paulsen2020} or protein engineering, where the end-states correspond to the different amino acids considered for one position in the protein.\cite{Shobana2000, Bieler2015B, Jespers2019B}
Each free-energy calculation involves the choice of a sampling approach, a free-energy estimator (e.g. thermodynamic integration (TI),\cite{Kirkwood1935} the Zwanzig equation,\cite{Zwanzig1954} or Bennett's acceptance ratio (BAR)\cite{Bennett1976}), and a representation of the end-states (i.e., molecules or substructures of molecules) during the simulation.

%%% The solutions in application
Several possible representations have been proposed in the past to build a coordinate and topology space of the end-states. 
Historically, two approaches emerged, which were termed ``single topology'' \cite{Pearlman1991, Pearlman1994} and ``dual topology'' \cite{Pearlman1991, Gao1989}.
Unfortunately, the terminology is not always clear in the literature and these terms are used ambiguously.\cite{Boresch1999, Rocklin2013, Fleck2021}
To distinguish the different representation options, we propose here a terminology based on the difference in the respective coordinate space (Figure \ref{fig:Topology Types}). These definitions may differ from the historical ones. The single topology approach contains a single set of coordinates for both end states. In contrast,  the dual topology approach involves a separate set of coordinates for each end state. The two approaches can be seen as opposite extremes. Three sub-variants of the dual topology approach can be found in the literature: linked, separated and unconstrained.
In addition, a ``hybrid topology'' approach was recently described,\cite{Jiang2019} which presents an intermediate between the single and dual approaches (Figure \ref{fig:Topology Types}). This scheme has been used in many studies for binding free energy calculations before but not called hybrid topology. In protein engineering however Shobana \textit{et al.} used a similar approach called hybrid topology.\cite{Shobana2000} The different representations vary with respect to sampling efficiency and the capability of handling complex transformations.

\begin{figure}[h!]
    \centering
    \includegraphics[width=\linewidth]{fig/theory/landscape_of_simulationApproaches.png}
    \caption{Three end-state representations can be distinguished based on the coordinate space. The ``single topology'' approach (left) contains a single set of coordinates for all end-states. The ``dual topology'' approach contains separate sets of coordinates for each end state (right). The ``hybrid topology'' approach (middle) combines atoms of common substructures into one coordinate set, while atoms that differ between the end-states are represented separately. It is therefore an intermediate between the two other representations. The dual topology approach can be further subdivided into three sub-variants: linked, separated, and unconstrained. The linked dual topology approach is closest to the single topology approach, as the coordinate overlap between the end-states is ensured with direct spatial restraints (e.g. distance restraints). The separated variant is connecting the molecules indirectly by restraining them spatially to the same area, whereas the unconstrained variant does not restrain the molecules at all and is therefore also the most difficult to sample.}
    \label{fig:Topology Types}
\end{figure}

%%%% automation
With the high-throughput application of RBFE calculations comes the need for automation.\cite{Christ2014} While there exist tools such as FESetup,\cite{Loeffler2015} ProtoCaller,\cite{Suruzhon2020} SMArt,\cite{Petrov2021} or LOMAP,\cite{Liu2013} to automatically set up single-topology RBFE calculations, the dual topology approaches are in principle the easiest to automate as any alchemical molecule transformation can be realized without requiring atom mapping.\cite{Rocklin2013}
For the unconstrained dual topology variant, an automatic set-up procedure is available in the packages pyAutoFEP\cite{Carvalho2021} and FEW\cite{Homeyer2013}.
When representing the end-states with a linked dual topology approach, the set-up is more difficult than in the unconstrained case as the distance restraints between the molecules need to be chosen.
For example, the QligFEP pipeline\cite{Jespers2019} provides an automatic system generation for the linked dual topology approach, where the distance restraints are placed in the perturbed common substructure of the end-states. These distance restraints only become active if the restrained atoms surpass a distance of $0.02~$ nm.
However, for more complex transformations (e.g. in scaffold hopping), a more flexible approach is needed to select the optimal distance restraints between molecules.

%outlook
In this work, we present a greedy algorithm to select optimal distance restraints for RBFE calculations with the linked dual topology approach, which is also applicable to molecule pairs without a common core. In addition, the algorithm is extended to solve the same problem for multi-state RBFE methods such as enveloping distribution sampling (EDS)\cite{Christ2007,Christ2008} and multi-site $\lambda$-dynamics,\cite{Knight2011} resulting in a linked multi-topology approach. Finally, we analyze the sampling behavior and performance of the approach for the calculation of relative hydration free energies. The algorithm is implemented in a Python package\\ (\textit{https://github.com/rinikerlab/restraintmaker}), which can be used as a scripting library or with a GUI inside PyMOL. \cite{Delano2020}

%================================================================================
\section{Theory}
%================================================================================

%------------------------------------------------------------
\subsection{Alchemical Free-Energy Calculations}
%------------------------------------------------------------



The goal of path-dependent alchemical free-energy calculations is
to evaluate the free-energy difference $\Delta G$
between two states $A$ and $B$ of a molecular system,
by introducing a coupling scheme relying on a parameter $\lam$,
and sampling along the so-defined $\lam$-path.
%
The two states have the same number $3N$ of degrees of 
freedom, but distinct Hamiltonian functions $\ham_A(\xv)$ and $\ham_B(\xv)$,
respectively, where $\xv = (\rv,\pv)$ is the $6N$-dimensional
phase-space vector representative of a microscopic system configuration, $\rv$ 
and $\pv$ being the corresponding coordinate and momentum vectors.
%
The coupling parameter is introduced into a hybrid Hamiltonian 
$\ham(\xv;\lam)$ 
satisfying the boundary conditions
$\ham(\xv;0)=\ham_A(\xv)$ and $\ham(\xv;1)=\ham_B(\xv)$,
where the semi-colon indicates a parametric dependence.
%
%
%Examples of methods involving the coupling-parameter approach are 
%multi-configuration\cite{ST91.1} thermodynamic integration\cite{KI33.1,KI34.2,KI35.1} (MCTI or, simply, TI)
%and multi-configuration\cite{JO08.3} free-energy perturbation\cite{ZW54.1,MA95.2,LI96.1,MA99.10,SC99.3,RA12.6,RA17.5,BO17.2} (MC-FEP or, simply, FEP)
%along with their HRE\cite{SU99.1,FU02.2,ZH16.2}/HRP\cite{IT13.1,IT13.2,YA17.2} variants, as well as $\lam$-dynamics\cite{KO96.1,DA01.7,GU03.1,KN09.1,KN11.2,DO11.2,AR15.2,HA17.1} ($\lam$D),
%along with its coordinate-transformed or/and $\lam$-biased variants\cite{GU98.1,GU98.2,SO01.1,WU11.1,KN11.2,DO11.2,KN11.1,ZH12.3,DO13.1,BI14.1,BI14.2,BI15.1,BI15.2} ({\em e.g.} the $\lam$-LEUS scheme\cite{BI14.1,BI14.2,BI15.1,BI15.2}).
%
\revphildel{[-- Redundant paragraph removed --]}
%



Since the proposed CBTI scheme encompasses features of both $\lam$D and TI,
these two approaches are summarized briefly in the \radd{next two subsections.
The following three subsections then describe in turn
the basis of the CBTI scheme, 
its free-energy estimator 
and the application of a biasing potential.}



%------------------------------------------------------------
\subsection{$\lam$-Dynamics ($\lam$D)}
%------------------------------------------------------------


In the $\lam$D scheme\cite{KO96.1,DA01.7,GU03.1,KN09.1,KN11.2,DO11.2,AR15.2,HA17.1}, 
the coupling parameter $\lam$ is assigned a mass
$m_\lam$ and a momentum $p_\lam$, and considered to be an additional
pseudo-conformational degree of freedom of the system.
%
The Hamiltonian of the extended system is defined as
%
\beq{lam_dyn_ext_ham}
  \ham^\star(\xv,\lam) = \ham(\xv;\lam) + \frac{p_\lam^2}{2m_\lam} \ ,
\eeq
%
where the star refers to an extended system in which $\lambda$ 
is now a variable and no longer a parameter 
(thus the replacement of the semi-colon by a comma).
%
This leads to the additional equation of motion
%
\beq{lam_dyn_eq_mot}
  \ddot{\lam} = \frac {\dot{p}_{\lam}} {m_{\lam}} 
              = -\frac {1} {m_{\lam}} 
                 \frac {\partial\ham(\xv;\lam)} {\partial\lam}
\eeq
%
for propagating the $\lam$-variable, \radd{where a dot over a variable indicates its time derivative}.
%


The free-energy difference $\Delta G$
between the two physical end-states can then in principle be
calculated based on a single thermostated MD simulation 
of the extended system, as
%
\beq{lam_dyn_dg}
  \Delta G = 
      - \frac{1}{\beta}\ln\frac{\left\langle \delta(\lam - 1)\right\rangle^\star}
      {\left\langle \delta(\lam)\right\rangle^\star} \ ,
\end{equation}
%
where 
$\beta=(k_BT)^{-1}$, 
$k_B$ being the Boltzmann constant and $T$ the absolute temperature,
$\delta$ is the Dirac delta function, 
and $\langle\cdot\cdot\cdot\rangle^\star$ denotes ensemble averaging 
for the extended system (\ie{} over the joint trajectories of $\xv$ and $\lam$).
%
\revphil{In practice, the $\delta$-functions in \refeq{lam_dyn_dg} must be replaced
      by two finite end-state bins ($\lambda$-cutoff), sufficiently large
      for proper statistics but also sufficiently small for avoiding 
      distortions due to averaging at the end-states\cite{KN11.1}.}
%
\revphildel{[-- Redundant paragraph removed --]}
%
%There are, however, a number of important practical issues 
%to be addressed if this scheme is to be accurate and efficient\cite{BI14.1}:
%%
%($i$) the variations of $\lam$ must be restricted to the range $[0,1]$ between the two physical end-states\cite{KO96.1};
%($ii$) the mass and thermostat-coupling scheme of the $\lam$-variable must be chosen appropriately
%      to ensure a correct temperature for this variable\cite{BI15.1};
%($iii$) the sampling along $\lam$ must generally be adjusted
%      to ensure adequate sampling of the A and B states
%      with a sufficient number of interconversion transitions\cite{BI14.1};
%($iv$) the $\delta$-functions in \refeq{lam_dyn_dg} must be replaced
%      by two finite end-state bins ($\lambda$-cutoff), sufficiently large
%      for proper statistics but also sufficiently small for avoiding 
%      averaging distortions at the end-states\cite{KN11.1}. %
%%
%Different implementation variants of $\lambda$D address these
%issues in a variety of ways, including the use of
%coordinate transformations\cite{KN11.2,DO11.2,WU11.1,KN11.1,DO13.1,ZH12.3,BI14.1,BI15.2},
%memory-based biasing potentials,\cite{GU98.1,GU98.2,SO01.1,WU11.1,BI14.1,BI15.2}
%separate thermostat coupling of the $\lam$-variable\cite{BI14.1,BI15.1,BI15.2},
%and 
%alternative free-energy estimators\cite{LU04.3,SH05.6,KA05.1,KA06.6,FA09.4,KA12.5,TA12.1,DI17.5,ZH17.6}.



%------------------------------------------------------------
\subsection{Thermodynamic Integration (TI)}
%------------------------------------------------------------


In the original TI scheme\cite{KI33.1,KI34.2,KI35.1}, a set of 
$K$ replicas of the system are simulated in parallel at fixed predefined 
$\lam$-values in the range $[0,1]$.
Since the replicas are entirely decoupled 
from each other, the simulations
can be performed serially as well. However, TI extensions 
including the HRE scheme\cite{SU99.1,FU02.2,ZH16.2} and the HRP scheme\cite{IT13.1,IT13.2,YA17.2}
introduce a coupling in the form of $\lam$-value exchanges,
in which case the simulations must really be carried out in parallel.
The same will apply to the proposed CBTI scheme, where the coupling involves a synchronization of
the dynamical $\lam$-variations. 


Considering all replicas $k=0 \dots K-1$ as the members of a replica system,
one may note the corresponding $6K\mkern-3mu\times\mkern-3mu N$-dimensional phase-space vector
as $\Xv=\{\xv_k\}$ and the corresponding $K$-dimensional vector containing the fixed $\lam$-values as 
$\lamv=\{\lam_k\}$.
%
In plain TI, the Hamiltonian of the replica system is defined as
%
\beq{ti_rep_ham}
  \ham^\dagger(\Xv;\lamv) = \sum_{k=0}^{K-1} \ham(\xv_k;\lam_k) \ ,
\eeq
%
where the dagger refers to a replica system,
and $\lamv$ is here a parameter vector (thus the semi-colon).
Because the Hamiltonian of \refeq{ti_rep_ham} involves no coupling 
term  between the replicas, the dynamics of a replica $k$
is independent from that of the other replicas and
solely depends on $\lam_k$.


The free energy difference $\Delta G$
between the two states can then be
calculated based on a single thermostated MD simulation 
of the replica system, as
%
\begin{multline}
  \label{eq:ti_formula}
  \Delta G = 
  \int\limits_0^1 \mathrm{d}\lam' \left\langle \frac{\partial{\ham(\xv;\lam)}}{\partial \lam} \right\rangle_{\lam'} 
    \approx
    \sum_{k=0}^{K-1}  w_k \left\langle \frac{\partial{\ham^\dagger(\Xv;\lamv)}}{\partial \lam_k} \right\rangle^\dagger\\ 
    = 
    \sum_{k=0}^{K-1}  w_k \left\langle \frac{\partial{\ham(\xv_k;\lam_k)}}{\partial \lam_k} \right\rangle^\dagger \ ,
\end{multline}
%
where 
the $w_k$ are quadrature weights for the numerical integration\cite{JO10.2,BR11.5,BR11.6},
$\langle\cdot\cdot\cdot\rangle_{\lam}$ denotes ensemble averaging 
for a single system \radd{(\ie{} over $\xv$) at the given $\lam$ value},
and
$\langle\cdot\cdot\cdot\rangle^\dagger$ denotes ensemble averaging 
for the replica system (\ie{} over $\Xv$).


\revphil{In the above form, TI has long been the workhorse of alchemical free-energy calculations.
%
The method is extremely robust in the sense 
that the accuracy of the calculated $\Delta G$
%free-energy change 
can always be systematically
improved (more $\lambda$-points, longer equilibration or/and sampling times).
However, it is not necessarily the most 
%practical and 
efficient 
method to determine $\Delta G$ up to a certain accuracy,
due to possible sub-optimalities in
the coupling scheme\cite{CR86.1,BL04.2,PH11.1,PH12.1,NA14.1,NA15.3},
the protocol design\cite{ME93.3,HU16.7,ME17.1,SU17.3,DA18.1}
the free-energy estimator\cite{LU04.3,SH05.6,SH08.7,FA09.4,TA12.1,DE16.9,DI17.5,ZH17.6},
and
the orthogonal sampling\cite{WO03.1,WO03.2,KH10.1,KH11.2,BI15.1}.
}
\revphildel{[-- Redundant paragraph removed --]}




%In the above form, TI has long been the workhorse 
%of alchemical free-energy calculations.
%%
%In practice, in addition to the choice of a coupling scheme\cite{CR86.1,BL04.2,PH11.1,PH12.1,NA14.1,NA15.3},
%a TI calculation requires the definition of a protocol including:
%($i$) the number $K$ and distribution (spacing) of the $\lambda$-points considered;
%($ii$) the initial configurations, equilibration times and sampling times
%    selected for each $\lambda$-point;
%($iii$) the choice of a quadrature scheme for the numerical integration.
%%
%The accuracy of the calculation will depend on the selection of these 
%parameters, the optimization of which may represent a non-trivial and 
%time-consuming task\cite{ME93.3,HU16.7,ME17.1,SU17.3,DA18.1}.
%%
%Thus, even though TI is extremely robust in the sense 
%that the accuracy of the calculated free-energy change can always be systematically
%improved (more $\lambda$-points, longer equilibration or/and sampling times),
%it is not necessarily the most efficient method to determine 
%$\Delta G$ up to a certain accuracy.
%
%Many improvements have been proposed for the TI scheme
%over the years.
%%
%Most prominently, the inclusion of $\lam$-exchanges in 
%%HRE\cite{SU99.1,FU02.2,ZH16.2}/HRP\cite{IT13.1,IT13.2,YA17.2} 
%\radd{the HRE scheme\cite{SU99.1,FU02.2,ZH16.2} and the HRP scheme\cite{IT13.1,IT13.2,YA17.2}}
%permits to 
%circumvent orthogonal barriers\cite{WO03.1,WO03.2,KH10.1,KH11.2}, a feature also shared by $\lambda$D schemes\cite{BI15.1}, and
%the use of alternative free-energy estimators like the
%EXTI\cite{DE16.9} scheme or the MBAR\cite{SH08.7} scheme permits to improve the statistical efficiency.



%------------------------------------------------------------
\subsection{Conveyor Belt Thermodynamic Integration (CBTI)}
%------------------------------------------------------------


The proposed CBTI scheme encompasses features of both $\lam$D and TI.
%
Similarly to TI, it is based on the simulation of a replica system involving 
$K$ copies of the molecular system of interest, where $K$ is taken to be even.
%
And similarly to $\lam$D, the individual replicas are extended 
systems, for which the associated $\lam_k$-variable is allowed to evolve
along the simulation.
%
However, the evolutions of these $\lam_k$-variables are not independent. 
They are coupled to each 
other by means of a sequence of hard constraints,
 so that they follow the course of a conveyor belt (CB).
Thus, they are entirely determined by a single dynamical
variable $\Lamb$, following 
the scenario depicted in \reffig{scheme} and discussed
in the Introduction section.



The variable $\Lamb$ is a continuous real variable
representing the overall advance of the CB, successive multiples
of $2\pi$ corresponding to as many full rotations.
%
%
Given $\Lamb$ and $K$, the $\lam$-value $\lam_k$ associated with a system $k$ on the CB
is obtained as
%
\beq{cb_lam_of_big_lam}
  \lam_k(\Lamb) = \zeta\left( \Lamb + k \Delta\Lamb  \right) \ ,
\eeq
%
\radd{with
\beq{delta_lamb}
\Delta\Lamb = 2\pi K^{-1} .
\eeq
}
\radd{Here, the function $\zeta$} is a continuous and periodic zig-zag function of period $2\pi$ and image range $[0,1]$,
defined over the reference period $[0,2\pi)$ as
%
\beq{zigzag_fct}
  \zeta(\theta) = \left\{
                    \begin{array}{ll}
                       \pi^{-1}\theta & \mathrm{if}\ \theta < \pi \\
                       2-\pi^{-1}\theta & \mathrm{if}\ \theta \geq \pi \\
                    \end{array} 
                  \right. \ \ \mathrm{for}\ \theta\in[0,2\pi) \ ,
\eeq
%
where the $[\cdot,\cdot)$ indicates an interval that is open to the right side, {\em e.g.} $[0,2\pi)$ includes $0$ but excludes $2\pi$.
%
\radd{An advance of the CB by $\Delta\Lamb$}
%
corresponds to a cyclic permutation of the $K$ replicas, each system moving by 
one position forward along the CB, \ie
$\lam_k(\Lamb+\Delta\Lamb) = \lam_{k+1}(\Lamb)$
for $k<K-1$, along with
$\lam_{K-1}(\Lamb+\Delta\Lamb) = \lam_0(\Lamb)$.
%
For this reason, the increment $\Delta \Lamb$ will be further referred to 
as one shift of the CB.
%
The system $k=0$ can be viewed as a reference system,
as $\lambda_0=\pi^{-1}\Lambda$ for $0\leq\Lambda<\pi$
and $\lambda_0=2-\pi^{-1}\Lambda$ for $\pi\leq\Lambda<2\pi$.
%
Since $K$ is chosen to be even, an increase of $\Lambda$
always corresponds to an increase of $\lambda_k$
for half of the systems (forward-moving side of the CB)
and a decrease of $\lam_k$ for the other half of the systems
(backward-moving side of the CB). This choice also implies that 
$\Lamb$-values which are integer multiples of the
CB shift $\Delta \Lamb$ 
correspond to situations where there is 
one system in state $A$ and one system in state $B$.
%
Since an advance of the CB variable by $2\pi$
leaves the replica system invariant, 
the variable $\Lamb$ will commonly be refolded into the 
reference period $[0,2\pi)$ when illustrating the
results of the CBTI method.

The Hamiltonian of the extended replica system is defined as
%
\beq{cb_ext_rep_ham}
  \ham^{\dagger\star}(\Xv,\Lamb) = \ham^\dagger(\Xv;\lamv)  +  \frac{p_\Lamb^2}{2m_\Lamb} 
  \quad \text{with} \quad \lamv=\lamv(\Lamb),
\eeq
%
%
where $\ham^{\dagger}$ is defined as in TI by \refeq{ti_rep_ham}
and $\lamv(\Lamb)$ by \refeq{cb_lam_of_big_lam}.
%
In analogy with \refeq{lam_dyn_eq_mot}, the resulting equation of motion for $\Lamb$ reads
%
\beq{cb_big_lam_eq_mot}
  \ddot{\Lamb} = \frac {\dot{p}_{\Lamb}} {m_{\Lamb}} 
              = -\frac {1} {m_{\Lamb}}  \frac {\partial\ham^\dagger(\Xv;\lamv)} {\partial \Lamb} 
\eeq
%
where
%
\begin{align}
  \label{eq:cb_big_lam_eq_mot_der}
   \frac{\partial\ham^\dagger(\Xv;\lamv)}{\partial\Lamb}
        &= \sum_{k=0}^{K-1} \frac {\partial\ham(\xv_k;\lam_k)} {\partial \lam_k} \frac{\mathrm{d} \lam_k}{\mathrm{d} \Lamb}\nonumber \\
        &= \sum_{k=0}^{K-1} \frac {\partial\ham(\xv_k;\lam_k)} {\partial \lam_k} \zeta'\left( \Lamb + 2\pi K^{-1} k \right) \ .
\end{align}
%
Here, the function $\zeta'$ is the derivative of the zig-zag function of \refeq{zigzag_fct}, 
given over the reference period $[0,2\pi)$ by
%
\beq{zigzag_der}
  \zeta'(\theta) = \left\{
                    \begin{array}{ll}
                       \pi^{-1} & \mathrm{if}\ \theta < \pi \\
                       -\pi^{-1} & \mathrm{if}\ \theta \geq \pi \\
                    \end{array} 
                  \right. \ \ \mathrm{for}\ \theta\in[0,2\pi) \ .
\eeq
%
Formally, the derivative is not defined when $\theta$
is an integer multiple of $\pi$, \ie{} for a system that is exactly 
in one of the physical end-states $A$ or $B$. In this case, the value of $\zeta'$ has been arbitrarily set to 
$\pi^{-1}$ for even multiples and $-\pi^{-1}$ for odd multiples.
This has a negligible impact in practice, as it only concerns a series
of infinitesimal points over the entire $\Lamb$-range, 
\ie{} infinitesimally few configurations along a CBTI simulation.
%
\revphil{For example, when using double-precision floating-point arithmetics
(including denormalized numbers),
their probability of occurrence is on the order of 10$^{-324}K$ 
({\em i.e.} a single expected occurrence over a simulation lasting about 10$^{292}K^{-1}$
times the age of the universe with a 2 fs timestep).
}
%
Neither does going over the discontinuity within a timestep represent a source of non-conservativeness. The concerned replica
will merely bounce back the corresponding physical end-state with a reversion of its velocity, akin to a particle reflected elastically
by a hard wall (delta-function force).
%
If desired, these \revphil{exceptional points could be handled
more formally by altering} the definition of $\zeta$,
\revphil{{\em e.g.} by smoothing its tips in a narrow range around $0$ and $\pi$}.


The $\lam_k$-dynamics of the individual systems is entirely
specified by \refeq{cb_big_lam_eq_mot} to propagate $\Lamb$
along with  \refeq{cb_lam_of_big_lam} to calculate the $\lam_k$-values from the current $\Lamb$.
Alternatively, one may write an equation of motion for the $\lambda_k$-variables of the 
individual replicas by combining the two equations (along with \refeq{cb_big_lam_eq_mot_der}) as
%
\begin{align}
  \label{eq:cb_lam_eq_mot}
  \ddot{\lam}_k &= \ddot{\Lamb} \frac{\mathrm{d} \lam_k}{\mathrm{d} \Lamb} + \dot{\Lamb}^2 \frac{\mathrm{d^2} \lam_k}{\mathrm{d} \Lamb^2} \nonumber\\
                &= -\frac {1} {m_{\Lamb}} \left( 
                       \sum_{l=0}^{K-1} \frac {\partial\ham(\xv_{l};\lam_{l})} {\partial \lam_{l}} \zeta'\left( \Lamb + 2\pi K^{-1} l \right) 
                       \right) \nonumber  \\
                   & \qquad \zeta'\left( \Lamb + 2\pi K^{-1} k \right) \ .
\end{align}
%
%
Note that the term in $\dot{\Lamb}^2$ vanishes since the second derivative $\zeta''$ of $\zeta$
is zero (except at the \radd{exceptional} singular points). However, it should not be overlooked 
if one decides to use a different function $\zeta$.
%
Introducing the vector $\Dv$ and the symmetric $\Lamb$-dependent matrix $\Cmat$ defined 
by their components as
%
\begin{multline}
  \label{eq:cb_def}
 D_k = \frac{\partial\ham(\xv_{k};\lam_{k})}{\partial \lam_{k}}
\ \ \ \mathrm{and}\\
C_{kl}(\Lambda) = \pi^2 \zeta'\left( \Lamb + 2\pi K^{-1} k \right) \zeta'\left( \Lamb + 2\pi K^{-1} l \right) \ ,
\end{multline}
%
\refeq{cb_lam_eq_mot} can be rewritten in an elegant matrix form as
%
\beq{cb_lam_eq_mot_mat}
  \ddot{\lamv} = -\frac {\Cmat(\Lamb)} {\pi^2 m_{\Lamb}}  \Dv \ .
\eeq
%
The elements of the symmetric matrix $\Cmat$
are either -1 (pair of systems currently on opposite sides of the CB,
and thus moving in opposite directions) or +1
(pair of systems currently on the same side of the CB, and thus moving in the same direction). The diagonal elements
are all +1, and the other +1 values surround the diagonal (line- and column-wise),
the rest being -1 values.
Because $K$ is even, the two types of values 
are always equally represented in the matrix, 
specific locations
depending on $\Lamb$.
%
Note that the variable $\Lamb$ itself still needs to be explicitly propagated
using \refeq{cb_big_lam_eq_mot}.

%%\revphil{Discussion in page 16-18 a little hard to get through.}

For a given configuration $\Xv$ of the replica system, the Hamiltonian
$\ham^\dagger$ of \refeq{ti_rep_ham} (together with \refeq{cb_lam_of_big_lam}) is periodic in $\Lamb$ 
with a period $2\pi$ corresponding to a full rotation of the CB.
%
However, because the Hamiltonians of the individual replicas are 
identical, upon ensemble averaging over $\Xv$, 
one expects the calculated properties to be periodic over $\Lamb$ 
with a smaller period $\Delta \Lamb$, corresponding to one shift of the CB.
%
\revphil{
This is in particular the case for the probability distribution $P(\Lamb)$
along $\Lamb$ and the associated free-energy profile $G_{\Lamb}(\Lamb)$,
given by
%
%\begin{align}
%\label{eq:fre_prof_lam_def}
%%\beq{fre_prof_lam_def}
%  G_{\Lamb}(\Lamb) &= G_{\Lamb}(0) +
%      \int\limits_0^\Lamb \mathrm{d}\Lamb' \left\langle \frac{\partial{\ham^{\dagger}(\Xv;\lamv)}}{\partial \Lamb} \right\rangle^{\dagger}_{\Lamb'} \\ \nonumber
%    &=  G_{\Lamb}(0) + \sum_{k=0}^{K-1}  \left[ G(\lam_k(\Lamb)) - G(\lam_k(0))\right] \quad \text{with} \  \lamv=\lamv(\Lamb) \ ,
%\end{align}
%
\begin{align}
  \label{eq:fre_prof_lam_def}
  G_{\Lamb}(\Lamb) &= G_{\Lamb}(0) +
      \int\limits_0^\Lamb \mathrm{d}\Lamb' \left\langle \frac{\partial{\ham^{\dagger}(\Xv;\lamv)}}{\partial \Lamb} \right\rangle^{\dagger}_{\Lamb'} \nonumber \\
    &=  \tilde{G}_{\Lamb}(0) + \sum_{k=0}^{K-1} G(\lam_k) \quad \text{with} \  \lamv=\lamv(\Lamb) \ ,
\end{align}
%
where $\lamv(\Lamb)$ is defined by \refeq{cb_lam_of_big_lam},
$\langle\cdot\cdot\cdot\rangle^{\dagger}_{\Lamb}$ denotes ensemble averaging 
for the replica system (\ie{} over $\Xv$) at the given $\Lamb$ value,
and the second equality follows from \refeqs{ti_rep_ham} and \refeqn{ti_formula}
(the unknown constant $\tilde{G}_{\Lamb}(0)$ 
is equal to $G_{\Lamb}(0)$ increased by a sum of $-G(\lam_k(0))$
offsets).
%
}
%


\radd{Owing to this periodicity over a smaller interval}, it is convenient to introduce a fractional advance variable $\tilde{\Lamb}$
defined as
%
\beq{tilde_lamb_def}
   \tilde{\Lamb} = \gamma( \Lamb, \Delta \Lamb ),
\eeq
%
where
%
\beq{gamma_def}
   \gamma(\theta,\theta_o)
     = \theta_o  \left( \theta_o^{-1}\theta - \lfloor \theta_o^{-1}\theta \rfloor \right)
\eeq
%
returns the part of $\theta$ in excess of the closest lower integer multiple of $\theta_o$.
%
In contrast to $\Lamb$, which is an unbounded variable, $\tilde{\Lamb}$ only spans a finite definition interval $[0,\Delta \Lamb)$.
%
At full convergence, 
any average property binned as a function 
of $\Lambda$ over the interval $[0,2\pi)$ will 
consist of $K$ successive repeats of the same property binned
as a function of $\tilde{\Lambda}$ over its definition interval $[0,\Delta \Lamb)$, as observed in \reffig{scheme:ene:sinus} for the free energy \radd{$G_{\Lamb}(\Lamb)$}.
%
Accordingly, in the absence of full convergence
along $\Lambda$, binning as a function of $\tilde{\Lambda}$
over the interval $[0,\Delta \Lamb)$ followed by $K$-fold replication
provides an efficient way to construct a more accurate representation of any 
$\Lambda$-resolved average quantity. In fact, the definition interval of $\tilde{\Lamb}$ could be further halved by noting that, upon ensemble averaging over $\Xv$ and for any $\Lamb$ value that is an integer multiple of $\Delta \Lamb$, a forward move of the CB produces the same result as a backward move of the same magnitude. Consequently, $\Lamb$-resolved average properties are even over successive $2\pi K^{-1}$ intervals, as also observed in \reffig{scheme:ene:sinus} for the free energy
\radd{$G_{\Lamb}(\Lamb)$}. The corresponding information is thus entirely encompassed in an interval of size $\Delta \Lamb /2$.

The normalized probability distribution $p(\lam)$ 
along the coupling variable $\lam$ considering all the replicas is defined by
%
\beq{proba_of_lam}
  p(\lam) = K^{-1} \sum_{k=0}^{K-1} \left\langle \delta(\lam_k-\lam) \right\rangle^{\dagger\star} \ ,
\eeq
%
\radd{where $\langle\cdot\cdot\cdot\rangle^{\dagger\star}$ denotes ensemble averaging 
for the extended replica system (\ie{} over the joint trajectories of $\Xv$ and $\lamv$).}
%
At full convergence, this probability over the interval $[0,1]$
will consist of $K/2$ successive
repeats of the corresponding distribution over the interval $[0,2 K^{-1})$.
%
\radd{More precisely, the distribution $p(\lam)$} is related to the distribution 
$\tilde{P}(\tilde{\Lamb})$ of $\tilde{\Lamb}$ over interval $[0,\Delta \Lamb)$
as
%
%\revdavid{Shouldn't there be a factor, \ie
\beq{proba_of_lam_from_that_of_big_lam}
%  p(\lam) = \tilde{P}( \pi \gamma( \lam, 2 K^{-1} )  ) \ .
  p(\lam) = \Delta \Lamb\,\tilde{P}( \pi \gamma( \lam, \pi^{-1}\Delta\Lamb )  )  \ .
\eeq
%}
%
%\beq{proba_of_lam_from_that_of_big_lam}
%%  p(\lam) = \tilde{P}( \pi \gamma( \lam, 2 K^{-1} )  ) \ .
%  p(\lam) = \tilde{P}( \pi \gamma( \lam, \pi^{-1}\Delta\Lamb )  ) \ .
%\eeq
%
In plain words, this means that $\tilde{P}(\tilde{\Lamb})$ is the relevant quantity
in terms of sampling along the coupling variable $\lam$.
If it is close to uniform over the range $[0,\Delta \Lamb)$,
then $p(\lam)$ will also be close to uniform over the range $[0,1]$.
Here again, it is noted that $p(\lam)$ is also even over the interval $[0,2 K^{-1})$,
and could be mapped to a $\tilde{\Lamb}$ value defined over an interval of size $\Delta \Lamb / 2$ instead of $\Delta \Lamb$ if desired, as

\beq{sym_proba_lam}
  p(\lam) = \left\{
                    \begin{array}{ll}
%                       \tilde{P}(\pi \gamma(\lam, 2 K^{-1})) & \mathrm{if}\ \gamma(\lam, 2 K^{-1}) < K^{-1} \\
                       \Delta \Lamb\,\tilde{P}(\pi \gamma(\lam, \pi^{-1}\Delta\Lamb)) & \mathrm{if}\ \gamma(\lam, \pi^{-1}\Delta\Lamb) < (2\pi)^{-1}\Delta\Lamb \\
%                       \tilde{P}(\pi \gamma(1-\lam,  2 K^{-1})) & \mathrm{otherwise} \\
                       \Delta \Lamb\,\tilde{P}(\pi \gamma(1-\lam,  \pi^{-1}\Delta\Lamb)) & \mathrm{otherwise} \\
                    \end{array} 
                  \right. .
\eeq


\radd{
%------------------------------------------------------
\subsection{CBTI Free-energy Estimator}
%------------------------------------------------------------
}


Due to the constraints coupling the $\lam_k$-values of the
$K$ replicas, the function $p(\lam)$ of \refeq{proba_of_lam}
is by no means a Boltzmann distribution in terms of the single-system
Hamiltonian. In fact, as seen above, it consists at full convergence
of $K/2$ successive repeats of the same even
curve.
In addition, compared to the Boltzmann distribution,
it will be
significantly flatter. On the one hand, the smaller amplitude of variations 
%and the reduced barriers 
are desired, as they will lead to more homogeneous sampling 
and \revphil{are expected to ease transitions along $\Lamb$ 
(up to the limit imposed by the speed of random diffusion)}.
%
%\revphil{
%Page 18: says that 'there is a smaller amplitude of variations and
%reduced barriers', but it's not clear dynamics are faster, since the
%dH/dl force is slowed down by (averaged). I think this is eventually
%explained by the fact that it becomes nearly diffusive behavior, but
%perhaps this can be brought up sooner.
%}
%
%
On the other hand, it is no longer possible to evaluate 
the free-energy difference $\Delta G$
directly from $p(\lam)$ in analogy with the $\lam$D expression of \refeq{lam_dyn_dg}.
%
However, since the dynamics remains Hamiltonian and the coupling 
between replicas does not
involve the configurational degrees of freedom, the change \radd{from TI to CBTI} does not affect 
the conditional probabilities $\mathcal{P}(\xv | \lam)$. Thus,
configurational ensemble averages sorted by $\lam$-values will remain
identical to those one would obtain from TI \radd{(or from HRE/HRP or $\lam$D)}.
%
As a result, $\Delta G$ can still be obtained 
\radd{by integrating over the average Hamiltonian derivative binned as a function 
of $\lambda$ considering all replicas simultaneously,}
in analogy with the TI expression
of \refeq{ti_formula}.
%
\revphil{Note that the exceptional points of the function $\zeta'$
(discussed previously in the context of \refeq{zigzag_der})
have no influence on the integration, as they represent
finite discontinuities over infinitesimal ranges.
}
%\revphil{SAY THAT DISCONT HAS NO EFFECT
%See Point 6 above. Similar considerations apply to the integration.
%We integrate $\partial\mathcal{H}/\partial\lam$. If the value is off
%at one (or a few) points, the integral is unaffected because the 
%neighborhood of the singularity is infinitesimal and its magnitude
%is finite ({\em i.e.} it is not a $\delta$ function, but just 
%a finite change in the function value).
%}



In practice, \revphil{$\Delta G$} is calculated here based on a single thermostated MD simulation 
of the extended replica system, as
%
\begin{align}
\label{eq:cbti_formula}
  \Delta G &= 
    \int_0^1 \mathrm{d}\lam' K^{-1} \sum_{k=0}^{K-1} \left\langle 
                 \frac{\ham(x_k;\lam_k)}{\partial \lam_k} \delta(\lam_k-\lam')
            \right\rangle^{\dagger\star}\\ \nonumber
   &\approx
%     K^{-1} 
\sum_{j=0}^{J-1}  \left \langle \frac{ \sum_{k=0}^{K-1} \frac{\ham(x_k;\lam_k)}{\partial \lam_k} \alpha(\lambda_k,j;J)}
                              { \sum_{k=0}^{K-1} \alpha(\lambda_k,j;J)}  \right \rangle^{\dagger \star}                                   ,
\end{align}
%
where 
%
\beq{binning_fct}
\alpha(\theta,j;J)=\begin{cases}
1 \qquad \text{if}\ j\leq J\theta < j+1 \\
0 \qquad \text{otherwise}
\end{cases}
\eeq
%
is a binning function  corresponding to a discretization of the $\lam$-interval $[0,1]$ using
$J$ bins. 
%
The approximation in \refeq{cbti_formula} corresponds to a 
simple forward rectangular quadrature, where the Hamiltonian derivative is 
averaged over the $J$ successive bins considering all replicas.
%
Since $\tilde{P}(\tilde{\Lamb})$, and thus $p(\lam)$, will typically be close to homogeneous, $J$ can be taken very large, 
resulting in a negligible quadrature error. 
For example, if $K$ replicas sample $L$ configurations each, the number of data points
per bin will be close to $K L/J$, with limited variations across bins. 
Defining
the maximal allowed value $J_{\mathrm{max}}$ as the highest value of $J$ for which 
empty bins (vanishing denominator in the ensemble average of \refeq{cbti_formula}) never 
occur, a graph of $\Delta G$ evaluated upon increasing $J$ from 1 to $J_{\mathrm{max}}$ 
will rapidly level off to a plateau when quadrature errors become negligible.
%sufficiently small. 
%
\revphil{
Two variants which do not require the specification 
of a number of bins are also proposed in \refsec{CH2C} (\refeqs{cbti_formula_app1} and \refeqn{cbti_formula_average}).}
%
\revphildel{[-- Variants moved to Appendix C --]}



%------------------------------------------------------------
\subsection{CBTI with Memory-based Biasing Potential}
%------------------------------------------------------------

%For a 
When using a large number of replicas, the sampling along $\lam$ afforded by the CBTI scheme will be close to homogeneous. However, for practical reasons (\eg{} number of processors available on a computer node), one may wish to use a small number of replicas. In this case, the sampling homogeneity can be enhanced by addition of a biasing potential. It is sufficient to apply this potential to the fractional advance variable $\tilde{\Lambda}$ over the range $[0,\Delta \Lamb)$. With inclusion of a biasing potential $\mathcal{B}$, \refeq{cb_big_lam_eq_mot} becomes 
%
  \begin{multline}
  \label{eq:cb_big_lam_eq_mot_bias}
  \ddot{\Lamb} = -\frac {1} {m_{\Lamb}} \frac {\partial}{\partial \Lamb} \left ( \ham^\dagger(\Xv;\lamv)  
  + \mathcal{B}(\tilde{\Lamb}) \right) \\ \text{with} \  \lamv=\lamv(\Lamb)\ \text{and}\ \tilde{\Lamb}=\tilde{\Lamb}(\Lamb) \ ,
  \end{multline}
%
where $\mathcal{H}^{\dagger}$ is defined by \refeq{ti_rep_ham}, $\lamv(\Lamb)$ by \refeq{cb_lam_of_big_lam} and $\tilde{\Lamb}(\Lamb)$ by \refeq{tilde_lamb_def}. 


In analogy with the $\lam$-LEUS scheme\cite{BI14.1,BI14.2,BI15.1,BI15.2}, this biasing potential can be expressed as a sum of local grid-based spline functions, built in a LE preoptimization phase and frozen in a subsequent US sampling phase. However, the duration of the LE phase can be considerably reduced compared to a single-system $\lam$-LEUS simulation, considering that $\tilde{P}(\tilde{\Lamb})$ is already close to homogeneity in the absence of biasing and that the support interval is reduced to the $\tilde{\Lamb}$-range $[0,\Delta \Lamb)$. The latter interval can actually be further restricted to $[0,\Delta \Lamb / 2]$ considering the even symmetry of $\tilde{P}(\tilde{\Lamb})$, \ie{} by enforcing an even symmetry of $\mathcal{B}$ as well.

Since the application of a biasing potential that only involves the $\lam_k$-variables does not alter the 
conditional probabilities $\mathcal{P}(\xv|\lam)$, \refeq{cbti_formula} \revphil{(or the variants of \refeqs{cbti_formula_app1} and \refeqn{cbti_formula_average})}
can still be employed without any modification to evaluate the free-energy change. In other words, 
in contrast to the $\lam$-LEUS scheme, the CBTI scheme with the presented TI-like free-energy estimator does not require any reweighting.


%================================================================================
\section{Computational Details}
%================================================================================


\subsection{Validation of the Restraint Selection Algorithm}
To assess the performance of the greedy algorithm for selecting optimal distance restraints between two molecules, it was first tested on toy models. These contained 12 to 30 particles that were randomly distributed in space. The particles were randomly assigned to two entities representing two molecules. A selection of four restraints was performed with no pre-processing steps. Different algorithmic approaches were compared: the developed greedy algorithm, an averaged random selection (100 repetitions), and two brute-force approaches. One of the brute-force approaches maximizes the sum of the restraint midpoint distances between the selected restraints by considering all possible quadruples of restraints explicitly (BF-maxD). The other one maximizes the CHV around the selected restraints (BF-maxCHV), as done for chaining in multi-state systems. Each number of particles was sampled 20 times (using different particle coordinates each time) to provide an uncertainty estimate. The scripts for this validation are available in the example folder of the GitHub repository (\textit{examples/publication/a\_benchmark\_algorithms}).

\subsection{Molecules with Hydration Free Energies}
The algorithm was applied for the calculation of relative hydration free energies $\Delta \Delta G_{\text{hyd}}$ (Figure \ref{fig:thermodynamic_cycle}).
A set of 16 molecules with experimentally available hydration free energies\cite{Wolfenden1987,Rizzo2006,Nicholls2008,Guthrie2009,Guthrie2014,Mobley2014} were selected (Table S1 in the Supporting Information). % SUPPL tab:SI_moleculelist
The topologies for these molecules were taken from the ATB server.\cite{Stroet2018} The selected molecules are small and possess a ring core. The corresponding pairwise transformations are nevertheless relatively complex, and involve R-group and ring-size changes as well as scaffold hopping-type transformations (e.g. benzene to cyclohexane).

\begin{figure}[h!]
    \centering
    \includegraphics[width=0.8\textwidth]{fig/theory/ThermCycle.png}
    \caption{Thermodynamic cycle for the calculation of relative hydration free energies $\Delta \Delta G_{\text{hyd}_{AB}}$. The direct way to obtain $\Delta \Delta G_{\text{hyd}_{AB}}$ employs two absolute free-energy calculations giving $\Delta G^\text{abs}_{\text{hyd}_A}$ and $\Delta G^\text{abs}_{\text{hyd}_B}$. The indirect way uses two alchemical or relative free-energy calculations giving $\Delta G^\text{rel}_{\text{vac}_{AB}}$ and $\Delta G^\text{rel}_{\text{wat}_{AB}}$.
    }
    \label{fig:thermodynamic_cycle}
\end{figure}

Pairwise TI calculations were carried out with a linked dual topology approach for the 16 molecules in a star-like scheme with molecule \textbf{12} as center, resulting in 15 relative hydration free energies (Figure \ref{fig: Pairwise_TI_M030_Graph}).
\begin{figure}[h]
    \centering
    \includegraphics[width=\textwidth]{fig/methods/StateGraph_TI_2D_enumerated.png}
    \caption{Set of 16 molecules with experimental hydration free energies available \cite{Stroet2018,Wolfenden1987,Rizzo2006,Nicholls2008,Guthrie2009,Guthrie2014,Mobley2014}. The black lines indicate the pairs of molecules for which TI calculations were performed. RestraintMaker was used to select pairwise distance restraints between the central molecule and all others (Figure S1 in the Supporting Information).} % SUPPL SIfig: Pairwise_TI_M030_Graph
    \label{fig: Pairwise_TI_M030_Graph}
\end{figure}

\subsection{Simulation Details}
All simulations were carried out using the MD software package GROMOS\cite{Schmid2012} version 1.5.0 (freely available on \textit{http://www.gromos.net}),\cite{Ries2021B} the Python RE-EDS pipeline\\ (\textit{https://github.com/rinikerlab/reeds}) 
and PyGromosTools\cite{Lehner2021}\\ (\textit{https://github.com/rinikerlab/PyGromosTools}). 

In order to compare our results with the absolute hydration free energies reported in the ATB server,\cite{Stroet2018} the same simulation setup was used.
The simple point-charge (SPC) model\cite{Berendsen1981} was employed for water. A single cutoff radius of $1.2$~nm was used for the calculation of the non-bonded interactions. The integration time step was set to $2$~fs and the pairlist was updated every five steps. Long-range nonbonded interactions were calculated using a reaction-field correction\cite{Tironi1995} with $\varepsilon_{\text{rf}}=1$ for the simulations in vacuum and $\varepsilon_{\text{rf}}=61$ for the simulations in water.\cite{Heinz2001} The force constant for the distance restraints was set to $5000$ kJ$/($mol$\cdot$nm$^2)$.

\subsubsection{Thermodynamic Integration}
The topologies and coordinate files of the single states were obtained from the ATB server\cite{Stroet2018} and were merged to pairs using PyGromosTools\cite{Lehner2021}. 
The coordinates of the different single molecules were aligned to each other using the common molecular skeleton of the molecules (only rings), with the \textit{align} function in RDKit\cite{Landrum2021}. After the alignment, RestraintMaker was used to place four restraints with $d_\text{res} = 0.1$~nm (Figures S1 and S2 in the Supporting Information). %SUPPL
The computational boxes for the simulations in water were generated with the GROMOS++ \cite{Eichenberger2011} program \textit{simbox} using a minimal solute-to-wall distance of $0.8~$nm, and relaxed by energy minimization. The scripts can be found in the example folder on Github (\textit{https://github.com/rinikerlab/restraintmaker/tree/main/examples/publication/\\b\_ATB\_solvationFreeEnergies}).
The TI calculations were carried out with 21 evenly spaced $\lambda$-points between 0 and 1, both for the molecules in water and in vacuum. Each $\lambda$-point was equilibrated for 1~ns, followed by a production run of 5~ns. The free-energy differences were calculated using the Simpson integration implemented in the SciPy library.\cite{Virtanen2020}

\subsection{Analysis}
The analysis of the simulations was carried using GROMOS++ \cite{Eichenberger2011} and PyGromosTools \cite{Lehner2021}. In addition, the following Python packages were employed for the statistical analysis and plotting: Pandas \cite{Mckinney2010}, Matplotlib \cite{Hunter2007}, NumPy \cite{Vanderwalt2011}, SciPy \cite{Virtanen2020}, mpmath \cite{Johansson2013}, and Jupyter notebooks.\cite{Kluyver2016}


%================================================================================
\section{Results and Discussion}
%================================================================================

The chosen model system of five inhibitors of CHK1 kinase exemplifies different core-hopping transformations (i.e. ring size change, ring opening/closing, ring extension) and R-group modifications \cite{Wang2017}, increasing the complexity compared to the systems previously studied with RE-EDS. Furthermore, the performance can be directly compared to the results obtained with FEP+ and OPLS3 in Ref.~\cite{Wang2017} as well as with QligFEP results in Ref.~\cite{Jespers2019}.

\subsection{Parameter Exploration and Parameter Optimization}
The RE-EDS workflow was started by estimating the lower bound for the $s$-distribution. Using the above mentioned undersampling criterion (see Methods section), a lower bound of $s=0.003$ was determined. 
%State Optimizations
Optimized coordinates were obtained for all five ligands, as verified by comparing the potential-energy distribution from the EDS simulation with the one extracted from a standard MD simulation of the respective ligand (Figure S1 in the Supporting Information). % SUPPL
From these same steps, the potential-energy thresholds for the occurrence sampling ($T_{i}^{\text{phys}}$) and undersampling ($T_{i}^{\text{us}}$) were estimated.

%Eoff:
The energy offsets $\vec{E}^R$ were estimated from a short RE-EDS simulation with the PEOE \cite{Sidler2016} scheme and are listed in Table \ref{tab:CHK1_set2_Eoff}.
For $s=1.0$, the energy offsets should ideally be equal to the free energy of the corresponding state (i.e. $\Delta E^R_{ji} = \Delta G_{ji}$) such that the partition function of the reference state is the sum of the partition functions of the end states \cite{Christ2008}. Therefore, the comparison between the relative estimated energy offsets in water and in complex ($\Delta \Delta E^R_{ji} = \Delta E^R_{ji,\text{complex}} - \Delta E^R_{ji,\text{water}}$) and the relative binding free energy $\Delta \Delta G^\text{bind}_{ji}$ can be used to (roughly) assess the quality of the estimated energy offsets. As shown in Figure S2 in the Supporting Information, % SUPPL
the energy offsets estimated from the SSM simulations are in better agreement with the experimental relative binding free energies than those estimated from the 1SS simulations.

\begin{table}[h]
	\caption{Energy offsets $\vec{E^R}$ estimated from a short RE-EDS simulation using the PEOE \cite{Sidler2016} scheme. The errors indicate the standard deviation over the different replicas in undersampling. All energy offsets were calculated relative to ligand L1. The starting coordinates were selected following the 1SS or the SSM approach (see Theory and Methods sections).}
	\label{tab:CHK1_set2_Eoff}
	\resizebox{\columnwidth}{!}{%
		\centering
		\begin{tabular}{ l | r r | r r }
			Ligand & \multicolumn{2}{c|}{RE-EDS 1SS}&\multicolumn{2}{c}{RE-EDS SSM}  \\ 
			&Water [kJ~mol$^{-1}$]&Complex [kJ~mol$^{-1}$]&Water [kJ~mol$^{-1}$]&Complex [kJ~mol$^{-1}$]\\ 
			\hline
			L1 & $0.0$ & $0.0$ & $0.0$ & $0.0$ \\ 
			L17 & $15.9 \pm 4.9 $ & $14.1 \pm 1.9 $ & $12.9 \pm 2.3 $ & $19.1 \pm 3.2$ \\
			L19 & $9.6 \pm 5.3 $ & $ -5.8 \pm 0.5 $ & $-5.4 \pm 4.7 $ & $ -2.3 \pm 3.1$ \\
			L20 & $-52.4 \pm 4.4$ & $ -48.5 \pm 3.0$ & $ -49.7 \pm 8.8 $ & $-55.0 \pm 1.5$\\
			L21 & $-69.9 \pm 1.8 $&$ -70.0 \pm 8.8$ & $ -72.5 \pm 6.0 $ & $ -72.9 \pm 3.0$\\
		\end{tabular}
	}
\end{table}

%S-Optimization
The optimization of the $s$-distribution was performed with the N-LRTO \cite{Sidler2017} algorithm, thereby minimizing the average round-trip time $\overline{\tau}$ in the replica graph. 
%
In the first iteration, the number of total round trips is relatively small and the average round-trip time is large for all simulations (Figure \ref{fig: CHK1_RingOpening_sOptimization}). The number of round trips is smaller in the complex than in water due to a more pronounced gap region \cite{Sidler2017}.
Already after the second iteration, the round-trip time is generally reduced. An exception was observed for 
RE-EDS 1SS in the complex, where no round trips occurred during the second iteration. The improvement of the $\overline{\tau}$ over the iterations can also be seen in Figure S4 in the Supporting Information. % SUPPL
%% s-replica placements
As can be seen in the third row of Figure \ref{fig: CHK1_RingOpening_sOptimization}, the optimization algorithm increases the density of the replicas around $s = 0.041$, where the major gap region lies.

\begin{figure}[h]
	\centering
	\includegraphics[width=\textwidth]{fig/results/ringOpening/paramOptimization/S-optimization_ringOpening.png}
	\caption{Optimization of the $s$-distribution with the N-LRTO \cite{Sidler2017} algorithm over eight iterations. The measured quality criteria were the number of round trips (1. row), the average round-trip time $\overline{\tau}$ (2. row), the placement of the replicas in $s$-space (3. row), and the sampling fractions of dominating states $f_{i}^{\text{domin}}$ (4. row).}
	\label{fig: CHK1_RingOpening_sOptimization}
\end{figure}

%% states sampling
In addition, we monitored the relative sampling of the end states at $s=1.0$ during the iterations.
Ideally, each end state should be sampled equally in an optimized RE-EDS simulation (see Eq. (\ref{eq: optimalDominationSamplingDist})).  The last row in Figure \ref{fig: CHK1_RingOpening_sOptimization} shows $f_{i}^{\text{domin}}$ as a function of the iteration. For all end states, the sampling fraction converges (slowly) towards the ideal value.
%% tau converge - Conclusion
The $s$-optimization for the ligands in water converged after the fourth iteration with $\Delta \overline{\tau} < 10~ps$ (Figure S4 in the Supporting Information). % SUPPL
This resulted in the final 36 replicas.
For the protein-ligands complex, the optimization converged after the fifth iteration, resulting in 41 replicas. 
The average round-trip time after convergence was $\overline{\tau} = 9.6 \pm 0.9$~ps for all simulations.

\subsection{Free-Energy Calculation}
After successfully optimizing the RE-EDS parameters, the production runs were performed for $4$~ns. 
%%Sampling++
Both in water and in complex, the potential-energy distributions of the end states match generally well the corresponding distributions from the standard MD simulations of the single end states (Figure \ref{fig:RingOpening_sampling_comparison}). 
%
The analysis of the dominating end states at $s=1.0$ shows that L19 is generally oversampled, while L20 and/or L21 are less sampled than expected (Figure S5 in the Supporting Information). % SUPPL
At $s=1.0$, $f_i^{\text{occur}}$ and $f_i^{\text{domin}}$ are very similar, which indicates sampling of clearly separated states in those simulations.

\begin{figure}[h]
	\centering
	\includegraphics[width=\columnwidth]{fig/results/ringOpening/FE/RingClosure_system_final_sampling.png}
	\caption{Comparison of the Boltzmann reweighted potential-energy distributions obtained from standard MD simulations of a given end state (black) and from the RE-EDS production runs (colored).}
	\label{fig:RingOpening_sampling_comparison}
\end{figure}

%%Accuracy
From the replica at $s=1.0$, the free-energy differences were calculated using Eq.~(\ref{EQ: Free Energy calculation via reference state}) and the resulting $\Delta \Delta G^\text{bind}_{ji}$ were compared with the experimental results taken from Ref.~\cite{Huang2012}. The results are shown graphically in Figure \ref{fig:CHK1_set2_FreeEnergyCalculation} and numerically in Table \ref{SItab: RE-EDS_FE_RingCycleOpening_ddF}. The individual free-energy differences are given in Table S3 in the Supporting Information. %SUPPL
The RMSE with RE-EDS 1SS is $7.3$~kJ~mol$^{-1}$ and the MAE is $5.75\pm4.4$~kJ~mol$^{-1}$. 
%
%how I calculate the MAE and RMSD:
%MAE = np.mean(np.abs(ddG_differences))
%std(MAE) = np.std(np.abs(ddG_differences)) #gives an impression of absolute deviation of all ddG_diffs
%RMSE = np.sqrt(np.mean(np.square(ddG_differences))
%
The main deviations stem from ligand L19 in the RE-EDS 1SS approach.

The performance was substantially improved using the SSM approach with RE-EDS, giving an RMSE of $2.5$~kJ~mol$^{-1}$ and an MAE of $2.1 \pm 1.3$~kJ~mol$^{-1}$. 
Only one value (L21-L17) deviates more than $4.184$~kJ~mol$^{-1}$ (i.e. $1$~kcal~mol$^{-1}$) from experiment.
The Spearman correlation coefficient for RE-EDS 1SS is $r^2_{\text{Spearman}}=0.87$ and for RE-EDS SSM $r^2_{\text{Spearman}}=0.84$.

%%Performance:
Next, we assessed the convergence of the $\Delta G_{ji}$ values as a function of simulation time (Figure S7 in the Supporting Information). % SUPPL
For the RE-EDS 1SS approach, all free-energy differences appeared converged after $1.48$~ns in water and after $1.04$~ns in the complex. For the RE-EDS SSM approach, convergence was observed after $0.52$~ns in water and after $0.88$~ns in the complex. These findings indicate that the use of different starting configurations representing all end states enhances sampling further and reduces the simulation time needed to obtain converged results.

\begin{figure}[h]
	\centering
	\begin{subfigure}{0.85\columnwidth}
		\includegraphics[width=\textwidth]{fig/results/ringOpening/FE/RingClosure_system_final_results_4ns.png}
	\end{subfigure}
	\begin{subfigure}{0.85\columnwidth}
		\includegraphics[width=\textwidth]{fig/results/ringOpening/FE/ddG_bind_paper_comparison_reeds_only_4nsSimulation.png}
	\end{subfigure}
	\caption{Free-energy differences estimated from the production run of $4$~ns length. (Top): Comparison between the experimental and calculated $\Delta \Delta G^\text{bind}_{ji}$ using RE-EDS 1SS and RE-EDS SSM. (Bottom): Graphical representation of the $\Delta \Delta G^\text{bind}_{ji}$ results with structures, inspired by the one in Ref.~\cite{Wang2017}.}
	\label{fig:CHK1_set2_FreeEnergyCalculation}
\end{figure}

%Comparison results with Schroedinger & Jespers
By applying the RE-EDS methodology to the same system of five CHK1 inhibitors as studied by Wang \textit{et. al.} \cite{Wang2017} and later on also Jespers \textit{et al.} \cite{Jespers2019}, a direct comparison with FEP+ and QligFEP is possible (Table \ref{SItab: RE-EDS_FE_RingCycleOpening_ddF}). Note that the quality metrics were calculated over all possible pairs of ligands, not only those directly calculated by FEP+ and QligFEP.
For FEP+, we obtained an RMSE of $2.4$~kJ~mol$^{-1}$ and an MAE of $1.8 \pm 1.2$~kJ~mol$^{-1}$ with a Spearman correlation coefficient of $r^2_{\text{Spearman}}=0.67$.
Including cycle closure correction (CC) \cite{Wang2017} reduced the RMSE to $2.1$~kJ~mol$^{-1}$ and the MAE to $1.9 \pm 1.0$~kJ~mol$^{-1}$. The Spearman correlation coefficient increased to $r^2_{\text{Spearman}}=0.73$.
Jespers \textit{et al.} \cite{Jespers2019} reported free-energy differences with QligFEP as an average over ten independent replicas, each with significantly less simulation time per $\lambda$-window than in Ref.~\cite{Wang2017}. For QligFEP, an RMSE of $2.3$~kJ~mol$^{-1}$, an MAE of $2.0 \pm 1.2$~kJ~mol$^{-1}$, and a Spearman coefficient of $r^2_{\text{Spearman}}=0.61$ was obtained.

Overall, the performance of RE-EDS SSM is comparable with the pairwise methods. The results with FEP+ CC and QligFEP showed a slightly higher accuracy compared to experiment, likely due to the different force fields used. The Spearman correlation coefficient is higher for both RE-EDS approaches than with the pairwise approaches, indicating a good ranking of the ligands with the RE-EDS method.
A strong correlation with experiment is of interest in drug design approaches, as the ranking of ligands in virtual screening is important to suggest the most promising drug candidates to be synthesized.

In terms of computational cost, the RE-EDS approach (with 4~ns per replica) resulted in about half the total simulation time (in ns) than reported for the FEP+ calculations in Ref.~\cite{Wang2017}. A major advantage of the simultaneous simulation of multiple ligands in a single RE-EDS simulation is that all $N(N-1)/2$ transformations are sampled directly, leading to low statistical errors and removing the need of a state graph. This advantage increases with increasing number of ligands. The current workflow of RE-EDS uses a relatively large amount of simulation time for parameter optimization. Future work will focus on further optimization of the workflow to reduce the pre-processing time. 

\begin{table}[h]
	\caption{Relative binding free energies $\Delta \Delta G^\text{bind}_{ji}$ from experiment and calculated with the RE-EDS 1SS and RE-EDS SSM approaches. For comparison, the results for FEP+ with and without cycle closure (CC) correction taken from Ref.~\cite{Wang2017} and the results for QligFEP taken from Ref.~\cite{Jespers2019} are listed. The free-energy differences of directly simulated paths were used to infer not directly simulated free-energy differences (marked in bold). If multiple indirect paths were possible, their average was used. The errors for QligFEP were determined in Ref.~\cite{Jespers2019} by calculating the standard deviation over ten replicas. For FEP+, the error of the results was taken from the used BAR \cite{Bennett1976} method and the FEP+ CC errors were obtained from the cycle closure analysis. For the RE-EDS approaches, the reported error is based on the statistical uncertainties of the $\Delta G_{ji}^{env}$ values estimated using Gaussian error approximation \cite{Christ2008}.}
	\begin{center}
		\footnotesize
		\resizebox{\columnwidth}{!}{%
			\begin{tabular}{ c c |c |c|c|c|c|c}
				\multicolumn{2}{c|}{Ligands} & \multicolumn{1}{c|}{Exp. \cite{Huang2012}} &\multicolumn{1}{c|}{FEP+ \cite{Wang2017}}&\multicolumn{1}{c|}{FEP+ CC \cite{Wang2017}}&\multicolumn{1}{c|}{QligFEP \cite{Jespers2019}}&\multicolumn{1}{c|}{RE-EDS 1SS}&\multicolumn{1}{c}{RE-EDS SSM}\\ 
				$i$ & $j$  & [kJ~mol$^{-1}$]  & [kJ~mol$^{-1}$] & [kJ~mol$^{-1}$] & [kJ~mol$^{-1}$] & [kJ~mol$^{-1}$] & [kJ~mol$^{-1}$]  \\
				\hline
				L17 &  L1 &   0.1 & -3.6 $\pm$ 0.4          & -2.9 $\pm$ 1.0         & -1.6 $\pm$ 1.7                                     &   1.2 $\pm$ 0.3 &  3.4 $\pm$ 0.2 \\
				L19 &  L1 &  -4.8 & -3.9 $\pm$ 0.3          & -4.0 $\pm$ 0.6         & -1.7 $\pm$ 2.0                                     & -14.0 $\pm$ 0.3 & -3.9 $\pm$ 0.3 \\
				L20 &  L1 &  -2.0 & -2.5 $\pm$ 0.1          & -3.1 $\pm$ 1.0         & -1.3 $\pm$ 1.3                                     &   2.6 $\pm$ 0.3 & -2.6 $\pm$ 0.4 \\
				L21 &  L1 &  -2.3 &\textbf{-3.4} $\pm$ \textbf{0.7}  &\textbf{-3.2} $\pm$ \textbf{1.3} & \textbf{-0.1} $\pm$ \textbf{3.5} &  -1.7 $\pm$ 0.4 & -3.6 $\pm$ 0.9 \\
				L19 &  L17 & -4.9 & -1.4 $\pm$ 0.3          & -1.1 $\pm$ 1.0         & \textbf{0.1} $\pm$ \textbf{2.6}                    & -15.2 $\pm$ 0.2 & -7.3 $\pm$ 0.2 \\
				L20 &  L17 & -2.1 &  0.3 $\pm$ 0.4          & -0.1 $\pm$ 0.8         & -1.3 $\pm$ 2.3                                     &   1.4 $\pm$ 0.3 & -6.0 $\pm$ 0.4 \\
				L21 &  L17 & -2.4 & -1.1 $\pm$ 0.4          & -0.9 $\pm$ 0.9         &\textbf{0.7} $\pm$ \textbf{2.6}                     &  -2.9 $\pm$ 0.4 & -7.0 $\pm$ 0.9 \\
				L20 &  L19 & 2.8  &\textbf{0.8} $\pm$ \textbf{0.6}   & \textbf{0.1} $\pm$ \textbf{1.3} & \textbf{-0.4} $\pm$ \textbf{3.7} &  16.6 $\pm$ 0.4 &  1.3 $\pm$ 0.4 \\
				L21 &  L19 & 2.5  & -0.1 $\pm$ 0.6         &  0.6 $\pm$ 0.1         &  0.6 $\pm$ 4.9                                      &  12.3 $\pm$ 0.5 &  0.3 $\pm$ 0.9 \\
				L21 &  L20 & -0.3 & -0.3 $\pm$ 0.8         & -0.6 $\pm$ 0.8         &  0.6 $\pm$ 1.1                                    &  -4.3 $\pm$ 0.5 & -1.0 $\pm$ 0.9 \\ 
				\hline
				\multicolumn{2}{c|}{RMSE} &                    & 2.4            & 2.1           &  2.3          & 7.3          & 2.5 \\
				\multicolumn{2}{c|}{MAE} &                     & 1.8 $\pm$ 1.2 & 1.9 $\pm$ 1.0 & 2.0 $\pm$ 1.2 & 5.8 $\pm$4.4 & 2.1 $\pm$ 1.3 \\
				%\multicolumn{2}{c|}{$r^2_{\text{Pearson}}$} & & 0.66            & 0.67          & 0.63          & 0.83          & 0.83 \\
				\multicolumn{2}{c|}{$r^2_{\text{Spearman}}$} & & 0.67           & 0.73          & 0.61          & 0.87          & 0.84 \\
			\end{tabular}
		}
	\end{center}
	\label{SItab: RE-EDS_FE_RingCycleOpening_ddF}
\end{table}

% 
%FEP+: 
%MAE 1.99 +- 1.4
%RMSE 2.43
%r^2 pearson 0.56 	 r^2 spearman 0.56
%
%FEP+ CC: 
%MAE 1.88 +- 1.05
%RMSE 2.15
%r^2 pearson 0.66 	 r^2 spearman 0.71

%QLigFEP
%MAE 1.97 +- 1.2
%RMSE 2.3
%r^2 pearson 0.63 	 r^2 spearman 0.61

%1SS: 
%MAE 5.75 +- 4.45
%RMSE 7.27 +- 7.27
%r^2 pearson 0.83 	 r^2 spearman 0.87

%SSM: 
%MAE 2.11 +- 1.32
%RMSE 2.49 +- 2.49
%r^2 pearson 0.83 	 r^2 spearman 0.84


%ComputationalCost
%FEP: 16 l-windwos*5ns*4 pairs * 2 approaches ==> 640ns
%QligFEP: 10 repetitions * (eq 131ps + 51lams * 10ps sim) * 4 pairs * 2 approaches ==> 51ns
%RE-EDS: 4ns * (41+36) ==> 308ns
%RE-EDS opt: eoff(21*0.8)+sop1(0.4ns*21)+sopt2(0.8ns*25)+sopt3(1.2ns*29)+sopt4(1.2ns*31) | == 234.4ns
%+Complex stop water +sopt5(1.2ns*35) == 42ns


\clearpage
\newpage

%================================================================================
\section{Conclusion}
%================================================================================

In this work, we presented an efficient algorithm for the appropriate placement of distance restraints in free-energy calculations performed with the linked dual topology approach. Linked dual topologies have the advantage that larger transformations can be simulated in a straightforward manner (e.g. no soft bonds are required), while reducing the sampling complexity. 
With the developed RestraintMaker Python package, distance restraint sets can be created from a script or at GUI level, and written out in the GROMOS and GROMACS formats or in JSON format.
The greedy algorithm is a graph-based approach and can be straightforwardly applied to molecules with (semi)rigid cores (typically aromatic or aliphatic rings). The only required user inputs are the number of restraints $n_\text{res}$ to be selected and the maximum distance between the restrained atoms $d_\text{res}$ .
%%%%%ToyModelFun
The performance of the algorithm was evaluated using toy systems (particle clouds) and compared to two brute-force approaches. In view of the results, the greedy algorithm represents a good trade-off between computing time and accuracy.

%%%%%free energy
RestraintMaker was used to select optimal distance restraints for the calculation of relative hydration free energies with both TI (pairwise) and RE-EDS (multi-state). In all cases, good agreement between the different free-energy methods and with experiment was observed. Detailed analysis of the conformational sampling also indicated that the effect of the possible distortions induced by the distance restraints on the conformations is negligible. Even when restraining the benzene core and the cyclohexane core of two molecules together, accurate free-energy differences were obtained and the distributions of the pseudo torsional angles of the cyclohexane ring were nearly identical with those from plain MD simulations.
The results with RE-EDS highlighted the superior sampling efficiency of the method. In the future, we will apply RestraintMaker in RBFE calculations with RE-EDS.

%As an outlook, we would like to point out that our future work on multistate methods will be expanding on this work, and we are looking forward to using the approach for more complex systems. Interesting challenges include increasing the number of states significantly or going to larger molecules.



\clearpage
\pagebreak

%\bibliography{4_chapter_2/ref/ref.bib}


\end{document}

