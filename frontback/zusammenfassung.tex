\chapter{Zusammenfassung}

Molekulardynamikcomputersimulationen (MD-\-Simulationen) sind 
heutzutage ein n\"utzliches Mittel, um Ph\"{a}nomene 
in der Chemie und Biologie zu studieren.
%
Die Geschichte chemischer Modelle sowie der methodische
und theoretische Basis von MD-Simulationen wird
im \ref{ch:intro}.~Kapitel pr\"{a}sentiert.

Kapitel~\ref{ch:resa} besch\"{a}ftigt sich mit
der Anwendung von MD-Simulationen auf Resorcin[4]arene in explizitem 
L\"{o}sungsmittel. Diese K\"{a}figmolek\"{u}le k\"{o}nnen
geschlossene/kontrahierte (\vase{}) und
offene/ausgestreckte (\kite{}) Konformationen annehmen.
%
Die Gleichgewichte zwischen \vase{} und \kite{} von Resorcin[4]arenen,
die auf Quinoxalin und Dinitrobenzen basieren, werden in verschiedenen
Umgebungen (Vakuum, Chloroform und Toluol) bei drei verschiedenen
Temperaturen (198.15, 248.15 und 298.15\unit{K}) untersucht.
%
Die Herausforderung, die Millisekunden-Zeitskala des \vase{}-\kite{}
\"Ubergangs in der Simulation zu beobachten, wird durch die Berechnung
relativer freier Enthalpien mit der \textit{ball-and-stick local-elevation 
umbrella-sampling} (B\&S-LEUS)
%, w\"ortlich \textit{Ball-und-Stab lokale-Hebung Schirm Sampling???) 
Methode gemeistert, die die \"{U}berg\"ange zwischen den Konformationen 
beschleunigt. Die berechneten freien Enthalpieunterschiede $\Delta G$
stimmen qualitativ mit den experimentellen Werten und Tendenzen \"uberein.


Kapitel~\ref{ch:cbti}-\ref{ch:cbus} pr\"{a}sentieren die Entwicklung
des \textit{F\"orderbandschemas} (engl. conveyor belt scheme), welches
eine Methode ist, um freie Enthalpie-Differenzen zu berechnen.
%
Die Arbeitsweise beruht auf dem Aneinanderkoppeln von $K$ Repliken eines Systems,
die an verschiedenen Werten des Kopplungsparameters $\lam$ simuliert werden.
%
Die Anzahl $K$ sollte gerade gew\"ahlt werden und die Repliken werden
im gleichen Abstand zueinander auf einem vor-und-zur\"uck Pfad angeordnet, was an ein F\"orderband
zwischen zwei Endpunkten erinnert.
%
Wie in der $\lam$-Dynamikmethode ($\lam$D) \"andern sich die $\lam$-Werte der einzelnen 
Systeme im Laufe der Simulation.
%
Dies geschieht allerdings in einer gemeinschaftlichen Art, welche durch die \"Anderung 
einer einzigen dynamischen F\"orderbandvariablen $\Lamb$ mit Periode $2\pi$ bestimmt wird.
%
Daher bewirkt eine \"Anderung von $\Lamb$ immer eine Vorw\"artsbewegung entlang $\lam$ von einer H\"alfte der Repliken und eine R\"uckw\"artsbewegung entlang $\lam$ von der anderen.
%
Daraus resultiert ein effektives freies Enthalpieprofil des Replikensystems entlang $\Lamb$,
welches sich durch geringere werdende Energiebarrieren auszeichnet, je gr\"osser $K$ ist.
%
Wenn eine gen\"ugende Anzahl $K$ an Repliken benutzt wird, werden diese Barrieren so klein,
dass eine durchg\"angige und nahezu homogene Erfassung des gesamten $\lam$-Bereichs durch die 
Repliken erreicht wird.

Kapitel~\ref{ch:cbti} wendet dieses Schema auf alchemische freie Enthalpieberechnungen an.
Deshalb wird es als \textit{F\"orderband thermodynamische Integration} (engl. conveyor belt thermodynamic integration, CBTI) bezeichnet. Eine
mathematische/physikalische Formulierung des Schemas wird vorgelegt bevor eine ersten Anwendung der Methode auf die Berechnung der freien Hydratisierungs-Enthalpie
f\"ur Methanol beschrieben wird.

In Kapitel~\ref{ch:ortho} wird die CBTI Methode auf zwei andere Systeme angewandt, 
n\"amlich auf ein Lys-X-Lys Tripeptid,
wobei die zentrale Aminos\"aure Glu in Gly mutiert wird, und auf 
Guanosintriphosphat, wobei ein Wasserstoffatom an der C8 Position in ein Bromatom 
mutiert wird.
%
Mit beiden Systemen wurden Samplingprobleme festgestellt wegen grossen
orthogonalen Barrieren entlang  der Torsionswinkel $\phi$ and $\psi$ der Peptidbindung (Tripeptid)
oder entlang des Ribosentorsionswinkels $\chi$ (Guanosine).
Die jeweiligen Vorz\"uge verschiedener Samplingmethoden, orthogonalen  Biases oder
Sch\"atzer, um die Konvergenz zu verbessern, werden untersucht.
%
Es wird gezeigt, dass die thermodynische Integration nachteilig ist,
weil die Simulationen bei konstantem $\lam$ stattfinden.
%
Durch eine \"Anderung des Sch\"atzers von der Simpson's Quadratur zu  MBAR 
lassen sich die Ergebnisse nicht signifikant verbessern.
%
Sowohl Hamiltonian-Austausch zwischen Repliken als auch CBTI verbessern die 
Resultate f\"ur das Tripeptid.
%
F\"ur Guanosintriphosphat wird eine Verbesserung nur duch die Anwendung
eines orthogonalen Biases erreicht, wiederum am Effizientesten f\"ur
den Hamiltonian-Austausch zwischen Repliken und CBTI.

Das F\"orderbandschema wird im \ref{ch:cbus}.~Kapitel auf konformationelle
freie Enthalpiedifferenzen ausgeweitet, was zu dem sogennanten
\textit{conveyor belt umbrella sampling} (\CBUS) Schema f\"uhrt.
%
CBUS wird hier als Erstes auf die Berechnung von 45 freie Standard-Bindungsenthalpien
zwischen 5 verschiedenen Alkaliionen und drei Kronenethern in drei verschiedenen
L\"osungsmitteln angewandt.
%
Neben dem Vorstellen und Testen des neuen Schemas wird es auch mit anderen Methoden
verglichen.
%
Wie erwartet f\"uhrt das einfache Z\"ahlen der Konformationen (direct counting)
zu Konvergenzproblemen in der
durchgef\"uhrten Simulationszeit und die erhaltenen Resultate sind daher
unzuverl\"assig.
%
Andererseits sind die Resultate, die durch die traditionelle Umbrella Sampling und 
die CBUS Methoden erhalten wurden, sehr \"ubereinstimmend.
%
Zus\"atzlich wurden exzellente \"Ubereinstimmung mit  mit fr\"uheren alchemischen 
Berechnungen \"uber einen alchemischen Pfad erreicht und die Tendenzen 
der experimentellen Daten wurden qualitativ reproduziert.

Eine Schlussfolgerung und ein Ausblick auf zuk\"unftige Entwicklungen sind
im Kapitel \ref{ch:outlook} aufgef\"uhrt.



