\chapter{Summary}

Molecular dynamics (MD) simulations offer nowadays a valuable 
tool for studying phenomena in chemistry and biology.
%
The history of models in chemistry, as well as the methodological 
and theoretical background of MD is presented in \refch{intro}.

\refch{resa} deals with the application of explicit-solvent molecular dynamics (MD)
simulations to resorcin[4]arene cavitands, which
can adopt a close/contracted (\vase{})
and an open/expanded (\kite{}) conformation.
%
The \vase-\kite{} equilibria of a quinoxaline- and a dinitrobenzene-based 
resorcin[4]arene are investigated in three solvent environments (vacuum, chloroform and toluene)
and at three temperatures (198.15, 248.15 and 298.15 K).
%
The challenge of sampling the millisecond-timescale \vase-\kite{} transition
%, which occurs experimentally
%on the milisecond timescale, represents a challenge in terms of sampling.
%
%It 
is addressed by calculating relative free energies using ball-and-stick local 
elevation umbrella sampling (B\&S-LEUS) to promote interconversion transitions.
%
The calculated \vase-to-\kite{} free-energy changes $\Delta G$
% between \vase{} and \kite{} calculated
are in qualitative agreement with the 
experimental magnitudes and trends.
%


\refchs{cbti}-\refchnn{cbus} present the development of the conveyor belt scheme,
a method to calculate free-energy differences.
%
The working principle relies on $K$ coupled replicas of the system that are 
simulated at different values of a coupling parameter $\lambda$.
%
The number $K$ is taken to be even and the replicas are equally spaced
on a forward-turn-backward-turn path, akin to a conveyor belt (CB)
between the two end-states.
%
As in $\lambda$-dynamics ($\lam$D), the $\lambda$-values associated with the 
individual systems evolve in time along the simulation.
%
However, they do so in a concerted fashion, determined by the evolution
of a single dynamical variable $\Lambda$ of period $2\pi$ controlling 
the advance of the entire CB.
%
Thus, a change of $\Lambda$ is always
associated with one half of the replicas moving forward and the other half moving backward along $\lambda$.
%
As a result, the effective free-energy profile of the 
replica system along $\Lambda$ is characterized with 
decrasing barriers upon increasing $K$, at least as $K^{-1}$ \radd{in the limit of large $K$}.
%
When \radd{a sufficient number of replicas is used}, 
these variations become small, 
which enables a complete and quasi-homogeneous coverage of the $\lambda$-range
by the replica system.
%


\refch{cbti} introduces this scheme with respect to
alchemical free-energy calculations. Therefore it is termed \textit{conveyor belt thermodynamic integration}.
It provides the mathematical/physical formulation of the scheme, along with an initial 
application of the method to the calculation of the hydration free energy of methanol.

In \refch{ortho}, the conveyor belt thermodynamic integration is applied to 
a Lys-X-Lys tripeptide, involving a side-chain mutation in the central residue,
and guanosine triphosphate, involving a hydrogen-to-bromine mutation.
With both systems, sampling issues have been encountered, due to 
the large orthogonal barriers either along the backbone 
dihedral angles $\phi$ and $\psi$ (tripeptide) or along 
ribose-base dihedral angle $\chi$ (guanosine). 
This relative merits of different sampling schemes, orthogonal biasing and estimators to improve the convergence 
are investigated.
%
%
The thermodynamic integration scheme is shown to suffer from the 
constraint of simulations at fixed $\lambda$.
%
There is no significant improvement upon changing 
from the Simpson's quadrature to the MBAR estimator.
%
Both Hamiltonian replica exchange and conveyor belt 
thermodynamic integration improve the results for
the tripeptide.
%
For the guanosine triphosphate, improvement
is only achieved upon application of an orthogonal biasing potential, 
most efficiently in combination with Hamiltonian replica exchange or
conveyor belt thermodynamic integration.
%

The conveyor belt scheme is extended to the calculation of
conformational free-energy differences in \refch{cbus},
resulting in a so-called conveyor belt umbrella sampling scheme (\CBUS).
%
CBUS is here initally applied to the calculation of 45 standard absolute binding free energies 
of five alkali cations to three crown ethers in three different solvents.
%
Besides introducing and testing the new scheme, it is compared to 
other methods.
%
Expectedly, the direct counting approach has convergence issues
on the accessible simulation timescale, and the corresponding results
are rather unreliable. On the other hand, the results obtained 
using traditional umbrella sampling and CBUS are
very consistent. 
%
Additionally, comparison of the results to
those of previous alchemical calculations {\em via} an
alchemical pathway reveals excellent 
consistency while the trends of available experimental data
are qualitatively reproduced.

Conclusions and an outlook into future developments
are given in \refch{outlook}.
