This chapter introduced PyGromosTools, an API for the GROMOS software package and the GROMOS++ package of programs that facilitates easy and fast set-up, running, and analysis of GROMOS MD simulations. The structure of the four modules (\textbf{file}, \textbf{simulation}, \textbf{analysis}, and \textbf{data}) was presented and illustrated with examples.  
The package is used already in multiple scientific projects and is a central element of the RE-EDS pipeline (Chapter \ref{ch:pyGrom}). 

Modern software development concepts are realized in PyGromosTools such that the API can be used for fast and sustainable development of complex solutions in different projects. Key elements of the package are (i) making data and functionality accessible in Python, and (ii) supporting the simulation set-up by providing classes for file management and job queueing on HPC clusters. With its underlying code structure and development decisions, PyGromosTools enables writing of readable and transferable code, which will result in increased shareability and enhanced reproducibility of scientific work.  \cite{Walters2020}

\FloatBarrier