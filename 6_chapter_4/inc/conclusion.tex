In this work, we introduced PyGromosTools, an API for GROMOS that allows easy and fast setting up, performing, and analyzing GROMOS simulations. The structure of the three Modules: Files, Simulation, and analysis, was presented and illustrated by examples. 
The package is used already in multiple scientific projects and will be further developed in the future. 

We believe that PyGromosTools shows the potential of combining scripting and programming languages in terms of code readability and shareability. Key elements are making data accessible on a scripting programming layer, supporting the simulation endeavor by providing classes for file management and HPC-Queueing.
These features will lead to faster project development and increase the reproducibility of results.  

Our general vision is to build a PyGROMOS package that integrates the GROMOSPP \cite{Eichenberger2011} into the Python layer if required under the usage of Numba to solve efficiency issues and to integrate GROMOSXX \cite{Schmid2012} tighter with the use of binding tools like pyBinds or SWIG. \cite{Wenzel2011, Beazley1996} These changes would lead to a more future-ready GROMOS environment that provides easier access to GROMOS and could even be shipped in one package and be compiled by the python package managing tools. 