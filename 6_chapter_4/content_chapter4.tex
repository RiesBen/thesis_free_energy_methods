\makeatletter
\def\input@path{{../}}
\makeatother
\documentclass[../main.tex]{subfiles}
\begin{document}
\renewcommand{\path}{3_chapter_1/}
\chapter[PyGromosTools: A Fast and Flexible API for the Molecular Dynamics Software Package GROMOS]{PyGromosTools: A Fast and Flexible API for the Molecular Dynamics Software Package GROMOS}
\chaptermark{PyGromosTools}
\label{ch:pyGrom}

\aquote{
    "The determined Real Programmer can write FORTRAN programs in any language."
}
{Ed Post\cite{Post1982}}



\begin{summary}
Making data accessible and reducing the complexity of process execution are key aspects of an application programming interface (API). Additionally, APIs reduce code duplication through code encapsulation and therefore increase code readability and re-usability. We have developed PyGromosTools to make the molecular dynamics (MD) software package GROMOS accessible in Python and facilitate its usage. PyGromosTools provides users with file classes and simulation functionality. In this chapter, we will discuss design aspects, code structure, and example usage of PyGromosTools. The code is open source and available on GitHub \url{https://github.com/rinikerlab/PyGromosTools}.
\end{summary}

\clearpage
\pagebreak

%%%%%%%%%%%%%%%%%%%%%%%%%%%%%%%%%%%%%%%%%%%%%%%%%%%%%%%%%%%%%%%%%%%%%
%% Start the main part of the manuscript here.
%%%%%%%%%%%%%%%%%%%%%%%%%%%%%%%%%%%%%%%%%%%%%%%%%%%%%%%%%%%%%%%%%%%%%
%================================================================================
\section{Introduction}
%================================================================================

%general idea - productivity through digitalization and code
Digitalization has been identified as one of the most promising tools to increase productivity in a vast range of disciplines.\cite{Tuomi2018}
Over the last four decades, computational techniques have also become increasingly relevant in chemistry, and many methods are now routinely used in academia and industry to e.g. predict the physicochemical properties of molecules, their 3D conformation and interactions with other molecules, or their chemical reactivity. A prominent application of computational chemistry is the highly interdisciplinary field of drug discovery, where computational approaches are employed in all stages of the development process.\cite{Chodera2011, Hansen2014, Abel2017, Cournia2017, Cournia2020, Meier2021}

%% Diversification of coding skill 
With the growing application of computational chemistry in industry, also the fraction of scientists that need to read or write software code increases. Typical tasks do not require writing highly optimized code in low-level languages such as Java\cite{Gosling2000}, Fortran\cite{Backus1957}, or C++\cite{Stroustrup1995}. As a consequence, more advanced concepts such as memory management, parallel programming, or advanced programming paradigms are not always considered during code development, possibly leading to problems later on. In most projects, it is a balance between long-term sustainable code and so-called technical debt\cite{Alfayez2018} to obtain results on the short term. This issue has led to the development of convenient-to-use scripting languages and APIs, which can be used to swiftly implement solutions to complex problems with minimal technical debt. By doing this, time is freed to focus on problems requiring specialized knowledge.\cite{Ayer2014} The scripting language with the highest impact in science nowadays is Python. Already in 2011, Python was declared the \textit{de facto} standard language in natural sciences and engineering, and from 2017 to today (2021), Python has remained the highest-ranked language in the \textit{IEEE Spectrum} journal. This ranking reflects the user interests measured by the internet community Stack Overflow, data available on GitHub, and IEEE Articles. \cite{Millman2011, Vanderwalt2011, Cass2017, Cass2018, Cass2019, Cass2020, Cass2021} 
%
Core features identified by Oliphant\cite{Oliphant2007} that make Python attractive to a widespread audience are: 
\begin{itemize}
    \item Intuitive syntax that is easy to read and learn, and thus allows rapid prototyping.
    \item Straightforward integration with other programming languages (especially C/C++\cite{Stroustrup1995}, enabled by tools like pybinds\cite{Wenzel2011}, Boost Python\cite{Koranne2011}, SWIG\cite{Beazley1996}, Cython\cite{Behnel2011}, or Numba\cite{Lam2015}).
    \item Large community supporting high-quality packages (e.g., NumPy\cite{Vanderwalt2011}, SciPy\cite{Virtanen2020}, Matplotlib\cite{Hunter2007}, Pandas\cite{Mckinney2010}, and Jupyter\cite{Kluyver2016}).
    \item High-quality tools for environment and package management (e.g. pip\cite{Pypi2021} and conda\cite{Anaconda2020}).
    \item Platform independence that enables development on different operating systems and computer architectures.
\end{itemize}

%APIs in Python -> Efficiency as API
Despite the boom of Python, one downside of this programming language is its limited computational performance in a native form, which is related to the dynamic typing concept. To address the efficiency issues of native Python code, many different solutions have been developed. One solution is to translate the Python code during run-time (just-in-time-compiling) into C code\cite{Kernighan2006} or directly into machine code\cite{Lattner2008}. Tools making this solution accessible are for example Cython\cite{Behnel2011} or Numba\cite{Lam2015}. 
Prominent packages like SciPy\cite{Vanderwalt2011}, NumPy\cite{Virtanen2020}, or Pandas\cite{Mckinney2010} follow a reverse approach and use Python as a wrapper to bind C or Fortran code. In this case, Python merely 'steers' the code execution and the user only interacts with the Python layer.\cite{Oliphant2007} 
Tools like pybind11\cite{Wenzel2017} or Boost Python\cite{Koranne2011} make this approach easy to implement and allow rapid construction of Python APIs. 
Many packages in computational chemistry make use of this concept. Examples are RDKit\cite{Landrum2021}, PyMol\cite{Delano2020}, PySCF\cite{Sun2018}, pyOpenMS\cite{Roest2014}, BioPython\cite{Cock2009}, and Pybel\cite{O'Boyle2008}. The most popular MD packages follow this trend and offer Python APIs. GROMACS \cite{Berendsen1995, Lindahl2001, Abraham2015} and AMBER\cite{Weiner1981, Pearlman1995, Case2005} provide for this purpose the packages gmxapi\cite{Irrgang2018} and ParmEd\cite{Shirts2017}, respectively. OpenMM\cite{Friedrichs2009, Eastman2010, Eastman2017} and LAMMPS,\cite{plimpton1995, Thompson2022} on the other hand, include Python APIs natively.\cite{Talirz2021}
%Multiple packages from our group follow this concept, for example the vsCNN repository for hierarchical clustering. \cite{Weiss2021}
Here, we introduce a Python API for the GROMOS software package \cite{Schmid2012} called PyGromosTools.\cite{Lehner2021} This API is a starting point for further development that may ultimately allow access to the entire functionality of GROMOS as well as GROMOS++ \cite{Eichenberger2011} from Python. Currently, PyGromosTools already provides access to the simulation trajectories and to a selection of GROMOS features, together with job queueing for high perfomance computing (HPC) clusters, and data analysis functionality. This chapter presents our rationale behind the design of PyGromosTools and shows examples of how the API can be used.



\FloatBarrier

%================================================================================
\section{Implementation}
%================================================================================

The implementation of  PyGromosTools tries to accomplish several goals that were identified to be crucial for good API development: \cite{Henning2009, Blanchette2008, Bloch2006}
\begin{itemize}
    \item An API is easy to learn and memorize, such that coding with it comes naturally. 
    \item The usage of an API should lead to better readable code.
    \item A well-designed API is hard to misuse and easy to extend.
    \item Last but not least, an API is complete and simple. However, this can develop over time.
    \item Design an API with context knowledge for the field of usage. 
\end{itemize}

As PyGromosTools is a package that needs to deal with the history of GROMOS and has a non-linear history, it needs to balance the amount of technical debt and the progress in projects very well. However, the package slowly converges to a consistent form. By now we believe, that the general structure of usage is established with long-term sustainability.
The rigidification of specific usage patterns on the user level is vital to sustaining a long-term experience that does not crash old code.  

If all criteria are fulfilled, the easy-to-use, reproducible, fast, expandable simulation setup and execution of molecular dynamics simulations will lead to a higher quality of scientific output in terms of openness, accessibility, and reproducibility.

\subsection{Coding Paradigms}
PyGromosTools follows several modern Coding styles. These are not enforced on users, but the style is enforced on PyGromosTools developers who would like to contribute. 

%code visibility/information hiding
Information hiding is an essential concept in coding for larger projects. It boils down to presenting only the required information to a reader. Usually, larger packages have thousands of lines of code, and presenting all of them at the same time gets very overwhelming. Therefore encapsulation of code into functions and classes or managing the accessibility of certain variables/functions is hugely important in code hiding. \cite{} 

In PyGromosTools, accessibility is managed with the provided methods by python. For example are private variables in the \textit{GROMOS\_System} assigned with a prefix "\_" like the attribute \textit{\_GROMOSPP} (see  \hyperlink{https://github.com/rinikerlab/PyGromosTools/blob/348439a326357b8172717f5d5ed7a5bbfa4564ab/pyGROMOS/files/GROMOS\_system/GROMOS\_system.py\#L83}{Code}). Note that this way of declaring a variable still  makes it easily accessible in practice. If a variable of function should clearly never be used externally name mangling is used with the prefix "\_\_" like \textit{\_\_ionDecorator} (see  \hyperlink{https://github.com/rinikerlab/PyGromosTools/blob/348439a326357b8172717f5d5ed7a5bbfa4564ab/pyGROMOS/files/GROMOS\_system/GROMOS\_system.py\#L994}{Code}) like defined in \hyperlink{https://www.python.org/dev/peps/pep-0008/}{PEP 8}.
The second aspect of encapsulation of code into functions and classes is the modularization of the code. This concept allows quick construction of more complex structures and therefore speeds up the development process and readability at the same time. \cite{}

\subsubsection{Variables, Signatures, and Classes}
PyGromosTools, in general, follows the principle of using descriptive variables. Each naming in the package should give a comprehensive description of the function of a definition. Abbreviations are therefore forbidden. Following this paradigm increases the general readability of the code, leading to a clearer understanding of code and, therefore, avoiding mistakes. 
Consequentially, the second style decision for PyGromosTools is annotating types of variables in classes and functions and function return types. This style follows the python enhancement proposals  \hyperlink{https://www.python.org/dev/peps/pep-0526/}{PEP 526}, \hyperlink{https://www.python.org/dev/peps/pep-0484//}{PEP 484} , and  \hyperlink{https://www.python.org/dev/peps/pep-3107/}{PEP 3107} which implemented the type annotation system of python3. This decision was made because of the more complex types in PyGromoTools, in order to provide quick tips to users and developers on which type is the expected type. The type annotations can be quickly accessed in the source code, are visualized in IDEs or interactive python sessions. 
A third style choice is the usage of keyword arguments used in function parameter passing. The keyword argument passing was introduced with \hyperlink{https://www.python.org/dev/peps/pep-0468/}{PEP468} the underlying reason for the usage of this feature is the increased readability of the code and making the code more robust versus code refactoring of function signatures.

\subsection{Documentation and Continous Integration}
In PyGromosTools, each function and module should contain a docstring description. The chosen docstring style is \hyperlink{https://numpydoc.readthedocs.io/en/latest/format.html}{numpydoc} from the NumPy package. \cite{VanDerWalt2011} The doc-string style is supported over various IDEs and is automatically collected in the documentation via sphinx.
Additionally to the documentation are several jupyter-notebooks provided exemplifying the usage of PyGromosTools on multiple layers. \cite{Kluyver2016}
Besides this, a continuous integration pipeline is implemented with a set of unit tests that ensure the base functionality of the package.


\subsection{Object Oriented and Functional Programming}
Python3 is a multi-paradigm language allowing for mixing OOP and functional programming styles. Generally, PyGromosTools focuses on OOP. 

On the one hand, OOP brings the benefits of inheritance the subsequent concept of polymorphism. These concepts are used in PyGromosTools with state-driven contexts like file representations or the submission system classes. Especially in the base class \textit{\_general\_GROMOS\_file} contains the, for example, the fundamental read/write procedures, that are the same for all GROMOS files and therefore is only implemented once. (see \hyperlink{https://github.com/rinikerlab/PyGromosTools/blob/348439a326357b8172717f5d5ed7a5bbfa4564ab/pyGROMOS/files/\_basics/\_general\_GROMOS\_file.py\#L15}{Code})

On the other hand, functional programming can be found in subsequent function implementations, realized with, for example, map, apply, and zip operations, or even as higher-level functional programming with python decorators. As an example for the Decorators, the principle of currying was realized for the GROMOSPP integration into the GROMOS system, such as the dynamically generated functions update the attribute files of the GROMOS system automatically, and those do not need to be provided to the function call (see Figure \ref{fig: GROMOSSystemSimulationExample}) (see  \hyperlink{https://github.com/rinikerlab/PyGromosTools/blob/348439a326357b8172717f5d5ed7a5bbfa4564ab/pyGROMOS/files/GROMOS\_system/GROMOS\_system.py\#L915}{code}). \cite{Curry1958}


\subsection{Code Structure}
PyGromosTools itself has two layers. 
The lowest layer is building interfaces/APIs to establish communication with the operating system, the queueing system, or the GROMOS wrappers. All these functions are used in the next layer to build more complex structures that fulfill more complex tasks. 

In the following, we want to discuss the implementation of the different modules of PyGromosTools. In general, one can split the package into four modules: Data, Files, Simulation, Analysis.  


%The fundamental idea of files
The file module is a collection of classes that represent the different GROMOS-files in python. The module's design is based on the object-oriented programming paradigm (OOP) and therefore makes extensive use of inheritance and polymorphism to build up a class structure in the sense of the GROMOS File structure with a minimal amount of code duplications.
In the GROMOS file structure, each file contains multiple blocks. These blocks contain either a table of data or a  list of values. The structure of files in PyGromosTools is compatible with this structure and makes the experience pattern similar to the design of GROMOS. The structural fundament is placed in the \textit{\_general\_GROMOS\_file} class that was created to contain fundamental functionality that is encoded in the general file structure. Resulting functions are \textit{read\_file}, \textit{write} or \textit{str}-operator overloading.
This generic file class encapsulates classes derived from \textit{\_generic\_GROMOS\_blocks} which again provides generic functions like read-  and write-functionalities as well as overloading operators. 
The most elementary structure is the one used by the generic blocks as content. This structure is in many blocks the \textit{\_generic\_field}, a direct primitive type or a pandas data frame. (see Figure \ref{subFig-GeneralFileGROMOSXX})

%Features
All these compartments generate specific classes for the different files used in the GROMOS environment. Features of these classes are: 
\begin{itemize}
    \item IO functionalities that not only allow writing files but also writing obj states or converting files.
    \item The direct accessibility of any type of GROMOS data in the objects allows a lean integration into python3 scripts.
    \item Additional functionality, that directly works on the file class like: calculating trajectory RMS or removing residues from \textit{CNF}s.
\end{itemize}

% generation of coordinates/atoms/molecules and topology generation of topology terms with correctness checking (is bond already present, etc. ) wrappers for simulation parameter blocks


%File Types
From the GROMOS environment, most files were implemented in PyGromosTools with an individually class derived from the \textit{\_general\_GROMOS\_file} class and sorted into the different categories of their functionality.
The class realizations span over aspects of the MD-Package, like coordinate files, topologies, simulation parameter files, trajectories, forcefield files, and other files like replica-exchange outputs or NMR GROMOSPP program output files. The Force Field Class ForceField System represents a whole forcefield in this environment and allows the capability of parametrizing molecules. (see Figure \label{subFig-FilesinModule})

The trajectory files translate the GROMOS trajectory in a pandas data frame that can quickly and efficiently be used on the python. The data frame is stored in the compressed hf5 format and, therefore, is very storage efficient.
%Todo: MTB files / lib files oder so trf, trs, trv

%%GROMOS System
All the different file formats can be stored centralized in the GROMOS System class. This class is the central structure of PyGromosTools, from which any simulation approach can be started. The class can be used with a minimal set of files, like coordinate file (.cnf), topology file (.top), and a simulation parameter file (.imd) or even just with a molecule SMILE/RDKit Mol string which leads to a complete parametrization with openFF.

% off

%% helper functions that adapt system, future flag, auto convert (oFF topo,conv, adaptIMD from smiles)
%%interface forcefield SYS

%conc Outlook
Due to the consequent inheritance structure, new blocks or files can be implemented very easily and quickly. 
In theory, this API could also introduce new file types that are easier to handle without omitting the old ones. For example, the Simulation parameters file could easily be translated into a JSON or XML data format, making the file handling and readability through the key-value policy much more accessible.

\begin{figure}[h]
    \centering
    \begin{subfigure}{1\textwidth}
        \includegraphics[width=\textwidth]{fig/implementation/Files.png} 
        \caption{}
        \label{subFig: FilesinModule}
    \end{subfigure}
    \caption{Files Module:  The PyGROMOS module Files is implemented with an OOP structure based on the GROMOS file structure. The base classes are the \textit{\_generic\_GROMOS\_block} and the \textit{\_general\_GROMOS\_file}. From these base classes all the different GROMOS blocks and GROMOS files are derived, with the exception of trajectories, in order to provide a consistent experience.
    In the figure only the implemented file classes are shown for clarity. As a central element of PyGromosTools, the GROMOSSystem class collects all files and makes them easily accessible for simulation approaches or other functionality that requires multiple GROMOS files.
    }
    \label{fig: FileModule}
\end{figure}

\subsection{Simulation Module}
The module simulations contain multiple submodules used to realize GROMOS simulations, ready for HPC-Queueing on different hardware setups. The submodules of simulations can be arranged into three layers. This structure allows fast adaptation of functionality in these modules (see Figure \ref{fig: SimulationModule}).

One foundation of this module is the two submodules GROMOS and HPC-Queuing. GROMOS contains the python API to the GROMOSXX and GROMOSPP C++ code, allowing quick access to their functionality. Currently is the API of GROMOSXX and GROMOSPP realized in the form of bash wrappers that provide the functionality. Nevertheless, we believe that proper C++ python integration would be beneficial to improve the communication between the layers. Recently a pyGROMOSPP compartment was added that contains efficient python implementations of several functionalities similar to the style of GROMOSPP. In general, we believe that the advantages of the python language and the possibility of writing more efficient code with modern python packages will lead to a growth of GROMOSPP functionality implemented directly into the python layer instead of a more complicated C++ implementation. The GROMOS module can be used isolated from all other modules to allow fast development in other parts of the packages or for specific projects.

The second foundation of this module is the HPC-Queuing module. It builds an interface to job-management tools like LSF and provides the functionality of job queueing for PyGromosTools scripts. The submodule is divided into a submission system and a job scheduling part. The submission system part is structured into an OOP-based structure with the parent class \textit{\_SubmissionSystem} that functions as an interface, ensuring the correct implementation of the child classes. Already implemented submission system classes are: 
\textit{DUMMY}, a class that only prints out strings, and therefore can be easily used for testing. \textit{LOCAL}, this class is executing every submitted job directly on the local machine via the operating system. Last, the class \textit{LSF} for the IBM Queueing system. This class allows the scheduling of simulations on the HPC cluster using the LSF queuing system. The class was optimized for the Euler cluster.
Besides the submission systems, the HPC-Module contains job scheduling tools that implement a scheduler worker pattern (see Figure \ref{fig: SimulationExecPattern}). In this pattern, a scheduler function submits worker sub-scripts dynamically generated from template workers to the queueing system, effectively executing the submitted steps in order. Present workers are the simulation, the clean-up, and the analysis worker. The separation of these three tasks is essential for guaranteeing the correct execution of the different tasks.

A good usage pattern for the submission systems is starting to develop a pipeline with the DUMMY or LOCAL submission system locally and then upscale the approach to the HPC-Cluster. This pattern proved to be a practical and straightforward approach as many problems/bugs can be already caught locally without wasting time queueing and stressing the Cluster with traffic.

\begin{figure}[h]
    \centering
    \includegraphics[width=\textwidth]{fig/implementation/SimulationExecutionManagment.png}
    \caption{HPC-queuing submission pattern: The implemented submission pattern of the HPC-Queuing module follows the pattern of using a scheduler function, that submits worker scripts that perform the whole work or a part of it.The worker script is directly executed or submitted to a job-manager tool based on the provided submission system class. The automatically generated file structure by the exemplified execution is shown in the right box. Note that the simulation directory is compressed in the process. The md folder was uncompressed in the figure to illustrate the concept.Generally, the user can delete this compressed folder, but it is only advised after finishing a project. The code for this example is shown in Figure \ref{fig: GROMOSSystemSimulationHPCExample}. }
    \label{fig: SimulationExecPattern}
\end{figure}

The two described submodules are combined in the simulation blocks, which can directly execute simulations. As required input, the simulation blocks take only a GROMOS system, which contains all input files. If no IMD file is provided, a template file for the simulation approach will be mapped on the system. By exchanging the default submission system (\textit{LOCAL}), the described upscaling can be performed easily. Besides automatic scheduling of tasks, the function also provides automatic file management for the simulation (see Figure \ref{fig: SimulationExecPattern}). The output of the simulation function is a copy of the input simulation adapted to contain the output files of the simulation. If the simulation was directly executed, these files are directly parsed into the class. However, if the simulation is queued, all the non-present output files are marked with an \textit{\_future} flag. This flag prevents the system from executing tasks on the object that requires the presence of the data. In more complex simulation chaining approaches, these tagged files can be used to submit follow-up simulation steps (e.g., reeds s-optimization or Eoff-rebalancing approach). If a file is existent at a later time point, the \textit{\_future\_promise} flag is removed by the \textit{\_check\_promises} and the file is automatically loaded into the GROMOS system object.
Additional minor features are checking if simulations or analysis scripts were already finished successfully or if the current job was already submitted to the queue, all leading to skipping the step. And the final concatenation of files and storing as compressed .h5 PyGromosTools-trajectory files in analysis/data (see Figure \ref{fig: SimulationExecPattern}).

%Simulation Approaches
\begin{figure}[h]
    \centering
    \includegraphics[width=\textwidth]{fig/implementation/Simulation.png}
    \caption{Simulation Module: The simulation module contains two submodules, HPC-queuing and GROMOS. GROMOS is the API to all GROMOS functionality written in C++ and a new module containing GROMOSPP like functionality in python. The HPC-queueing Submission System classes can be used to adapt a simulation approach to a different environment quickly  (e.g., local execution or queuing with the LSF-Job management tool on the Euler cluster). This adaptation is possible due to the commonly shared parent class \textit{\_SubmissionSystem}. The simulation module provides functions that wrap HPC-queuing and GROMOS functionality providing quick access to performing GROMOS Simulations. Additionally are higher-level approaches present that can be used for simulating. Here dashed box borders imply based on functions, and bold box borders imply an underlying class structure.}
    \label{fig: SimulationModule}
\end{figure}

The final compartment of the simulations module is the approaches directory. In this submodule, high-level approaches are stored that fulfill a complete simulation approach, like calculating free energies for evaporation or thermodynamic integration simulations (used for Restraintmaker).

\subsection{Analysis \& Module}
The analysis module contains functionality for analyzing and visualizing properties and structures. The analysis functions can calculate properties over trajectories like estimating free energies, calculating coordinate RMSDs, and more. The visualization tools use py3dmol and enable visualization of coordinate files or trajectories in jupyter notebooks. The analysis package is the youngest module of PyGromosTools and will grow in the future. As a bonus, the data module provides the package with parameters for force fields like the GROMOS 54A7 force field and template simulation parameter files to have a good start for a simulation (see Figure \ref{fig: GROMOSSystemExample} or \ref{fig: GROMOSSystemSimulationExample})


%================================================================================
\section{Applications and Examples}
%================================================================================

\subsection{Gromos System and Simulation Modules}
\begin{figure}[h!]
    \centering
    \includegraphics[width=\textwidth]{fig/ApplicationExamples/codeExample_GROMOSSystemSolvBoxTut.png}
    \caption{PyGromosTools code to perform the simulation set-up the pentapeptide in water from the GROMOS tutorial\cite{Lier2020}. The \textit{Gromos\_System} is the central object for the system generation, and all functions can be called from there. The visualization of the start structure is called in the last line of code. The function \textit{rebase\_files()} triggers an automatic file management function writing out all files, that are indlcuded in the \textit{Gromos\_System}.}
    \label{fig: GROMOSSystemExample}
\end{figure}

Figure \ref{fig: GROMOSSystemExample} shows how PyGromosTools can be used to perform the simulation set-up of a short peptide in solution, following the example from the official GROMOS tutorial \cite{Lier2020}. The set-up steps include (i) generating a topology file for the given peptide residue sequence, (ii) generating a GROMOS coordinate file, (iii) solvating the peptide in water, and (iv) adding two CL- ions to the system to neutralize the overall charge. In this procedure, all actions on files contained in the \textit{Gromos\_System} are directly accessed and stored in the \textit{Gromos\_System}, leading to a simplified function call of the GROMOS++ functions. 
%The function call change can be visualized by inspecting the function signature of, for example, make\_top called from GROMOSPP and called from GROMOS system. This modification results in the simplified code, and the user does not need to think about the file management of GROMOS++ functions while building systems. (see Figure \ref{fig: GROMOSSystemExample} vs Figure \ref{fig: GROMOSWrappers})



\begin{figure}[h!]
    \centering
    \includegraphics[width=\textwidth]{fig/ApplicationExamples/codeExample_GROMOSSystemSimulation.png}
    \caption{PyGromosTools code to perform an energy minimization (emin) and MD simulation (md) using the default settings (top left). After the simulation, the output coordinates can be visualized in the Jupyter notebook (bottom right), and the energy trajectory can be directly analyzed (bottom left). The simulation approach is realized with automatic file management in the background that is easy to understand (top right).}
    \label{fig: GROMOSSystemSimulationExample}
\end{figure}

After the system generation, the \textit{Gromos\_System} can be combined with the \textbf{simulations} module to carry out different types of simulations. For standard simulations, the default settings provided in the simulation block can be used. An example is shown in Figure \ref{fig: GROMOSSystemSimulationExample} for the energy minimization and production run. If a more complex set-up is required, the user can manually set a simulation parameter file (.imd) in the \textit{Gromos\_System} and modify it as required. An important feature of the \textit{simulation} function is that the given \textit{Gromos\_System} will not be modified by it (immutability principle). Only the returned system will be a modified version of the initial input. This immutability approach for \textit{Gromos\_System} in the \textit{simulation} function avoids confusion when the system state changes during scripting. 

\begin{figure}[h!]
    \centering
    \includegraphics[width=0.8\textwidth]{fig/ApplicationExamples/codeExample_GROMOSSystemHPCSimulation.png}
    \caption{PyGromosTools code to perform a longer MD simulation on a HPC cluster in three parts (\textit{simulation\_runs}=3). The adaptation of the code is minimal, as only a submission system has to be changed to \textit{LSF}.}
    \label{fig: GROMOSSystemSimulationHPCExample}
\end{figure}

To perform longer simulations on a HPC cluster, the simulation can be split into multiple parts, which are submitted by changing the number of \textit{simulation\_runs} (Figure \ref{fig: GROMOSSystemSimulationHPCExample}). This parameter controls how often the simulation with the given parameters is executed. To submit the jobs to the queue, only the \textit{submission\_system} parameter needs to be changed from \textit{LOCAL} to \textit{LSF}.



%--------------------------------
\FloatBarrier

\subsection{Further Examples}
\subsubsection{Example of File Handling}
%The power of the PyGromosTools file module lies in making data directly accessible in the Python environment with the correct type. 
The PyGromosTools package is able to read and write GROMOS files, modify parameter values, or use output files to perform further analysis. In Figure \ref{fig: GROMOSSystemSimulationExample}, a code example is given in which the template simulation parameter file (.imd) from PyGromosTools is modified by the user to change the number of time steps and set the temperature to $600~K$. Finally, the parameter file is written out (e.g. to start a GROMOS simulation in the command shell). 
%In addition, functions can be defined that support the user by adapting dependent attributes at the same time, like, for example, the force group number while changing the force groups in the \textit{FORCE} block (see Figure \ref{fig: FileHandlingExampleIMD}). Other files like the energy trajectory (\textit{tre}) can give direct access to the simulation data and can be used for further analysis or plotting (see Figure \ref{fig: GROMOSSystemSimulationExample}).

\begin{figure}[h]
    \centering
    \includegraphics[width=0.8\textwidth]{fig/ApplicationExamples/codeExample_GROMOSFiles.png}
    \caption{PyGromosTools code to modify the template system parameter file (.imd) as desired (e.g., changing the number of time steps or setting the temperature). The different parameters can be directly accessed and modified via their GROMOS name. Afterwards, the object can be written out.}
    \label{fig: FileHandlingExampleIMD}
\end{figure}


\subsubsection{GROMOS Wrappers}
The GROMOS API provides users with many functions from the GROMOS environment, including documentation and reasonable defaults suited for most simulation set-ups. The GROMOS and GROMOS++ binaries are used from the operating system \textit{PATH} variable or can be redirected by providing a binary directory path to the object construction. After constructing the wrappers, the full functionality is accessible from the object. The return value of the functions will always be the output file generated by the command. The API functionality of these wrappers is used throughout PyGromosTools to accomplish more complex tasks (Figure \ref{fig: GROMOSWrappers}).

\begin{figure}[h]
    \centering
    \includegraphics[width=0.8\textwidth]{fig/ApplicationExamples/codeExample_GROMOSWrapper.png}
    \caption{The GROMOS and GROMOS++ wrappers facilitate the usage of GROMOS functionality from Python. After constructing the GROMOS wrapper classes, the programs of GROMOS++ and GROMOS are accessible as a function of the object. Note that the GROMOS 54A7 force-field parameters are taken directly from the data module inside the package.}
    \label{fig: GROMOSWrappers}
\end{figure}


%--------------------------------
\FloatBarrier

\clearpage
\newpage

%================================================================================
\section{Conclusion and Outlook}
%================================================================================

This chapter introduced PyGromosTools, an API for the GROMOS software package that facilitates easy and fast set-up, running, and analysis of GROMOS MD simulations. The structure of the three modules (file, simulation, and analysis) was presented and illustrated with examples.  
The package is used already in multiple scientific projects and is a central element of the RE-EDS pipeline \cite{Ries2021B}. 

Modern software development concepts are realized in PyGromosTools such that the API can be used for fast and sustainable development of complex solutions in different projects. Key elements of the package are (i) making data and functionality accessible in Python, and (ii) supporting the simulation set-up by providing classes for file management and HPC-queueing. With its underlying code structure and development decisions, PyGromosTools enables writing of readable and transferable code, which will result in increased shareability and enhanced reproducibility of scientific work.  \cite{Walters2020}

\FloatBarrier

\clearpage
\pagebreak

%\bibliography{6_chapter_4/ref/ref.bib}


\end{document}
