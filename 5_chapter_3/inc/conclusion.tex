This chapter reports the recent developments for the multistate free-energy method RE-EDS, which omits the definition of alchemical transition paths. The automatic workflow for RE-EDS was improved in robustness, and was applied to estimate the relative binding free energies of five CHK1 inhibitors containing typical core-hopping transformations. This system was investigated previously with FEP+ and QligFEP, allowing for a direct comparison of RE-EDS with state-of-the-art pairwise free-energy methods.
Using different starting configurations representing all end states (SSM approach) in the parameter optimization of the RE-EDS workflow improved the sampling, convergence, and the accuracy of the resulting free-energy differences. The performance of RE-EDS SSM was found to be comparable with FEP+ and QligFEP, and shows that RE-EDS with a ``dual topology" approach can be readily applied to challenging ligand transformations like ring size change, ring opening/closing, and ring extension.

In terms of computational efficiency, the total production run time with RE-EDS ($3.5$~ns per replica) was about a quarter of that reported for FEP+ with this system. As multiple ligands are simulated simultaneously in a single RE-EDS simulation, this sampling enhancement will increase with increasing number of ligands. 
However, the pre-processing phase in the RE-EDS workflow currently uses a relatively large amount of simulation time. Making these steps more efficient will be addressed in future work.
