In this work, we presented an efficient algorithm for the placement of distance restraints in a linked dual topology approach that can easily be applied or incorporated into pipelines with the provided git repository and a python anaconda environment \cite{anaconda} \textit{\hyperlink{https://github.com/rinikerlab/restraintmaker}{https://github.com/rinikerlab/restraintmaker}}. The algorithm can be used in python scripts or interactively with a representation of the molecules in PyMol 2.0 (or higher). Distance restraint files can be written in GROMOS and GROMACS format, or in an easily accessible and human-readable JSON format that can be parsed by any JSON parser. \cite{DeLano2020}

The presented algorithm is a graph-based approach and can be straightforwardly applied to molecules with a rigid core. With this, the only required user input is the number of restraints assigned to link two molecules.

%%%%%ToyModelFun
The introduced algorithm was tested, and its time and result performance were analyzed with toy models and compared to brute force approaches. This test found that the algorithm represents a good trade-off between time and algorithmic complexity compared to the brute force approaches.

%%%%%TI
Relative hydration free energy calculations were presented for pairwise TI approaches, showing the influence of the restraints on the convergence of the simulations, and the sampling behavior of molecules containing a rigid core, like commonly used in pharmaceutical drug discovery, is minimal.

%%%%%%REEDS
For multistate methods, we expanded our scheme in order to match the requirements of such approaches. It could be shown that the relative hydration free energies obtained with RE-EDS results in a similar accuracy as the pairwise TI calculations, but using a shorter simulation time.

As an outlook, we would like to point out that our future work on multistate methods will be expanding on this work, and we are looking forward to using the approach for more complex systems. Interesting challenges include increasing the number of states significantly or going to larger molecules.