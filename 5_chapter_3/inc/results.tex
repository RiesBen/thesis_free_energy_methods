\labsec{results}
%----------
\subsection{Reference TI Calculations}
%----------

The curve for the average Hamiltonian derivative as a function of $\lambda$
corresponding to the aqueous methanol-to-dummy mutation is shown in \reffig{ti:dhdl} (blue curve, scale on the left),
as averaged over the ten repeats of the $100\unit{ns}$ TI calculation 
with 129 equispaced $\lambda$-points.
%($0.775\unit{ns}$ sampling per $\lambda$-point in each repeat).
The corresponding running integral, \ie{} the free-energy profile $G(\lam)$, is also shown (orange curve, scale on the right).
%
In view of the large amount of statistics (total  $1\,\mu$s sampling,
{\em i.e.} clearly an overkill for such a calculation), the curve 
is perfectly smooth and extremely well converged.
%
The average Hamiltonian derivative is large and positive below about $0.5$ and
becomes smaller and negative thereafter, leading to a maximum in
the free-energy profile. The shape of these curves is determined
both by the hydration physics of methanol and by the choice of
the employed (soft-core) coupling scheme.
%

The convergence properties of these TI calculations
%estimate with 129 $\lam$-points
as a function of
the total sampling time per repeat are illustrated in \reffig{ti:conv},
in the form of running $\Delta G$ estimates for each of the
ten repeats (dashed curves), along with the associated 
mean and error (95\% confidence interval) on the mean over the repeats (thick blue curve and shaded area).
%
\radd{The latter mean and error at full sampling time, reported in \reftab{repeats},}
evaluate to $21.43 \pm 0.21\,\mathrm{kJ}\,\mathrm{mol}^{-1}$.
This result \radd{(with a minus sign)} is in quantitative agreement with the experimental value 
\radd{for 
%minus 
the} hydration free 
energy of methanol at $298.15\unit{K}$ and $1\unit{bar}$, namely\cite{CA81.1} $-21.4\unit{kJ\,mol^{-1}}$.


\begin{figure}
  \centering
  \subgraph{fig:ti:dhdl}{dhdl_dg_ti_ene_ana.pgf}{0.32}%
  \hfill%
  \subgraph{fig:ti:conv}{conv_mad_129_ene_ana.pgf}{0.32}%
  \hfill%
  \subgraph{fig:ti:lampts}{conv_lampts_simps_ene_ana.pgf}{0.32}\\
  \caption{\footnotesize
  \captitital{Convergence properties of the TI calculations for the aqueous methanol-to-dummy mutation at 298.15 K and 1 bar with the 2016H66 force field.}
%
           \revphil{This figure provides the reference free-energy profile for the perturbation considered,
                   and investigates the convergence properties of TI with different numbers $K_{\mathrm{TI}}$
                     of equidistant $\lambda$-points (based on ten repeats of $100\unit{ns}$ TI calculations).}
%
           Panel (a) shows the average Hamiltonian derivative as a function of 
                     $\lambda$ (blue curve, scale on the left) and the corresponding free-energy 
                     profile (orange, scale on the right). The curves are
                     averaged over ten repeats of $100\unit{ns}$ TI calculations 
                     with 129 equidistant $\lambda$-points.
% ($0.775\unit{ns}$ sampling per $\lam$-point
%                     for each repeat). 
           Panel (b) shows the running $\Delta G$ estimates for each of the ten repeats (dashed curves)
                     as a function of the total sampling time $t$ per repeat, 
                     along with the corresponding mean and error on the mean \radd{(95\% confidence interval)} over the repeats 
                     (thick blue curve and shaded area).
           Panel (c) shows $\Delta G$ estimates considering ten repeats of $100\unit{ns}$
                     TI calculations using $K_{\mathrm{TI}}=2^n+1$ equidistant
                     $\lambda$-points with $n$ = 1,2,...,7 
%($100/K_{\mathrm{TI}}\unit{ns}$ sampling
%                     per $\lam$-point for each repeat; 
                     (individual colored circles),
                     along with the corresponding mean and error on the mean \radd{(95\% confidence interval)} over the repeats 
                     (blue crosses and error bars).
%
%           All $\Delta G$ estimates rely on Simpson quadrature.
%
%           The error on the mean is calculated \radd{from the standard deviation $\sigma$ of the mean
%           as $\epsilon = 2.262 \sigma$},
%%           where $\sigma$ is the standard deviation of the mean,
%           corresponding to a two-sigma confidence interval of 95\%.
           %
           Figs. equivalent to Panel (b) for the TI calculations
           with fewer $\lam$-points are provided in \reffig{avgti}.
%
           Numerical values at full sampling time are reported in \reftab{repeats}
           and \reftab{ti}.
%           In (c), the total sampling time per replica is determined by the number points selected
%                      as $t = 100 M_{TI}/129$ ns.
          }
  \label{fig:ti}
\end{figure}


Corresponding graphs for TI calculations involving different numbers of equidistant $\lam$-points
($K_{\mathrm{TI}}=2^n+1$ with $n=1,2,\dots,7$) at identical total sampling time 
($100\unit{ns}$ for each of the ten repeats) are shown in \reffig{avgti}.
%
The \radd{associated $\Delta G$ means and errors} at full sampling time are reported numerically in \reftab{repeats} 
(see \reftab{ti} for \radd{the values of the individual repeats}). 
The results of \reftab{repeats} also include a comparison of different error estimates
(all excluding the Student $t$-factor), namely:
($i$) \radd{the standard deviation $\sigma$ of the mean over the repeats;}
($ii$) \radd{a bootstrap error $\sigma_{\mathrm{B}}$ calculated from the concatenated trajectories of the repeats at each $\lam$-point;}
($iii$) \radd{an alternative error $\sigma_{\mathrm{G}}$} obtained similarly 
            by calculating the bootstrap errors on the average Hamiltonian derivatives 
            at all $\lam$-points, and propagating them assuming a Gaussian distribution of 
            the estimates, as proposed in Refs. \citenum{BH17.1,BH18.1}.
%
The three values turn out to be very similar, so that further discussions 
will refer to the error $\epsilon$ defined as $\sigma$ amplified by the Student $t$-factor.
%


\begin{table}
  \centering
\caption{\footnotesize
  \captitital{Repeat statistics over TI and CBTI calculations for the aqueous methanol-to-dummy mutation at $298.15\unit{K}$ and $1\unit{bar}$ with the
  2016H66 force field.}
  %
  \revphil{This table investigates the convergence properties 
           of the calculated free-energy change $\Delta G$
           considering
           TI calculations with different numbers of $\lambda$-points
           along with a CBTI simulation with 16 replicas
           (average and various error estimates, based on ten calculation repeats
           involving each at a constant total single-system sampling time of 100 ns) 
         }
  %
\radd{The TI calculations involved $K_{\mathrm{TI}}=2^n+1$ equispaced $\lam$-points 
      with $n=1,2,...,7$.
      The CBTI simulations involved $K=16$ 
      replicas, along with $m_\Lamb=160\,\mathrm{u}\,\mathrm{nm}^2$ and $\tau_{\Lamb}=0.5\,\mathrm{ps}$ (no biasing potential).
      In both cases, the total single-system sampling time was $100\unit{ns}$,
      and the calculation was repeated ten times.
%
      Considering each set of repeats, the successive entries are:
      the free-energy change $\Delta G_{\mathrm{avg}}$ averaged over the repeats;
      the corresponding standard deviation $\sigma$ of the mean;
      the associated error $\epsilon$ on the mean;
      alternative error estimates $\sigma_{\mathrm{B}}$ and $\sigma_{\mathrm{G}}$ calculated by bootstrapping \radd{(no Student $t$-factor included)}.
%
  The quantity $\sigma$ is calculated by scaling the standard deviation of the ten $\Delta G$ values by the square-root of nine.
%
  The error estimate $\epsilon$ is then calculated as $\epsilon=2.262\sigma$, corresponding to a two-sigma confidence interval of $95\%$.
%
  The estimate $\sigma_{\mathrm{B}}$ corresponds to a bootstrap error calculated from the concatenated 
  trajectories of the ten repeats \radd{(for TI, concatenated separately at each $\lambda$-point)}.
%
  The estimate $\sigma_{\mathrm{G}}$ (TI only) is obtained by calculating the bootstrap 
  errors on the average Hamiltonian derivatives at all $\lam$-point, and propagating them assuming 
  a Gaussian distribution of the estimates, as proposed in Refs. \citenum{BH17.1,BH18.1}.
  %
}
  The $\Delta G$ values for the individual repeats can be found in \reftabs{ti} and \reftabn{cvbseed}.
}   
\label{tab:repeats}
\resizebox{\textwidth}{!}{
\begin{tabular}{*{8}{c} | c}
\hline
 & \multicolumn{7}{c}{TI} & CBTI \\
\hline
   $n$ &                1 &                2 &                3 &               4 &                 5 &                6 &                7 &              - \\
   $K_{\mathrm{TI}}$ or $K$ &                3 &                5 &                9 &               17 &               33 &               65 &              129 &             16 \\
\hline
\hline
$\Delta G_{\mathrm{avg}}\,[\mathrm{kJ\,mol^{-1}}]$ &  15.61 &   4.81 &  18.97 &  21.65 &  21.45 &  21.40 &  21.43 &  21.39 \\
$\epsilon\,[\mathrm{kJ\,mol^{-1}}]$ &   0.05 &   0.18 &   0.13 &   0.15 &   0.20 &   0.17 &   0.21 &   0.10 \\
$\sigma\,[\mathrm{kJ\,mol^{-1}}]$ &   0.02 &   0.08 &   0.06 &   0.07 &   0.09 &   0.08 &   0.09 &   0.04 \\
$\sigma_{\mathrm{B}}\,[\mathrm{kJ\,mol^{-1}}]$ &   0.03 &   0.06 &   0.05 &   0.05 &   0.07 &   0.08 &   0.09 &   0.05 \\
$\sigma_{\mathrm{G}}\,[\mathrm{kJ\,mol^{-1}}]$ &   0.03 &   0.06 &   0.05 &   0.05 &   0.05 &   0.05 &   0.05  & -\\
\end{tabular}
}
\end{table}



%
The effect of $K_{\mathrm{TI}}$ on the convergence properties is 
illustrated graphically in \reffig{ti:lampts},
which shows the $\Delta G$ estimates at full sampling time as a function of $K_{\mathrm{TI}}$
for each of the ten repeats (individual colored circles),
along with the associated mean and error on the mean over the repeats (blue crosses and error bars).
%
Using too few $\lambda$-points leads to quadrature errors, which are very significant for $n\leq 4$ ($K_{TI} \leq 17$), 
{\em i.e.} the shape of the \radd{average} Hamiltonian derivative curve is not 
well captured by the Simpson quadrature.
%
For $n \geq 5$ ($K_{\mathrm{TI}} \geq 33$), the quadrature error becomes essentially negligible
compared to the sampling error. 


Interestingly, at constant total sampling time, the latter error appears 
to be essentially insensitive to the number of $\lam$-points employed, 
with very similar $\Delta G$ estimates and error bars
of $21.45\pm 0.20, 21.40\pm 0.17$ and $21.43\pm 0.21\unit{kJ\, mol^{-1}}$ 
%for 
%$n=5$, 6 or 7 (
\radd{for $K_{\mathrm{TI}}=33$, 65 and 129, respectively (\reftab{repeats}).} 
%
Thus, 
%in principle, 
\revphil{for the system considered here and at constant total sampling time},
the use of a larger 
number of $\lam$-points with a reduced simulation time 
is advantageous 
%in TI 
as it reduces the quadrature error without incurring any penalty in terms of sampling error.
%
However, it also induces an extra cost in the form of the preequilibration dead time, which is proportional 
to the number of $\lam$-points employed. 
%
\revphil{
In addition, the insensitivity of the sampling error to the number of 
$\lam$-points may still break down when the simulation times 
at the different $\lam$-points become shorter than the corresponding
orthogonal relaxation times.
}
%


The use of numerous $\lam$-points with insufficient preequilibration is the plague affecting 
the slow-growth (SG) method\cite{BE85.3,ST86.1}, which is rather inaccurate in 
pratice\cite{PE89.1,HE91.1,MA94.12} (large error and hysteresis), unless the results are exponentially 
averaged over \revphil{repeats} with different \radd{Boltzmann-distributed} initial conditions,
as in fast growth\cite{JA97.3,CR00.2,HE01.4,HU02.2} (FG). 
%
In essence, the CBTI scheme aims at achieving this limit of a very large number of $\lam$-points, 
but without suffering from issues related to the preequilibration dead time (TI),
to an insufficient preequilibration (SG), or to the requirement of \radd{exponential} averaging over initial conditions (FG).
%



%------------------------------------------------------------
\subsection{Results using the CBTI Scheme}
%------------------------------------------------------------

The exploration of the influence of the CBTI 
mass parameter $m_\Lambda$ and thermostat coupling 
time $\tau_\Lambda$ considering different numbers $K$ 
of replicas is presented in details in \refsec{othersim}. The results are summarized in \reftab{screen} (entries 1-10).
%
They show that when selected within reasonable ranges,
the parameters $m_\Lambda$ and $\tau_{\Lambda}$ have only a limited 
influence on the kinetic-energy exchange between the CB advance variable
$\Lambda$ and the conformational degrees of freedom, on the 
\radd{average}
temperature $T_\Lambda$, on the diffusion constant $D_{\Lamb}$, and on the 
calculated free-energy change $\Delta G$.
%
Based on this exploration, \revphil{a working}
choice
$m_{\Lamb}=40 K^{1/2}\,\mathrm{u}\,\mathrm{nm}^2$ 
and $\tau_{\Lamb}=0.5\,\mathrm{ps}$ was selected for the following CBTI calculations.



Using this parameter setting, CBTI calculations were performed using
$K=8,16,32,64$ or 128 replicas \radd{along with $256K^{-1}\unit{ns}$ simulation} time for the replica system,
\ie{} $256\unit{ns}$ total single-system sampling time in all cases.
The detailed results are shown graphically in \reffig{thermo:04_cvb_thermo:008:10} - \reffig{thermo:04_cvb_thermo:128:10}
and summarized numerically in \reftab{screen} (entries 11-15).
Illustrative results for $K=8,32$ or 128 (entries 11, 13 and 15)
are also shown in \reffigs{lam} and \reff{dynamics}, and discussed in details below.

The time series and probability distributions of 
 $\Lamb$ for these three simulations
are displayed in \reffig{lam:caplam}.
%
For $K=8$ replicas (left panel), $\Lambda$ covers the entire range
$[0,2\pi)$, but presents 8 regularly spaced probability peaks
at integer multiples of $2\pi/8$. This corresponds
to CB configurations where 
two of the replicas are at the physical end-states $\lam=0$ and $\lam=1$,
the six others being by pairs at $\lam=0.25$, $0.5$ and $0.75$.
%
\radd{The occurrence of such probability peaks} is not surprising considering the shape of the free-energy profile
(\reffig{ti:dhdl}). This profile presents minima at $\lam=0$ and $\lam=1$, the former one being
the fully interacting methanol state, which is particularly favored.
%
This significant inhomogeneity in the sampling of $\Lamb$ correlates with a 
stepwise (hopping) dynamics in the $\Lambda$ time series.

In contrast, for $K=128$ replicas (right panel), $\Lambda$ only spans a limited 
subrange of the interval $[0,2\pi)$ and no longer presents regularly 
spaced probability peaks.
%
%At full convergence, 
\radd{For longer simulations},
one would
expect the \radd{probability} curve to also cover the entire $[0,2\pi)$ range 
and present regularly spaced peaks 
at integer multiples of $2\pi/128$,
but these would be of much smaller magnitudes compared to the $K=8$ case.
%
For $K=128$, the effect of 
\radd{these residual variations}
%this residual inhomogeneity 
becomes almost negligible, {\em i.e.} there is essentially 
no driving force on the $\Lambda$ variable. As a result, instead of a 
hopping dynamics, the $\Lambda$ variable now follows
a purely diffusive (random-walk) trajectory.
%


The situation for $K=32$ replicas (middle panel) is intermediate.
The $[0,2\pi)$ range is fully covered, but the distribution is 
irregular, presenting some small indentations at $2\pi / 32$ spacing.
%

The reduction of the coverage and homogeneity in terms of $\Lamb$
over the interval $[0,2\pi)$ when going from $K=8$ to 32, and then to 128,
is not due to a progressively slower (hopping {\em vs.} diffusive)
dynamics, but simply to the different time spans of the simulations
(32, 8 and 2 ns, respectively), imposed by the constant total single-system sampling 
time \radd{of $256\unit{ns}$} in the present simulations. 
%
This can be seen in \reffig{dynamics}.


\begin{figure}
    \centering
    \subgraph{fig:lam:caplam}{caplambda_ts_hist_ene_ana.pgf}{.99}\\[-1em]
    \subgraph{fig:lam:tildelam}{tildelambda_ts_hist_ene_ana.pgf}{.99}\\[-1em]
    \subgraph{fig:lam:smalllam}{lambda_ts_hist_ene_ana.pgf}{.99}\\
    \caption{\footnotesize\captitital{Time series and probability distributions of the relevant variables
             in unbiased CBTI simulations of the aqueous methanol-to-dummy mutation
             at 298.15 K and 1 bar with the 2016H66 force field.}
%
           \revphil{This figure compares the dynamics and distributions of the 
                    CB variables $\Lambda$ (a), $\tilde{\Lambda}$ (b) and $\lambda$ (c)
                    in CBTI simulations with $K=8$ (left), $32$ (middle) or $128$ (right) replicas
                    (based on simulations with a total single-system sampling time of 256 ns).
           }
%
%             The left, middle and right graphs correspond to simulations 
%
             Panel (a) shows the time series $\Lamb(t)$ and probability distribution $P(\Lamb)$
                       of the CB advance variable $\Lamb$.
%
             Panel (b) shows the time series $\tilde{\Lamb}(t)$ and probability distribution $\tilde{P}(\tilde{\Lamb})$
                       of the CB fractional advance variable $\tilde{\Lamb}$ (\refeq{tilde_lamb_def}).
%
             Panel (c) shows the time series $\lam(t)$ and probability distribution $p(\lambda)$
                       of the coupling variable $\lambda$ for all replicas (\refeq{proba_of_lam}).
%
               In Panel (c), the time series is shown as a blue curve for one reference 
                       replica $k=0$, and as individual colored points at $0.5\unit{ns}$ interval
                       for the $K-1$ other replicas, \radd{and the probability distribution 
             $p(\lambda)$ is calculated considering all replicas according to \refeq{proba_of_lam}}. 
%
               All probability distributions are normalized to one.
%
     \radd{The CBTI simulations relied on $m_\Lamb=40 K^{1/2}\,\mathrm{u}\,\mathrm{nm}^2$ and $\tau_{\Lamb}=0.5\,\mathrm{ps}$ (no biasing potential).}
%
            See also \reftab{screen} entries 11, 13 and 15 for relevant numerical results. 
            Corresponding graphs also including $K=16$ and 64 can be found 
            in \reffigs{thermo:04_cvb_thermo:008:10} - \reffign{thermo:04_cvb_thermo:128:10}.
%
   }
\label{fig:lam}
\end{figure}






%
\reffig{dynamics:vel} shows the time series and probability distribution of $\dot{\Lamb}$, the velocity associated with the $\Lamb$-variable.
\radd{The 
%expected 
Maxwell-Boltzmann} distribution of \refeq{ana_vel} is also displayed for comparison.
The observed width $\sigma_{\dot{\Lamb}}$ of the distribution is reported numerically in \reftab{screen}. 
As expected from \refeq{ana_vel}, it is \radd{proportional to $m_{\Lamb}^{-1/2}$}  which, 
in the present simulations, was itself made proportional to $K^{1/2}$ \revphil{(an arbitrary parameter
choice justified in \refsec{othersim})}. Thus, the average magnitude of $\dot{\Lamb}$ decreases somewhat when going from $K=8$ to 32, and then to 128 (scaling \radd{factor $2^{-1/2} \approx 0.7$} each time).
\reffig{acf} shows the \radd{normalized} autocorrelation functions $c_{\dot{\Lambda}}$ of $\dot{\Lambda}$ for different $m_{\Lamb}$ in the absence of thermostat coupling (entries 1-5 of \reftab{screen}). 
The corresponding correlation times $\tau_{\dot{\Lambda}}$ are reported numerically in \reftab{screen}. Here, one observes that the correlation time 
increases with \radd{$m_{\Lamb}$} as a result of the higher inertia of the CB.
In addition, for the two lowest values \radd{of $m_{\Lamb}$}, the initial decay of $c_{\dot{\Lamb}}$ to negative values suggests that the velocities are frequently reversed at short time, when barriers cannot be crossed.
\radd{Note, however, that this connection between $m_{\Lamb}$ and $\tau_{\dot{\Lambda}}$
is} somewhat altered by the thermostat coupling in the other simulations (including those of entries 11-15 in \reftab{screen}).
%
\begin{figure}
    \centering
    \begin{minipage}[t]{2.2in}
        \hfill \\
    \input{\path/plt/autocorr_vel_ene_ana.pgf}%
  \end{minipage}%
    \begin{minipage}[t]{1.2in}
        \hfill \\
        \input{\path/plt/autocorr_vel_legend.pgf}
  \end{minipage}\\

    \caption{\footnotesize \captitital{\radd{Velocity autocorrelation} functions of the CB advance variable in unbiased CBTI
      simulations of the aqueous methanol-to-dummy mutation at $298.15\unit{K}$
      and $1\unit{bar}$ with the 2016H66 force field.}
      %
           \revphil{This figure shows the increase in the autocorrelation time
                    of the CB velocity upon increasing the 
                    mass parameter in CBTI simulations with $K=16$ replicas
                    and no thermostat coupling.
                   }
%
      The normalized autocorrelation function \radd{$c_{\dot{\Lambda}}$} of the velocity $\dot{\Lambda}$
      is shown for different values of the mass-parameter $m_{\lambda}$, \radd{in the absence of}
      thermostating of the $\Lambda$-variable ($\tau_{\Lambda}\rightarrow \infty$).
%
      See also \reftab{screen} entries 1-5 for relevant numerical results, including
      the corresponding autocorrelation times $\tau_{\dot{\Lamb}}$.
%
      Corresponding graphs for all simulations listed in \reftab{screen} are shown 
      in \reffigs{thermo:03_cvb_screen:016:1_0.01} - \reffign{thermo:04_cvb_thermo:128:10}
   }
\label{fig:acf}
\end{figure}

%
\revphil{Because the two above trends act in opposite directions,
the actual rate of diffusion along $\Lamb$ is not very different in the three simulations.}
%
\revphildel{[-- Redundant sentence removed --]}



\radd{This is seen in \reffig{dynamics:dif}}, which shows the mean-square displacement \radd{$d_{\Lambda}$}
of $\Lamb$ as a function of time. The curves are
essentially linear, as expected from the Einstein model (\refeq{einstein}). 
The corresponding diffusion constants $D_{\Lamb}$ are reported in \reftab{screen}, namely 
$4.4$, $6.2$ and $2.3\unit{ns^{-1}}$ 
for $K=8$, 32 and 128 replicas, respectively, \ie{} not dramatically different.
A similar observation holds for the probability distribution of $\ddot{\Lambda}$, the acceleration associated with the $\Lamb$-variable, shown in \reffig{dynamics:acc}.
%As discussed in Appendix B, 
The choice of taking $m_{\Lambda}$ proportional to $K^{1/2}$ 
was made here precisely to achieve a CB acceleration that is essentially independent of $K$ \radd{(see discussion in \refsec{quad})}.

\begin{figure}
    \centering
    \subgraph{fig:dynamics:vel}{vel_ts_hist_ene_ana.pgf}{.99}\\[-1em]
    \subgraph{fig:dynamics:acc}{acc_ts_hist_ene_ana.pgf}{.99}\\[-1em]
    \subgraph{fig:dynamics:dif}{diffusion_ene_ana.pgf}{.99}
    \caption{\footnotesize
    \captitital{Dynamical properties of the CB advance variable in unbiased CBTI calculations of
      the aqueous methanol-to-dummy mutation at 298.15 K and 1 bar with the 2016H66 force field.}
           \revphil{This figure compares the dynamics and distributions of the 
                    CB variables $\dot{\Lambda}$ (a) and $\ddot{\Lambda}$ (b), as well as the 
                    mean-square displacement $d_{\Lambda}$ along $\Lambda$ (c),
                    in CBTI simulations with $K=8$ (left), $32$ (middle) or $128$ (right) replicas
                    (based on simulations with a total single-system sampling time of 256 ns).
           }
%
%
%             The left, middle and right graphs correspond to simulations 
%             with $K=8$, $32$ or $128$ replicas, respectively.
%
             Panel (\subref{fig:dynamics:vel}) shows the time series $\dot{\Lamb}(t)$ and probability distribution $P_{\dot{\Lamb}}(\dot{\Lamb})$
                       of the velocity associated with the CB advance variable $\Lamb$.
%
             Panel (\subref{fig:dynamics:acc}) shows the time series $\ddot{\Lamb}(t)$ and probability 
                       distribution $P_{\ddot{\Lamb}}(\ddot{\Lamb})$ of the corresponding acceleration.
                       % As can be seen the distribution of $P_{\ddot{\Lamb}}(\ddot{\Lamb})$ is 
%roughly the same independent of $K$ due to scaling of $m_{\Lamb} \propto M^{1/2}$.
             Panel (\subref{fig:dynamics:dif}) shows the mean-square displacement 
             $d_{\Lamb}(t)$ of $\Lamb$ (\refeq{einstein}).
%
             In Panel (\subref{fig:dynamics:vel}), the probability distribution from the simulation (blue curve) is shown 
together with the
             analytical Maxwell-Boltzmann distribution $P^{\mathrm{MB}}_{\dot{\Lamb}}(\dot{\Lamb})$ of \refeq{ana_vel} (orange curve).
             %
             In Panel (\subref{fig:dynamics:dif}), the mean-square displacement $d_{\Lamb}$ from the simulation (blue curve)
             is shown along with a linear least-squares fit over the interval 0 - $0.15\unit{ns}$ (brown dashed line).
             %
            All probability distributions are normalized to one. 
%
     \radd{The CBTI simulations relied on $m_\Lamb=40 K^{1/2}\,\mathrm{u}\,\mathrm{nm}^2$ and $\tau_{\Lamb}=0.5\,\mathrm{ps}$ (no biasing potential).}
%            The CBTI simulations were performed using $m_{\Lamb}=40 M^{1/2}\,\mathrm{u}\,\mathrm{nm}^2$,            $\tau_{\Lamb}=0.5\,\mathrm{ps}$ and no biasing potential.
          See also \reftab{screen},
            entries 11, 13 and 15 for relevant numerical results. 
            Corresponding graphs also including $K=16$ and 64 can be found in 
            \reffigs{thermo:04_cvb_thermo:008:10} - \reffign{thermo:04_cvb_thermo:128:10}.
%
   }
\label{fig:dynamics}
\end{figure}

%
The evolution of $D_{\Lamb}$ with $K$ is clearly non-monotonic (see values for entries 11-15 in \reftab{screen}), 
which results in large part from the two opposite effects mentioned above
\radd{concerning the effect of $m_{\Lamb}$ on $\sigma_{\dot{\Lamb}}$ and $\tau_{\dot{\Lambda}}$.}
%
\revphil{On the one hand, $\sigma_{\dot{\Lamb}}$ ({\em i.e.} the average magnitude of the CB velocity)
decreases upon increasing $m_{\Lamb}$,
%(made here proportional to $K^{1/2}$), 
and the dynamics along the $\Lamb$-variable becomes intrinsically slower.
%
On the other hand, $\tau_{\dot{\Lambda}}$ ({\em i.e.} the dynamical inertia of the CB)
increases upon increasing $m_{\Lamb}$,
and the ability to cross residual barriers in the $G_{\Lamb}(\Lamb)$ free-energy profile is enhanced.
}
%
The magnitude of these barriers itself decreases upon increasing $K$. 
In addition, inertia along $\Lamb$ is also affected by the thermostat coupling. 
The combination of these effects is complex and the resulting trends in $D_{\Lamb}$ 
upon varying $K$ would be difficult to rationalize in details. 
Most importantly, however, they appear not to be extremely pronounced. 


The time series and probability distributions
of the CB fractional advance variable $\tilde{\Lamb}$ are shown in 
\reffig{lam:tildelam}.
% for the same three simulations.
%
In contrast to the distributions of $\Lamb$ over the range $[0,2\pi)$ \radd{(\reffig{lam:caplam})}, the distributions of $\tilde{\Lamb}$
over the range $[0,\Delta \Lamb)$ are converged and cover the entire range, also for $K=128$. 
%
    For $K=8$, the distribution of $\tilde{\Lamb}$ is significantly biased towards 0 and $2\pi/8$,
    \ie{} CB configurations with two of the replicas at the physical end-states.
    For $K=32$, it is still somewhat biased towards 0 and $2\pi/32$, which corresponds to the same
    situation.
    And for $K=128$, the probability distribution becomes very close to homogeneous.
%


Finally, the time series and probability distributions
of the coupling variable $\lambda$ are shown in \reffig{lam:smalllam} for all replicas.
%in the same three simulations.
%
The time series is shown for one reference replica $k=0$ (blue curve) as well as 
 the other replicas $k=1\,...\,K-1$ (individual colored points at 0.5\unit{ns} interval).
%
Compared to the reference replica, these representative points are merely 
\radd{shifted by $2K^{-1}k$}, refolded by periodicity into the range $[0,2)$, 
and \radd{the points in the range $[1,2)$ mirrored into the range $[0,1]$} 
\radd{according to \refeqs{cb_lam_of_big_lam}-\refeqn{zigzag_fct}}.
%
The probability distribution $p(\lambda)$ is calculated 
according to \refeq{proba_of_lam}, {\em i.e.} considering all replicas.
%
Although the $\Lamb$-space \radd{in the range $[0,2\pi)$} is \radd{sampled only partially given the simulation lengths} (\reffig{lam:caplam}), 
the $\lam$-space in the range $[0,1]$ (\reffig{lam:smalllam}), just like the $\tilde{\Lamb}$-space
in the range $[0,\Delta \Lamb)$ (\reffig{lam:tildelam}), is completely sampled in all cases.
%
However, for $K=8$, the barriers are still too high
and the sampling is considerably biased towards
$\lambda$-values that are integer multiples of $1/4$.
The corresponding bias towards integer multiples of $1/16$ 
is far less pronounced for $K=32$, and the simulation
with $K=128$ is very close to a homogeneous sampling
of $\lam$.


In summary, the sampling of $\Lamb$ only needs to cover 
a $\Delta \Lamb$ range to ensure the sampling 
of all possible 
$\tilde{\Lamb}$-values
by the CB
and thus,
of all possible $\lam$-values
by at least one replica.
%
However, the homogeneity of the sampling
in $\lam$ depends crucially
on $K$, quasi-homogeneous sampling requiring 
a large number of replicas.
%

%
This can be seen in \reffig{deltag},
which shows in a logarithmic form the height $G^\star_{\Lamb}$ of the peaks
 in $G_{\Lamb}(\Lamb)$ as a function
of the number $K$ of replicas (see also \reffig{gprof} for the individual curves).
%
The dependency is approximately linear with a slope
of minus one, suggesting that $G^\star_{\Lamb}$
scales as $K^{-1}$. Such a scaling is expected based on the considerations
made in \refsec{quad}. Note that the proportionality constant depends on the derivatives of $G(\lam)$
at the two physical end-states. \radd{In a situation where these derivatives vanished,
a scaling as $K^{-2}$ would be expected instead}.
%

\begin{figure}
    \centering
    \input{\path/plt/deltag_ene_ana.pgf}
    \caption{\footnotesize
    \captitital{Height $G^\star_{\Lamb}$ of the residual barriers in the 
      free-energy profile $G_{\Lamb}(\Lamb)$ for unbiased CBTI
      simulations of the aqueous methanol-to-dummy mutation at $298.15\unit{K}$
      and $1\unit{bar}$ with the 2016H66 force field.}
%
           \revphil{This figure shows the progressive decrease in the height of the 
                    barriers along $\Lambda$
                    upon increasing the number $K$ of replicas in a CBTI simulation.
                   }
%
      The data is represented in logarithmic form and a linear line of slope $-1$ is fitted to the filled 
      circles (intercept $1.75$ \radd{at $\log_{10}(K)=0$}). The open circle was omitted 
      from the fit because  $G^\star_{\Lamb}$ cannot be determined sufficiently accurately (\radd{within the} noise).
      %
      The simulations considered rely on $K=8,16,32,64$ or $128$ replicas,
      \radd{along with $m_\Lamb=40 K^{1/2}\,\mathrm{u}\,\mathrm{nm}^2$ and $\tau_{\Lamb}=0.5\,\mathrm{ps}$ (no biasing potential).}
%
      The corresponding free-energy profiles $G_{\tilde{\Lamb}}(\tilde{\Lamb})$ are shown graphically in \reffig{gprof}.
   }
\label{fig:deltag}
\end{figure}

%------------------------------------------------------------
\subsection{CBTI with Biasing Potential}
%------------------------------------------------------------

As seen above, simulations employing only few replicas may still be affected by a
significant sampling inhomogeneity along $\lam$ due to high residual barriers along $\Lamb$.
To prevent the undersampling of specific $\lam$-ranges in this case,
a memory-based biasing potential can be employed. This renders the CBTI 
method more robust since it becomes insensitive to the chosen number of replicas,
and more compatible with possible constraints related to \eg{} the number of cores in a specific computing 
node during the calculation.
%
As discussed in the Theory section, it is sufficient to define such a biasing potential 
over a $[0,\Delta \Lamb / 2]$ range of $\tilde{\Lamb}$,
and its magnitude is limited by the already largely homogeneous sampling of $\tilde{\Lamb}$.
These features speed up the build-up procedure drastically, leading to a comparatively small
 simulation time for the LE build-up phase relative to single-system $\lam$-LEUS.


\reffig{leus} illustrates the sampling in \radd{a biased 
compared to that in the unbiased CBTI simulation with
$K=8$ replicas}.
%
%\revdavid{[DELETE (already in methods): 
\radd{The biasing relied on $34$ 
%equispaced 
basis functions 
over half the $\tilde{\Lamb}$-range $[0,\pi/4]$ 
(17 free coefficients) 
and a build-up phase of $0.15~\mathrm{ns}$ 
(\ie{} only $1.2\,\mathrm{ns}$ total single-system build-up time).
During this time, the replica system performed three double sweeps over all grid-points,
leading to a biasing potential  
with an energetic resolution of $10^{-6}\,\mathrm{kJ\,mol^{-1}}$ (\reffigs{leus:tildebias} and \reff{leus:bias}).
}

The $\Lamb$-sampling in the biased CBTI simulation is 
much closer to uniform compared to that in the unbiased 
one
%simulation 
(\reffig{leus:lhist} \textit{vs.} \reff{leus:lhistleus}),
\ie{} the relative probabilities between most and least sampled $\Lamb$-values has been reduced
from 70 to 4.6. 
The time series of $\lam$ and the sampling distribution $p(\lam)$ show that the 
motion of the conveyor belt was hindered in the unbiased simulation,
which it is no longer the case in the biased one (\reffig{leus:llhist} 
\textit{vs.} \reff{leus:llhistleus}).

Alternative  build-up protocols were also
explored for $K=8$ and $K=16$, differing in the number $N_{\mathrm{gp}}$ of grid points,
in the duration $t_{\mathrm{LE}}$ of the build-up phase, and in the build-up parameters
$c_{\mathrm{LE}}$ and $f_{\mathrm{red}}$. The corresponding results are shown graphically in \reffigs{leus:8} and \reffign{leus:16} and summarized numerically
in \reftab{leus}. All the tested protocols reduce the sampling inhomogeneity to an acceptable level,
although residual inhomogeneities are difficult to remove entirely. Compared to the TI results
with a sufficient number of $\lam$-points (\reftab{repeats}, entries with $K_{\mathrm{TI}}\geq 33$) and the unbiased
CBTI results with a sufficient
% high
number of replicas (\reftab{screen}, entries 12-15 with $K\geq 16$), the biased
CBTI results with $K=8$ or 16 (entries 16 and 17 in \reftab{screen}) 
deliver similar $\Delta G$ values and error bars, provided that 
an appropriate
%a 
%sufficiently thorough 
build-up protocol is employed. 

\begin{figure}
  \centering
  \subgraph{fig:leus:tildebias}{tildememory_ene_ana.pgf}{0.4}
  \subgraph{fig:leus:bias}{memory_ene_ana.pgf}{0.4}\\[-1.2em]%
  \subgraph{fig:leus:lhist}{lambda_hist_ene_ana.pgf}{0.4}%
  \subgraph{fig:leus:lhistleus}{lambda_leus_hist_ene_ana.pgf}{0.4}\\[-1.2em]
  \subgraph{fig:leus:llhist}{lambda_ts_hist_008_ene_ana.pgf}{0.4}%
  \subgraph{fig:leus:llhistleus}{lambda_ts_hist_008_leus_ene_ana.pgf}{0.4}\\[-1.2em]
  \caption{\footnotesize
    \captitital{Biasing potential and simulation results from
             unbiased {\em vs.} biased CBTI calculations of the aqueous methanol-to-dummy mutation
             at 298.15 K and 1 bar with the 2016H66 force field.}
%
           \revphil{This figure illustrates how the use of a biasing potential 
                    permits to drastically reduce the sampling inhomogeneity of 
                    CBTI simulations performed with a small number of replicas
                    (here, $K=8$ replicas).
}
%
%
             Panels (\subref{fig:leus:tildebias}) and (\subref{fig:leus:bias}) show the biasing potential
             along the $\tilde{\Lambda}$- and $\Lambda$-ranges, respectively.
%
             Panels (\subref{fig:leus:lhist}) and (\subref{fig:leus:lhistleus}) show the
             probability distribution $P(\Lamb)$ of the CB advance variable $\Lamb$
             in the unbiased and biased simulation, respectively.
%
             Panels (\subref{fig:leus:llhist}) and (\subref{fig:leus:llhistleus}) show
             the time series $\lam(t)$ and probability distribution $p(\lambda)$
             of the coupling variable $\lambda$ for all replicas
             in the unbiased and biased simulation, respectively.
%
             In Panel (\subref{fig:leus:tildebias}), the symmetric biasing potential (blue curve) is represented
             using 34 basis functions (curves of different colors) involving 17 free coefficients.
%
             In Panels (\subref{fig:leus:llhist}) and (\subref{fig:leus:llhistleus}), the time series is shown as a blue curve for one reference 
             replica $k=0$, and as single colored points at $0.5\unit{ns}$ spacing 
             for the $K-1$ other replicas, and the probability distribution 
             $p(\lambda)$ is calculated considering all replicas according to \refeq{proba_of_lam}.
%
            All probability distributions are normalized to one. 
%
            The CBTI simulations were performed with or without biasing potential,
            using $m_{\Lamb}=113\,\mathrm{u}\,\mathrm{nm}^2$ and $\tau_{\Lamb}=0.5\,\mathrm{ps}$.
%
            For the biased simulation, the biasing potential was constructed in a build-up phase
            of $0.15\unit{ns}$ duration.
%
            Results for biased simulations with $K=16$ and alternative build-up protocols are shown
            in \reffig{leus:8} and \reffig{leus:16}, the numerical results being
            reported in \reftab{leus}.
      }
\label{fig:leus}
\end{figure}




%------------------------------------------------------------
\subsection{CBTI Free-Energy Estimator} 
%------------------------------------------------------------





The calculation of free-energy differences from CBTI simulations according to
\refeq{cbti_formula} relies on the choice of a number $J$ of bins. This number can be taken very large, thereby
reducing the (rectangular) quadrature error to an essentially negligible amount,
but should not exceed a threshold $J_{\mathrm{max}}$ where empty bins occur. 
The influence of $J$ is illustrated in \reffig{dhdl}, which shows the average 
Hamiltonian derivative curve obtained from the
%\radd{$3.125\unit{ns}$ of the} 
unbiased CBTI simulation with $K=32$ replicas (entry 13 in \reftab{screen}, \radd{restricted to 
total 100 ns single-system sampling time})
for $J=10$ (\reffig{dhdl:10}), $J=500$ (\reffig{dhdl:500}) and $J=5000$ (\reffig{dhdl:5000}).
%
A too low number of bins (\eg{} $J=10$) will lead to \radd{an insufficient number of integration points,
thereby increasing the quadrature error}. 
%
A too high number of bins will lead to insufficient sampling of each bin, thereby
increasing the statistical error. The optimal value of $J$ will depend on the
%\revdavid{[DELETE: number of $\lam$-values sampled] 
total sampling time and on the degree of homogeneity of the sampling
along $\lam$.

\begin{figure}
    \centering
    \subgraph{fig:dhdl:10}{dhdl_0_ene_ana.pgf}{0.32}%
    \subgraph{fig:dhdl:500}{dhdl_1_ene_ana.pgf}{0.32}%
    \subgraph{fig:dhdl:5000}{dhdl_2_ene_ana.pgf}{0.32}\\
    \subgraph{fig:conv:bins}{conv_bins_ene_ana.pgf}{.5}\hfill\\
    \caption{\footnotesize
    \captitital{Influence of the number of bins used in the free-energy estimator for 
      CBTI simulations of the aqueous methanol-to-dummy 
      mutation at $298.15\unit{K}$ and $1\unit{bar}$ with the 2016H66 force field.}
      %
           \revphil{This figure illustrates how the number $J$ of bins employed in the estimator affects
                   the quadrature and sampling quality (top)
                   and the resulting estimated $\Delta G$ (bottom).
}
%
%, a mass-parameter $m_{\Lamb}=226\,\mathrm{u\,nm^2}$ and a coupling time $\tau_{\Lamb}=0.5\unit{ps}$.
      %
      Panels (\subref{fig:dhdl:10}), (\subref{fig:dhdl:500}) and (\subref{fig:dhdl:5000})
      show the \radd{average} Hamiltonian derivative curve constructed with different numbers $J$ of bins,
      \radd{namely $J=10$ (\subref{fig:dhdl:10}), 100 (\subref{fig:dhdl:500}) or 5000 (\subref{fig:dhdl:5000})}.
%
      The curves are based on the
%$3.125\unit{ns}$ 
  CBTI simulation \radd{using $K=32$ replicas} (entry 13 in \reftab{screen}).
%
      Panel (\subref{fig:conv:bins}) shows the $\Delta G$ estimate from \refeq{cbti_formula} 
      \radd{for CBTI simulations with $K=8$, 16, 32, 64 or 128 replicas (the two former either unbiased or biased; entries 11-17 in \reftab{screen})}
      as a function of $J$ from 1 to $J_{\mathrm{max}}$, where $J_{\mathrm{max}}$ corresponds to 
      the first occurence of empty bins. 
      %
\radd{All CBTI simulations were performed 
            using $m_{\Lamb}=40K^{1/2}\,\mathrm{u}\,\mathrm{nm}^2$ and $\tau_{\Lamb}=0.5\,\mathrm{ps}$,
            and their analysis is restricted here to 100 ns total single-system sampling time}.
%
%            The CBTI simulations were performed with or without biasing potential,
%            using $m_{\Lamb}=113\,\mathrm{u}\,\mathrm{nm}^2$ and $\tau_{\Lamb}=0.5\,\mathrm{ps}$.
%
%%%      \radd{All the results are based on CBTI simulations of 100 ns total single-system sampling time.}
    }
\label{fig:dhdl}
\end{figure}


\radd{To investigate the influence of $J$ on the calculated $\Delta G$,
this number was systematically increased from 1 to $J_{\mathrm{max}}$ in increments ranging 
from 10 to 1000.
%
The results are shown in \reffig{conv:bins}},
% shows the dependence of the calculated $\Delta G$ value on $J$ from 1 to $J_{\mathrm{max}}$
considering the unbiased CBTI simulations with $K=8,16,32,64$ or 128 
(entries 11-15 in \reftab{screen}, \radd{restricted to $100\unit{ns}$ total single-system sampling time})
as well as the biased CBTI simulations with $K=8$ or 16 (entries 16 and 17 in \reftab{screen}, same restriction). 
%
For the unbiased simulation with $K=8$ (blue curve), the convergence of $\Delta G$ upon increasing $J$
is irregular and the value of $J_{\mathrm{max}}$ is comparatively low (about 600),
due to the inhomogeneous sampling along $\lam$. All other curves present qualitatively similar convergence
features. For a low $J$ ($J<100$), the quadrature error leads to strong variations.
For intermediate $J$ ($100\leq J \leq 3000$), the curve presents a plateau, \ie{} 
the $\Delta G$ estimate becomes insensitive to changes in $J$.
For larger $J$ ($3000< J \leq J_{\mathrm{max}}$, where $J_{\mathrm{max}}$ ranges from about 5000 to 8000)
the curves become somewhat noisy again due to undersampled bins.
%
These results suggest that, as a rule of thumb, the value of $J$ for the application 
of \refeq{cbti_formula} should be selected between at least about 100 and at most about $J_{\mathrm{max}}/2$.
For most of the $\Delta G$ results presented in this chapter, \radd{a default value 
%of 
$J=500$  was selected
as
leading to a reasonably smooth Hamiltonian derivative curve (\reffig{dhdl:500}) while keeping the quadrature error negligible
(except for the 
%unbiased 
CBTI simulation
with $K=8$,
where $J=200$ was used).
}
%
\revphil{The results for the variants $\Delta G_{\mathrm{alt}}$ and $\Delta G_{\mathrm{app}}$ to
$\Delta G$ from \refeq{cbti_formula} which do not require the specification of a number 
of bins (\refeqs{cbti_formula_app1} and \refeqn{cbti_formula_average})
are also included in \reftab{screen} for comparison. They are discussed in \refsec{CH2C}.}
%
\revphildel{[-- Variants moved to Appendix C --]}
%%
%\revphil{The estimates} $\Delta G_{\mathrm{alt}}$ and $\Delta G_{\mathrm{app}}$ were also calculated \revphil{for comparison},
%using \refeqs{cbti_formula_app1} and \refeqn{cbti_formula_average} \revphil{of Appendix C}.
%%The latter one is only expected to be accurate for quasi-homogeneous sampling of the $\lam$-range. 
%%


%------------------------------------------------------------
\subsection{CBTI Convergence Properties}
%------------------------------------------------------------

The convergence properties of TI and CBTI are compared in \reffigs{ti:conv:repeat} and \reff{cbti:conv},
which shows the $\Delta G$ values calculated from ten simulation 
repeats (individual dashed curves) along with the corresponding
mean and error \radd{(95\% confidence interval)} on the mean (thick blue curve and shaded area) 
as a function of the total single-system sampling time per repeat.
%
The simulation protocols compared are TI with $K_{TI}=$129 $\lambda$-points 
(\reffig{ti:conv:repeat}, which is identical to \reffig{ti:conv}) 
and unbiased CBTI with $K=16$ (\reffig{cbti:conv}).
%
Note that in the latter case, the curves do not start at time zero
because the application of \refeq{cbti_formula} with $J=500$ to estimate 
$\Delta G$ only becomes possible when $J_{\mathrm{max}}$ exceeds $500$
({\em i.e.} each of the 500 bins encompasses at least
one sampled $\lambda$-value).
%
The \radd{mean $\Delta G$ values over the repeats} and the associated error bars at full
single-system sampling time ($100\unit{ns}$ per repeat) are 
reported in \reftab{repeats} (last two columns;
see also \reftab{ti} and \reftab{cvbseed} for the results
of the individual repeats).
%

\begin{figure}
  \centering
  \begin{minipage}{1.67in}
  \subgraph{fig:ti:conv:repeat}{conv_mad_129_wide_ene_ana.pgf}{1.0}\\
  \subgraph{fig:cbti:conv}{conv_mad_cvb_ene_ana.pgf}{1.0}
  \end{minipage}%
  \begin{minipage}{2.86in}
  \subgraph{fig:comp:conv}{conv_datapts_ene_ana.pgf}{.585}%
  \begin{minipage}[t]{1.14in}
  \centering
  \hfill \\
  \scalebox{0.71}{\input{\path/plt/conv_datapts_legend.pgf}}
  \end{minipage}\\
  \subgraph{fig:comp:errors}{conv_errors_datapts_ene_ana.pgf}{.585}%
  \begin{minipage}[t]{1.14in}
  \centering
  \hfill \\
  \scalebox{0.71}{\input{\path/plt/conv_errors_datapts_legend.pgf}}
  \end{minipage}
  \end{minipage}
  \caption{\footnotesize \captitital{Convergence properties of the CBTI calculations compared to other methods 
    for the aqueous methanol-to-dummy 
    mutation at $298.15\unit{K}$ and $1\unit{bar}$ with the 2016H66 force field.}
    %
           \revphil{This figure compares the convergence properties 
           of the calculated free-energy change $\Delta G$
           considering TI (a) {\em vs.} CBTI (b) based on ten repeats
           of calculations involving 100 ns total single-system sampling time,
           or considering CBTI (biased or unbiased) {\em vs.} TI, HRE, TI/EXTI and TI/MBAR
           in terms of estimated $\Delta G$ (c) and error (d)
           based on one calculation involving 100 ns total single-system sampling time.
}
%
    Panel (\subref{fig:ti:conv:repeat}) is identical to \reffig{ti:conv}  and
    shows the convergence of $\Delta G$ for TI calculations with $K_{\mathrm{TI}}=129$ $\lam$-points.
%
    Panel (\subref{fig:cbti:conv}) displays the corresponding
      convergence of $\Delta G$ for unbiased CBTI calculations with $K=16$ replicas. 
%
    Panels (\subref{fig:ti:conv:repeat}) and (\subref{fig:cbti:conv})
show the running $\Delta G$ estimates for each of the ten repeats (dashed curves)
                     as a function of the total sampling time $t$ per repeat, 
                     along with the corresponding mean and error on the mean \radd{(95\% confidence interval)} over the repeats 
                     (thick blue curve and shaded area).
%
The final $\Delta G$ values for the individual 
    repeats are reported in the \reftab{cvbseed},
\radd{and the statistics over the repeats can be found in \reftab{repeats}}.
%
    Panel (\subref{fig:comp:conv}) compares convergence properties in terms of $\Delta G$  
    based on single simulations.
    %
    Panel (\subref{fig:comp:errors}) displays the associated errors $\sigma_B$ calculated by bootstrapping \radd{(no Student $t$-factor included)}.
    %
\radd{ Panels (\subref{fig:comp:conv}) and (\subref{fig:comp:errors}) 
consider 
unbiased CBTI ($K=8, 32$ or 128),
biased CBTI ($K=8$)
HRE ($K_{\mathrm{HRE}}=65)$,
along with TI with EXTI
or MBAR estimators ($K_{\mathrm{TI}}=17$ and 129 virtual $\lambda$-points in both cases).
}
%
    % 
    The corresponding final $\Delta G$ estimates are reported in \revphil{\reftab{comp}}.
    %
    %% For TI (red), the convergence of 1 simulation using 129 equidistant $\lam$-points is shown. 
    %% The Hamiltonian derivative at every $\lam$-point was integrated using Simpson's quadrature.
    %% %
    %% For HRE (dark green) employed $K=65$ replicas with exchange trials every 
    %% $\tau_{\mathrm{HRE}}=0.2\,\mathrm{ps}$. The analysis was performed analogous to TI.
    %% %
    %% For EXTI (pink) and BAR (yellow), 17 equidistant $\lam$-points were simulated, but the 
    %% Hamiltonian and Hamiltonian derivatives were evaluated at 129 equidistant $\lam$-points.
    %% %
    %% The corresponding values after $100\unit{ns}$ sampling are reported in \reftab{screen}.
  }
\label{fig:conv:pts}
\end{figure}


The comparison clearly evidences an enhanced convergence \radd{for CBTI at identical sampling time},
both in terms of the running averages and of the final values.
%
These final values are $21.43\pm0.21\unit{kJ\,mol^{-1}}$ for TI, compared to $21.39\pm0.10\unit{kJ\,mol^{-1}}$ for CBTI,
{\em i.e.} identical within error bars, but with \radd{an error reduced by about a factor two}
for CBTI.
%
This improved convergence is likely due to the improved orthogonal sampling,
{\em i.e.} the diffusion of the systems along $\lam$ permits to circumvent orthogonal
barriers present at specific $\lam$-values.


The same observation holds for the CBTI simulations involving different 
numbers of replicas or/and a biasing potential. This is shown in \reffigs{comp:conv} and \reff{comp:errors}
for $K=8$, 32 or 128 (unbiased) and for $K=8$ (biased). Here, a single simulation
was performed (no repeats), and the running $\Delta G$ value (\reffig{comp:conv})
is displayed along with the corresponding bootstrap error estimate
(\reffig{comp:errors}; no Student $t$-factor included) as a function of the total single-system
sampling time.
%
The $\Delta G$ estimates and bootstrap error bars 
%\radd{(no Student $t$-factor included)} 
at full single-system sampling time \radd{($100\unit{ns}$)} are 
reported in \reftab{comp} (second and third columns).
%
Except for the unbiased CBTI simulation with $K=8$,
where the sampling along $\lambda$ is still heterogeneous,
all the CBTI protocols follow very similar error curves upon increasing
the sampling time \radd{(\reffig{comp:errors})}, with final errors of $0.12-0.16\unit{kJ\,mol^{-1}}$.
%
In contrast, the convergence of TI is significantly slower,
with corresponding final errors of $0.17-0.30\unit{kJ\,mol^{-1}}$ \radd{(fourth column in \reftab{comp})}.
%


\begin{table}
  \caption{\footnotesize\captitital{Free-energy estimates from the CBTI calculations compared to other methods 
    for the aqueous methanol-to-dummy mutation at $298.15\unit{K}$ and $1\unit{bar}$ with the 2016H66 force field.}
    %
  \revphil{This table compares the convergence properties 
           of the calculated free-energy change $\Delta G$
           considering CBTI (biased or unbiased), TI, HRE, EXTI and MBAR      
           calculations with different numbers of replicas (all at a constant total single-system sampling time of 100 ns).
  }
%
\radd{
The CBTI calculations involved $K$ replicas, along with
$m_{\Lamb}=40 K^{1/2}\,\mathrm{u}\,\mathrm{nm}^2$ and  $\tau_{\Lamb}=0.5\,\mathrm{ps}$ (with or without biasing potential).
The TI calculations involved $K_{\mathrm{TI}}$ equispaced $\lam$-points.
The HRE calculations involve $K_{\mathrm{HRE}}$ equispaced $\lam$-points, along with $\tau_{\mathrm{HRE}}=0.2\,\mathrm{ps}$.
%
For EXTI and MBAR, the calculations relied on $K_{\mathrm{TI}}$ equispaced real $\lambda$-points,
and consideration of 129 equispaced virtual $\lam$-points.
%
      In all cases, the total single-system sampling time was $100\unit{ns}$.
}
%
%    All calculations relied on $100\,\mathrm{ns}$ total single-system sampling time. 
%    The protocols considered different numbers of replicas $K$, $K_{\mathrm{TI}}$ or $K_{\mathrm{HRE}}$.
%    %
    The corresponding convergence properties are illustrated graphically in \reffig{conv:pts}.
%
    Additional numerical results regarding the CBTI simulations are reported in \reftab{screen}, entries 11-15
    (unbiased CBTI) and entries 16-17 (biased CBTI).
    %
%%     CBTI-LEUS simulations employed $m_{\Lamb}=113$ for $K=8$ and $m_{\Lamb}=160$ for $K=16$. A build-up force constant
%% of $c_{\mathrm{LE}}=0.001\,\mathrm{kJ\,mol^{-1}}$ was used, which was multiplied with a factor $f_{\mathrm{red}}=0.1$ after every double sweep over the range $[0,\pi/M[$. 
%% The total number of gridpoints used for the construction of the biasing potential were 264 ($K=8$) and 
%% 272 ($K=16$) based on 17  ($K=8$) and 9 ($K=16$) coefficients.
%% The biasing potential was allowed to build up during a simulation time of $t_{\mathrm{LE}}=150\,\mathrm{ps}$ 
%% (\ie{} $1.2\,\mathrm{ns}$ total simulation time) for $K=8$ and $t_{\mathrm{LE}}=70\,\mathrm{ps}$ 
%% (\ie{} $1.12\,\mathrm{ns}$ total simulation time) for $K=16$, during which the CB performed 3 double sweeps over the range $[0,\pi/M[$ 
%% (both replica systems). 
%% %
%% For TI (first repeat of the 10 repeats), 129 equidistant $\lam$-points were simulated. The Hamiltonian
%% derivative was integrated using Simpson quadrature rule considering 9, 17, 33, 65 and 129 equidistant points, 
%% where the total simulation time of $100\unit{ns}$ was distributed equally over all considered points.
%% %
%% HRE (column 5) were run using  $K=17,33~\text{or}~65$ replicas with exchange trials every 
%% $\tau_{\mathrm{HRE}}=0.2\,\mathrm{ps}$. The analysis was performed analogous to TI.
%% %
%% EXTI (column 6) relied on 9 or 17 simulated $\lam$-points, but the Hamiltoninan derivative was predicted at
%%  129 equidistant $\lam$ points. The Hamiltonian derivative curve was integrated using trapezoidal quadrature.
%% %
%% Likewise, the MBAR (column 7) employs 9 or 17 simulated $\lam$-points, but the Hamiltoninan is predicted 
%% for every configuration at 129 equidistant $\lam$ points. The energies of all $\lam$-points 
%% are finally combined using the MBAR-formula to lead to the final free-energy.
  }
  \label{tab:comp}
\begin{center}
\resizebox{\textwidth}{!}{
  \begin{tabular}{ccccccc}
\hline
& \multicolumn{6}{c}{$\Delta G\,[\mathrm{kJ\,mol^{-1}}]$}\\
\hline
 $K$, $K_{\mathrm{TI}}-1$ &              CBTI &          CBTI &               TI &              HRE &             EXTI &             MBAR  \\
 or $K_{\mathrm{HRE}}-1$  &            (unbiased)   &          (biased) \\
\hline
\hline
%   8 & 21.69 (0.40) & 21.48 (0.16) & 19.22 (0.18) &      -       & 21.24 (0.12) & 21.37 (0.14) \\
%  16 & 21.42 (0.16) & 21.30 (0.13) & 21.87 (0.17) & 21.52 (0.16) & 21.12 (0.05) & 21.17 (0.14) \\
%  32 & 21.44 (0.14) &         -    & 21.48 (0.20) & 21.39 (0.15) &       -      & -            \\
%  64 & 21.32 (0.14) &         -    & 21.18 (0.24) & 21.47 (0.13) &       -      & -            \\
% 128 & 21.43 (0.12) &         -    & 21.45 (0.30) &      -       &       -      & -            \\
% UPD ON 12.11
    8 & 21.69 (0.40) & 21.48 (0.16) & 19.22 (0.18) &      -       & 21.24 (0.37) & 21.37 (0.15) \\
   16 & 21.42 (0.16) & 21.30 (0.13) & 21.87 (0.17) & 21.52 (0.16) & 21.12 (0.14) & 21.17 (0.13) \\
   32 & 21.44 (0.14) &         -    & 21.48 (0.20) & 21.39 (0.15) &       -      & -            \\
   64 & 21.32 (0.14) &         -    & 21.18 (0.24) & 21.47 (0.13) &       -      & -            \\
  128 & 21.43 (0.12) &         -    & 21.45 (0.30) &      -       &       -      & -            \\
%
  \end{tabular}
}
\end{center}
\end{table}

%------------------------------------------------------------
\subsection{Comparison of CBTI with other Methods}
%------------------------------------------------------------


The $\Delta G$ values and associated bootstrap errors (no Student $t$-factor included)
calculated using various CBTI protocols are compared to the 
results of other approaches in \reftab{comp}.
%
The comparison involves calculations relying on $100\,\mathrm{ns}$ total 
single-system sampling time, {\em i.e.} essentially identical computational costs.
%
The protocols compared are
unbiased CBTI (with $K=8, 16, 32, 64$ or 128),
biased CBTI (with $K=8$ or $16$),
TI (with $K_{\mathrm{TI}}=9, 17, 33, 65$ or 129),
HRE (with $K_{\mathrm{HRE}}=17, 33$ or 65),
\radd{along with TI/EXTI and TI/MBAR
(both with $K_{\mathrm{TI}}=9$ or $17$, and using 129 virtual $\lambda$-points).}
%
More details on the HRE, TI/EXTI and TI/MBAR calculations can 
be found in \reffigs{hre} - \reffign{barlam}.




\radd{As previously discussed,
the results of the TI calculations with $K_{\mathrm{TI}}=9$ or 17 
are affected by large quadrature errors (see \reffig{ti:lampts} and \reftab{repeats}), 
and those of the unbiased CBTI simulation with $K=8$,
are affected by a significant heterogeneity of the 
sampling along $\lambda$ (see \reffig{lam}).}
% and second column of \reftab{comp}).}
%
Excepting these three calculations, all free-energy estimates \radd{of \reftab{comp}}
are essentially compatible, taking into account that the error bars exclude a
Student $t$-factor.
%


The convergence of the HRE ($K_{\mathrm{HRE}}=65)$, TI/EXTI ($K_{\mathrm{TI}}=17$) and TI/MBAR ($K_{\mathrm{TI}}=17$)
protocols are also compared to those of TI ($K_{\mathrm{TI}}=129$), unbiased CBTI ($K=8, 32$ or 128) and biased CBTI ($K=8$) 
in  \reffigs{comp:conv} and \reff{comp:errors}.
%
Considering \reffig{comp:errors}, the uncertainty of the unbiased CBTI simulation
with $K=8$ is found to be particularly large, owing to the sampling heterogeneity. 
Excepting this simulation, the uncertainty \radd{is the largest for the TI estimate.}
%
The different CBTI protocols reduce the error by about a factor two
compared to TI, presumably due to enhanced orthogonal sampling.
%
This sampling advantage is shared by HRE, for which the error curve is very similar.
%
Advanced free-energy estimators such as EXTI and MBAR also improve the convergence 
over plain TI in a different way, namely by increasing the statistical 
%but for a different reason, 
efficiency of the $\Delta G$ estimation. 
%
\revphil{The error reduction is also about a factor two compared to TI,
which is probably not entirely coincidental.
%
If two fixed-$\lambda$ simulations are trapped in different 
orthogonal configurational wells in TI, the variable-$\lambda$
approaches (CBTI, HRE) will promote well-transitions,
whereas the advanced-estimator approaches (EXTI, MBAR)
will import statistical information on the other well
from the neighboring $\lambda$-point.
%
One might refer to these two types of effects
orthogonal-sampling {\em vs.} orthogonal-statistics
advantages, respectively.
%
For the simple system considered here,
both results in comparable convergence improvements.
%
This suggests that the two effects may not be cumulative
if one generalizes the current
%ly used 
CBTI estimator of \refeq{cbti_formula} to an EXTI- or MBAR-type estimator.
}
%may 
%might further enhance the accuracy of the calculated $\Delta G$ estimates
%by increasing the statistical efficiency of the estimation.}
%
%\revphil{
%The fact that the corresponding error reduction is also 
%about a factor two compared to TI is probably coincidental
%}.
%
%%%However, the error in EXTI may be underestimated since the predicted $\lam$-points  are possibly correlated which was neglected in the error calculation.
%
%\radd{Thus, if CBTI benefits like HRE from an orthogonal sampling advantage,
