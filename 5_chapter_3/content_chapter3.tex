\makeatletter
\def\input@path{{../}}
\makeatother
\documentclass[../main.tex]{subfiles}
\begin{document}
\renewcommand{\path}{4_chapter_3/}
\chapter[Relative Free-Energy Calculations for Scaffold Hopping-Type Transformations with an Automated RE-EDS Sampling Procedure]{Relative Free-Energy Calculations for Scaffold Hopping-Type Transformations with an Automated RE-EDS Sampling Procedure 
    \footnote{\label{footnoteChapter3CopyRight} 
    This Chapter is reproduced in part from  Benjamin Ries, Karl Normak, R. Gregor Wei{\ss} , Salom\'e Rieder, Em\'ilia P. de Barros, Candide Champion, Gerhard K\"onig, Sereina Riniker, J. Comput.-Aided Mol. Des., in press (2021), licensed by Creative Commons CC BY.}
}
\chaptermark{RE-EDS Sampling Procedure}
\label{ch:fereeds}

\aquote{
    "During my undergraduate work I concluded that electrostatics is unlikely to be important [for enzymes]"
}{Arieh Warshel, Nobel lecture 2013}


\begin{summary}
The calculation of relative free-energy differences between different compounds plays an important role in drug design to identify potent binders for a given protein target. 
Most rigorous methods based on molecular dynamics (MD) simulations estimate the free-energy difference between pairs of ligands. 
Thus, the comparison of multiple ligands requires the construction of a ``state graph'', in which the compounds are connected by alchemical transformations. 
The computational cost can be optimized by reducing the state graph to a minimal set of transformations.
However, this may require individual adaptation of the sampling strategy if a transformation process does not converge in a given simulation time. 
In contrast, path-free methods like replica-exchange enveloping distribution sampling (RE-EDS) allow the sampling of multiple states within a single simulation without the pre-definition of alchemical transition paths.
To optimize sampling and convergence, a set of RE-EDS parameters needs to be estimated in a pre-processing step.
Here, we present an automated procedure for this step that determines all required parameters, improving the robustness and ease of use of the methodology.
To illustrate the performance, the relative binding free energies are calculated for a series of checkpoint kinase 1 (CHK1) inhibitors containing challenging transformations in ring size, opening/closing, and extension, which reflect changes observed in scaffold hopping. The simulation of such transformations with RE-EDS can be conducted with conventional force fields and, in particular, omit the need for soft bond-stretching terms.
%
\end{summary}

\clearpage
\pagebreak

%%%%%%%%%%%%%%%%%%%%%%%%%%%%%%%%%%%%%%%%%%%%%%%%%%%%%%%%%%%%%%%%%%%%%
%% Start the main part of the manuscript here.
%%%%%%%%%%%%%%%%%%%%%%%%%%%%%%%%%%%%%%%%%%%%%%%%%%%%%%%%%%%%%%%%%%%%%
%================================================================================
\section{Introduction}
%================================================================================

Rigorous free-energy calculations using MD simulations have become an important tool to estimate binding free energies of novel compounds for lead optimization in drug discovery \cite{Cournia2017, Armacost2020,Cournia2020}. Although computationally relatively expensive, these methods are needed to properly account for entropic contributions introduced by protein/ligand conformational changes, entropy-enthalpy compensation, and the desolvation of a ligand \cite{Chodera2013}.

Computational free energy calculations typically make use of thermodynamic cycles, i.e., the transitive difference relations of idealized states of the system of interest that are representable by a graph. For instance, to estimate the binding free energy of five compounds, a ``state graph'' can be constructed (Figure~\ref{fig: StateGraph}), where the nodes represent the end states and the edges the free-energy differences between them. Although not impossible \cite{Aldeghi2016}, the direct calculation of (absolute) binding free-energies ($\Delta G^\text{bind}_i$) is generally very challenging to achieve computationally \cite{Cournia2017}. A simpler alternative is to calculate the alchemical free-energy differences between two compounds $i$ and $j$ in a given environment ($\Delta G_{ji}^{env}$) and then compare the relative binding free energy $\Delta \Delta G^\text{bind}_{ji}$ with the difference of the $\Delta G^\text{bind}_i$ obtained from experiments \cite{Jorgensen1988b, Merz1991},
\begin{equation}
    \Delta \Delta G^\text{bind}_{ji} = \Delta G_{ji}^{\text{protein}} - \Delta G_{ji}^{\text{water}}
    = \Delta G^\text{bind}_j - \Delta G^\text{bind}_i
\end{equation}

Conventional free-energy methods such as TI \cite{Kirkwood1935} and FEP \cite{Zwanzig1954} introduce a coupling parameter $\lambda$ to define a pathway from end state $i$ ($\lambda=0$) to end state $j$ ($\lambda=1$). In practice, simulations at discrete intermediate $\lambda$-points are performed to obtain converged free-energy differences.

%
If a (large) series of $N$ compounds is investigated, the free-energy difference for all $(N(N-1))/2$ pairs of ligands would in principle have to be calculated. To reduce the computational cost, automatic schemes have been developed to identify the edges in the state graph (Figure \ref{fig: StateGraph}) with the smallest perturbations such that all nodes (for a given environment) are connected \cite{Liu2013,Wang2015,Yang2020}. It is thereby important to include some cycles as cycle closure is a frequently used measure to assess convergence. 
Nevertheless, manual optimizations may sometimes be required to determine the best sampling strategy \cite{Jespers2019}.
Furthermore, calculating only a subset of the edges leads to a larger uncertainty in the estimated free-energy difference for pairs that are no longer directly connected. As $\Delta \Delta G^\text{bind}_{ji}$ values are often relatively small, the increased uncertainty may negatively impact the usefulness of such calculations in practical applications. 

\begin{figure}[h!]
    \centering
    \includegraphics[width=\columnwidth]{fig/intro/State_graph.png}
    \caption{State graph to calculate relative binding free energies, where the nodes represent specific compounds $A$ - $E$ in a particular environment (water/protein). 
    The connecting (directed) edges describe the transformations from one end state to another. The dashed-dotted arrows denote the direct calculation of the (absolute) binding free energy of compound $i$ to the protein, $\Delta G_{i}^\text{bind}$, whereas solid arrows indicate alchemical transformations between compound $i$ to compound $j$ in a given environment. From the resulting $\Delta G_{ji}^\text{env}$, $\Delta \Delta G^\text{bind}_{ji}$ can be calculated and compared with the value obtained from the difference of the experimentally determined $\Delta G_{i}^\text{bind}$ (gray dashed arrows).
    In pathway-dependent methods, each edge between two end states is calculated separately. With (RE-)EDS, all end states in a given environment can be considered simultaneously in a single simulation of a reference state (green circles).}
    \label{fig: StateGraph}
\end{figure}

%The advantage of pathlessness
An attractive and more efficient alternative to path-dependent methods is to simulate a reference state, which includes all $N$ end states simultaneously, without the specification of pathways (green rings in Figure \ref{fig: StateGraph}). Such a reference state is provided by the EDS \cite{Christ2007, Christ2008, Christ2009, Riniker2011} method. The EDS reference state can be further tuned for optimal sampling with parameters. Note that cycle closure is guaranteed by definition in this approach.
%
In order to enhance sampling further, combinations of EDS with enhanced sampling methods were developed such as replica-exchange EDS (RE-EDS) \cite{Lee2014, Sidler2016, Sidler2017} and accelerated EDS \cite{Perthold2018, Perthold2020}.

%conclusion:
In this study, we present an improved automated workflow for RE-EDS simulations that was restructured into two phases. The first phase aims to automatically estimate method parameters that otherwise had to be provided by the user. The second phase automatically optimizes the estimates from the first phase to retrieve a robust parameter set. The final production phase calculates the relative binding free energies of multiple ligands from a single simulation per environment. 
The robustness and versatility of the RE-EDS workflow are demonstrated on a series of five inhibitors of human checkpoint kinase 1 (CHK1) \cite{Huang2012}.
These ligands were selected by Wang \textit{et al.} \cite{Wang2017} as a challenging benchmarking set for FEP calculations since the changes between these ligands exemplify different types of core-hopping transformations (i.e. ring size change, ring opening/closing, and ring extension). Special soft bond-stretching terms were developed to be able to handle these transformations \cite{Wang2017}. In contrast to many other methods, no such special soft bonds are required with RE-EDS as we can use the linked dual topology approach \cite{Riniker2011} in a straightforward manner. 
\FloatBarrier

%================================================================================
\section{Theory}
%================================================================================

The here-defined algorithm is based on classical graph algorithms. Its target is to find a suitable placement for distance restraints between two molecules $m_i$ and $m_j$ for a linked dual topology approach.

The following assumptions are made towards this process:
\begin{enumerate}
    \item $m_i$ and $m_j$ strongly overlap in the coordinate space; i.e., are aligned to each other
    \item optimal restraints placement for an MD simulation fulfills:
    \begin{enumerate}
        \item restrained atom pairs are maximally distributed over the two molecules
        \item restrained atoms have a minimal distance to each other ($d_{\text{res}} <= 1.0~\text{\AA}$)
        \item restraints are not hindering the molecule sampling
    \end{enumerate}
    \item given the number of required restraints $n_{\text{res}}$, it holds that $n_{\text{res}} << n_{\text{atoms}_{m_i}} \wedge~n_{res} << n_{\text{atoms}_{m_j}}$ 
\end{enumerate}
As a result of these assumptions, only relatively rigid areas of the molecules, such as ring-atoms, should be selected for the restraint search space. While restraining non ring atoms might be favorable for the distribution of the atoms over the molecules, it will be more likely to negatively affect the sampling behavior of the molecules.

\subsection{Assigning distance Restraints  to a pair of molecules}
\begin{figure}
    \centering
    \includegraphics[width=\textwidth]{fig/theory/AlgorithmScheme.png}
    \caption{Algorithm scheme going stepwise through the algorithm}
    \label{fig:algorithmScheme}
\end{figure}

\subsubsection{Translation of the problem}
In order to be able to apply the algorithm to a pair of molecules, a set of pre-processing operations is required to translate the problem into a graph problem. 
The formulated approach is based on a graph representation of the restraint space. To be able to solve the graph problem of selecting a good set of atoms to form distance restrains, it needs to be translated into a graph fulfilling:
\begin{equation}
    G(N, E, \omega), E \subseteq \{\{x,y\}\mid x,y\in N\;{\textrm {and}}\;x\neq y\},
\end{equation}. \cite{}

with $N$ as a set of nodes, $E$ as a set of edges, and $\omega$ as a set of weights.

Here, we start with a molecule pair $mp$ consisting of a molecule $m_i$ and a molecule $m_j$ and their sets of atoms, $A_i$ and $A_j$, respectively, for which the relative free energy shall be determined. 

%%Alignment
In a first step, which will not be covered here in detail, the two molecules are aligned to each other, such that the overlap of the van der Waals surfaces of both molecules is maximal. This can be achieved using third-party tools such as the RDKit or PyMol. \cite{landrum2021, DeLano2020}

%%Filter
Next, the sets of possible atoms $A_i$ and $A_j$, that could be selected for the distance restraints are reduced to atoms  $A^{ring}_i$ and  $A^{ring}_j$, that are incorporated in molecule substructures forming a ring. 

%% Restraints with a Distance cutoff
$A^{ring}_i$ and  $A^{ring}_j$ are used to find potential restraints $R$,  with a user-defined cutoff distance $d_{a_i, a_j} \leq 1.0~\text{\AA}$ between the atoms. 
A potential restraint, therefore, is a pair of atoms $(a^{ring}_i, a^{ring}_j)$ that fulfills the distance criterion.

%% Building a Graph
All potential $R$ are used as nodes $N$ to construct a fully connected graph $G$. 
Each $r$ is represented by the center of geometry (COG) of the two involved atoms.
The undirected edges of the fully connected graph retrieve a weight as the euclidean distance $d_{\vec{r}_j,\vec{r}_i}=\omega_{ji}$ between two coordinates. 

\subsubsection{Solving the graph problem}
From the generated fully connected graph, only a selection of restraints is required to fulfill the assumptions. Many different algorithms could be applied to the graph in order to retain a set of restraints. Here, we decided to use a Min-Max decision scheme to build a spanning tree $R_{opt}$ within a greedy Prim-like approach.

%Algorithm Definition
%%bit more literature research
%initial move
The algorithm starts by picking the edge of $G$ with the largest $\omega_{ji}$/distance in the graph, i.e. the two restraints whose centers of geometry are the furthest away from each other. 

%iterate
%%update min
After this initial selection of two restraints for $R_{opt}$, the weights of $G$ are updated with the minimal distance of all $r$ in $R_{opt}$ to a respective node $n_i$. Subsequently, all edges and nodes are removed that contain atoms that are already selected in $R_{opt}$.

%% select max
After the edge update, the $e$ with maximal $\omega_{ji}$, and not connecting two nodes already in $R_{opt}$,  is added to  $R_{opt}$.

%%termination
This procedure is repeated till either all present nodes are connected or $|R_{opt}| = n_{res}$. 

\subsubsection{Mapping back}
The retrieved set of restraints $R_{opt}$ is translated back, such that the atoms forming the restraints can be selected and used to translate the distance restraints to a format usable for different MD packages like GROMOS or GROMACS. Additionally a JSON format is provided, allowing importing the results with any standardized JSON-Parser. \cite{Schmid2012, Abraham2015}


\subsubsection{Tie-Breaker}
Due to numerical accuracy and non-perfect alignment, small absolute differences in the priority calculation for the picking step were observed, leading to suboptimal results. These practical problems were solved by adding a tie-breaker that detects if multiple priorities in one iteration are within a range of $0.2$~\AA. 
Such a tie is broken by calculating the COG for all present restraints and using the distance to all current possible restraints as a secondary criterion. This distance should be maximal.


\subsection{Assigning distance Restraints to a multistate system}
In the case of multistate methods such as Replica Exchange Enveloping Distribution Sampling (RE-EDS) or $\lambda$-Dynamics, multiple molecules need to be restrained to each other. \cite{Sidler2016, knight2011} It was decided to form a ring of connecting restraints that links all molecules to two neighboring molecules in our approach.

The approach uses the described pairwise greedy approach to calculate all possible molecule pair restraints. The possible sets of restraints will be compared to each other by calculating the convex hull around the restraint coordinates. The convex hull volume (chv) is used to build up a fully connected graph connecting all molecules. The optimal connections are retrieved by applying a greedy algorithm, inspired by the Kruskal Algorithm, to the graph of ligand pairs (Figure \ref{fig: convexHull}). This algorithm picks the edges providing the largest chvs for chaining the pairs of restraints together. The following rules are applied during this: no node has larger connectivity than two, and no cycles in the generated graph are allowed. The cycle closure is achieved in the end by tying the loose chain ends together.

\begin{figure}[h]
    \centering
    \includegraphics[width=\textwidth]{fig/theory/MultistateChainingScheme.png}
    \caption{Multistate Molecule chaining by maximizing the CHV}
    \label{fig: convexHull}
\end{figure}

\subsection{Free energy calculation methods}
In this work, two free energy calculation methods are applied with the linked dual topology approach.

\subsubsection{TI-Simulation}
Thermodynamic integration is a standard method for calculating free energies. It is used to sample a $\lambda$-dependent path between two end states A and B. The potential system energy is constructuted as follows:\cite{Kirkwood1935}
\begin{equation}
    V(\vec{r}; \lambda) = (1-\lambda) ~ V_A(\vec{r}) + \lambda ~ V_B(\vec{r})
    \label{eq: TI-Potential}
\end{equation}

In a simulation, state A is represented if $\lambda = 0$, and state B if $\lambda = 1$. Usually, the free energy is calculated as a path over multiple intermediate state simulations, in between the two extreme $\lambda$ values 0 and 1:
\begin{equation}
    \Delta F = \int^{1}_{0} \left< \frac{\delta V(\lambda)}{\delta \lambda} \right>_{\lambda} \,d\lambda
    \label{eq: TI-Integration}
\end{equation}

\subsubsection{RE-EDS}
%% Hamiltonian Construction
RE-EDS is a combination of Hamiltonian replica exchange (H-RE) and enveloping distribution sampling (EDS).\cite{Hansmann1997,Sugita2000, Christ2007, Sidler2016,Sidler2017,Ries2021} In EDS, a reference state Hamiltonian $V_R$ is sampled.\cite{Christ2007} $V_R$ combines $N$ end states as
\begin{align}
    V_R\left(\vec{r};s,\vec{E}^R\right) = -\frac{1}{\beta s}\ln\left[\sum\limits_{i=1}^N e^{-\beta s\left(V_i(\vec{r})-E_i^R\right)}\right]
\end{align}
with the smoothness parameter $s$ and a set of energy offsets $E_i^R$. 

The force on a particle $k$ can be calculated as \cite{Christ2008}

\begin{align}
    \vec{f}_k(t)=-\frac{\partial V_R(\vec{r}; s, \vec{E}^R)}{\partial \vec{r}_k} = \sum^N_{i=1}\frac{e^{-\beta s(V_i(\vec{r}) -E_i^R)}}{\sum^N_{j=1}{e^{-\beta s (V_j(\vec{r})-E_j^R)}}}  \left( -\frac{\partial V_i(\vec{r})}{\partial \vec{r}_k} \right) \,.
\end{align}

For high $s$ values, the sampling of the reference state is dominated by the state with the lowest $(V_i(\vec{r}) - E_i^R)$, whereas for low $s$ values, all states actively contribute to the system sampling resulting in the so called 'undersampling'. \cite{Riniker2011} 

EDS allows the calculation of the relative free energy difference of any end state pair in the system from a single simulation as

\begin{align}
    \Delta G_{BA} = -\frac{1}{\beta}\ln\frac{\langle e^{-\beta(V_B-V_R)}\rangle_R}{\langle e^{-\beta(V_A-V_R}\rangle_R} \, .
\end{align}

For EDS, an optimal choice of $s$ is essential to achieving good sampling of all end states in the system. RE-EDS reduces the difficulty of choosing an optimal $s$-value by simulating several replicas with different $s$-values and performing Hamiltonian replica exchanges.\cite{Sidler2016}


%================================================================================
\section{Computational Details}
%================================================================================

\subsection{Model System}
To showcase the performance of RE-EDS, a system of five inhibitors (L1, L17, L19, L20 and L21) of checkpoint kinase 1 (CHK1) taken from Ref.~\cite{Huang2012} was chosen (Figure \ref{fig:Ligands/Protein}). The numbering of the compounds is according to Ref.~\cite{Huang2012}. The same system was studied in Ref.~\cite{Wang2017} as part of a series of scaffold hopping systems. Although the five ligands share a common substructure, they were considered to exemplify different types of core-hopping transformations (i.e. ring size change, ring opening/closing, ring extension) and R-group modifications \cite{Wang2017}.

\begin{figure}[h]
    \centering
    \includegraphics[width=\columnwidth]{fig/methods/CHK1_ring_opening_system_condense.png}
    \caption{(Top): 3D depiction of the five CHK1 inhibitors L1, L17, L19, L20, and L21 (numbering according to Ref.~\cite{Huang2012}). The selected locations of the distance restraints are indicated by the silver spheres. (Bottom): CHK1 protein in complex with the ligand bundle (PDB ID:3U9N).}
    \label{fig:Ligands/Protein}
\end{figure}

For the protein, the GROMOS 54A7 force field \cite{Schmid2011} was used. For the ligands, topologies were generated using the parametrization by the ATB server \cite{Mark2011} as an initial guess. The bonded terms were manually harmonized and adjusted to match the parameterization of similar functional groups in the GROMOS 54A7 force field. Partial charges were generated with our previous machine learning approach \cite{Bleiziffer2018} ($\epsilon$ = 4) and manually arranged into charge groups. The input files can be retrieved from: \\ https://github.com/rinikerlab/reeds/tree/main/examples/systems.

\subsection{System Preparation}%%initialize
The crystal structure of CHK1 in complex with ligand L1 (PDB ID:3U9N) was used as starting structure. The initial coordinates for ligands L17, L19, L20, L21 were generated with the {\tt{ConstrainedEmbed()}} functionality in the RDKit \cite{Landrum2021}, where the common part was kept fixed in the crystal conformation. The coordinates of each ligand and those of the protein were subsequently energy minimized in vacuum using the steepest descent \cite{Ruder2016} approach implemented in the GROMOS software package \cite{Schmid2012}. 

%%build eds_system
A ``dual topology'' approach was used for the RE-EDS simulations, i.e. each ligand is present in the system separately \cite{Riniker2011}. Thus, each end state comprises of one active ligand and $N-1$ inactive (dummy) ligands. To avoid spatial drifting of the dummy ligands, eight distance restraints per ligand pair were defined within the common substructure (Figure \ref{fig:Ligands/Protein}) to connect all ligands in a ring with the help of the RestraintMaker program (https://github.com/rinikerlab/restraintmaker) (order: -L1-L17-L19-L20-L21-). The reference distance was set to 0.0~nm and the force constant to $1000$~kJ~mol$^{-1}$~nm$^{-2}$.
The combined topology file was generated with the program {\tt{prep\_eds}} in the GROMOS++ \cite{Eichenberger2011} package. 
The EDS system was solvated in a cubic box of SPC \cite{Berendsen1981} water (resulting in 1'848 solvent molecules for the ligands in water and 15'639 solvent molecules for the protein-ligands complex). 
An energy minimization was carried out with the steepest descent algorithm \cite{Ruder2016}, where all solute atoms were position restrained with a force constant of $25'000$~kJ~mol$^{-1}$~nm$^{-2}$. 

\subsection{Simulation Details}
All simulations were performed with the GROMOS software package \cite{Schmid2012} (freely available on http://www.gromos.net).
The equilibrations and production runs were carried out under isothermal-isobaric (NPT) conditions using the leap-frog integration algorithm \cite{Hockney1970} and a time step of $2$~fs. 
Bond lengths were constrained with SHAKE \cite{Ryckaert1977} using a tolerance of $10^{-4}$. 
The nonbonded contributions were calculated with a twin-range scheme using a short-range cutoff of $0.8$~nm and a long-range cutoff of $1.4$~nm. 
The electrostatic nonbonded contributions beyond the long-range cutoff were calculated with the reaction-field \cite{Tironi1995} approach and a dielectric permittivity of 66.7 \cite{Glattli2002} for water. 

The temperature was kept constant at $300$~K using the weak coupling scheme \cite{Berendsen1984} and a coupling time of $0.1$~ps$^{-1}$. The pressure was kept at $1.031$~bar ($1$~atm) with the same type of algorithm and a coupling time of $0.5$~ps and an isothermal compressibility of $4.575 \cdot 10^{-4}$~(kJ~mol$^{-1}$~nm$^{-3}$)$^{-1}$.
Rotation and translation of the center of mass of the simulation box were removed every $2$~ps. 
Energies were written to file every $20$ steps and coordinates every $5'000$ steps.
In the RE-EDS simulations, replica exchanges was attempted every $20$ steps.

\subsection{RE-EDS Workflow}
The new Python code to manage the RE-EDS workflow, including the analysis steps, can be retrieved from: https://github.com/rinikerlab/reeds.
The workflow starts with the energy-minimized coordinates of the EDS system (all $N$ ligands plus environment, maximally contributing end state is L20) into the parameter exploration step, which is used as equilibration phase.
A RE-EDS simulation of 0.2~ns length was performed with 21 logarithmically distributed replicas between $s=1.0$ and $10^{-5}$ and all energy offsets set to zero.  
The thresholds $T_{i}^{\text{us}}$ were estimated from replicas with very low $s$-values.
Undersampling was observed when each end state occurred with a fraction $f_{i}^{\text{occur,us}} \ge 0.75$ during the simulation period.
To be conservative, the lower bound of the $s$-parameters for the following steps was set to the $s$-value two levels below the highest replica with undersampling.

To optimize the coordinates of the system for each end state, an EDS simulation of 2~ns length was performed for each end state $i$ with $s=1.0$ and $E^R_i=500$~kJ~mol$^{-1}$ while the energy offsets of all other end states were set to $-500$~kJ~mol$^{-1}$. L20 was the initial maximally contributing end state in the starting configuration. 
The coordinates were considered to be optimized when the desired end state was constantly sampled as the maximally contributing state in the last 30~\% of the simulation. 

%%Eoff
To determine the energy offsets, a $1.5$~ns RE-EDS simulation was carried out with $12$ logarithmically distributed replicas for the ligands in water and $17$ for the protein-ligands complex between $s=1.0$ and the lower bound (determined above). The first $0.4$~ns of the simulation were discarded as equilibration. This simulation was performed in two manners: (i) using the final coordinates from the lower-bound determination as starting configuration for all replicas (1SS approach), or (ii) using the different optimized coordinates from the previous substep for the replicas in an alternating way (SSM approach). 
For the PEOE \cite{Sidler2016} scheme, the following parameters were used: fraction $f_{i}^{\text{us}} \ge 0.9$ and the potential thresholds determined in the lower bound exploration $T_{i}^{\text{us}}$.

%param optimization
%%S-opt
The iterative optimization of the $s$-distribution with the N-LRTO \cite{Sidler2017} algorithm was started with the energy offsets and the final coordinates of the previous substep.
Four replicas were added per iteration. 
The simulation length of the first iteration was $0.5$~ns, and subsequently increased by $0.5$~ns at each iteration until a maximum length of $1.5$~ns was reached.

%%% Eoff Rebalancing
The iterative optimization of the $f_i^{\text{mc}}$ distribution was carried out with the described scheme.
The scheme used short $0.5$~ns simulations, and adjusted in each step the energy offsets $E^R$ with a pseudo-count intensity factor $x = 30$.

The optimization was considered converged here when all end states were sampled as maximally contributing states at $s=1.0$, the number of round trips per ns was above zero, and the improvement of the round-trip time was below $\overline{\tau}/nRT < 0.5$~ns.

The production run with constant reference-state parameters was performed for $3.5$~ns.

\subsection{Simulation of Single States}
The input coordinates for the simulations of the individual end states were extracted from the RE-EDS starting coordinates and subsequently energy minimized. Next, a production run of $4$~ns was performed. 

\subsection{Analysis}
Free-energy differences were calculated with the program {\tt{dfmult}} from the \textit{GROMOS++} \cite{Eichenberger2011} package.
Statistical analysis and handling of the workflow steps are based on the Python packages pandas \cite{Mckinney2010}, Matplotlib \cite{Hunter2007}, NumPy \cite{Vanderwalt2011}, SciPy \cite{Virtanen2020}, and PyGromosTools \cite{Ries2021}.

%================================================================================
\section{Results and Discussion}
%================================================================================

The chosen model system of five inhibitors of CHK1 kinase exemplifies different core-hopping transformations (i.e. ring size change, ring opening/closing, ring extension) and R-group modifications \cite{Wang2017}, increasing the complexity compared to the systems previously studied with RE-EDS. Furthermore, the performance can be directly compared to the results obtained with FEP+ and OPLS3 in Ref.~\cite{Wang2017} as well as with QligFEP results in Ref.~\cite{Jespers2019}.

\subsubsection{Parameter Exploration}
The RE-EDS workflow was started by estimating the lower bound for the $s$-distribution. Using the above mentioned undersampling criterion (see Methods section), a lower bound of $s=0.01$ was determined for the protein-ligands complex and $s=0.0056$ for the ligands in water. 

%State Optimizations
A fast transition of the initial maximally contributing end state to the desired maximally contributing end state was observed in the End State Generation. This process was monitored by the maximally contributing end state metric over time.
The transition occurred latest after $1.3$~ns, and the system remained in the biased end state for the rest of the simulation time.
In both water and complex simulations, the desired end state was sampled about $~99\%$ of the simulation time with the exception of L19 in water (Table \ref{SItab:RingCycleOpenin_sampling_fraction_optimizedStates}).
Optimized coordinates were obtained for all five ligands, as verified by comparing the potential-energy distribution from the EDS simulation with the one extracted from a standard MD simulation of the respective ligand (Figure \ref{SIfig:CHK1_RingOpening_soptimization_efficiency}). 
From these same steps, the potential-energy thresholds for the occurrence sampling ($T_{i}^{\text{phys}}$) and undersampling ($T_{i}^{\text{us}}$) were estimated (Table \ref{SItab:RingCycleOpenin_PotentialTresholds}).

\begin{table}[H]
\centering
\caption{Fraction of the simulation time $f_i^{\text{mc}}$ (in \%) that the desired end state was sampled as the maximally contributing state during the EDS simulation to optimize the coordinates for a desired end state.}
\label{SItab:RingCycleOpenin_sampling_fraction_optimizedStates}
\begin{adjustbox}{max width=\textwidth}

\begin{tabular}{ l | c c }
 Ligand & Water  & Complex \\ 
 \hline
     L1 & 99.84 & 99.97 \\ 
     L17 & 99.99 & 99.97\\
     L19 & 36.07 &  99.98\\
     L20 & 99.99 & 100\\
     L21 & 100 & 99.97 \\
\end{tabular}
\end{adjustbox}
\end{table}

To inspect if the optimized state simulations' results sufficiently represent the target states, a comparison between the target state obtained potential-energy distributions in the EDS simulations with MD simulations consisting of only the target state was conducted (Figure \ref{fig:CHK1_set2_stateOptimization_EnergyDistribution}). 


\begin{table}[H]
\centering
\caption{Potential thresholds for occurrence sampling ($T_{i}^{\text{phys}}$) and undersampling ($T_{i}^{\text{us}}$) determined during the parameter exploration (in kJ~mol$^{-1}$).}
\label{SItab:RingCycleOpenin_PotentialTresholds}
\begin{adjustbox}{max width=\textwidth}
\begin{tabular}{ l | c c |c c| }
 Ligand &\multicolumn{2}{c|}{Water} & \multicolumn{2}{c|}{Complex}\\ 
  & \multicolumn{1}{c}{$T^{\text{phys}}$}& \multicolumn{1}{c|}{$T^{\text{us}}$}&  \multicolumn{1}{c}{$T^{\text{phys}}$}& \multicolumn{1}{c|}{$T^{\text{us}}$} \\ 
 \hline
     L1  & -582.96 & -436.05 & -737.37 & -516.41\\ 
     L17 & -572.41 & -419.16 & -717.95 & -492.83\\
     L19 & -579.13 & -415.91 & -738.95 & -483.78\\
     L20 & -636.00 & -492.75 & -759.01 & -549.35\\
     L21 & -656.22 & -488.43 & -805.30 & -539.78\\
\end{tabular}
\end{adjustbox}
\end{table}


%Eoff:
The energy offsets $\vec{E}^R$ were estimated from a short RE-EDS simulation with the PEOE \cite{Sidler2016} scheme and are listed in Table \ref{tab:CHK1_set2_Eoff}.
For $s=1.0$, the energy offsets should ideally be equal to the free energy of the corresponding state (i.e. $\Delta E^R_{ji} = \Delta G_{ji}$) such that the partition function of the reference state is the sum of the partition functions of the end states \cite{Christ2008}. Therefore, the comparison between the relative estimated energy offsets in water and in complex ($\Delta \Delta E^R_{ji} = \Delta E^R_{ji,\text{complex}} - \Delta E^R_{ji,\text{water}}$) and the relative binding free energy $\Delta \Delta G^\text{bind}_{ji}$ can be used to (roughly) assess the quality of the estimated energy offsets. As shown in Figure \ref{fig:CHK1_set2_stateOptimization_EnergyDistribution},
the energy offsets estimated from the SSM simulations are in better agreement with the experimental relative binding free energies than those estimated from the 1SS simulations.
The relative energy offsets $\Delta \Delta E^R_{ji}$ are compared with the experimental relative binding free energies $\Delta \Delta G^\text{bind}_{ji}$ in Figure \ref{SIfig:Eoff_experiment_corr_RingOpening}. 
The RMSE between $\Delta \Delta E^R_{ji}$ obtained with RE-EDS 1SS and $\Delta \Delta G^\text{bind}_{ji}$ is $12.6$~kJ~mol$^{-1}$. Outliers are mainly related to L19.
With the RE-EDS SSM approach, the RMSE was reduced to $7.0$~kJ~mol$^{-1}$. No clear outliers were observed in this case. Thus, the use of the SSM approach is recommended for RE-EDS simulations.

\begin{table}[h]
\caption{Energy offsets $\vec{E^R}$ estimated from a short RE-EDS simulation using the PEOE \cite{Sidler2016} scheme. The errors indicate the standard deviation over the different replicas in undersampling. All energy offsets were calculated relative to ligand L1. The starting coordinates were selected following the 1SS or the SSM approach (see Theory and Methods sections).}
\label{tab:CHK1_set2_Eoff}
\centering
\begin{adjustbox}{max width=\textwidth}
\begin{tabular}{ l | r r | r r }
 Ligand & \multicolumn{2}{c|}{Water}&\multicolumn{2}{c}{Complex}  \\ 
  &RE-EDS 1SS &RE-EDS SSM &RE-EDS 1SS &RE-EDS SSM \\ 
  & [kJ~mol$^{-1}$]& [kJ~mol$^{-1}$]& [kJ~mol$^{-1}$]& [kJ~mol$^{-1}$]\\
 \hline
     L1 & $0.0$ & $0.0$ & $0.0$ & $0.0$ \\ 
     L17 & $11.07 \pm 7.61 $ & $17.81 \pm 0.69 $ & $20.03 \pm 5.04 $ & $18.19 \pm 3.43$ \\
     L19 & $-9.38 \pm 6.85 $ & $ -12.37 \pm 5.23 $ & $-2.09 \pm 1.56 $ & $ 2.4 \pm 1.56$ \\
     L20 & $-53.15 \pm 2.95 $ & $ -56.01 \pm 13.67 $ & $ -58.73 \pm 4.87 $ & $-52.2 \pm 2.6$\\
     L21 & $-76.75 \pm 5.79 $& $-69.15 \pm 3.74 $ & $ -77.29 \pm 3.12 $ & $ -77.9 \pm 3.4$\\
\end{tabular}
\end{adjustbox}
\end{table}

\begin{figure}[h]
\centering
     \includegraphics[width=\textwidth]{fig/results/ringOpening/paramExploration/single_state_energy_sampling.png}
    \caption{Comparison of the potential-energy distribution obtained from a standard MD simulation of a given end state (black) and from an EDS simulation with the given end state favoured (colored) from the first step of the RE-EDS workflow.}
     \label{fig:CHK1_set2_stateOptimization_EnergyDistribution}
\end{figure}


\begin{figure}[h]
\centering
  \includegraphics[width=0.8\textwidth]{fig/results/ringOpening/paramOptimization/RingClosure_system_Eoff_final_results.png}
\caption{Comparison of the relative energy offsets $\Delta \Delta E^R_{ji}$ in water and complex with the experimental relative binding free energies $\Delta \Delta G^\text{bind}_{ji}$. The energy offsets were estimated from RE-EDS simulations using the 1SS (green) or SSM (blue) approach to select the starting configurations of the replicas.} \label{SIfig:Eoff_experiment_corr_RingOpening}
\end{figure}

%---------------------------
\FloatBarrier
\clearpage 

\subsection{Parameter Optimization}
%S-Optimization
The optimization of the $s$-distribution was performed with the N-LRTO \cite{Sidler2017} algorithm, thereby minimizing the average round-trip time $\overline{\tau}$ in the replica graph. For the 1SS complex system, four optimization iterations were used. For the other systems, three iterations were used. 

%EoffRB
In the first iteration, the total number of observed round trips was very low or zero for all approaches. In the following iterations, this quantity increased, and the average round-trip time decreased for all simulations (Figure \ref{fig: CHK1_RingOpening_sOptimization}). The number of round trips was generally smaller in the complex than in water due to a more pronounced gap region \cite{Sidler2017}.
Already after the second iteration, the round-trip time was reduced in all approaches. The improvement of the $\overline{\tau}$ over the iterations can also be seen in Figure \ref{SIfig:CHK1_RingOpening_soptimization_efficiency}.

\begin{figure}[h!]
\centering
\includegraphics[width=\linewidth]{fig/results/ringOpening/paramOptimization/RingOpening_optimization_RTstat.png}
\caption{Average round-trip time as a function of the optimization steps $i$ ($\overline{\tau}_i$) on a logarithmic scale. The red line indicates the switch from $s$-optimization to energy offset rebalancing.}
\label{SIfig:CHK1_RingOpening_soptimization_efficiency}
\end{figure}

%% s-replica placements
As can be seen in the third row of Figure \ref{fig: CHK1_RingOpening_sOptimization}, the optimization algorithm increases the density of the replicas around $s = 0.041$, where the major gap region lies.

\begin{figure}[h]
\centering
\includegraphics[width=\textwidth]{fig/results/ringOpening/paramOptimization/S-optimization_ringOpening.png}
\caption{Optimization steps of the $s$-distribution with the N-LRTO \cite{Sidler2017} algorithm followed by the energy offset rebalancing scheme (start indicated by the red horizontal line). The measured quality criteria were the number of round trips (1. row), the average round-trip time $\overline{\tau}$ (2. row), the placement of the replicas in $s$-space (3. row), and the sampling fractions of maximally contributing states $f_{i}^{\text{mc}}$ (4. row). The light colored bars of $f_{i}^{\text{mc}}$ indicate $s$-optimization iterations, whereas the fully colored bars indicate energy offset rebalancing steps. }
\label{fig: CHK1_RingOpening_sOptimization}
\end{figure}


%% tau converge - Conclusion
The $s$-optimization was stopped after a sufficiently high number of round trips and low round-trip time was reached. 
This resulted in 20 replicas for the ligands in water after three $s$-optimization iterations.
For the protein-ligands complex, the fourth $s$-optimization iteration was chosen for the 1SS approach, and the third iteration for the SSM approach, resulting in 29 and 25 replicas, respectively. 
The average round-trip time after convergence was $\overline{\tau} = 0.4 \pm 0.2$~ns for all simulations.

After the $s$-optimization, the energy offset rebalancing scheme was applied to improve the state sampling. 

%RT analysis
During the rebalancing steps, no further replicas were added to the $s$-distribution. It is essential for the success of the rebalancing scheme that round trips occur. Therefore, the number of round trips and average round-trip time were monitored. In all systems, the number of round trips and $\overline{\tau}$ remained relatively stable over the four rebalancing steps. For the RE-EDS 1SS approach in water, the number of round trips slightly decreased but never dropped to zero.

%% states sampling
Across the optimization steps, also the sampling of the end states as maximally contributing states at $s=1.0$ was monitored.
During the $s$-optimization, some end states ``vanish'' and are no longer sampled as maximal contributing states. This leakage effect can occur when the initially estimated $E^{\text{R}}$ are not exactly optimal \cite{Sidler2016}. 
With energy offset rebalancing, the sampling of each end state can be recovered, and the sampling distribution approaches the ideal case.
%After the s-optimization the MAE($P^{\text{maxContrib}}$) was for all approaches approximately at $25\%$, with the exception of the 1SS water system, here it was $20\%$. 
After rebalancing, all end states showed a $f_i^{\text{mc}} > 0$ and the mean absolute deviation of the sampling distribution from ideal decreased from $20-25\%$ to approximately $7-12\%$ (Figure \ref{SIfig:CHK1_RingOpening_optimization_fractOptSampMAE}). 
 


\begin{figure}[h]
\centering
\includegraphics[width=\linewidth]{fig/results/ringOpening/paramOptimization/RingOpening_optimization_fractOptSampMAE.png}
\caption{Mean absolute deviation (MAE, in percentage) of the observed state sampling $f_i^{\text{mc}}$ from the ideal equal distribution $f_i^{\text{mc,ideal}}$ during the short optimization simulations. The red line indicates the switch from $s$-optimization to energy offset rebalancing.}
\label{SIfig:CHK1_RingOpening_optimization_fractOptSampMAE}
\end{figure}

%---------------------------
\FloatBarrier
\clearpage

\subsection{Free-Energy Calculation}
After successfully optimizing the RE-EDS parameters, the production runs were performed for $3.5$~ns. 


%%Sampling++
Both in water and in complex, the potential-energy distributions of the end states generally match well the corresponding distributions from the standard MD simulations of the single end states (Figure \ref{fig:RingOpening_sampling_comparison}). Only in the complex 1SS approach, a deviation can be seen for L17, with a slight shift to higher potential energies. This is due to insufficient sampling of L17 in this case (see below). 
%
The analysis of the maximally contributing end states at $s=1.0$ shows that in water all end states were sampled close to the ideal equal distribution (Figure \ref{SIfig:CHK1_RingOpening_soptimization_final_Sampling_s1}). 
\begin{figure}[H]
\centering
\includegraphics[width=\textwidth]{fig/results/ringOpening/FE/Reeds_RingOpening_production_sampling_s1.png}
\caption{Sampling of the end states in the final production run at replica $s=1.0$. Sampling was assessed by monitoring the maximally contributing end state (top panels) and by counting all end states a potential energy below $T_{i}^{phys}$ (see Table \ref{SItab:RingCycleOpenin_PotentialTresholds}) (bottom panels). Ideally, the sampling fraction as maximally contributing end state should be 1/$N$ (Eq. (8) in the main text) for all end states, indicated as a black horizontal line.}
\label{SIfig:CHK1_RingOpening_soptimization_final_Sampling_s1}
\end{figure}


In the simulation of the protein-ligands complex, there are still differences in sampling. Especially with the 1SS approach, L19 is generally sampled too much, while L17 is not sampled enough. The situation is improved with the SSM approach.
Comparing $f_i^{\text{occur}}$ and $f_i^{\text{mc}}$ in Figure \ref{SIfig:CHK1_RingOpening_soptimization_final_Sampling_s1} indicates that the end states in the CHK1 system are clearly separated (i.e. no phase-space overlap).

\begin{figure}[h]
    \centering
    \includegraphics[width=\columnwidth]{fig/results/ringOpening/FE/RingClosure_system_final_sampling.png}
    \caption{Comparison of the Boltzmann reweighted potential-energy distributions obtained from standard MD simulations of a given end state (black) and from the RE-EDS production runs of the 1SS (green) and SSM (turquoise, dashed) approaches.}
    \label{fig:RingOpening_sampling_comparison}
\end{figure}

%%Accuracy
From the replica at $s=1.0$, the free-energy differences were calculated using Eq.~(\ref{EQ: Free Energy calculation via reference state}) and the resulting $\Delta \Delta G^\text{bind}_{ji}$ were compared with the experimental results taken from Ref.~\cite{Huang2012}. The results are shown graphically in Figure \ref{fig:CHK1_set2_FreeEnergyCalculation} and numerically in Table \ref{tab: RE-EDS_FE_RingCycleOpening_ddF}. The individual free-energy differences are given in Table \ref{SItab: RE-EDS_FE_RingCycleOpening_dFs}.
The RMSE with RE-EDS 1SS is $4.4$~kJ~mol$^{-1}$ and the MAE is $3.9\pm2.8$~kJ~mol$^{-1}$. 

\begin{table}[H]
\caption{Free-energy differences in water and in complex calculated from the production run of 3.5~ns of length with the RE-EDS 1SS and RE-EDS SSM approaches.}
\label{SItab: RE-EDS_FE_RingCycleOpening_dFs}
\begin{center}
\begin{adjustbox}{max width=\textwidth}
\begin{tabular}{ c c |c c |c c}
  \multicolumn{2}{c|}{Ligand} & \multicolumn{2}{c|}{RE-EDS 1SS} &\multicolumn{2}{c}{RE-EDS SSM}\\ 
  J & I  & water [kJ~mol$^{-1}$] & complex [kJ~mol$^{-1}$]  & water [kJ~mol$^{-1}$] & complex [kJ~mol$^{-1}$] \\
  \hline
        L17 &         L1 &       11.9 $\pm$ 0.0&        17.0 $\pm$ 0.8&   12.4 $\pm$ 0.5&   9.4 $\pm$ 1.9\\
        L19 &         L1 &        2.7 $\pm$ 0.0&        5.7 $\pm$ 1.0&     3.1 $\pm$ 0.0&   8.0 $\pm$ 0.0\\
        L20 &         L1 &      -47.8 $\pm$ 0.0&      -47.6 $\pm$ 0.9& -47.7 $\pm$ 0.0&  -48.1 $\pm$ 0.0\\
        L21 &         L1 &      -61.7 $\pm$ 0.06&     -63.1 $\pm$ 0.8& -61.7 $\pm$ 0.0&  -64.8 $\pm$ 0.0\\
        L19 &         L17 &      -9.2 $\pm$ 0.0&      -11.3 $\pm$ 0.6&  -9.3 $\pm$ 0.5&   -1.4 $\pm$ 1.9\\
        L20 &         L17 &     -59.6 $\pm$ 0.0&      -64.5 $\pm$ 0.1& -60.1 $\pm$ 0.5&  -57.6 $\pm$ 1.9\\
        L21 &         L17 &     -73.6 $\pm$ 0.0&      -80.1 $\pm$ 0.1& -74.1 $\pm$ 0.5&  -74.3 $\pm$ 1.9\\
        L20 &         L19 &     -50.5 $\pm$ 0.0&      -53.2 $\pm$ 0.6& -50.7 $\pm$ 0.0&  -56.2 $\pm$ 0.0\\
        L21 &         L19 &     -64.4 $\pm$ 0.0&      -68.8 $\pm$ 0.6& -64.7 $\pm$ 0.0&  -72.9 $\pm$ 0.0\\
        L21 &         L20 &     -13.9 $\pm$ 0.0&      -15.5 $\pm$ 0.2& -14.0 $\pm$ 0.08& -16.7 $\pm$ 0.0 \\
\end{tabular}
\end{adjustbox}
\end{center}
\end{table}

The main deviations stem from ligand L17 in the RE-EDS 1SS approach, which can be explained by the insufficient sampling of L17 in the complex (see Figure \ref{fig:RingOpening_sampling_comparison} and Figure \ref{SIfig:CHK1_RingOpening_soptimization_final_Sampling_s1}).

The performance was substantially improved using the SSM approach with RE-EDS, giving an RMSE of $3.3$~kJ~mol$^{-1}$ and an MAE of $2.8 \pm 1.7$~kJ~mol$^{-1}$. 
Only two values (L21-L11) and (L21-L19) deviate more than $4.184$~kJ~mol$^{-1}$ (i.e. $1$~kcal~mol$^{-1}$) from experiment.
The Spearman correlation coefficient for RE-EDS 1SS is $r_{\text{Spearman}}=0.01$ and for RE-EDS SSM $r_{\text{Spearman}}=0.69$.

%%Performance:
Next, we assessed the convergence of the $\Delta G_{ji}$ values as a function of simulation time (Figure \ref{SIfig:CHK1_RingOpening_dF_convergence}).
\begin{figure}[h]
\centering
\includegraphics[width=\textwidth]{fig/results/ringOpening/FE/dF_RingOpening_Convergence.png}
\caption{Convergence analysis of the RE-EDS production runs (total 3.5~ns): The free-energy results are plotted as a function of the simulation time. The vertical lines indicate when a particular $\Delta G_{ji}$ value was found to be converged (deviation below 1~kJ~mol$^{-1}$).}
\label{SIfig:CHK1_RingOpening_dF_convergence}
\end{figure}

For the RE-EDS 1SS approach, all free-energy differences appeared converged after $2.5$~ns in water and after $2.7$~ns in the complex. For the RE-EDS SSM approach, convergence was observed after $2.5$~ns in water and after $2.9$~ns in the complex.

\begin{figure}[h]
    \centering
    \begin{subfigure}{0.85\columnwidth}
        \includegraphics[width=\textwidth]{fig/results/ringOpening/FE/RingClosure_system_final_results_4ns_comparison.png}
        \end{subfigure}
    \begin{subfigure}{0.85\columnwidth}
        \includegraphics[width=\textwidth]{fig/results/ringOpening/FE/ddG_bind_paper_comparison_reeds_only_4nsSimulation.png}
        \end{subfigure}
    \caption{Free-energy differences estimated from the production run of $3.5$~ns length. (Top): Comparison between the experimental and calculated $\Delta \Delta G^\text{bind}_{ji}$ using RE-EDS 1SS and RE-EDS SSM. The results were calculated with all possible pairwise transformations (forward and backward). (Bottom): Graphical representation of the $\Delta \Delta G^\text{bind}_{ji}$ results with structures, inspired by the one in Ref.~\cite{Wang2017}.}
    \label{fig:CHK1_set2_FreeEnergyCalculation}
\end{figure}

%Comparison results with Schroedinger & Jespers
By applying the RE-EDS methodology to the same system of five CHK1 inhibitors as studied by Wang \textit{et. al.} \cite{Wang2017} and later on also Jespers \textit{et al.} \cite{Jespers2019}, a direct comparison with FEP+ and QligFEP is possible (Table \ref{tab: RE-EDS_FE_RingCycleOpening_ddF}). Note that the quality metrics were calculated over all possible pairs of ligands and in both directions, not only those directly calculated by FEP+ and QligFEP.
For FEP+, we obtained an RMSE of $2.4$~kJ~mol$^{-1}$ and an MAE of $1.8 \pm 1.2$~kJ~mol$^{-1}$ with a Spearman correlation coefficient of $r_{\text{Spearman}}=0.67$.
Including cycle closure correction (CC) \cite{Wang2017} reduced the RMSE to $2.1$~kJ~mol$^{-1}$ and the MAE to $1.9 \pm 1.0$~kJ~mol$^{-1}$. The Spearman correlation coefficient increased to $r_{\text{Spearman}}=0.73$.
Jespers \textit{et al.} \cite{Jespers2019} reported free-energy differences with QligFEP as an average over ten independent replicas, each with significantly less simulation time per $\lambda$-window than in Ref.~\cite{Wang2017}. For QligFEP, an RMSE of $2.3$~kJ~mol$^{-1}$, an MAE of $2.0 \pm 1.2$~kJ~mol$^{-1}$, and a Spearman coefficient of $r_{\text{Spearman}}=0.61$ was obtained.


Overall, the performance of RE-EDS SSM is comparable with the pairwise methods. The results with FEP+ CC and QligFEP showed a slightly higher accuracy compared to experiment, likely due to the different force fields used. The Spearman correlation coefficient is comparable with the other methods for the RE-EDS SSM approach.

\begin{table}[h]
\caption{Relative binding free energies $\Delta \Delta G^\text{bind}_{ji}$ from experiment and calculated with the RE-EDS 1SS and RE-EDS SSM approaches. For comparison, the results for FEP+ with and without cycle closure (CC) correction taken from Ref.~\cite{Wang2017} and the results for QligFEP taken from Ref.~\cite{Jespers2019} are listed. The free-energy differences of directly simulated paths were used to infer not directly simulated free-energy differences (marked in bold). If multiple indirect paths were possible, their average was used. The errors for QligFEP were determined in Ref.~\cite{Jespers2019} by calculating the standard deviation over ten replicas. For FEP+, the error of the results was taken from the used BAR \cite{Bennett1976} method and the FEP+ CC errors were obtained from the cycle closure analysis. For the RE-EDS approaches, the reported error is based on the statistical uncertainties of the $\Delta G_{ji}^{env}$ values estimated using Gaussian error approximation \cite{Christ2008}. The uncertainty estimate of the RMSE was obtained by a 100-fold bootstrapping approach. }
\begin{center}
\footnotesize
\begin{adjustbox}{max width=\textwidth}
\begin{tabular}{ c c |c |c|c|c|c|c}
  \multicolumn{2}{c|}{Ligands} & \multicolumn{1}{c|}{Exp. \cite{Huang2012}} &\multicolumn{1}{c|}{FEP+ \cite{Wang2017}}&\multicolumn{1}{c|}{FEP+ CC \cite{Wang2017}}&\multicolumn{1}{c|}{QligFEP \cite{Jespers2019}}&\multicolumn{1}{c|}{RE-EDS 1SS}&\multicolumn{1}{c}{RE-EDS SSM}\\ 
    $i$ & $j$  & [kJ~mol$^{-1}$]  & [kJ~mol$^{-1}$] & [kJ~mol$^{-1}$] & [kJ~mol$^{-1}$] & [kJ~mol$^{-1}$] & [kJ~mol$^{-1}$]  \\
  \hline
        L17 &  L1 &   0.1 & -3.6 $\pm$ 0.4          & -2.9 $\pm$ 1.0         & -1.6 $\pm$ 1.7                                     &    5.1 $\pm$ 0.8 &  3.0 $\pm$ 2.0 \\
        L19 &  L1 &  -4.8 & -3.9 $\pm$ 0.3          & -4.0 $\pm$ 0.6         & -1.7 $\pm$ 2.0                                     &    3.0 $\pm$ 1.0 & -5.0 $\pm$ 0.1\\
        L20 &  L1 &  -2.0 & -2.5 $\pm$ 0.1          & -3.1 $\pm$ 1.0         & -1.3 $\pm$ 1.3                                     &    0.2 $\pm$ 0.9 &  0.5 $\pm$ 0.1\\
        L21 &  L1 &  -2.3 &\textbf{-3.4} $\pm$ \textbf{0.7}  &\textbf{-3.2} $\pm$ \textbf{1.3} & \textbf{-0.1} $\pm$ \textbf{3.5} &   -1.4 $\pm$ 0.8 &  3.2 $\pm$ 0.1\\
        L19 &  L17 & -4.9 & -1.4 $\pm$ 0.3          & -1.1 $\pm$ 1.0         & \textbf{0.1} $\pm$ \textbf{2.6}                    &   -2.1 $\pm$ 0.6 & -7.9 $\pm$ 1.9\\
        L20 &  L17 & -2.1 &  0.3 $\pm$ 0.4          & -0.1 $\pm$ 0.8         & -1.3 $\pm$ 2.3                                     &   -4.9 $\pm$ 0.1 & -2.5 $\pm$ 1.9\\
        L21 &  L17 & -2.4 & -1.1 $\pm$ 0.4          & -0.9 $\pm$ 0.9         &\textbf{0.7} $\pm$ \textbf{2.6}                     &  -6.5 $\pm$ 0.1 &  0.2 $\pm$ 1.9\\
        L20 &  L19 & 2.8  &\textbf{0.8} $\pm$ \textbf{0.6}   & \textbf{0.1} $\pm$ \textbf{1.3} & \textbf{-0.4} $\pm$ \textbf{3.7} &  -2.7 $\pm$ 0.6 &  5.4 $\pm$ 0.1\\
        L21 &  L19 & 2.5  & -0.1 $\pm$ 0.6         &  0.6 $\pm$ 0.1         &  0.6 $\pm$ 4.9                                      &  -4.4 $\pm$ 0.6 &  8.2 $\pm$ 0.1\\
        L21 &  L20 & -0.3 & -0.3 $\pm$ 0.8         & -0.6 $\pm$ 0.8         &  0.6 $\pm$ 1.1                                    &    -1.6 $\pm$ 0.1 &   -2.7 $\pm$ 0.1\\ 
    \hline
        \multicolumn{2}{c|}{RMSE} &                    & 2.4  $\pm$ 0.3           & 2.1  $\pm$ 0.2          &  2.3  $\pm$ 0.38      & 4.8 $\pm$ 0.5         & 3.3  $\pm$ 0.3 \\
        \multicolumn{2}{c|}{MAE} &                     & 1.8 $\pm$ 1.2 & 1.9 $\pm$ 1.0 & 2.0 $\pm$ 1.2 & 3.9 $\pm$2.8 & 2.8 $\pm$ 1.7 \\
        \multicolumn{2}{c|}{$r_{\text{Spearman}}$} & & 0.67           & 0.73          & 0.61          & -0.01           & 0.69 \\
        \multicolumn{2}{c|}{$t_{simulation} [ns]$} & & 640          &  640         &  51        & 171.5         & 157.5  \\
\end{tabular}
\end{adjustbox}
\end{center}
\label{tab: RE-EDS_FE_RingCycleOpening_ddF}
\end{table}

In terms of computational cost, the RE-EDS approach (with $3.5$~ns per replica) resulted in about a quarter of the total simulation time (in ns) than reported for the FEP+ calculations in Ref.~\cite{Wang2017} (Table \ref{tab: RE-EDS_FE_RingCycleOpening_ddF}). However the QligFEP approach is the approach with the lowest simulation time consumption. A major advantage of the simultaneous simulation of multiple ligands in a single RE-EDS simulation is that all $N(N-1)/2$ transformations are sampled directly, leading to low statistical errors and removing the need for a state graph. This advantage increases with increasing number of ligands. The current workflow of RE-EDS uses a relatively large amount of simulation time for parameter optimization. Future work will focus on further optimization of the workflow to reduce the pre-processing time. 

From the calculated relative binding free energies, $\Delta G_{i}^{\text{bind}}$ can be obtained by using one experimental value as anchor point. This allows us to generate a ranking of the five ligands. To avoid any bias from the selected experimental anchor point, all possibilities were calculated and the resulting values averaged (Table \ref{tab:RE-EDS_FE_RingCycleOpening_absoluteShiftDF}). While the RMSE is generally low for all approaches ($<$ 1 kcal mol$^{-1}$ = 4.184 kJ mol$^{-1}$), the ranking of the ligands as measured by $r_{\text{Spearman}}$ is not very good. 
%A strong correlation with experiment is of interest in drug design approaches, as the ranking of ligands in virtual screening is important to suggest the most promising drug candidates to be synthesized.
This observation is not uncommon for ligand series with small differences in binding free energy \cite{Wang2015,Schindler2020}.
Note that the uncertainties of the individual values have increased compared to the relative binding free energies due to the anchoring and averaging procedure.

\begin{table}[h]
\caption{Absolute binding free energies $\Delta G_{i}^{\text{bind}}$ and ranking of the ligands derived from the relative binding free energies. The values were calculated from the relative binding free energies using an experimental binding free energy as anchor point, and then averaged over the five possibilities. The errors are standard deviations over the possible outcomes. For comparison, the results for FEP+ with and without cycle closure (CC) correction taken from Ref.~\cite{Wang2017} and the results for QligFEP taken from Ref.~\cite{Jespers2019} are shown (calculated with the same procedure). The uncertainty estimate of the RMSE was obtained by a 100-fold bootstrapping approach.}
\begin{center}
\footnotesize
\begin{adjustbox}{max width=\textwidth}
\begin{tabular}{ c |c |c|c|c|c|c}
  Ligands & \multicolumn{1}{c|}{Exp. \cite{Huang2012}} &\multicolumn{1}{c|}{FEP+ \cite{Wang2017}}&\multicolumn{1}{c|}{FEP+ CC \cite{Wang2017}}&\multicolumn{1}{c|}{QligFEP \cite{Jespers2019}}&\multicolumn{1}{c|}{RE-EDS 1SS}&\multicolumn{1}{c}{RE-EDS SSM}\\ 
    Molecule & [kJ~mol$^{-1}$]  & [kJ~mol$^{-1}$] & [kJ~mol$^{-1}$] & [kJ~mol$^{-1}$] & [kJ~mol$^{-1}$] & [kJ~mol$^{-1}$]  \\
  \hline
        L1 &   -40.7 & -41.7 $\pm$ 1.7         & -41.7 $\pm$ 0.9         & -38.5 $\pm$ 1.5 &   -40.0 $\pm$ 3.4 &    -38.0 $\pm$ 2.0 \\
        L17 &  -40.8 &  -38.0 $\pm$ 1.0         & -38.2 $\pm$ 1.1         & -38.6 $\pm$ 1.3 &    -33.7 $\pm$ 1.3 & -41.7 $\pm$ 2.3 \\
        L19 &  -35.9  & -38.1 $\pm$ 0.9         & -38.3 $\pm$ 1.8         & -38.3 $\pm$ 1.0 &   -37.6 $\pm$ 3.3 &  -33.0 $\pm$ 2.0 \\
        L20 &  -38.6 & -38.6  $\pm$ 1.6         & -38.3 $\pm$ 1.4         & -39.2 $\pm$ 1.7 &    -40.4 $\pm$ 3.3 & -39.1 $\pm$ 2.3 \\
        L21 &  -38.4 & -37.7  $\pm$ 1.4         & -37.8 $\pm$ 1.3         & -39.4 $\pm$ 1.9 &    -42.4 $\pm$ 2.9 &    -42.5 $\pm$ 1.4 \\
    \hline
        RMSE &                    & 1.7  $\pm$  0.4        & 1.7   $\pm$ 0.4        & 1.7 $\pm$ 0.4          & 3.8 $\pm$ 1.3         & 2.6 $\pm$ 0.6 \\
        MAE &                     & 1.3 $\pm$ 1.0  & 1.4 $\pm$ 0.9 & 1.4 $\pm$ 0.9 & 3.0 $\pm$ 2.3 & 2.2 $\pm$ 1.6 \\
        $r_{\text{Spearman}}$ &  & 0.20           & 0.10          & -0.21         &  -0.40           & 0.30 \\
\end{tabular}
\end{adjustbox}
\end{center}
\label{tab:RE-EDS_FE_RingCycleOpening_absoluteShiftDF}
\end{table}

\FloatBarrier
\clearpage

\clearpage
\newpage

%================================================================================
\section{Conclusion}
%================================================================================

This study reports the recent developments for the multistate free-energy method RE-EDS, which omits the definition of alchemical transition paths. The automatic workflow for RE-EDS was improved in robustness, and was applied to estimate the relative binding free energies of five CHK1 inhibitors containing typical core-hopping transformations. This system was investigated previously with FEP+ and QligFEP, allowing for a direct comparison of RE-EDS with state-of-the-art pairwise free-energy methods.
Using different starting configurations representing all end states (SSM approach) in the parameter optimization of the RE-EDS workflow improved the sampling, convergence, and the accuracy of the resulting free-energy differences. The performance of RE-EDS SSM was found to be comparable with FEP+ and QligFEP, and shows that RE-EDS with a ``dual topology" approach can be readily applied to challenging ligand transformations like ring size change, ring opening/closing, and ring extension.

In terms of computational efficiency, the total production run time with RE-EDS ($3.5$~ns per replica) was about a quarter of that reported for FEP+ with this system. As multiple ligands are simulated simultaneously in a single RE-EDS simulation, this sampling enhancement will increase with increasing number of ligands. 
However, the pre-processing phase in the RE-EDS workflow currently uses a relatively large amount of simulation time. Making these steps more efficient will be addressed in future work. In addition, further automatization of the dual topology approach with distance restraints is ongoing.


\clearpage
\pagebreak

%\bibliography{5_chapter_3/ref/ref.bib}


\end{document}
