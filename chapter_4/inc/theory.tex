%------------------------------------------------------------
\subsection{Alchemical Free-Energy Calculations}
%------------------------------------------------------------



The goal of path-dependent alchemical free-energy calculations is
to evaluate the free-energy difference $\Delta G$
between two states $A$ and $B$ of a molecular system,
by introducing a coupling scheme relying on a parameter $\lam$,
and sampling along the so-defined $\lam$-path.
%
The two states have the same number $3N$ of degrees of 
freedom, but distinct Hamiltonian functions $\ham_A(\xv)$ and $\ham_B(\xv)$,
respectively, where $\xv = (\rv,\pv)$ is the $6N$-dimensional
phase-space vector representative of a microscopic system configuration, $\rv$ 
and $\pv$ being the corresponding coordinate and momentum vectors.
%
The coupling parameter is introduced into a hybrid Hamiltonian 
$\ham(\xv;\lam)$ 
satisfying the boundary conditions
$\ham(\xv;0)=\ham_A(\xv)$ and $\ham(\xv;1)=\ham_B(\xv)$,
where the semi-colon indicates a parametric dependence.
%
%
%Examples of methods involving the coupling-parameter approach are 
%multi-configuration\cite{ST91.1} thermodynamic integration\cite{KI33.1,KI34.2,KI35.1} (MCTI or, simply, TI)
%and multi-configuration\cite{JO08.3} free-energy perturbation\cite{ZW54.1,MA95.2,LI96.1,MA99.10,SC99.3,RA12.6,RA17.5,BO17.2} (MC-FEP or, simply, FEP)
%along with their HRE\cite{SU99.1,FU02.2,ZH16.2}/HRP\cite{IT13.1,IT13.2,YA17.2} variants, as well as $\lam$-dynamics\cite{KO96.1,DA01.7,GU03.1,KN09.1,KN11.2,DO11.2,AR15.2,HA17.1} ($\lam$D),
%along with its coordinate-transformed or/and $\lam$-biased variants\cite{GU98.1,GU98.2,SO01.1,WU11.1,KN11.2,DO11.2,KN11.1,ZH12.3,DO13.1,BI14.1,BI14.2,BI15.1,BI15.2} ({\em e.g.} the $\lam$-LEUS scheme\cite{BI14.1,BI14.2,BI15.1,BI15.2}).
%
\revphildel{[-- Redundant paragraph removed --]}
%



Since the proposed CBTI scheme encompasses features of both $\lam$D and TI,
these two approaches are summarized briefly in the \radd{next two subsections.
The following three subsections then describe in turn
the basis of the CBTI scheme, 
its free-energy estimator 
and the application of a biasing potential.}



%------------------------------------------------------------
\subsection{$\lam$-Dynamics ($\lam$D)}
%------------------------------------------------------------


In the $\lam$D scheme\cite{KO96.1,DA01.7,GU03.1,KN09.1,KN11.2,DO11.2,AR15.2,HA17.1}, 
the coupling parameter $\lam$ is assigned a mass
$m_\lam$ and a momentum $p_\lam$, and considered to be an additional
pseudo-conformational degree of freedom of the system.
%
The Hamiltonian of the extended system is defined as
%
\beq{lam_dyn_ext_ham}
  \ham^\star(\xv,\lam) = \ham(\xv;\lam) + \frac{p_\lam^2}{2m_\lam} \ ,
\eeq
%
where the star refers to an extended system in which $\lambda$ 
is now a variable and no longer a parameter 
(thus the replacement of the semi-colon by a comma).
%
This leads to the additional equation of motion
%
\beq{lam_dyn_eq_mot}
  \ddot{\lam} = \frac {\dot{p}_{\lam}} {m_{\lam}} 
              = -\frac {1} {m_{\lam}} 
                 \frac {\partial\ham(\xv;\lam)} {\partial\lam}
\eeq
%
for propagating the $\lam$-variable, \radd{where a dot over a variable indicates its time derivative}.
%


The free-energy difference $\Delta G$
between the two physical end-states can then in principle be
calculated based on a single thermostated MD simulation 
of the extended system, as
%
\beq{lam_dyn_dg}
  \Delta G = 
      - \frac{1}{\beta}\ln\frac{\left\langle \delta(\lam - 1)\right\rangle^\star}
      {\left\langle \delta(\lam)\right\rangle^\star} \ ,
\end{equation}
%
where 
$\beta=(k_BT)^{-1}$, 
$k_B$ being the Boltzmann constant and $T$ the absolute temperature,
$\delta$ is the Dirac delta function, 
and $\langle\cdot\cdot\cdot\rangle^\star$ denotes ensemble averaging 
for the extended system (\ie{} over the joint trajectories of $\xv$ and $\lam$).
%
\revphil{In practice, the $\delta$-functions in \refeq{lam_dyn_dg} must be replaced
      by two finite end-state bins ($\lambda$-cutoff), sufficiently large
      for proper statistics but also sufficiently small for avoiding 
      distortions due to averaging at the end-states\cite{KN11.1}.}
%
\revphildel{[-- Redundant paragraph removed --]}
%
%There are, however, a number of important practical issues 
%to be addressed if this scheme is to be accurate and efficient\cite{BI14.1}:
%%
%($i$) the variations of $\lam$ must be restricted to the range $[0,1]$ between the two physical end-states\cite{KO96.1};
%($ii$) the mass and thermostat-coupling scheme of the $\lam$-variable must be chosen appropriately
%      to ensure a correct temperature for this variable\cite{BI15.1};
%($iii$) the sampling along $\lam$ must generally be adjusted
%      to ensure adequate sampling of the A and B states
%      with a sufficient number of interconversion transitions\cite{BI14.1};
%($iv$) the $\delta$-functions in \refeq{lam_dyn_dg} must be replaced
%      by two finite end-state bins ($\lambda$-cutoff), sufficiently large
%      for proper statistics but also sufficiently small for avoiding 
%      averaging distortions at the end-states\cite{KN11.1}. %
%%
%Different implementation variants of $\lambda$D address these
%issues in a variety of ways, including the use of
%coordinate transformations\cite{KN11.2,DO11.2,WU11.1,KN11.1,DO13.1,ZH12.3,BI14.1,BI15.2},
%memory-based biasing potentials,\cite{GU98.1,GU98.2,SO01.1,WU11.1,BI14.1,BI15.2}
%separate thermostat coupling of the $\lam$-variable\cite{BI14.1,BI15.1,BI15.2},
%and 
%alternative free-energy estimators\cite{LU04.3,SH05.6,KA05.1,KA06.6,FA09.4,KA12.5,TA12.1,DI17.5,ZH17.6}.



%------------------------------------------------------------
\subsection{Thermodynamic Integration (TI)}
%------------------------------------------------------------


In the original TI scheme\cite{KI33.1,KI34.2,KI35.1}, a set of 
$K$ replicas of the system are simulated in parallel at fixed predefined 
$\lam$-values in the range $[0,1]$.
Since the replicas are entirely decoupled 
from each other, the simulations
can be performed serially as well. However, TI extensions 
including the HRE scheme\cite{SU99.1,FU02.2,ZH16.2} and the HRP scheme\cite{IT13.1,IT13.2,YA17.2}
introduce a coupling in the form of $\lam$-value exchanges,
in which case the simulations must really be carried out in parallel.
The same will apply to the proposed CBTI scheme, where the coupling involves a synchronization of
the dynamical $\lam$-variations. 


Considering all replicas $k=0 \dots K-1$ as the members of a replica system,
one may note the corresponding $6K\mkern-3mu\times\mkern-3mu N$-dimensional phase-space vector
as $\Xv=\{\xv_k\}$ and the corresponding $K$-dimensional vector containing the fixed $\lam$-values as 
$\lamv=\{\lam_k\}$.
%
In plain TI, the Hamiltonian of the replica system is defined as
%
\beq{ti_rep_ham}
  \ham^\dagger(\Xv;\lamv) = \sum_{k=0}^{K-1} \ham(\xv_k;\lam_k) \ ,
\eeq
%
where the dagger refers to a replica system,
and $\lamv$ is here a parameter vector (thus the semi-colon).
Because the Hamiltonian of \refeq{ti_rep_ham} involves no coupling 
term  between the replicas, the dynamics of a replica $k$
is independent from that of the other replicas and
solely depends on $\lam_k$.


The free energy difference $\Delta G$
between the two states can then be
calculated based on a single thermostated MD simulation 
of the replica system, as
%
\begin{multline}
  \label{eq:ti_formula}
  \Delta G = 
  \int\limits_0^1 \mathrm{d}\lam' \left\langle \frac{\partial{\ham(\xv;\lam)}}{\partial \lam} \right\rangle_{\lam'} 
    \approx
    \sum_{k=0}^{K-1}  w_k \left\langle \frac{\partial{\ham^\dagger(\Xv;\lamv)}}{\partial \lam_k} \right\rangle^\dagger\\ 
    = 
    \sum_{k=0}^{K-1}  w_k \left\langle \frac{\partial{\ham(\xv_k;\lam_k)}}{\partial \lam_k} \right\rangle^\dagger \ ,
\end{multline}
%
where 
the $w_k$ are quadrature weights for the numerical integration\cite{JO10.2,BR11.5,BR11.6},
$\langle\cdot\cdot\cdot\rangle_{\lam}$ denotes ensemble averaging 
for a single system \radd{(\ie{} over $\xv$) at the given $\lam$ value},
and
$\langle\cdot\cdot\cdot\rangle^\dagger$ denotes ensemble averaging 
for the replica system (\ie{} over $\Xv$).


\revphil{In the above form, TI has long been the workhorse of alchemical free-energy calculations.
%
The method is extremely robust in the sense 
that the accuracy of the calculated $\Delta G$
%free-energy change 
can always be systematically
improved (more $\lambda$-points, longer equilibration or/and sampling times).
However, it is not necessarily the most 
%practical and 
efficient 
method to determine $\Delta G$ up to a certain accuracy,
due to possible sub-optimalities in
the coupling scheme\cite{CR86.1,BL04.2,PH11.1,PH12.1,NA14.1,NA15.3},
the protocol design\cite{ME93.3,HU16.7,ME17.1,SU17.3,DA18.1}
the free-energy estimator\cite{LU04.3,SH05.6,SH08.7,FA09.4,TA12.1,DE16.9,DI17.5,ZH17.6},
and
the orthogonal sampling\cite{WO03.1,WO03.2,KH10.1,KH11.2,BI15.1}.
}
\revphildel{[-- Redundant paragraph removed --]}




%In the above form, TI has long been the workhorse 
%of alchemical free-energy calculations.
%%
%In practice, in addition to the choice of a coupling scheme\cite{CR86.1,BL04.2,PH11.1,PH12.1,NA14.1,NA15.3},
%a TI calculation requires the definition of a protocol including:
%($i$) the number $K$ and distribution (spacing) of the $\lambda$-points considered;
%($ii$) the initial configurations, equilibration times and sampling times
%    selected for each $\lambda$-point;
%($iii$) the choice of a quadrature scheme for the numerical integration.
%%
%The accuracy of the calculation will depend on the selection of these 
%parameters, the optimization of which may represent a non-trivial and 
%time-consuming task\cite{ME93.3,HU16.7,ME17.1,SU17.3,DA18.1}.
%%
%Thus, even though TI is extremely robust in the sense 
%that the accuracy of the calculated free-energy change can always be systematically
%improved (more $\lambda$-points, longer equilibration or/and sampling times),
%it is not necessarily the most efficient method to determine 
%$\Delta G$ up to a certain accuracy.
%
%Many improvements have been proposed for the TI scheme
%over the years.
%%
%Most prominently, the inclusion of $\lam$-exchanges in 
%%HRE\cite{SU99.1,FU02.2,ZH16.2}/HRP\cite{IT13.1,IT13.2,YA17.2} 
%\radd{the HRE scheme\cite{SU99.1,FU02.2,ZH16.2} and the HRP scheme\cite{IT13.1,IT13.2,YA17.2}}
%permits to 
%circumvent orthogonal barriers\cite{WO03.1,WO03.2,KH10.1,KH11.2}, a feature also shared by $\lambda$D schemes\cite{BI15.1}, and
%the use of alternative free-energy estimators like the
%EXTI\cite{DE16.9} scheme or the MBAR\cite{SH08.7} scheme permits to improve the statistical efficiency.



%------------------------------------------------------------
\subsection{Conveyor Belt Thermodynamic Integration (CBTI)}
%------------------------------------------------------------


The proposed CBTI scheme encompasses features of both $\lam$D and TI.
%
Similarly to TI, it is based on the simulation of a replica system involving 
$K$ copies of the molecular system of interest, where $K$ is taken to be even.
%
And similarly to $\lam$D, the individual replicas are extended 
systems, for which the associated $\lam_k$-variable is allowed to evolve
along the simulation.
%
However, the evolutions of these $\lam_k$-variables are not independent. 
They are coupled to each 
other by means of a sequence of hard constraints,
 so that they follow the course of a conveyor belt (CB).
Thus, they are entirely determined by a single dynamical
variable $\Lamb$, following 
the scenario depicted in \reffig{scheme} and discussed
in the Introduction section.



The variable $\Lamb$ is a continuous real variable
representing the overall advance of the CB, successive multiples
of $2\pi$ corresponding to as many full rotations.
%
%
Given $\Lamb$ and $K$, the $\lam$-value $\lam_k$ associated with a system $k$ on the CB
is obtained as
%
\beq{cb_lam_of_big_lam}
  \lam_k(\Lamb) = \zeta\left( \Lamb + k \Delta\Lamb  \right) \ ,
\eeq
%
\radd{with
\beq{delta_lamb}
\Delta\Lamb = 2\pi K^{-1} .
\eeq
}
\radd{Here, the function $\zeta$} is a continuous and periodic zig-zag function of period $2\pi$ and image range $[0,1]$,
defined over the reference period $[0,2\pi)$ as
%
\beq{zigzag_fct}
  \zeta(\theta) = \left\{
                    \begin{array}{ll}
                       \pi^{-1}\theta & \mathrm{if}\ \theta < \pi \\
                       2-\pi^{-1}\theta & \mathrm{if}\ \theta \geq \pi \\
                    \end{array} 
                  \right. \ \ \mathrm{for}\ \theta\in[0,2\pi) \ ,
\eeq
%
where the $[\cdot,\cdot)$ indicates an interval that is open to the right side, {\em e.g.} $[0,2\pi)$ includes $0$ but excludes $2\pi$.
%
\radd{An advance of the CB by $\Delta\Lamb$}
%
corresponds to a cyclic permutation of the $K$ replicas, each system moving by 
one position forward along the CB, \ie
$\lam_k(\Lamb+\Delta\Lamb) = \lam_{k+1}(\Lamb)$
for $k<K-1$, along with
$\lam_{K-1}(\Lamb+\Delta\Lamb) = \lam_0(\Lamb)$.
%
For this reason, the increment $\Delta \Lamb$ will be further referred to 
as one shift of the CB.
%
The system $k=0$ can be viewed as a reference system,
as $\lambda_0=\pi^{-1}\Lambda$ for $0\leq\Lambda<\pi$
and $\lambda_0=2-\pi^{-1}\Lambda$ for $\pi\leq\Lambda<2\pi$.
%
Since $K$ is chosen to be even, an increase of $\Lambda$
always corresponds to an increase of $\lambda_k$
for half of the systems (forward-moving side of the CB)
and a decrease of $\lam_k$ for the other half of the systems
(backward-moving side of the CB). This choice also implies that 
$\Lamb$-values which are integer multiples of the
CB shift $\Delta \Lamb$ 
correspond to situations where there is 
one system in state $A$ and one system in state $B$.
%
Since an advance of the CB variable by $2\pi$
leaves the replica system invariant, 
the variable $\Lamb$ will commonly be refolded into the 
reference period $[0,2\pi)$ when illustrating the
results of the CBTI method.

The Hamiltonian of the extended replica system is defined as
%
\beq{cb_ext_rep_ham}
  \ham^{\dagger\star}(\Xv,\Lamb) = \ham^\dagger(\Xv;\lamv)  +  \frac{p_\Lamb^2}{2m_\Lamb} 
  \quad \text{with} \quad \lamv=\lamv(\Lamb),
\eeq
%
%
where $\ham^{\dagger}$ is defined as in TI by \refeq{ti_rep_ham}
and $\lamv(\Lamb)$ by \refeq{cb_lam_of_big_lam}.
%
In analogy with \refeq{lam_dyn_eq_mot}, the resulting equation of motion for $\Lamb$ reads
%
\beq{cb_big_lam_eq_mot}
  \ddot{\Lamb} = \frac {\dot{p}_{\Lamb}} {m_{\Lamb}} 
              = -\frac {1} {m_{\Lamb}}  \frac {\partial\ham^\dagger(\Xv;\lamv)} {\partial \Lamb} 
\eeq
%
where
%
\begin{align}
  \label{eq:cb_big_lam_eq_mot_der}
   \frac{\partial\ham^\dagger(\Xv;\lamv)}{\partial\Lamb}
        &= \sum_{k=0}^{K-1} \frac {\partial\ham(\xv_k;\lam_k)} {\partial \lam_k} \frac{\mathrm{d} \lam_k}{\mathrm{d} \Lamb}\nonumber \\
        &= \sum_{k=0}^{K-1} \frac {\partial\ham(\xv_k;\lam_k)} {\partial \lam_k} \zeta'\left( \Lamb + 2\pi K^{-1} k \right) \ .
\end{align}
%
Here, the function $\zeta'$ is the derivative of the zig-zag function of \refeq{zigzag_fct}, 
given over the reference period $[0,2\pi)$ by
%
\beq{zigzag_der}
  \zeta'(\theta) = \left\{
                    \begin{array}{ll}
                       \pi^{-1} & \mathrm{if}\ \theta < \pi \\
                       -\pi^{-1} & \mathrm{if}\ \theta \geq \pi \\
                    \end{array} 
                  \right. \ \ \mathrm{for}\ \theta\in[0,2\pi) \ .
\eeq
%
Formally, the derivative is not defined when $\theta$
is an integer multiple of $\pi$, \ie{} for a system that is exactly 
in one of the physical end-states $A$ or $B$. In this case, the value of $\zeta'$ has been arbitrarily set to 
$\pi^{-1}$ for even multiples and $-\pi^{-1}$ for odd multiples.
This has a negligible impact in practice, as it only concerns a series
of infinitesimal points over the entire $\Lamb$-range, 
\ie{} infinitesimally few configurations along a CBTI simulation.
%
\revphil{For example, when using double-precision floating-point arithmetics
(including denormalized numbers),
their probability of occurrence is on the order of 10$^{-324}K$ 
({\em i.e.} a single expected occurrence over a simulation lasting about 10$^{292}K^{-1}$
times the age of the universe with a 2 fs timestep).
}
%
Neither does going over the discontinuity within a timestep represent a source of non-conservativeness. The concerned replica
will merely bounce back the corresponding physical end-state with a reversion of its velocity, akin to a particle reflected elastically
by a hard wall (delta-function force).
%
If desired, these \revphil{exceptional points could be handled
more formally by altering} the definition of $\zeta$,
\revphil{{\em e.g.} by smoothing its tips in a narrow range around $0$ and $\pi$}.


The $\lam_k$-dynamics of the individual systems is entirely
specified by \refeq{cb_big_lam_eq_mot} to propagate $\Lamb$
along with  \refeq{cb_lam_of_big_lam} to calculate the $\lam_k$-values from the current $\Lamb$.
Alternatively, one may write an equation of motion for the $\lambda_k$-variables of the 
individual replicas by combining the two equations (along with \refeq{cb_big_lam_eq_mot_der}) as
%
\begin{align}
  \label{eq:cb_lam_eq_mot}
  \ddot{\lam}_k &= \ddot{\Lamb} \frac{\mathrm{d} \lam_k}{\mathrm{d} \Lamb} + \dot{\Lamb}^2 \frac{\mathrm{d^2} \lam_k}{\mathrm{d} \Lamb^2} \nonumber\\
                &= -\frac {1} {m_{\Lamb}} \left( 
                       \sum_{l=0}^{K-1} \frac {\partial\ham(\xv_{l};\lam_{l})} {\partial \lam_{l}} \zeta'\left( \Lamb + 2\pi K^{-1} l \right) 
                       \right) \nonumber  \\
                   & \qquad \zeta'\left( \Lamb + 2\pi K^{-1} k \right) \ .
\end{align}
%
%
Note that the term in $\dot{\Lamb}^2$ vanishes since the second derivative $\zeta''$ of $\zeta$
is zero (except at the \radd{exceptional} singular points). However, it should not be overlooked 
if one decides to use a different function $\zeta$.
%
Introducing the vector $\Dv$ and the symmetric $\Lamb$-dependent matrix $\Cmat$ defined 
by their components as
%
\begin{multline}
  \label{eq:cb_def}
 D_k = \frac{\partial\ham(\xv_{k};\lam_{k})}{\partial \lam_{k}}
\ \ \ \mathrm{and}\\
C_{kl}(\Lambda) = \pi^2 \zeta'\left( \Lamb + 2\pi K^{-1} k \right) \zeta'\left( \Lamb + 2\pi K^{-1} l \right) \ ,
\end{multline}
%
\refeq{cb_lam_eq_mot} can be rewritten in an elegant matrix form as
%
\beq{cb_lam_eq_mot_mat}
  \ddot{\lamv} = -\frac {\Cmat(\Lamb)} {\pi^2 m_{\Lamb}}  \Dv \ .
\eeq
%
The elements of the symmetric matrix $\Cmat$
are either -1 (pair of systems currently on opposite sides of the CB,
and thus moving in opposite directions) or +1
(pair of systems currently on the same side of the CB, and thus moving in the same direction). The diagonal elements
are all +1, and the other +1 values surround the diagonal (line- and column-wise),
the rest being -1 values.
Because $K$ is even, the two types of values 
are always equally represented in the matrix, 
specific locations
depending on $\Lamb$.
%
Note that the variable $\Lamb$ itself still needs to be explicitly propagated
using \refeq{cb_big_lam_eq_mot}.

%%\revphil{Discussion in page 16-18 a little hard to get through.}

For a given configuration $\Xv$ of the replica system, the Hamiltonian
$\ham^\dagger$ of \refeq{ti_rep_ham} (together with \refeq{cb_lam_of_big_lam}) is periodic in $\Lamb$ 
with a period $2\pi$ corresponding to a full rotation of the CB.
%
However, because the Hamiltonians of the individual replicas are 
identical, upon ensemble averaging over $\Xv$, 
one expects the calculated properties to be periodic over $\Lamb$ 
with a smaller period $\Delta \Lamb$, corresponding to one shift of the CB.
%
\revphil{
This is in particular the case for the probability distribution $P(\Lamb)$
along $\Lamb$ and the associated free-energy profile $G_{\Lamb}(\Lamb)$,
given by
%
%\begin{align}
%\label{eq:fre_prof_lam_def}
%%\beq{fre_prof_lam_def}
%  G_{\Lamb}(\Lamb) &= G_{\Lamb}(0) +
%      \int\limits_0^\Lamb \mathrm{d}\Lamb' \left\langle \frac{\partial{\ham^{\dagger}(\Xv;\lamv)}}{\partial \Lamb} \right\rangle^{\dagger}_{\Lamb'} \\ \nonumber
%    &=  G_{\Lamb}(0) + \sum_{k=0}^{K-1}  \left[ G(\lam_k(\Lamb)) - G(\lam_k(0))\right] \quad \text{with} \  \lamv=\lamv(\Lamb) \ ,
%\end{align}
%
\begin{align}
  \label{eq:fre_prof_lam_def}
  G_{\Lamb}(\Lamb) &= G_{\Lamb}(0) +
      \int\limits_0^\Lamb \mathrm{d}\Lamb' \left\langle \frac{\partial{\ham^{\dagger}(\Xv;\lamv)}}{\partial \Lamb} \right\rangle^{\dagger}_{\Lamb'} \nonumber \\
    &=  \tilde{G}_{\Lamb}(0) + \sum_{k=0}^{K-1} G(\lam_k) \quad \text{with} \  \lamv=\lamv(\Lamb) \ ,
\end{align}
%
where $\lamv(\Lamb)$ is defined by \refeq{cb_lam_of_big_lam},
$\langle\cdot\cdot\cdot\rangle^{\dagger}_{\Lamb}$ denotes ensemble averaging 
for the replica system (\ie{} over $\Xv$) at the given $\Lamb$ value,
and the second equality follows from \refeqs{ti_rep_ham} and \refeqn{ti_formula}
(the unknown constant $\tilde{G}_{\Lamb}(0)$ 
is equal to $G_{\Lamb}(0)$ increased by a sum of $-G(\lam_k(0))$
offsets).
%
}
%


\radd{Owing to this periodicity over a smaller interval}, it is convenient to introduce a fractional advance variable $\tilde{\Lamb}$
defined as
%
\beq{tilde_lamb_def}
   \tilde{\Lamb} = \gamma( \Lamb, \Delta \Lamb ),
\eeq
%
where
%
\beq{gamma_def}
   \gamma(\theta,\theta_o)
     = \theta_o  \left( \theta_o^{-1}\theta - \lfloor \theta_o^{-1}\theta \rfloor \right)
\eeq
%
returns the part of $\theta$ in excess of the closest lower integer multiple of $\theta_o$.
%
In contrast to $\Lamb$, which is an unbounded variable, $\tilde{\Lamb}$ only spans a finite definition interval $[0,\Delta \Lamb)$.
%
At full convergence, 
any average property binned as a function 
of $\Lambda$ over the interval $[0,2\pi)$ will 
consist of $K$ successive repeats of the same property binned
as a function of $\tilde{\Lambda}$ over its definition interval $[0,\Delta \Lamb)$, as observed in \reffig{scheme:ene:sinus} for the free energy \radd{$G_{\Lamb}(\Lamb)$}.
%
Accordingly, in the absence of full convergence
along $\Lambda$, binning as a function of $\tilde{\Lambda}$
over the interval $[0,\Delta \Lamb)$ followed by $K$-fold replication
provides an efficient way to construct a more accurate representation of any 
$\Lambda$-resolved average quantity. In fact, the definition interval of $\tilde{\Lamb}$ could be further halved by noting that, upon ensemble averaging over $\Xv$ and for any $\Lamb$ value that is an integer multiple of $\Delta \Lamb$, a forward move of the CB produces the same result as a backward move of the same magnitude. Consequently, $\Lamb$-resolved average properties are even over successive $2\pi K^{-1}$ intervals, as also observed in \reffig{scheme:ene:sinus} for the free energy
\radd{$G_{\Lamb}(\Lamb)$}. The corresponding information is thus entirely encompassed in an interval of size $\Delta \Lamb /2$.

The normalized probability distribution $p(\lam)$ 
along the coupling variable $\lam$ considering all the replicas is defined by
%
\beq{proba_of_lam}
  p(\lam) = K^{-1} \sum_{k=0}^{K-1} \left\langle \delta(\lam_k-\lam) \right\rangle^{\dagger\star} \ ,
\eeq
%
\radd{where $\langle\cdot\cdot\cdot\rangle^{\dagger\star}$ denotes ensemble averaging 
for the extended replica system (\ie{} over the joint trajectories of $\Xv$ and $\lamv$).}
%
At full convergence, this probability over the interval $[0,1]$
will consist of $K/2$ successive
repeats of the corresponding distribution over the interval $[0,2 K^{-1})$.
%
\radd{More precisely, the distribution $p(\lam)$} is related to the distribution 
$\tilde{P}(\tilde{\Lamb})$ of $\tilde{\Lamb}$ over interval $[0,\Delta \Lamb)$
as
%
%\revdavid{Shouldn't there be a factor, \ie
\beq{proba_of_lam_from_that_of_big_lam}
%  p(\lam) = \tilde{P}( \pi \gamma( \lam, 2 K^{-1} )  ) \ .
  p(\lam) = \Delta \Lamb\,\tilde{P}( \pi \gamma( \lam, \pi^{-1}\Delta\Lamb )  )  \ .
\eeq
%}
%
%\beq{proba_of_lam_from_that_of_big_lam}
%%  p(\lam) = \tilde{P}( \pi \gamma( \lam, 2 K^{-1} )  ) \ .
%  p(\lam) = \tilde{P}( \pi \gamma( \lam, \pi^{-1}\Delta\Lamb )  ) \ .
%\eeq
%
In plain words, this means that $\tilde{P}(\tilde{\Lamb})$ is the relevant quantity
in terms of sampling along the coupling variable $\lam$.
If it is close to uniform over the range $[0,\Delta \Lamb)$,
then $p(\lam)$ will also be close to uniform over the range $[0,1]$.
Here again, it is noted that $p(\lam)$ is also even over the interval $[0,2 K^{-1})$,
and could be mapped to a $\tilde{\Lamb}$ value defined over an interval of size $\Delta \Lamb / 2$ instead of $\Delta \Lamb$ if desired, as

\beq{sym_proba_lam}
  p(\lam) = \left\{
                    \begin{array}{ll}
%                       \tilde{P}(\pi \gamma(\lam, 2 K^{-1})) & \mathrm{if}\ \gamma(\lam, 2 K^{-1}) < K^{-1} \\
                       \Delta \Lamb\,\tilde{P}(\pi \gamma(\lam, \pi^{-1}\Delta\Lamb)) & \mathrm{if}\ \gamma(\lam, \pi^{-1}\Delta\Lamb) < (2\pi)^{-1}\Delta\Lamb \\
%                       \tilde{P}(\pi \gamma(1-\lam,  2 K^{-1})) & \mathrm{otherwise} \\
                       \Delta \Lamb\,\tilde{P}(\pi \gamma(1-\lam,  \pi^{-1}\Delta\Lamb)) & \mathrm{otherwise} \\
                    \end{array} 
                  \right. .
\eeq


\radd{
%------------------------------------------------------
\subsection{CBTI Free-energy Estimator}
%------------------------------------------------------------
}


Due to the constraints coupling the $\lam_k$-values of the
$K$ replicas, the function $p(\lam)$ of \refeq{proba_of_lam}
is by no means a Boltzmann distribution in terms of the single-system
Hamiltonian. In fact, as seen above, it consists at full convergence
of $K/2$ successive repeats of the same even
curve.
In addition, compared to the Boltzmann distribution,
it will be
significantly flatter. On the one hand, the smaller amplitude of variations 
%and the reduced barriers 
are desired, as they will lead to more homogeneous sampling 
and \revphil{are expected to ease transitions along $\Lamb$ 
(up to the limit imposed by the speed of random diffusion)}.
%
%\revphil{
%Page 18: says that 'there is a smaller amplitude of variations and
%reduced barriers', but it's not clear dynamics are faster, since the
%dH/dl force is slowed down by (averaged). I think this is eventually
%explained by the fact that it becomes nearly diffusive behavior, but
%perhaps this can be brought up sooner.
%}
%
%
On the other hand, it is no longer possible to evaluate 
the free-energy difference $\Delta G$
directly from $p(\lam)$ in analogy with the $\lam$D expression of \refeq{lam_dyn_dg}.
%
However, since the dynamics remains Hamiltonian and the coupling 
between replicas does not
involve the configurational degrees of freedom, the change \radd{from TI to CBTI} does not affect 
the conditional probabilities $\mathcal{P}(\xv | \lam)$. Thus,
configurational ensemble averages sorted by $\lam$-values will remain
identical to those one would obtain from TI \radd{(or from HRE/HRP or $\lam$D)}.
%
As a result, $\Delta G$ can still be obtained 
\radd{by integrating over the average Hamiltonian derivative binned as a function 
of $\lambda$ considering all replicas simultaneously,}
in analogy with the TI expression
of \refeq{ti_formula}.
%
\revphil{Note that the exceptional points of the function $\zeta'$
(discussed previously in the context of \refeq{zigzag_der})
have no influence on the integration, as they represent
finite discontinuities over infinitesimal ranges.
}
%\revphil{SAY THAT DISCONT HAS NO EFFECT
%See Point 6 above. Similar considerations apply to the integration.
%We integrate $\partial\mathcal{H}/\partial\lam$. If the value is off
%at one (or a few) points, the integral is unaffected because the 
%neighborhood of the singularity is infinitesimal and its magnitude
%is finite ({\em i.e.} it is not a $\delta$ function, but just 
%a finite change in the function value).
%}



In practice, \revphil{$\Delta G$} is calculated here based on a single thermostated MD simulation 
of the extended replica system, as
%
\begin{align}
\label{eq:cbti_formula}
  \Delta G &= 
    \int_0^1 \mathrm{d}\lam' K^{-1} \sum_{k=0}^{K-1} \left\langle 
                 \frac{\ham(x_k;\lam_k)}{\partial \lam_k} \delta(\lam_k-\lam')
            \right\rangle^{\dagger\star}\\ \nonumber
   &\approx
%     K^{-1} 
\sum_{j=0}^{J-1}  \left \langle \frac{ \sum_{k=0}^{K-1} \frac{\ham(x_k;\lam_k)}{\partial \lam_k} \alpha(\lambda_k,j;J)}
                              { \sum_{k=0}^{K-1} \alpha(\lambda_k,j;J)}  \right \rangle^{\dagger \star}                                   ,
\end{align}
%
where 
%
\beq{binning_fct}
\alpha(\theta,j;J)=\begin{cases}
1 \qquad \text{if}\ j\leq J\theta < j+1 \\
0 \qquad \text{otherwise}
\end{cases}
\eeq
%
is a binning function  corresponding to a discretization of the $\lam$-interval $[0,1]$ using
$J$ bins. 
%
The approximation in \refeq{cbti_formula} corresponds to a 
simple forward rectangular quadrature, where the Hamiltonian derivative is 
averaged over the $J$ successive bins considering all replicas.
%
Since $\tilde{P}(\tilde{\Lamb})$, and thus $p(\lam)$, will typically be close to homogeneous, $J$ can be taken very large, 
resulting in a negligible quadrature error. 
For example, if $K$ replicas sample $L$ configurations each, the number of data points
per bin will be close to $K L/J$, with limited variations across bins. 
Defining
the maximal allowed value $J_{\mathrm{max}}$ as the highest value of $J$ for which 
empty bins (vanishing denominator in the ensemble average of \refeq{cbti_formula}) never 
occur, a graph of $\Delta G$ evaluated upon increasing $J$ from 1 to $J_{\mathrm{max}}$ 
will rapidly level off to a plateau when quadrature errors become negligible.
%sufficiently small. 
%
\revphil{
Two variants which do not require the specification 
of a number of bins are also proposed in \refsec{CH2C} (\refeqs{cbti_formula_app1} and \refeqn{cbti_formula_average}).}
%
\revphildel{[-- Variants moved to Appendix C --]}



%------------------------------------------------------------
\subsection{CBTI with Memory-based Biasing Potential}
%------------------------------------------------------------

%For a 
When using a large number of replicas, the sampling along $\lam$ afforded by the CBTI scheme will be close to homogeneous. However, for practical reasons (\eg{} number of processors available on a computer node), one may wish to use a small number of replicas. In this case, the sampling homogeneity can be enhanced by addition of a biasing potential. It is sufficient to apply this potential to the fractional advance variable $\tilde{\Lambda}$ over the range $[0,\Delta \Lamb)$. With inclusion of a biasing potential $\mathcal{B}$, \refeq{cb_big_lam_eq_mot} becomes 
%
  \begin{multline}
  \label{eq:cb_big_lam_eq_mot_bias}
  \ddot{\Lamb} = -\frac {1} {m_{\Lamb}} \frac {\partial}{\partial \Lamb} \left ( \ham^\dagger(\Xv;\lamv)  
  + \mathcal{B}(\tilde{\Lamb}) \right) \\ \text{with} \  \lamv=\lamv(\Lamb)\ \text{and}\ \tilde{\Lamb}=\tilde{\Lamb}(\Lamb) \ ,
  \end{multline}
%
where $\mathcal{H}^{\dagger}$ is defined by \refeq{ti_rep_ham}, $\lamv(\Lamb)$ by \refeq{cb_lam_of_big_lam} and $\tilde{\Lamb}(\Lamb)$ by \refeq{tilde_lamb_def}. 


In analogy with the $\lam$-LEUS scheme\cite{BI14.1,BI14.2,BI15.1,BI15.2}, this biasing potential can be expressed as a sum of local grid-based spline functions, built in a LE preoptimization phase and frozen in a subsequent US sampling phase. However, the duration of the LE phase can be considerably reduced compared to a single-system $\lam$-LEUS simulation, considering that $\tilde{P}(\tilde{\Lamb})$ is already close to homogeneity in the absence of biasing and that the support interval is reduced to the $\tilde{\Lamb}$-range $[0,\Delta \Lamb)$. The latter interval can actually be further restricted to $[0,\Delta \Lamb / 2]$ considering the even symmetry of $\tilde{P}(\tilde{\Lamb})$, \ie{} by enforcing an even symmetry of $\mathcal{B}$ as well.

Since the application of a biasing potential that only involves the $\lam_k$-variables does not alter the 
conditional probabilities $\mathcal{P}(\xv|\lam)$, \refeq{cbti_formula} \revphil{(or the variants of \refeqs{cbti_formula_app1} and \refeqn{cbti_formula_average})}
can still be employed without any modification to evaluate the free-energy change. In other words, 
in contrast to the $\lam$-LEUS scheme, the CBTI scheme with the presented TI-like free-energy estimator does not require any reweighting.
