%------------------------------------------------------------
\subsection{Test System}
%------------------------------------------------------------

As an initial application of the proposed CBTI scheme, we considered here 
a relatively simple perturbation, namely the conversion of methanol
from a fully interacting molecule to a dummy skeleton (no intermolecular interactions) 
in an aqueous environment at $P=1\unit{bar}$ and $T=298.15\unit{K}$.
%
The calculations were performed using 
\radd{a modified version of the 
GROMOS11 program \cite{VA11.7,SC12.1,KU12.1} along with}
the parameters of the GROMOS-compatible 2016H66 force field\cite{HO16.1} for methanol\cite{HO11.1} (united atom, rigid bonds, flexible bond angle)
and the simple point charge (SPC) model\cite{BE81.1} for water (fully rigid).
%
Since the dummy skeleton retains the intramolecular interactions (here, only the bond angle),
the calculated free-energy change $\Delta G$ corresponds directly to minus the hydration 
free energy of methanol.


Possible issues related to the existence of a singularity\cite{SI93.1,BE94.1,ZA94.1} 
and the insufficient solute-solvent kinetic-energy exchange\cite{SH03.4,SH05.7,MO07.2,LI08.8} 
close to $\lambda=1$ were alleviated in the usual way,
by means of a soft-core scheme\cite{BE94.1} for the alchemical coupling 
and of stochastic dynamics\cite{LA08.6,VA88.1} (SD) for thermostating the solute and solvent conformational 
degrees of freedom.
%
In most CBTI simulations, the \radd{instantaneous} temperature $T_\Lambda$ of the CB advance variable
$\Lambda$ was also controlled separately by means of a 
Nos\'e-Hoover chain thermostat\cite{MA92.1} 
\radd{at 298.15 K} \revphil{(eight successive thermostat variables)},
with a coupling time $\tau_{\Lamb}$.

%------------------------------------------------------------
\subsection{Simulations Sets}
%------------------------------------------------------------

The exploration of the CBTI scheme and the comparison of its
performance with that of existing methods was carried out in five successive 
steps:
%
($1$) establishing reference TI results;
($2$) analyzing the influence of the CBTI parameters (number $K$ of replicas along with the
       mass-parameter $m_\Lambda$ and thermostat coupling time $\tau_\Lambda$ 
       of the CB advance variable);
($3$) investigating the use of a biasing potential;
($4$) examining the features of the TI-like free-energy estimator (effect of the number $J$ of integration bins and use of equations without specification of $J$);
($5$) comparing the results of CBTI with those of existing methods.


The reference TI calculations (Step 1) were performed using $K_{\mathrm{TI}}=2^n+1$ 
equidistant $\lambda$-points covering the range $[0,1]$  with $n$ = 1,2,...,7.
They involved initial configurations equilibrated for $0.2\,\mathrm{ns}$ 
starting from the equilibrated configuration at the previous $\lam$-point,
\radd{and a simulation time of $100 K_{\mathrm{TI}}^{-1}$ ns per $\lambda$-point.}
%
Each of these calculations, \radd{involving a total single-system sampling time of 100 ns},
was repeated ten times using different random initial velocities. 
%
The integration over the average Hamiltonian derivative 
was 
%then 
performed based on \refeq{ti_formula} using the Simpson quadrature rule\cite{JO10.2,BR11.5,BR11.6}.


To explore the influence of the CBTI parameters (Step 2), various 
combinations of $K$, $m_\Lambda$ and $\tau_\Lambda$ were considered in 
three series of calculations, namely:
%
($i$) the choices $m_{\Lambda}=$16,160,800,1600 or 3200$\,\mathrm{u}\,\mathrm{nm}^2$
       (where $\mathrm{u}$ stands for atomic mass unit, {\em i.e.} g mol$^{-1}$),
      along with $K=16$ replicas in the absence of thermostat coupling 
      for $\Lambda$, {\em i.e.} with $\tau_\Lambda\rightarrow\infty$;
%%      ($10\unit{ns}$ simulations of the replica system after $0.2\unit{ns}$ equilibration);
%
($ii$) the choices $\tau_\Lambda={0.05,0.1,0.5,1~\text{or}~2}\,\mathrm{ps}$
       along with $K=16$ replicas and $m_{\Lambda}=160\,\mathrm{u}\,\mathrm{nm}^2$;
%%      ($10\unit{ns}$ simulations of the replica system after $0.2\unit{ns}$ equilibration);
%
($iii$) the choices $K=8,16,32,64~\text{or}~128$
       along with 
       $m_{\Lamb}=40 K^{1/2}\,\mathrm{u}\,\mathrm{nm}^2$ 
       and $\tau_{\Lambda}=0.5\,\mathrm{ps}$;
%% ($256 K^{-1}\unit{ns}$ simulations of the replica system after $0.2\unit{ns}$ equilibration).
%
%
The parameters (and results) of these three series of simulations, including their durations $t_{\mathrm{sim}}$, are summarized  in \reftab{screen} (entries 1-15).
%
\radd{All these simulations were preceded by $0.2\unit{ns}$ equilibration}.
%
\revphil{
For the the third series (entries 11-15), $m_{\Lambda}$ was made proportional 
to $K^{1/2}$, an arbitrary parameter choice justified by arguments provided 
in \refsec{othersim}, and the five simulations relied on the 
the same total single-system sampling time of 256 ns.}
%
%
This exploration showed that the CBTI method is rather robust with respect to the choice of its parameters.
The values $K=16$, $m_\Lamb=160\,\mathrm{u}\,\mathrm{nm}^2$ and $\tau_{\Lamb}=0.5\,\mathrm{ps}$ 
were retained as 
%an optimal combination
\revphil{a good combination}
for the alchemical perturbation considered.
\radd{For comparison with the TI results of Step 1,}
ten repeats of the calculation 
involving this specific choice were performed using different random initial velocities
and 
\radd{a total single-system sampling time of 100 ns after $0.2\unit{ns}$ equilibration}.
%
%($6.25\unit{ns}$ simulations of the replica system after $0.2\unit{ns}$ equilibration, \ie{}  total $100\unit{ns}$ single-system sampling time).
%

The application of CBTI with a biasing potential (Step 3)
was investigated in the context of simulations with $K=8$ or 16, 
both with $m_{\Lamb}=40 K^{1/2}\,\mathrm{u}\,\mathrm{nm}^2$ and  $\tau_{\Lamb}=0.5\,\mathrm{ps}$.
%
For $K=8$, the biasing potential $\mathcal{B}$ (\refeq{cb_big_lam_eq_mot_bias}) was constructed 
using $N_{\mathrm{gp}}=34$ basis functions centered at equidistant grid-points \radd{$i=0,...,N_{\mathrm{gp}}-1$}
over the $\tilde{\Lamb}$-range $[0,\pi/4)$. 
%
The coefficients of the basis-functions $i$ \radd{and $N_{\mathrm{gp}}-1-i$}
\radd{with $i=0,...,N_{\mathrm{gp}}/2-1$}
were constrained to be identical, 
considering the expected even symmetry of $\tilde{P}(\tilde{\Lamb})$.
%
In terms of the CB advance variable $\Lamb$, this means that the biasing potential relied in effect on $K(N_{\mathrm{gp}}-1)=264$ local functions covering the $\Lamb$-range $[0,2\pi)$, these functions being defined by only 17 independent coefficients. 
For $K=16$, $\mathcal{B}$ 
relied on $N_{\mathrm{gp}}=18$ basis functions 
%centered 
%at equidistant grid-points 
over the $\tilde{\Lamb}$-range $[0,\pi/8)$, leading to $272$ functions over the $\Lamb$-range $[0,2\pi)$ 
%and
defined by 9 independent coefficients.
%
Second-order splines\cite{DE01.6,HA07.12} (of range $\pm 2 \delta$ with $\delta=\pi/132$ or $\pi/136$ for $K=8$ and $16$, respectively) were 
employed\cite{BI14.1} as basis functions. An initial build-up 
force constant $c_{\mathrm{LE}}=10^{-3}\,\mathrm{kJ\,mol^{-1}}$ was used, which was multiplied by a reduction factor $f_{\mathrm{red}}=0.1$ 
after each double-sweep of \radd{half} the \radd{$\tilde{\Lambda}$-range $[0,\pi K^{-1}]$.}
% 
The duration $t_{\mathrm{LE}}$ of the LE build-up phase for 
the replica system 
\radd{was $=0.15\,\mathrm{ns}$ for $K=8$ and 
$0.07\,\mathrm{ns}$ for $K=16$,
corresponding to only $1.1-1.2\,\mathrm{ns}$ 
total single-system simulation time.
%\revdavid{[DELETE:,
%which was sufficient three double-sweeps in both cases.]}
}
%
The parameters (and results) of these two simulations are summarized in \reftab{screen} (entries 16 an 17).
%
The duration $t_{\mathrm{sim}}$ of the US sampling phases for 
the replica system were $22\unit{ns}$,
corresponding to total single-system sampling 
times of $176\unit{ns}$ and $352\unit{ns}$ for $K=8$ and $16$, respectively. 
%


\begin{table}
\caption{
  \textit{Influence of the CBTI parameters in 
%\revdavid{[DELETE: unbiased]}
 simulations 
  of the aqueous methanol-to-dummy mutation at 298.15 K and 1 bar with the 
  2016H66 force field.}
%
  \revphil{This table investigates the influence of the parameters selected for the CBTI scheme
           on the temperature and dynamics of the CB advance 
           variable $\Lambda$ and on the calculated free-energy change $\Delta G$
           (for the latter, considering a constant total single-system sampling time of 100 ns).
           }
%
  \radd{For each simulation,} the successive entries are:
  the index of the simulation (sim),
  the number $K$ of replicas,
  the simulation time $t_{\mathrm{sim}}$ for the replica system,
  the mass-parameter $m_{\Lambda}$,
  the thermostat coupling time $\tau_{\Lambda}$ ($\infty$ indicates that no coupling is applied),
  the average temperature $T_\Lambda$,
  the root-mean-square fluctuation ${\sigma}_{\dot{\Lamb}}$ of $\dot{\Lambda}$,
  the autocorrelation time $\tau_{\dot{\Lamb}}$ of $\dot{\Lamb}$,
  the diffusion coefficient $D_\Lambda$ (\refeq{einstein}),
  the free-energy difference $\Delta G$ calculated using \refeq{cbti_formula} with $J=500$ (except for entry 11, $J=200$),
  the alternative free-energy difference $\Delta G_{\mathrm{alt}}$ calculated
  using \refeq{cbti_formula_app1},
  and the approximate free-energy difference $\Delta G_{\mathrm{app}}$ calculated
  using \refeq{cbti_formula_average}.
  %
  Error estimates obtained by bootstrapping \radd{(no Student $t$-factor included)} are also reported between parentheses
  for $\Delta G$, $\Delta G_{\mathrm{alt}}$
  and $\Delta G_{\mathrm{app}}$.
%
  \radd{Note that the simulations differ in terms of total single-system sampling time $Kt_{\mathrm{sim}}$.
        To enable a fair comparison, the free energy-changes and associated errors have been calculated 
        after truncating the all simulations to $100\unit{ns}$ single-system sampling time evenly distributed 
        over all replicas.}
  %
%%
%  To make the free-energy differences and associated errors comparable, the values are calculated for all simulations
%  considering only $100\unit{ns}$ single-system sampling time evenly distributed over all replicas.
%
  Associated graphs for the distributions \radd{$P(\Lambda)$ of $\Lambda$}, 
$P_{\dot{\Lambda}}$ of $\dot{\Lamb}$ and $P_{\ddot{\Lamb}}$ of $\ddot{\Lamb}$, as well as the mean-square displacements
  $d_{\Lamb}$ of $\Lamb$ and autocorrelation functions $c_{\dot{\Lamb}}$ of $\dot{\Lamb}$
  can be found in \reffigs{lam}, \reff{dynamics} \radd{and \reff{leus}, or} in Figs. \reffig{thermo:03_cvb_screen:016:1_0.01} - 
\reffig{thermo:04_cvb_thermo:128:10} \radd{and \reffig{leus:8}-\reffig{leus:16}}.
%
 Simulations $11$, $13$, $15$, \radd{$16$ and $17$} are discussed in the main text.
 The other simulations are discussed in \refsec{othersim}.
}   
\label{tab:screen}
\resizebox{\textwidth}{!}{
\begin{tabular}{*{12}{c}}
\hline
sim & $K$ & $t_{\mathrm{sim}}\,[\mathrm{ns}]$ & $m_{\Lambda} [\mathrm{u\,nm^2}]$ & $\tau_{\Lambda} [\mathrm{ps}]$ & $T_{\Lambda} [\mathrm{K}]$ & $\sigma_{\dot{\Lambda}}\,[\mathrm{ps^{-1}}]$ & $\tau_{\dot{\Lambda}}\,[\mathrm{ps}]$ & $D_{\Lambda} [\mathrm{ns^{-1}}]$ & $\Delta G\,[\mathrm{kJ\,mol^{-1}}]$ &  $\Delta G_{\mathrm{alt}}\,[\mathrm{kJ\,mol^{-1}}]$ & $\Delta G_{\mathrm{app}}\,[\mathrm{kJ\,mol^{-1}}]$ \\
%&&&&&&&&&  (\refeq{cbti_formula})  & (\refeq{cbti_formula_app1}) &  (\refeq{cbti_formula_average}) \\
\hline
    1 &    16 &    10 &    16 & $\infty$ &      308.0 &       0.40 &       0.02 &       16.7 &      21.14 (0.16) &      21.23 (0.17) &      20.21 (0.32) \\ 
    2 &    16 &    10 &   160 & $\infty$ &      297.9 &       0.12 &       0.61 &       13.2 &      21.27 (0.17) &      21.32 (0.15) &      20.29 (0.31) \\ 
    3 &    16 &    10 &   800 & $\infty$ &      296.3 &       0.05 &       1.90 &       10.2 &      21.25 (0.15) &      21.34 (0.16) &      20.27 (0.32) \\ 
    4 &    16 &    10 &  1600 & $\infty$ &      304.0 &       0.04 &       3.47 &       10.3 &      21.28 (0.15) &      21.25 (0.17) &      20.32 (0.32) \\ 
    5 &    16 &    10 &  3200 & $\infty$ &      297.7 &       0.03 &       5.63 &        7.4 &      21.23 (0.16) &      21.24 (0.18) &      20.11 (0.31) \\ 
\hline
    6 &    16 &    10 &   160 &  0.05 &      284.0 &       0.12 &       0.37 &        6.9 &      21.23 (0.16) &      21.04 (0.15) &      20.22 (0.31) \\ 
    7 &    16 &    10 &   160 &  0.10 &      291.7 &       0.12 &       0.36 &        7.4 &      21.48 (0.13) &      21.41 (0.16) &      20.47 (0.29) \\ 
    8 &    16 &    10 &   160 &  0.50 &      296.4 &       0.12 &       0.41 &        9.1 &      21.42 (0.14) &      21.48 (0.16) &      20.45 (0.30) \\ 
    9 &    16 &    10 &   160 &  1.00 &      299.4 &       0.12 &       0.49 &       12.9 &      21.57 (0.16) &      21.48 (0.17) &      20.59 (0.30) \\ 
   10 &    16 &    10 &   160 &  2.00 &      296.8 &       0.12 &       0.56 &       13.1 &      21.48 (0.13) &      21.73 (0.15) &      20.45 (0.29) \\ 
\hline
   11 &     8 &    32 &   113 &  0.50 &      298.4 &       0.15 &       0.23 &        4.4 &      21.69 (0.40) &      21.69 (0.44) &      15.65 (0.31) \\ 
   12 &    16 &    16 &   160 &  0.50 &      298.3 &       0.12 &       0.41 &       10.2 &      21.42 (0.16) &      21.48 (0.17) &      20.45 (0.33) \\ 
   13 &    32 &     8 &   226 &  0.50 &      300.6 &       0.10 &       0.38 &        6.2 &      21.44 (0.14) &      21.33 (0.15) &      21.34 (0.32) \\ 
   14 &    64 &     3 &   320 &  0.50 &      297.2 &       0.09 &       0.31 &        2.5 &      21.32 (0.14) &      21.27 (0.14) &      21.29 (0.30) \\ 
   15 &   128 &     2 &   452 &  0.50 &      304.1 &       0.08 &       0.24 &        2.3 &      21.43 (0.12) &      21.61 (0.14) &      21.41 (0.32) \\ 
\hline
   16 &     8 &    22 &   113 &  0.50 &      296.4 &       0.15 &       0.43 &       15.3 &      21.48 (0.16) &      21.61 (0.17) &      19.65 (0.34) \\ 
   17 &    16 &    22 &   160 &  0.50 &      299.3 &       0.12 &       0.43 &        9.4 &      21.30 (0.13) &      21.43 (0.16) &      20.61 (0.34) \\ 
\end{tabular}
}
\end{table}



%
To examine the features of the TI-like free-energy estimator (Step 4),
the number $J$ of integration bins
used in the rectangular quadrature to calculate $\Delta G$ (\refeq{cbti_formula}) was varied  considering the simulations of Steps 2 and 3 above (17 simulations of \reftab{screen}). 
%
%
%
The resulting $\Delta G$ values were also compared to those 
of the variants $\Delta G_{\mathrm{alt}}$ (\refeq{cbti_formula_app1}) 
and $\Delta G_{\mathrm{app}}$ (\refeq{cbti_formula_average}) \revphil{proposed in \refsec{CH2C}}.



Finally, the results of the CBTI simulations were compared to 
those of other methods (Step 5), namely TI or HRE using Simpson quadrature
as estimator as well as TI using EXTI and MBAR as estimator.
%
%
%XXX MOVED FROM BELOW
%Finally, the results of CBTI were compared to those of existing methods,
%namely TI and HRE using Simpson quadrature as free-energy estimator or TI
%using EXTI and MBAR.
%
\radd{For Simpson quadrature,} the TI simulations relied on $K_{\mathrm{TI}}=3,5,9,17,65$ or 129 equispaced $\lam$-points,
and the HRE simulations relied on $K_{\mathrm{HRE}}=17,33~\text{or}~65$ equispaced replicas 
with exchange attempts every $\tau_{\mathrm{HRE}}=0.2\,\mathrm{ps}$.
%
\radd{The use of EXTI and MBAR was explored based on TI-like
simulations relying on $K_{\mathrm{TI}}=9$ and $17$ equispaced $\lam$-points.
%
For EXTI, the \radd{average} Hamiltonian derivative was extrapolated 
%\revdavid{[DELETE: using OSP ]
}
during the $K_{\mathrm{TI}}$ simulations to 129 equispaced virtual $\lam$-points,
and the latter 129 values used in the Simpson quadrature.
%
For MBAR, the Hamiltonian was calculated at 129 equispaced virtual
$\lam$-points considering all the configurations sampled in the $K_{\mathrm{TI}}$, 
and the data combined using the MBAR equation\cite{SH08.7} as implemented in pymbar\cite{CH17.16}.
%
All the above comparisons were performed at a total single-system 
sampling time of $100\unit{ns}$ distributed evenly over all replicas.



\radd{Error bars on the calculated $\Delta G$ values
were estimated in two different ways.}
%
For the \radd{calculations involving ten repeats (all TI calculations plus one CBTI simulation),
%
the standard deviation $\sigma$ of the mean was calculated 
by scaling that of the ten estimates by the square-root of nine,
and the error $\epsilon$ on the mean was calculated 
as $\epsilon=2.262 \sigma$, where $2.262$
is the Student $t$-factor\cite{ST08.9} for nine degrees of freedom 
and a two-sigma confidence interval of $95\%$.
}
%
For the \radd{individual calculations that were not repeated},
the statistical error was estimated
by bootstrapping\cite{EF79.1,EF98.1} \radd{(no Student $t$-factor included)} 
\radd{using 100 bootstrap samples.
%
If $K$ replicas (CBTI) or $\lambda$-points (all other methods) 
have generated as many sets of $L$ data points, a sample consists 
here of $K$ sets of  $L$ data points selected randomly 
(possibly multiple times) from the $K$ original data sets.
}
%
\revphil{Note that for TI/EXTI and TI/MBAR, it is essential to perform 
the bootstrapping based on the data from the $K_{\mathrm{TI}}$ real 
$\lambda$-points, and not from the $129 K_{\mathrm{TI}}$ predicted values
(the latter procedure would result in underestimated errors 
due to correlation in the derived data).}
%
\radd{The bootstrapping error will only be accurate 
provided that the data from the simulations, written to file every $2\unit{ps}$, 
is uncorrelated in time.}
%
\radd{Normalized autocorrelation} functions and characteristic times for the \radd{average} Hamiltonian derivative
in the different TI-simulations are provided in
%
\reffig{ti:tcf} and \reftab{tcf} to support this assumption. 



%
%> The result of EXTI is now 21.12 (0.14) kJ mol^-1, which is in about the same
%> order of magnitude as all other methods.
%>
%> There is still a little issue:
%> I used 100 bootstrap samples of the simulated points and predicted the
%> non-simulated points from the same bootstrap sample (as discussed last
%> Thursday). Now, I calculated the error in two different ways:
%> - I took the 100 bootstrap samples per simulated point and calculated the
%> error on the dH/dl on all 129 (predicted) lam-points. Then I calculated the
%> error on the final Delta G by error propagation. *This still leads to an
%> error which is much smaller (0.03kJ mol^-1)* I assume, that the error
%> propagation is not correct because it neglects the cross-correlation between
%> the lambda-points which is definitely there because of the prediction.
%> - For each bootstrap sample of all simulated lam-points, I calculated a dHdl
%> curve and a Delta G. Then the bootstrap error is the standard deviation  of
%> the 100 Delta G values. This yields to the 0.14 kJ mol^-1. This is cleaner,
%> you don't have to do error propagation and it is the same way as I calculated
%> the error for the other methods.
%>
%> In the plots and tables I used error #2, which is more trustworthy in my
%> opinion. I will have a look whether I can improve the error propagation that
%> error #1 is also in the same range.
%>
%>
%






%------------------------------------------------------------
\subsection{Simulation Parameters}
%------------------------------------------------------------


The simulations involved a cubic computational box containing one methanol
and 1000 water molecules under periodic boundary conditions
in the isothermal-isobaric ensemble at 
$P=1\unit{bar}$ and $T=298.15\unit{K}$.
%
They were performed using SD 
by integrating the Langevin equation of motion\cite{LA08.6}
using the leap-frog scheme\cite{HO70.1} (SD variant\cite{VA88.1}) with a timestep $\Delta t=2\,\mathrm{fs}$
and a friction coefficient $\gamma=10\,\mathrm{ps^{-1}}$.
%
Since the kinetics of the system is irrelevant in this work, SD instead of thermostated MD was used to 
avoid problems related to insufficient solute-solvent kinetic-energy exchange\cite{SH03.4,SH05.7,MO07.2,LI08.8}
close to $\lambda=1$ (dummy-skeleton state).
The value of $\gamma$ corresponds 
%effectively 
to the  coupling time of $0.1\unit{ps}$
commonly employed in GROMOS simulations\cite{VA96.1,VA11.7} relying on a weak-coupling\cite{BE84.1} thermostat.
%
The average pressure was maintained close to its reference value
by isotropic weak coupling\cite{BE84.1} using a molecular virial, 
a coupling time $\tau_{P}=0.5\,\textrm{ps}$
and a compressibility $\kappa=4.575\cdot10^4\,\mathrm{kJ}\,\mathrm{mol}^{-1}\,\mathrm{nm}^{-3}$
as commonly used in GROMOS for aqueous biomolecular systems\cite{VA96.1,VA11.7}.
%
The bond rigidity  of methanol and the full rigidity of water were enforced by
application of the SHAKE algorithm\cite{RY77.1} with a relative 
geometric tolerance of $10^{-4}$. 
%
The energies and 
%\radd{average} 
Hamiltonian derivatives
were written to file every $2\unit{ps}$ for analysis.
%


The non-bonded interactions were handled by means of a molecule-based
twin-range cutoff scheme\cite{BE86.3}
with short- and long-range cutoff distances set
to $0.8$ and $1.4\unit{nm}$, respectively, and an update frequency of
5 timesteps for the short-range pairlist and intermediate-range
interactions. The molecule center was the center of geometry for methanol and the oxygen atom for water.
A reaction-field correction\cite{BA73.1,TI95.1} was applied to account for the mean effect of the electrostatic interactions beyond the long-range cutoff distance, using a relative
dielectric permittivity of $61$ as appropriate for the SPC model~\cite{HE01.1}.
%
To alleviate issues related to the existence of a singularity\cite{SI93.1,BE94.1,ZA94.1} 
close to $\lambda=1$ (dummy-skeleton state), the alchemical transformation
relied on a soft-core scheme\cite{BE94.1},
applied with the parameters $\alpha_{\mathrm{LJ}}=0.5$ and $\alpha_{\mathrm{CRF}}=0.5\,\mathrm{nm}^2$.
%

For the CBTI calculations, the propagation of the $\Lamb$ variable
(\refeq{cb_big_lam_eq_mot}) preceded that 
of the conformational degrees of freedom, and was performed with the same 
timestep $\Delta t$.
%
More precisely, the following 
leap-frog steps were carried out in sequence:
$\dot{\Lamb}(t-\Delta t/2) \rightarrow \dot{\Lamb}(t+\Delta t/2)$,
$\Lamb(t) \rightarrow \Lamb(t+\Delta t)$,
calculate $\lamv$ from $\Lambda$ using \refeq{cb_lam_of_big_lam},
$\dot{\rv}(t-\Delta t/2) \rightarrow \dot{\rv}(t+\Delta t/2)$
and
$\rv(t) \rightarrow \rv(t+\Delta t)$.
%
Unless otherwise specified (explorative simulations), 
\radd{the CBTI simulations 
relied on a
mass-parameter $m_\Lambda$
set to $m_{\Lamb}=40 K^{1/2}\,\mathrm{u}\,\mathrm{nm}^2$,
%where $K$ is the number of replicas, and
%the temperature of the $\Lambda$-variable was controlled separately
%by means of a Nos\'e-Hoover chain thermostat\cite{MA92.1}
%involving eight successive thermostat variables at a reference temperature of $298.15\unit{K}$
%and with a 
on a coupling time $\tau_{\Lambda}$ set to $0.5\unit{ps}$,
and 
%
%And unless otherwise specified 
%(investigation of the effect of $J$ and unbiased CBTI simulation with $K=8$), 
on the use of $J=500$ bins 
for evaluating
$\Delta G$ based on \refeq{cbti_formula}
($J=200$ for the unbiased simulation with $K=8$).
}


%------------------------------------------------------------
\subsection{Trajectory Analysis}
%------------------------------------------------------------

The dynamics of the replica system in the CBTI simulations was characterized by monitoring
the distribution $P$ of the $\Lambda$ variable,
the average temperature $T_{\Lamb}$, 
the distribution $P_{\dot{\Lamb}}$ of $\dot{\Lamb}$, 
the associated root-mean-square fluctuation $\sigma_{\dot{\Lamb}}$, 
the \radd{normalized} autocorrelation function $c_{\dot{\Lambda}}$ of $\dot{\Lamb}$, 
the associated autocorrelation time $\tau_{\dot{\Lambda}}$, 
\radd{the mean-square displacement $d_{\Lamb}$ of $\Lamb$ as a function of time},
the associated diffusion coefficient $D_{\Lambda}$, 
and 
the distribution $P_{\ddot{\Lambda}}$ of $\ddot{\Lambda}$.
%
%
%


The distribution $P_{\dot{\Lamb}}$ of $\dot{\Lamb}$ can be compared to the analytical one-dimensional Maxwell-Boltzmann velocity distribution\cite{MA60.1}
%
\beq{ana_vel}
P^{\mathrm{MB}}_{\dot{\Lamb}}(\dot{\Lamb})=\left ( \frac{\beta m_{\Lamb}}{2\pi} \right )^{\frac{1}{2}} \mathrm{e}^{-\frac{\beta m_{\Lamb}}{2} \dot{\Lamb}^2}.
\eeq
%


The diffusion coefficient $D_{\Lamb}$ was calculated from the mean-square displacement $d_{\Lamb}$ of $\Lamb$ as a function of time $t$ according to the one-dimensional Einstein equation\cite{EI05.1}
%
\begin{multline}
  \label{eq:einstein}
  D_{\Lamb}=\lim\limits_{t\rightarrow \infty} \frac{d_{\Lamb}(t)}{2t} \\
  \text{with} \qquad d_{\Lamb}(t)=\left \langle \left[\Lamb(\tau + t) -\Lamb (\tau)\right ]^2 \right \rangle_{t},
\end{multline}
%
where $\langle \cdot\cdot\cdot \rangle_{t}$ denotes averaging over $\tau$ (all possible time origins) at constant $t$. Note that this equation must be applied to the unbounded variable $\Lamb$, \ie{}  without refolding to the reference interval $[0,2\pi)$.
The infinite-time limit was replaced in practice by a linear least-squares fit over the time range 0 to $0.15\unit{ns}$.
%
\revphildel{[-- Redundant paragraph removed --]}


%To investigate the influence of the number $J$ of bins used 
%to calculate the free-energy differences $\Delta G$ in CBTI according to \refeq{cbti_formula},
%this number was systematically increased from 1 to $J_{\mathrm{max}}$ in increments ranging 
%from 10 to 1000, where $J_{\mathrm{max}}$ is the highest value of $J$ for which empty bins never occur.
%%
%%
%\revphil{The estimates} $\Delta G_{\mathrm{alt}}$ and $\Delta G_{\mathrm{app}}$ were also calculated \revphil{for comparison},
%using \refeqs{cbti_formula_app1} and \refeqn{cbti_formula_average} \revphil{of Appendix C}.
%%The latter one is only expected to be accurate for quasi-homogeneous sampling of the $\lam$-range. 
%%
%


All the graphs presented in this chapter were generated with Python (www.python.org) and the Matplotlib library.\cite{HU07.7}

