\makeatletter
\def\input@path{{../}}
\makeatother
\documentclass[../main.tex]{subfiles}
\begin{document}
\renewcommand{\path}{3_chapter_1/}
\chapter[Cyclic Peptides Permeability change based on stereocenter changes]{Modulation of the Passive Permeability of Semipeptidic Macrocycles based on the stereochemistry change of a single atom. \footnote{\label{footnoteChapter5CopyRight} 
    Reprinted (adapted) with permission from Christian Comeau$^{**}$, Benjamin Ries$^{**}$, Thomas Stadelmann, and et al.,
    J. Med. Chem., \textbf{64}, 5365-5383 (2021). Copyright 2021 American Chemical Society.
    $^{**}$ equal contribution}
}
\chaptermark{Cyclic Peptides}
\label{ch:cycPep}

\aquote{
``... everything that living things do can be understood in terms of the jigglings and wigglings of atoms ''
} 
{Richard Feynman, Lectures on Physics\cite{Karplus2002}}

\begin{summary}
Incorporating small modifications to peptidic macrocycles can have a major influence on their properties. For instance, N-methylation has been shown to impact permeability. A better understanding of the relationship between permeability and structure is of key importance as peptidic drugs are often associated with unfavorable pharmacokinetic profiles. 

A library consisting of 36 compounds of semipeptidic macrocycles was generated, investigating two small structure changes: peptide-to-peptoid substitution and various methyl placements on the nonpeptidic linker. 
The permeability of those compounds was assessed in parallel artificial membrane permeability assay (PAMPA) and Caco-2 assays.A permeability cliff was identified triggered by the stereochemistry change based on a single methyl placement. This cliff was studied using a combination of MD simulations and NMR measurements, resulting in a hypothesis on how the change modifies the conformational behavior.
\end{summary}

\clearpage
\pagebreak

%%%%%%%%%%%%%%%%%%%%%%%%%%%%%%%%%%%%%%%%%%%%%%%%%%%%%%%%%%%%%%%%%%%%%
%% Start the main part of the manuscript here.
%%%%%%%%%%%%%%%%%%%%%%%%%%%%%%%%%%%%%%%%%%%%%%%%%%%%%%%%%%%%%%%%%%%%%
%================================================================================
\section{Introduction}
%================================================================================

Macrocycles have recently gathered increased interest in medicinal chemistry as beyond rule-of-5 (bRO5) molecules. \cite{Driggers2008, Mallinson2012, Doak2014, Dougherty2017, Marsault2011, Abdalla2018, Marsault2017, Caron2021}
A key feature of these molecules is their conformational complexity that can be leveraged in drug design to target protein-protein interactions. \cite{ Chene2006, Janin2008, Jones13, Scott2016, Modell2016}
Such protein-protein interactions are typically characterized by large flat binding sites that are difficult to target with small molecules. \cite{Doak2016}
If the macrocycles are peptidic, their toxicity is often relatively low. \cite{Zorzi2017}
Most Food and Drug Administration (FDA)-approved macrocyclic drugs belong to natural products (e.g., erythromycin, tacrolimus) or peptides (e.g., sandostatin, eptifibatide). \cite{Giordanetto2014}
Peptidic or semipeptidic scaffolds bridge the gap between small molecules and biologics. An advantage of this molecule class is that they are relatively easy to synthesis and allow a broad choice of natural and non-natural amino acids required for rapid and thorough pharmacophoric exploration. 
The main challenge with peptides resides in their physicochemical and pharmacokinetics-ADME (absorption, distribution, metabolism, and excretion) properties. 
While cyclic peptides are typically more stable to proteases compared to their linear counterparts, their high polarity often translates into low bioavailability.\cite{Naylor2017, Fosgerau2015}
However, some cyclic peptides were found to cross cell membranes.\cite{Naylor2017, Wang2014, Nielsen2014} 
Developing tools and knowledge to optimize and better predict their structure–permeability relationship is, therefore, a requirement for the field. Such quest found inspiration in studies of the cyclic undecamer cyclosporine A, which is administered orally. 
One prominent structural feature of this natural macrocycle is the high number of N-methylated residues (7 out of 11) and its dynamic structural adaptation to its environment described as chameleonic behavior. \cite{Whitty2016, Danelius2020, Witek2017}
The effect of N-methylation on the permeability of cyclic hexa- and heptapeptides has been systematically investigated since the number and position of N-methylations may be beneficial or detrimental for permeability. \cite{Nielsen2014, Raeder2018, White2011, Beck2012, Biron2008, White2011} 
%
Less explored are the N-alkylated glycines -- aka peptoids -- in which the side chain has been moved from the $\alpha$-carbon to the amide nitrogen. \cite{Schwochert2015} 
Similar to N-methylation, this modification removes one H-bond donor and removes one stereogenic center, and induces glycine-like secondary structures.
The peptoid amide also facilitates cis-trans isomerization compared to the corresponding N-methylation.\cite{Sui2007} 


More recently, the impact of the dynamics of macrocycles in response to their environment, which can range from polar in water, nonhomogeneous in the presence of its target, to lipophilic in the membrane, has been appreciated.\cite{Danelius2020, Witek2017, Riniker2019, Witek2019, Wang2021}
Studying macrocycles with computational methods leads to multiple criteria identified as being possibly essential for chameleonic behavior. Examples of these criteria are the presence of intramolecular H-bonds, 3D polar surface areas (3D-PSA), or kinetic Markov models as metrics for how macrocyclic structures yield polar atoms and rigidification of the backbone cycle into certain polar/apolar states. \cite{Witek2016, Witek2017, Tyagi2018, Witek2019, Wang2021}
A powerful tool to modulate the properties of peptidic macrocycles is the inclusion of a nonpeptidic tether unit.\cite{Marsault2007, Hoveyda2011, Roux2020} 
This tether can serve multiple purposes: in the context of a target interacting with a specific sequence, various tethers can be screened without modifying the peptide recognition sequence, while providing a simple handle for modulating affinity and pharmacokinetic properties. 
Small modifications in size, shape, or functional groups on the tether can dramatically influence on this kind of constrained system.\cite{Appavoo2019}
%
The relationship between structure and permeability is known to be elusive for this class of compounds, with small structural modifications often yielding permeability cliffs. \cite{Wang2014, Raeder2018, Beck2012, White2011, Roux2020, Bockus2015, Hewitt2015, Rezai2006, Over2016}


To investigate the structural effects of a tether with a length of five atoms and the peptide-peptoid change on the compound permeability, our collaborators synthesis a collection of 36 semipeptidic macrocycles. \cite{Comeau2021, Roux2020}
The structure of the compounds was composed of a tripeptide tethered head-to-tail with a nonpeptidic linker (Figure \ref{fig:MolDes}). 
Two classes of modifications were explored: single peptoid replacement and regio- and stereocontrolled linker C-methylation. 
%
\begin{figure}[h!]
    \centering
    \includegraphics[width=\textwidth]{7_chapter_5/fig/intro/MoleculeDesign.jpeg}
    \caption{Synthesis strategy of our collaborators for model compound (\textbf{A}) and two types of modifications: Nala, Nleu, and Nphe peptoids (\textbf{B} showing Nleu) and regio/stereocontrolled C-methylation (\textbf{C} showing 2R methylation).\cite{Comeau2021}}
    \label{fig:MolDes}
\end{figure}

\begin{figure}[h!]
    \centering
    \includegraphics[width=\textwidth]{7_chapter_5/fig/intro/pampa.jpeg}
    \caption{Permeability results in the form of heatmaps. For heatmaps 1–3, the values are expressed as $−log(P_{\text{app}})$, so lower values mean higher permeability (in order of increasing permeability: blue, white, red, and black). Heatmap 4 shows the BA/AB ratio, which represents a measure of efflux.}
    \label{fig:permAssays}
\end{figure}
The passive permeability of the resulting macrocycles was measured by our collaborators in the parallel artificial membrane permeability assay (PAMPA)\cite{Ottaviani2006,Di2015} and their cellular permeability in the Caco-2 assay\cite{Fogh1977,Di2015} (Figure \ref{fig:permAssays}).\cite{Comeau2021}

%
\begin{figure}
    \centering
    \includegraphics[width=\textwidth]{7_chapter_5/fig/intro/permCliffMols.jpeg}
    \caption{Four semipeptidic macrocycles were selected from the collection. In contrast to the pair Nleu-2R/S (bottom), the pair Nleu-5R/S (top) behave significantly  different in the permeability assays. All molecules were  studied with experimental NMR analysis and molecular dynamics (MD) simulations in a polar and apolar environment.}
    \label{fig:permCMols}
\end{figure}
Based on the permeability data, we selected two pairs of diastereomers that differ only by their stereochemistry of the tether methyl group (Figure \ref{fig:permCMols}). While one pair (Nleu-5R/S) differs greatly in their passive permeability behavior, the second one (Nleu-2R/S) does not. 
Prior studies on cyclosporine A showed that the conformational behavior of cyclic peptides in the context of membrane permeability can be studied by performing extensive molecular dynamics (MD) simulations in apolar and polar environments (e.g., chloroform and water) to mimic the behavior outside and inside a membrane. \cite{Witek2016,Witek2017, Witek2019, Wang2021}
Therefore, we carried out MD simulations of each of the four selected macrocycles in water and chloroform. The simulations results were validated by comparing to solution NMR measurements of the compounds. \cite{Balazs2019,Stadelmann2020}
Finally, we used different metrics such as torsional angles, hydrogen-bond formation, and 3D-PSA\cite{Sebastiano2018} to assess and compare the conformational behavior of the compounds. 


%================================================================================
\section{Computational Details}
%================================================================================

\subsection{Simulation}
In the computational studies, two pairs of structurally similar cyclic peptides were selected, i.e., Nleu-5R/Nleu-5S and Nleu-2R/Nleu-2S. 
The first pair presents a “permeability cliff”, i.e., the two peptides show a large difference in the passive permeability in the PAMPA assay (Nleu-5R: $5.39$; Nleu-5S: $7.21$), despite a high structural similarity. 
In contrast, the second pair is similar in both structure and permeability (Nleu-2R: $6.14$; Nleu-2S: $5.80$). 
For each of these four peptides, $250$ starting coordinates were generated using the macrocycle variant of the OMEGA conformer generator from OpenEye. \cite{Hawkins2012, Hawkins2010, Poongavanam2018}
Conformers were energy-minimized for maximum $2000$ steps with the steepest descent \cite{Ruder2016} approach using the GROMOS software package \cite{Schmid2012} with the GROMOS 54A7 force field. \cite{Schmid2011} 
Each minimized starting conformation was solvated in a cubic box of simple-point-charge (SPC) water \cite{Berendsen1981} (on average, $4172$ solvent molecules) or chloroform \cite{Tironi1994} (on average, $980$ solvent molecules). 
For each system, a molecular dynamics (MD) simulation of $101~ns$ length was performed under isothermal–isobaric (NPT) conditions with the leap-frog integration algorithm \cite{Hockney1970, Gunsteren1988} and a time step of $2~fs$. 
The first 1 ns was discarded as equilibration. Bond lengths were constrained with SHAKE \cite{Ryckaert1977} and a tolerance of $10^{–4}~nm$. 
Nonbonded interactions were calculated using a twin-range scheme with a short-range cutoff of $0.8~nm$ and a long-range cutoff of $1.4~nm$. 
The electrostatic nonbonded contributions beyond the long-range cutoff were calculated with the reaction-field \cite{Tironi1995} (67) approach, setting the dielectric permittivity to 61 \cite{Heinz2001} for water, and to 4.8 \cite{Tironi1994} for chloroform. 
The temperature was kept constant at $300~K$ using the weak coupling scheme \cite{Berendsen1984} and a coupling time of $0.1 ps^{–1}$. 
The pressure was kept at $1.031~bar$ ($1~atm$) with the same type of algorithm, a coupling time of $0.5~ps^{–1}$, and an isothermal compressibility of $0.001654~bar^{–1}$ for chloroform and $0.0004575~bar^{–1}$ for water. 
Translational motion of the center of mass of the simulation box was removed every $2~ps$. Energies and coordinates were written every $5~ps$.

Trajectory analysis was performed with PyEmma \cite{Scherer2015} and MDTraj \cite{Mcgibbon2015}. 
The selection of features for the clustering consisted of the distances between all pairs of polar atoms and the backbone torsional angles, which resulted in $57$ features. 
This selection was reduced to three to five dimensions (depending on the peptide) with TICA \cite{Molgedey1994} using a cumulative variance of $0.9$ as criterium and a TICA correlation lag time of $50~ps$. 
Based on these TICs, the frames were clustered with a common nearest neighbor (CNN) algorithm \cite{Keller2010, Weiß2021} using a cutoff of $0.2$ and a similarity of $20$. 
Comparison of selected clusters with NMR experiments was performed with the GROMOS++ package of programs. \cite{Eichenberger2011}
The coefficients for the Karplus curve were taken from Vögeli et al. \cite{Voegeli2015}
Analysis of hydrogen bonds and torsional angles was performed with MDTraj. 
The 3D polar surface area (3D-PSA) was calculated with our implementation \cite{Witek2019} of the workflow in ref \cite{Tyagi2018} using PyMol \cite{Delano2020}.
Statistical analysis of all results was carried out using the Python packages pandas, NumPy and SciPy.\cite{Virtanen2020}

\subsection{NMR Measurements}
The selected peptides Nleu-5R, Nleu-5S, Nleu-2R, and Nleu-2S were characterized by NMR in chloroform-d (Cambridge Isotope Laboratories).
The following spectra were recorded if not stated otherwise: 1H NMR, total correlation spectroscopy (TOCSY), double-quantum filtered correlation spectroscopy (COSY), multiplicity edited 13C heteronuclear single quantum coherence (HSQC), 13C heteronuclear multiple bond correlation (HMBC), and NOESY.
All spectra were measured at 25 °C on a Bruker Avance III HD 600 MHz spectrometer equipped with a N2-cooled Prodigy triple resonance probe.
13C HSQC and TOCSY spectra were recorded with sensitivity enhancement. TOCSY was run with an 80 ms DIPSI2 isotropic mixing time.
The mixing time for the NOESY experiments was set to 300 ms if not stated otherwise. 
For compound Nleu-5R, an EASY-ROESY (78) spectrum with 100 ms mixing time was recorded instead of a NOESY. 
For all spectra, the time domain in both dimensions was extended to twice its size by zero filling, apodized with a cos2 function, and the baseline of the resulting spectra was corrected with a polynomial of fifth order or using the Whittacker smoother algorithm. (79)
Processing was done with Bruker TopSpin version 4.0 (Bruker Biospin AG) and MestReNova 12.0 (Mestrelab Research). Resonance assignment and volume integration of the ROESY cross-peaks were performed with SPARKY 3.115. (80)
The assignments are summarized in Table S1 in the SI.
3JHN–Hα coupling constants for compounds Nleu-5R, Nleu-5S, Nleu-2R, and Nleu-2S were extracted directly from the 1H spectrum with MestReNova and are summarized in Table S2 in the SI.
Volumes were extracted from NOESY and ROESY spectra by integration of the cross-peaks with a Gaussian function (eq 1).(1)
V1,2 is the volume of the cross-peak between proton 1 and 2, a and b are fitting parameters, and r1,2 is the corresponding interatomic distance.
A two-point calibration was done with the averaged interatomic distance (e.g., on both sides of the diagonal) between the diastereotopic protons NLeu Hα1–Hα2 (1.8 Å) and the distance between Hα and Hβ* in the alanine residue (2.65 Å). (81)
In the second case, the volume was previously divided by 3 to account for the three protons in the methyl group. 
Cross-peaks integrated on both sides of the diagonal were averaged, and error bonds of ±20\% were added to the calculated distance. 
Since the GROMOS++ programs can do multiplicity correction and averaging over indistinguishable protons automatically, the reported data do not account for that. 
The volumes and the corresponding distances can be found in Tables S3–S6 in the SI.

%================================================================================
\section{Results and Discussion}
%================================================================================

From the PAMPA and Caco-2 assays performed by our collaborators,\cite{Comeau2021} a large ``permeability cliff''  between Nleu-5R and 5S could be identified (Figure \ref{fig:permAssays}). Permeability cliffs were also observed between some other pairs of epimers, but to a lesser extent (e.g., Nala-4R vs 4S with $−log(P_e)$~=~$7.6$ and $6.6$, Nphe-2R vs 2S with $−log(P_e)$~=~$7.2$ and $6.3$, and Nphe-3R vs 3S with $−log(P_e)$~=~$7.5$ and $6.5$). 
Extensive MD simulations were carried out in a polar and apolar environment (i.e., water and chloroform) to study the influence of the stereocenter change on the conformational behavior of the molecules. As a negative control group, we used the molecule pair Nleu-2R and 2S. 

\FloatBarrier
%-------------------------------------------

\subsection{Starting Configurations} 
The starting conformations used for the MD simulations were generated with the macrocycle variant of the OMEGA conformer generator from OpenEye.  \cite{Hawkins2012, Hawkins2010, Poongavanam2018}
The generated conformers showed similar distributions in terms of hydrogen bonds (H-bonds) and backbone torsional angles for the enantiomer pairs (Tables \ref{tab: SIhbondRatios} and \ref{tab: SIhbondAmountRatios}).
Therefore, we started the MD simulations for each enantiomer pair from similar starting points.
In these conformer ensembles, approximately $50\%$ of the structures had a trans-peptoid bond for each molecule, and $50\%$ had a cis-peptoid bond.

\begin{table}[h!]
\centering
\caption{Hydrogen-bond occurrence in percentage for the starting conformations of Nleu-5R, Nleu-5S, Nleu-2R, and Nleu-2S.}
\label{tab: SIhbondRatios}
  \begin{adjustbox}{max width=\textwidth}
  \begin{tabular}{lcccc}
Hydrogen bond {[}\%{]} & Nleu-2R      & Nleu-2S      & Nleu-5R      & Nleu-5S      \\
\hline
NLEU-O tether-NH       & 14         & 15         &  9           & 9        \\
ALA-O tether NH        & \textless{}1 & \textless{}1  & \textless{}1  & \textless{}1 \\
PHE-O Ala-NH           & \textless{}1 & \textless{}1 & \textless{}1  & \textless{}1 \\
ALA-O PHE-NH           & 24           & 7            & 21 & 19 \\
    \hline
\end{tabular}%
\end{adjustbox}
\end{table}


\begin{table}[h!]
\centering
\caption{Percentage of starting conformations with zero, one, or two hydrogen bonds for Nleu-5R, Nleu-5S, Nleu-2R, and Nleu-2S.}
\label{tab: SIhbondAmountRatios}
  \begin{adjustbox}{max width=\textwidth}
\begin{tabular}{llll}
Hydrogen bond {[}\%{]} & 0      & 1          & 2        \\
    \hline
Nleu-2R        & 36             & 52         &  1       \\
Nleu-2s        & 39             & 51         &  1       \\
Nleu-5R        & 34             & 52         &  13      \\
Nleu-5S        & 37             & 51         &  12  \\
   \hline

\end{tabular}%
\end{adjustbox}
\end{table}

\FloatBarrier
%-------------------------------------------

\subsection{CNN Clustering} 
The cumulative 25~$\mu$s simulation data for each peptide and solvent were clustered separately based on the backbone dihedrals and the polar atom distances. The resulting clusters could be structurally classified depending on the conformation of the peptoid bond (i.e., cis or trans; see Tables \ref{tab:SIClusterTransCHCL3} and \ref{tab: SIClusterTransWater}). The number of generated clusters varied but usually the size of the clusters decreased rapidly. Conformations with a cis or trans-peptoid bond were cleanly separated into different clusters.

\begin{table}[h!]
\center
\caption{List of clusters identified in the simulations in chloroform with the  respective population (in percentage). Clusters with the peptoid bond in trans-conformation are labeled with *. In the other clusters, the peptoid bond is in cis-conformation.}
\label{tab:SIClusterTransCHCL3}
  \begin{adjustbox}{max width=\textwidth}
\begin{tabular}{c|lc||c|lc}
Molecule                  & Cluster & Size $[\%]$ & Molecule                  & Cluster & Size $[\%]$ \\
\hline
\multirow{11}{*}{NLeu-5R} & 1*      & 36.2        & \multirow{11}{*}{NLeu-5S} & 1*      & 41.0        \\
                          & 2       & 18..1       &                           & 2       & 16.0        \\
                          & 3       & 16.1        &                           & 3       & 9.2         \\
                          & 4*      & 15.4        &                           & 4       & 7.4         \\
                          & 5       & 3.5         &                           & 5*      & 6.6         \\
                          & 6       & 1.5         &                           & 6*      & 0.6         \\
                          & 7       & 0.5         &                           & 7       & 0.6         \\
                          & 8       & 0.3         &                           & 8       & 0.2         \\
                          & 9       & 0.2         &                           & 9       & 0.2         \\
                          & Noise   & 8.2         &                           & 10      & 0.1         \\
                          &         &             &                           & Noise   & 18.0        \\
\hline
\multirow{6}{*}{Nleu-2R}  & 1*      & 53.3        & \multirow{13}{*}{Nleu-2S} & 1*      & 26.2        \\
                          & 2       & 17.6        &                           & 2*      & 18.3        \\
                          & 3       & 11.7        &                           & 3       & 15.7        \\
                          & 4       & 10.2        &                           & 4       & 6.5         \\
                          & 5       & 2.8         &                           & 5       & 6.3         \\
                          & Noise   & 4.4         &                           & 6       & 5.2         \\
                          &         &             &                           & 7*      & 5.1         \\
                          &         &             &                           & 8       & 1.3         \\
                          &         &             &                           & 9*      & 0.6         \\
                          &         &             &                           & 10*     & 0.5         \\
                          &         &             &                           & 11      & 0.3         \\
                          &         &             &                           & 12*     & 0.3         \\
                          &         &             &                           & Noise   & 13.0       
\end{tabular}%
\end{adjustbox}
\end{table}

%SI-tab: 10
\begin{table}[h!]
\center
\caption{List of clusters identified in the simulations in water with the respective population (in percentage). Clusters with the peptoid bond in trans-conformation are labeled with *. In the other clusters, the peptoid bond is in cis-conformation.}
\label{tab: SIClusterTransWater}
  \begin{adjustbox}{max width=\textwidth}
\begin{tabular}{c|lc||c|lc}
Molecule                  & Cluster & Size $[\%]$ & Molecule                  & Cluster & Size $[\%]$ \\
\hline
\multirow{11}{*}{NLeu-5R} & 1*      & 46.7        & \multirow{11}{*}{NLeu-5S} & 1*      & 51.3        \\
                          & 2       & 22.4       &                            & 2       & 16.8        \\
                          & 3       & 20.8        &                           & 3       & 8.7         \\
                          & 4       & 6.1        &                            & 4       & 8.0         \\
                          & 5       & 1.5        &                            & 5       & 3.7         \\
                          & 6*      & 0.6         &                           & 6       & 3.6         \\
                          & 7       & 0.4         &                           & 7       & 1.7         \\
                          & 8*      & 0.4         &                           & 8*      & 1.7         \\
                          & 9*      & 0.2         &                           & 9*      & 0.8         \\
                          & Noise   & 1.0         &                           & 10      & 0.4         \\
                          &         &             &                           & 11*     & 0.2         \\
                          &         &             &                           & Noise   & 3.2        \\
\hline
\multirow{6}{*}{Nleu-2R}  & 1*      & 50.2        & \multirow{13}{*}{Nleu-2S} & 1*      & 42.0        \\
                          & 2       & 21.0        &                           & 2       & 19.0        \\
                          & 3       &  6.6        &                           & 3       & 12.0        \\
                          & 4       &  5.6        &                           & 4       &  7.7        \\
                          & 5       &  4.7        &                           & 5       &  6.3        \\
                          & 6       &  2.6        &                           & 6       &  3.5     \\
                          & 7       &  1.6        &                           & 7       &  2.7        \\
                          & Noise   &  7.8        &                           & 8*      &  1.4        \\
                          &         &             &                           & 9*      &  0.1        \\
                          &         &             &                           & Noise   &  5.0       
\end{tabular}%
\end{adjustbox}
\end{table}


The cis-trans isomerization represents a very slow process in the simulations, which occurred only rarely (Table \ref{tab: SICisTransTrans}). Due to the low number of transitions, the process could not be modeled robustly. Therefore, the clusters with the cis- and trans-peptoid bond are analyzed separately in the following.
\begin{table}[h!]
\centering
\caption{Occurrence of cis-trans isomerization of the peptoid bond during the MD 
simulations (25~$\mu$s in water and chloroform, respectively).}
\label{tab: SICisTransTrans}
\begin{adjustbox}{max width=\textwidth}
\begin{tabular}{r|cc|cc}
\multirow{2}{*}{Molecule} & \multicolumn{2}{l}{Chloroform} & \multicolumn{2}{|l}{Water}        \\
    & Cis \rightarrow Trans & Trans \rightarrow Cis & Cis \rightarrow Trans & Trans \rightarrow Cis \\
    \hline
    Nleu-5R    & 14    & 1    & 3     & 0   \\
    Nleu-5S    & 9     & 0    & 0    & 1    \\
    Nleu-2R    & 15    & 7    & 0    & 0    \\
    Nleu-2S    & 6    & 1    & 9    & 5       \\
    \hline
\end{tabular}%
\end{adjustbox}
\end{table}

\FloatBarrier
%-------------------------------------------

\subsection{NMR Validation}
The NMR experiments in chloroform-d (CDCl$_3$) showed that the four compounds adopt at least two different conformations in solution. \cite{Comeau2021}
The major conformer was identified with all amides in trans conformation. 
It was not possible to assign the minor conformers due to signal overlap and low intensity. In the case of Nleu-5R and Nleu-5S, a third conformer could be identified based on exchange spectroscopy (EXSY) cross-peaks in the nuclear Overhauser enhancement spectroscopy (NOESY) spectrum, which is barely detectable in the $^1$H spectrum. The corresponding conformer ratios are listed in Table \ref{tab: nmrConfRatios}.  

\begin{table}[h!]
    \centering
    \caption{Ratios of conformer population observed in the NMR spectra (CDCl$_3$).}
    \label{tab: nmrConfRatios}
    \begin{adjustbox}{max width=\textwidth}
    \begin{tabular}{cc}
    Compound & Ratio \\
    \hline
    Nleu-2R &	100:8 \\
    Nleu-2S &   100:3 \\
    Nleu-5R &   100:4:0 \\
    Nleu-5S &	100:16:1 \\
    \hline
    \end{tabular}
    \end{adjustbox}
\end{table}

The results from the MD simulations were compared to the NMR data of the major conformer, in particular the $^3J_{\text{HN–H}\alpha}$ coupling constants (Figure \ref{fig: j3NMRConfClusterAna}) and the NOE-derived interproton distances (Tables \ref{fig: SINOE violations Nleu-5R} - \ref{fig: SINOE violations Nleu-2SII}).

\begin{figure}[h!]
    \centering
    \includegraphics[width=\textwidth]{7_chapter_5/fig/results/j3NMRConfClusterAna.png}
    \caption{Root-mean-square deviation (RMSD, in hertz) between $^3$J$_{\text{HN–H}\alpha}$ coupling constants in chloroform from NMR measurements and from MD simulations. Clusters with the peptoid bond in trans conformation are shown in green.}
    \label{fig: j3NMRConfClusterAna}
\end{figure}

The clusters with all amides in trans conformation are in good agreement with the  $^3J_{\text{HN–H}\alpha}$ coupling constants (Figure \ref{fig: j3NMRConfClusterAna}), whereas the clusters containing the cis-peptoid bond deviate significantly from the experimental values. For Nleu-2R, the $^3J_{\text{HN–H}\alpha}$ coupling analysis is missing as we could not determine the $^3J_{\text{HN–H}\alpha}$ couplings reliably due to line broadening in the spectrum. The NOE-derived upper distance bounds are also generally reproduced in these clusters (Figures \ref{fig: SINOE violations Nleu-5S} -\ref{fig: SINOE violations Nleu-2SII}). Based on these findings, we focused the analysis in the following on those clusters, which have a reasonable agreement with the NMR data (i.e., clusters 1 and 4 for Nleu-5R, clusters 1, 5, and 6 for Nleu-5S, cluster 1 for Nleu-2R, and clusters 1 and 2 for Nleu-2S).

\begin{figure}[h!]
    \centering
    \includegraphics[width=\textwidth]{7_chapter_5/fig/results/NMR_5R.png}
    \caption{Violations of the NOE-derived upper distance bounds of Nleu-5R in chloroform by the clusters identified in the simulations in chloroform. Distances between residues across the backbone ring are colored. Distances between neighboring residues are shown in black. The dashed line indicates the expected uncertainty of the experimental upper bounds.}
    \label{fig: SINOE violations Nleu-5R}
\end{figure}

\begin{figure}[h!]
    \centering
    \includegraphics[width=\textwidth]{7_chapter_5/fig/results/NMR_5S.png}
    \caption{Violations of the NOE-derived upper distance bounds of Nleu-5S in chloroform by the clusters identified in the simulations in chloroform. Distances between residues across the backbone ring are colored. Distances between neighboring residues are shown in black. The dashed line indicates the expected uncertainty of the experimental upper bounds.}
    \label{fig: SINOE violations Nleu-5S}
\end{figure}

\begin{figure}[h!]
    \centering
    \includegraphics[width=\textwidth]{7_chapter_5/fig/results/NMR_2R.png}
    \caption{Violations of the NOE-derived upper distance bounds of Nleu-2R in chloroform by the clusters identified in the simulations in chloroform. Distances between residues across the backbone ring are colored. Distances between neighboring residues are shown in black. The dashed line indicates the expected uncertainty of the experimental upper bounds.}
    \label{fig: SINOE violations Nleu-2R}
\end{figure}

\begin{figure}[h!]
    \centering
    \includegraphics[width=\textwidth]{7_chapter_5/fig/results/NMR_2R.png}
    \caption{Violations of the NOE-derived upper distance bounds of Nleu-2S in chloroform by clusters 1–6 identified in the simulations in chloroform. Distances between residues across the backbone ring are colored. Distances between neighboring residues are shown in black. The dashed line indicates the expected uncertainty of the experimental upper bounds.}
    \label{fig: SINOE violations Nleu-2S}
\end{figure}

\begin{figure}[h!]
    \centering
    \includegraphics[width=\textwidth]{7_chapter_5/fig/results/NMR_2Sb.png}
    \caption{Violations of the NOE-derived upper distance bounds of Nleu-2S in chloroform by clusters 7–12 identified in the simulations in chloroform. Distances between residues across the backbone ring are colored. Distances between neighboring residues are shown in black. The dashed line indicates the expected uncertainty of the experimental upper bounds.}
    \label{fig: SINOE violations Nleu-2SII}
\end{figure}

\FloatBarrier
%-------------------------------------------

\subsection{Conformation Analysis}
A necessary condition for good membrane permeability is the adoption of conformations that shield polar groups optimally from the apolar environment. \cite{Sebastiano2018, Alex2011, Tyagi2018}
Therefore, we first analyzed the hydrogen-bonding patterns in the clusters in chloroform. For the peptides in this study, a maximum number of two H-bonds can be formed in a conformation due to ring strain. As can be seen in Table \ref{tab: hbondsratio}, the percentage of sampled conformations with two H-bonds differs significantly between Nleu-5R ($30\%$) and Nleu-5S ($7\%$). At the same time, the percentage of conformations without a H-bond is increased for Nleu-5S ($25\%$) compared to Nleu-5R ($8\%$). For the other pair, Nleu-2R and Nleu-2S, the percentages are more similar and in between those of Nleu-5R and Nleu-5S.


\begin{table}[h!]
    \centering
    \caption{Percentage of sampled conformations with zero, one, or two hydrogen bonds in chloroform. Analysis was restricted to the clusters with the trans-peptoid bond.}
    \label{tab: hbondsratio}
    \begin{adjustbox}{max width=\textwidth}
    \begin{tabular}{lccc}
    Number of hydrogen bonds [\%] &	0 &	1 &	2 \\
    \hline
    Nleu-5R  &	8	& 63	& 30 \\
    Nleu-5S  &	25	& 68	& 7  \\
    Nleu-2R  &	15	& 64	& 21 \\
    Nleu-2S  &	13	& 74	& 13 \\
    \hline
    \end{tabular}
    \end{adjustbox}
\end{table}

For a given molecule in an apolar environment, having access to conformations in which polar groups are shielded -- such as by H-bonding -- should be energetically favorable. 
To assess this effect, we extracted the potential energy of the peptides (i.e., intramolecular and peptide-solvent contributions) from the trajectories. 
The normality of each potential-energy distribution was confirmed by the Shapiro–Wilk test \cite{Shapiro1965} (Table \ref{tab: SIstatTestingNorm}). 

\begin{table}[h!]
\centering
\caption{Average potential energy of the peptides (i.e., sum of the intramolecular $\langle \text{V} \rangle$ contributions and the peptide–solvent contributions)  together with the $p$-value of the Shapiro-Wilk test for the simulations in chloroform and water, respectively. The significance limit for the $p$-value was 0.05.}
\label{tab: SIstatTestingNorm}
\begin{adjustbox}{max width=\textwidth}
\begin{tabular}{r|cc|cc}
\multirow{2}{*}{Molecule} & \multicolumn{2}{l}{Chloroform} & \multicolumn{2}{l}{Water}        \\
    & $\langle \text{V} \rangle [\text{kJ}/\text{mol}]$ & $\text{p}_{\text{Shapiro-Wilk}}$ & $\langle \text{V} \rangle [\text{kJ}/\text{mol}]$ & $\text{p}_{\text{Shapiro-Wilk}}$  \\
    \hline
    Nleu-5R    & -217.08    & \~0.0          & -117.77  & \~0.0         \\
    Nleu-5S    & -208.39    &   $5.9*10^{-8}$  & -115.64  & $5.24*10^{-10}$ \\
    Nleu-2R    & -211.17    & \~0.0          & -117.84  & \~0.0         \\
    Nleu-2S    & -216.63    &   $7.6*10^{-39}$ & -116.91  & \~0.0   \\
    \hline
\end{tabular}%
\end{adjustbox}
\end{table}

\begin{table}[h!]
\centering
\caption{Results of the Fisher t-test to validate the significance of the deviations in the average potential energy of the peptides. The significance limit for the $p$-value was 0.05.}
\label{tab: SIstatTestingDiff}
\begin{adjustbox}{max width=\textwidth}
\begin{tabular}{lcc}
Molecule          & Chloroform & Water     \\
\hline
Nleu-5R - Nleu-5S & $\sim$0.0  & $\sim$0.0 \\
Nleu-2R - Nleu-2S & $\sim$0.0  & $2.7*10^{-77}$\\
\hline
\end{tabular}%
\end{adjustbox}
\end{table}

The Fisher t-test \cite{Kotz1998} was employed to determine if the means of the distributions differ statistically significantly ($p < 0.05$). 
This was found to be the case for each pair of distributions (Table \ref{tab: SIstatTestingDiff}). 
On average, the potential energy of Nleu-5R is $9$~kJ/mol lower (i.e., more favorable) in chloroform compared to Nleu-5S, whereas the difference in the average potential energy between Nleu-2R and Nleu-2S is $6$~kJ/mol. In many studies in the literature, it was found that the 3D-PSA is a good measure for the degree of polar shielding in conformations. \cite{Roux2020, Sebastiano2018, Vorherr2018, Peraro2018} 
However, for the present set of four peptides, no correlation was observed between the 3D-PSA and the potential energy (Figure \ref{fig: SI3DPSAANA}). 
The ring strain in the relatively small backbone cycle of the peptides affects the geometry of the intramolecular H-bonds, which is likely not reflected appropriately in the 3D-PSA calculation. 
\begin{figure}[h!]
    \centering
    \includegraphics[width=\textwidth]{7_chapter_5/fig/results/3dPSA.png}
    \caption{Correlation between the 3D-PSA and the potential energy of the corresponding conformation (i.e., sum of intramolecular contributions and peptide–solvent contributions) for Nleu-5R and Nleu-5S in chloroform. The 100 structures closest to the cluster center were taken for the clusters with the trans-peptoid bond. The trend for an expected linear correlation is shown as gray line. The legend contains the cluster population in percentage and the Spearman correlation coefficient $r$.}
    \label{fig: SI3DPSAANA}
\end{figure}

In summary, the ranking Nleu-5R $<$ Nleu-2S $<$ Nleu-2R $<$ Nleu-5S, which was found in terms of both hydrogen-bonding patterns and potential energies, matches well with the experimental permeability data.
The findings described above indicate that the change in stereochemistry of the methyl group in position $5$ between Nleu-5R and Nleu-5S leads to different conformational behavior. 
A detailed analysis of the H-bonds showed that only Nleu-5S forms a H-bond between Ala-O and the tether-NH with an occurrence of $24\%$ in chloroform (Table \ref{tab: hbondsrationCLCH3}). 
This H-bond across the ring of Nleu-5S prevents the formation of other H-bonds (Figure \ref{fig: HbondExamples}B).
Such a conformation with a single H-bond is likely less favorable (compared to one with more H-bonds) in chloroform because less polar groups are shielded. In the dominant conformation of Nleu-5R, on the other hand, two H-bonds can be formed across the ring (Figure \ref{fig: HbondExamples}A).

\begin{figure}[h!]
    \centering
    \includegraphics[width=\textwidth]{fig/results/ExampleHbonds.png}
    \caption{Snapshots of Nleu-5R (\textbf{A}) and Nleu-5S (\textbf{B}) from MD simulations in chloroform. Hydrogen bonds are shown with their percentage of the absolute occurrence in chloroform in the trans-peptoid clusters. Pictures were generated with PyMol. \cite{Delano2020}}
    \label{fig: HbondExamples}
\end{figure}

\begin{table}[h!]
    \centering
    \caption{Hydrogen bond occurrence in percentage for the sampled conformations in chloroform. Analysis was restricted to the clusters with the trans-peptoid bond.}
    \label{tab: hbondsrationCLCH3}
    \begin{adjustbox}{max width=\textwidth}
    \begin{tabular}{lcccc}
    H-bond  [\%] &	Nleu-2R &	Nleu-2S &	Nleu-5R &	Nleu-5S  \\
    \hline
    Nleu-O tether-NH &	74 &	37 &	28 &	33 \\
    Ala-O tether-NH &	\textless{}1 & \textless{}1 &	\textless{}1 &	24 \\
    Phe-O Ala-NH    &	\textless{}1 &	35 &	57 &	\textless{}1 \\
    Ala-O Phe-NH    &	27 &	25 &	36 &	17 \\
    \hline\\
    \end{tabular}
    \end{adjustbox}
\end{table}

Next, we analyzed the torsional-angle distributions in the backbone ring of the peptides. The change in stereochemistry of the methyl group at position 5 leads to a shift in the torsional-angle distributions of the tether units for Nleu-5S compared to Nleu-5R (Figure \ref{fig: dihedralDist}A). This shift results in a bent conformation of the ring (Figure \ref{fig: dihedralDist}B), which allows only one H-bond to form between Ala-O and tether-NH (Figure \ref{fig: dihedralDist}B). There is also a shift in the backbone torsional-angle distributions between Nleu-2R and Nleu-2S, however, to a much smaller extent (Figure \ref{fig: SITorsion2RS}). Noteably a longrange effect in the dihedral distribution for the phenylalanine residue was detected. The missing hydrogen bond and the resulting rotation of the carbonyl group seem to hinder the free rotation of the phenylalanine rest (Figure \ref{fig: dihedralDistSubst}).

\begin{figure}[h!]
    \centering
    \includegraphics[width=\textwidth]{7_chapter_5/fig/results/dihedral_dist.png}
    \caption{(\textbf{A}) Torsional-angle distributions of the tether in Nleu-5R (blue) and Nleu-5S (orange) in chloroform. The analysis was restricted to the clusters with the trans-peptoid bond. (\textbf{B}) Torsional angles of the tether (shown in cyan and orange) corresponding to the peaks of the distributions. Pictures were generated with PyMol. \cite{Delano2020} The change in the stereocenter also affects the $\chi_1$-angle of the phenylalanine residue as the tether conformation hinders the rotation around this torsion due to a steric clash with the carbonyl group that is facing out of the backbone ring (Figure \ref{fig: dihedralDistSubst}).
    }
    \label{fig: dihedralDist}
\end{figure}

\begin{figure}[h!]
    \centering
    \includegraphics[width=\textwidth]{7_chapter_5/fig/results/dihedral_dist_subs.png}
    \caption{(\textbf{A}) Torsional-angle distributions of the $\chi_1$ torsional angle of the phenylalanine residue in Nleu-5R (blue) and Nleu-5S (orange) in chloroform. Analysis was restricted to the clusters with the trans-peptoid bond. (\textbf{B}) $\chi_1$ torsional angle of the phenylalanine residue (shown in purple) corresponding to the peaks of the distributions. The backbone carbonyl interferes with the rotation around this torsion is highlighted with a red circle. Pictures were generated with PyMol. \cite{Delano2020}}
    \label{fig: dihedralDistSubst}
\end{figure}

\begin{figure}[h!]
    \centering
    \includegraphics[width=\textwidth]{7_chapter_5/fig/results/dihedral_dist_2RS.png}
    \caption{Torsional-angle distributions of the tether in Nleu-2R (blue) and Nleu-2S (orange) in chloroform. Analysis was restricted to the clusters with the trans-peptoid bond. The distribution density is plotted as a function of the torsional angle, with a bin size of $0.5~\text{deg}$.}
    \label{fig: SITorsion2RS}
\end{figure}


\FloatBarrier
%-------------------------------------------

\begin{table}[h!]
    \centering
    \caption{Percentage of sampled conformations with zero, one, or two hydrogen bonds for Nleu-5R, Nleu-5S, Nleu-2R, and Nleu-2S in water. Analysis was restricted to the clusters with the trans-peptoid bond.}
    \label{tab: hbondsratiowater}
    \begin{adjustbox}{max width=\textwidth}
    \begin{tabular}{lccc}
    Number of hydrogen bonds [\%] &	0 &	1 &	2 \\
    \hline
    Nleu-5R  &	87	& 12	& 1 \\
    Nleu-5S  &	74	& 26	& 0  \\
    Nleu-2R  &	78	& 22	& 0 \\
    Nleu-2S  &	76	& 24	& 0 \\
    \hline
    \end{tabular}
    \end{adjustbox}
\end{table}

The analysis of the hydrogen-bonding patterns in water showed a significant decrease for the intramolecular H-bond populations, as they competed with the intermolecular H-bonds to the water. Specifically, Nleu-5R has a higher percentage (about $10\%$) of conformers with no H-bonds compared to Nleu-2R, Nleu-2S, and Nleu-5S (Table \ref{tab: hbondsratiowater}). The conformations containing two intramolecular H-bonds nearly vanished.
The positions of the intramolecular H-bonds are mainly focused on the Nleu-O and tether-NH position (Table \ref{tab: SIhbondRatiosWater}).
Nevertheless, it could be observed that the Ala-O and tether-NH, which was unique to Nleu-5S in the apolar environment, is again most present in water for Nleu-5S in contrast to the other possible intramolecular H-bonds. 
In general, however, no significant differences between the peptides could be observed in water.

\begin{table}[h!]
\centering
\caption{Hydrogen-bond occurrence in percentage for the sampled conformations in
water for Nleu-5R, Nleu-5S, Nleu-2R, and Nleu-2S. The analysis was restricted to the clusters with the trans-peptoid bond.}
\label{tab: SIhbondRatiosWater}
  \begin{adjustbox}{max width=\textwidth}
  \begin{tabular}{lcccc}
Hydrogen bond {[}\%{]} & Nleu-2R      & Nleu-2S      & Nleu-5R      & Nleu-5S      \\
\hline
Nleu-O tether-NH       & 14.5       & 18.3       &  12           & 9.75        \\
Ala-O tether-NH        & 5.5        & 3.6        & \textless{}1  & 15.25 \\
Phe-O Ala-NH           & \textless{}1 & \textless{}1 & \textless{}1  & 1 \\
Ala-O Phe-NH           & 2            & 1            & \textless{}1  & \textless{}1 \\
    \hline
\end{tabular}%
\end{adjustbox}
\end{table}


\FloatBarrier
%-------------------------------------------

The findings, taken together, suggest that the permeability cliff observed between Nleu-5R and Nleu-5S is related to their propensity for conformations with a maximized number of intramolecular H-bonds in the apolar environment. Their ability to adopt such conformations is in turn affected by the stereochemistry of the methyl group at position $5$ in the tether as it determines the preferred torsional angles of the tether.



\clearpage
\newpage

%================================================================================
\section{Conclusion}
%================================================================================

Combining the data generated by our collaborator, NMR measurements, and molecular dynamics simulations allow us to draw some conclusions on how the structural changes in the sub-selected molecules influence their membrane permeability and therefore create the permeability cliff. 
The pair of Nleu-2R/S did not show a significant change in permeability depending on the stereo-center change. In contrast, the second pair shows a significant effect on permeability behavior. Nleu-5R is the most permeable compound from our collaborator's initial library, while its counterpart, Nleu-5S is the exception among the Nleu compounds for its low permeability. 
A detailed analysis of torsion angles points analogously to the work of our collaborator at intramolecular H-bonds.  \cite{Comeau2021}
Finally, we found different H-bond patterns of Nleu-5R and Nleu-5S in the chloroform and water simulations.  Here we could show that the conformational populations in the different environments are more diverse in the Nleu-5R case than the Nleu-5S, which is partially caught in one intramolecular hydrogen bond in the water environment.
Additionally, in our chloroform simulations, we observed a long-range effect based on the stereochemistry change modifying the rotational freedom of the phenylalanine side chain.
We could not retrieve insightful data from applying the 3D-PSA criterium. We assume that the molecules studied are too small for significant polarity hiding effects.  Therefore the main effect introduced by the stereocenter change is a change in the torsion distribution that leads to the different hydrogen bond patterns.

In conclusion, we studied the relationship between small structural changes and the resulting permeability behavior for the selected compounds. 


\clearpage
\pagebreak

\bibliography{7_chapter_5/ref/ref.bib}


\end{document}
