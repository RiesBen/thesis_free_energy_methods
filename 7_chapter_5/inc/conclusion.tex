Combining the permeability data generated by our collaborators, NMR measurements, and MD simulations allowed us to draw some conclusions on how the structural differences between the selected macrocycles influence their membrane permeability. 
The pair of Nleu-2R/S did not show a significant change in permeability depending on the stereocenter change. In contrast, the second pair Nleu-5R/S showed a significant effect on permeability behavior. Nleu-5R is the most permeable compound from the compound collection synthesized by our collaborators, while Nleu-5S is with its low permeability the exception among the Nleu compounds. 
In the MD simulations, we observed different H-bond patterns for Nleu-5R and Nleu-5S in the chloroform. While Nleu-5R frequently adopted a conformation with the maximum number of two H-bonds (optimal shielding of the polar groups), such a conformation was rare for Nleu-5S. A detailed analysis of the torsional angle preferences highlighted the underlying steric effects.

In contrast to other studies, we could not retrieve a correlation between the 3D-PSA and the PAMPA permeability for the four selected macrocycles. The backbone cycle of the peptides is relatively small, thus minor structural changes affecting the geometry of the intramolecular H-bonds are likely not reflected appropriately in the 3D-PSA calculation.
In summary, we studied the relationship between small structural changes and the resulting permeability behavior for four semipeptidic macrocycles. The location and especially the stereochemistry of the methyl group played an important role in the intramolecular hydrogen-bonding pattern, impacting the passive membrane permeability of the compounds.
