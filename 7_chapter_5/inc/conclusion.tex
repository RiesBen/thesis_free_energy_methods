Combining the data generated by our collaborator, NMR measurements, and molecular dynamics simulations allow us to draw some conclusions on how the structural changes in the sub-selected molecules influence their membrane permeability and therefore create the permeability cliff. 
The pair of Nleu-2R/S did not show a significant change in permeability depending on the stereo-center change. In contrast, the second pair shows a significant effect on permeability behavior. Nleu-5R is the most permeable compound from our collaborator's initial library, while its counterpart, Nleu-5S is the exception among the Nleu compounds for its low permeability. 
A detailed analysis of torsion angles points analogously to the work of our collaborator at intramolecular H-bonds.  \cite{Comeau2021}
Finally, we found different H-bond patterns of Nleu-5R and Nleu-5S in the chloroform and water simulations.  Here we could show that the conformational populations in the different environments are more diverse in the Nleu-5R case than the Nleu-5S, which is partially caught in one intramolecular hydrogen bond in the water environment.
Additionally, in our chloroform simulations, we observed a long-range effect based on the stereochemistry change modifying the rotational freedom of the phenylalanine side chain.
We could not retrieve insightful data from applying the 3D-PSA criterium. We assume that the molecules studied are too small for significant polarity hiding effects.  Therefore the main effect introduced by the stereocenter change is a change in the torsion distribution that leads to the different hydrogen bond patterns.
In conclusion, we studied the relationship between small structural changes and the resulting permeability behavior for the selected compounds. The placement and especially the stereochemistry of the methyl group played an important role in the intramolecular hydrogen bond pattern and was observed to also impact the membrane permeability in experiment.
