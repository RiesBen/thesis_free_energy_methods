A total of 42 macrocycles were synthesized and their permeability assessed in the PAMPA and Caco-2 assays. 
The combination of these data, NMR measurements, and molecular dynamics simulations allows us to draw some conclusions that are hopefully applicable to other systems. 
The systematic higher permeability of macrocycles bearing an Nleu peptoid is striking and well above statistical significance. Our experiments suggest this effect is due to the removal of this specific H-bond donor, thus working similarly to the more widely used N-methylation strategy. 
This systematic effect shows that “masking” H-bond donors should be considered early in the design of cyclic peptides. 

Possibly, it is a matter of finding the right one(s), i.e., those that allow for the most favorable H-bonding patterns in the rest of the macrocycle.
The methyl position on the tether had little effect in most cases, with a few notable exceptions.
Nala-2S has the lowest passive permeability, while its epimer is average. 
Conversely, Nleu-5R is the most permeable compound from our initial library, while its counterpart Nleu-5S is the exception among the Nleu compounds for its low permeability. 
A detailed analysis of torsion angles points once more at intramolecular H-bonds. 
Nleu-5R and Nleu-5S have different intramolecular H-bonding patterns. 
It seems likely that the 2 and 5 positions have the highest potential to introduce significant conformational changes due to their proximity to H-bond partners (the tether’s carbonyl and nitrogen, respectively).
These positions might also have more impact due to the flexible nature of the tether we used, as they are close to the sp2-like amides.
It is also noteworthy that a simple inversion of stereochemistry was shown to exert long-distance influence, modifying the phenylalanine’s rotation.
Altogether, this study sheds light on the relationship between structure and permeability in this class of compounds.
The two seemingly very different substitutions we explored were both found to affect permeability through a change in the intramolecular H-bonding pattern.
