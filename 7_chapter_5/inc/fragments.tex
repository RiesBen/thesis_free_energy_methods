Macrocycles have recently gathered increasing levels of interest in medicinal chemistry.(1−6) \cite{Driggers2008, Mallinson2012, Dougherty2017, Marsault2011, Abdalla2018, Marsault2017}
Their unique combination of conformationally constrained structure and high level of structural information allows for the design of large, organized structures suitable to interact with extended and featureless binding sites such as those found in protein–protein interactions.\cite{Janin2008, Chène2006, Scott2016, Modell2016} 
Most Food and Drug Administration (FDA)-approved macrocyclic drugs belong to natural products (e.g., erythromycin, tacrolimus) or peptides (e.g., sandostatin, eptifibatin). \cite{Giordanetto2014}
Peptidic or semipeptidic scaffolds bridge the gap between small molecules and biologics, allying synthetic ease and broad choice of natural and non-natural amino acids required for rapid and thorough pharmacophoric exploration. 
The main challenge with peptides resides in their physicochemical and pharmacokinetics-absorption, distribution, metabolism, and excretion (PK-ADME) properties. 
While cyclic peptides are typically more stable to proteases compared to their linear counterparts, their high polarity often translates into low bioavailability.\cite{Naylor2017, Fosgerau2015}
Nonetheless, some cyclic peptides cross cell membranes.\cite{Naylor2017, Wang2014, Nielsen2014} 
Developing tools and knowledge to optimize and better predict their structure–permeability relationship is therefore a requirement for the field. Such quest found inspiration in studies of the natural cycloundecapeptide cyclosporine A, which is administered orally. 
One prominent structural feature of this natural macrocycle is its high number of N-methylated residues (7 out of 11) and its dynamic structural adaptation to its environment (also known as (aka) chameleonic properties). \cite{Whitty2016, Danelius2020, Witek2017}
The effect of N-methylation on permeability of cyclic hexa- and heptapeptides has been systematically investigated since the number and position of N-methylations may be beneficial or detrimental for permeability. \cite{Nielsen2014, Räder2018, White2011, Beck2012, Biron2008, White2011} 
Less explored are the N-alkylated glycines—aka peptoids—in which side chain has been moved from the α carbon to the amide nitrogen. \cite{Schwochert2015} 
Similarly to N-methylation, this modification removes one H-bond donor, yet it also removes one stereogenic center and induces glycine-like secondary structures.
The peptoid amide also facilitates cis–trans isomerization compared to the corresponding N-methylation.\cite{Sui2007} 
Synthetically, the inclusion of peptoids is also compatible with solid-phase protocols and allows for an almost unlimited variety of side chains, where virtually any primary amine can be used.\cite{Culf2010}
More recently, the impact of the dynamics of macrocycles in response to their environment, which can range from polar in water, nonhomogeneous in the presence of its target, to lipophilic in the membrane, has been appreciated.\cite{Danelius2020, Witek2017, Riniker2019, Witek2019}
A powerful tool to modulate the properties of peptidic macrocycles is the inclusion of a nonpeptidic tether unit.\cite{Marsault2007, Hoveyda2011, Roux2020} 
This tether can serve multiple purposes: in the context of a target interacting with a specific sequence, various tethers can be screened without modifying the peptide recognition sequence, while providing a simple handle for modulating affinity and PK properties. 
Small modifications in size, shape, or functional groups on the tether can dramatically influence on this kind of constrained system.\cite{Appavoo2019}
Additionally, a tether may facilitate macrocyclization, which can be challenging synthetically. \cite{Marsault2017} 
The relationship between structure and permeability is known to be elusive for this class of compounds, with small structural modifications often yielding permeability cliffs. \cite{Wang2014, Räder2018, Beck2012, White2011, Roux2020, Bockus2015, Hewitt2015, Rezai2006, Over2016}
To support our efforts in this direction \cite{Roux2020} and pinpoint the effects of conformational modulation on permeability, we synthesized a library of closely related compounds based on chemotype A composed of a tripeptide tethered head-to-tail with a nonpeptidic linker (Figure \ref{fig:}). 
Two classes of modifications were implemented on chemotype A: single peptoid replacement (B, Figure \ref{fig:}) or regio- and stereocontrolled linker C-methylation (C, Figure \ref{fig:}). 
All of the possible combinations of these variations were generated, providing a total of $4 × 9 = 36$ compounds with identical molecular weights (except for non-methylated tether derivatives), nearly identical sequence and identical ring sizes, leaving as little room as possible for confounding factors. 
The passive permeability of the resulting macrocycles was measured in the parallel artificial membrane permeability assay (PAMPA) and their cellular permeability in the Caco-2 assay.\cite{Di2015}
We then selected two pairs of diastereomers that differ only by their stereochemistry of the tether methyl group yet either differ greatly in passive permeability or not, and performed molecular dynamics (MD) simulations coupled with solution NMR to rationalize the origin of these differences.\cite{Balazs2019}






% METHODS NMR
\subsection{NMR Measurements}
The selected peptides Nleu-5R, Nleu-5S, Nleu-2R, and Nleu-2S were characterized by NMR in chloroform-d (Cambridge Isotope Laboratories).
The following spectra were recorded if not stated otherwise: $^1$H NMR, total correlation spectroscopy (TOCSY), double-quantum filtered correlation spectroscopy (COSY), multiplicity edited $^{13}$ heteronuclear single quantum coherence (HSQC), $^{13}$C heteronuclear multiple bond correlation (HMBC), and NOESY.
All spectra were measured at $25~^\circ$C on a Bruker Avance III HD $600~$MHz spectrometer equipped with a N$_2$-cooled Prodigy triple resonance probe.
$^{13}$C HSQC and TOCSY spectra were recorded with sensitivity enhancement. TOCSY was run with an $80~$ms DIPSI2 isotropic mixing time.
The mixing time for the NOESY experiments was set to $300~$ms if not stated otherwise. 
For compound Nleu-5R, an EASY-ROESY \cite{78} spectrum with $100~$ms mixing time was recorded instead of a NOESY. 
For all spectra, the time domain in both dimensions was extended to twice its size by zero filling, apodized with a cos2 function, and the baseline of the resulting spectra was corrected with a polynomial of fifth order or using the Whittacker smoother algorithm. \cite{79}
Processing was done with Bruker TopSpin version 4.0 (Bruker Biospin AG) and MestReNova 12.0 (Mestrelab Research). Resonance assignment and volume integration of the ROESY cross-peaks were performed with SPARKY 3.115. \cite{80}
The assignments are summarized in Table S1 .
$^3$J$_\text{\text{HN–H}\alpha}$ coupling constants for compounds Nleu-5R, Nleu-5S, Nleu-2R, and Nleu-2S were extracted directly from the $^1$H spectrum with MestReNova and are summarized in Table S2 .
Volumes were extracted from NOESY and ROESY spectra by integration of the cross-peaks with a Gaussian function (eq 1).\cite{1}
$V_{1,2}$ is the volume of the cross-peak between proton 1 and 2, a and b are fitting parameters, and $r_{1,2}$ is the corresponding interatomic distance.
A two-point calibration was done with the averaged interatomic distance (e.g., on both sides of the diagonal) between the diastereotopic protons NLeu H$\alpha_1$–H$\alpha_2$ ($1.8~$Å) and the distance between H$\alpha$ and H$\beta^*$ in the alanine residue ($2.65~$Å). \cite{81}
In the second case, the volume was previously divided by 3 to account for the three protons in the methyl group. 
Cross-peaks integrated on both sides of the diagonal were averaged, and error bonds of $\pm 20\%$ were added to the calculated distance. 
Since the GROMOS++ programs can do multiplicity correction and averaging over indistinguishable protons automatically, the reported data do not account for that. 
The volumes and the corresponding distances can be found in Tables S3–S6 .


