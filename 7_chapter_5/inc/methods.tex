\subsection{Simulation}
In the computational studies, two pairs of structurally similar cyclic peptides were selected, i.e., Nleu-5R/Nleu-5S and Nleu-2R/Nleu-2S. 
The first pair presents a “permeability cliff”, i.e., the two peptides show a large difference in the passive permeability in the PAMPA assay (Nleu-5R: $5.39$; Nleu-5S: $7.21$), despite a high structural similarity. 
In contrast, the second pair is similar in both structure and permeability (Nleu-2R: $6.14$; Nleu-2S: $5.80$). 
For each of these four peptides, $250$ starting coordinates were generated using the macrocycle variant of the OMEGA conformer generator from OpenEye. \cite{Hawkins2012, Hawkins2010, Poongavanam2018}
Conformers were energy-minimized for maximum $2000$ steps with the steepest descent \cite{Ruder2016} approach using the GROMOS software package \cite{Schmid2012} with the GROMOS 54A7 force field. \cite{Schmid2011} 
Each minimized starting conformation was solvated in a cubic box of simple-point-charge (SPC) water \cite{Berendsen1981} (on average, $4172$ solvent molecules) or chloroform \cite{Tironi1994} (on average, $980$ solvent molecules). 
For each system, a molecular dynamics (MD) simulation of $101~ns$ length was performed under isothermal–isobaric (NPT) conditions with the leap-frog integration algorithm \cite{Gunsteren1988} and a time step of $2~fs$. 
The first 1 ns was discarded as equilibration. Bond lengths were constrained with SHAKE \cite{Ryckaert1977} and a tolerance of $10^{–4}~nm$. 
Nonbonded interactions were calculated using a twin-range scheme with a short-range cutoff of $0.8~nm$ and a long-range cutoff of $1.4~nm$. 
The electrostatic nonbonded contributions beyond the long-range cutoff were calculated with the reaction-field \cite{tironi1995} (67) approach, setting the dielectric permittivity to 61 \cite{Heinz2001} for water, and to 4.8 \cite{Tironi1994} for chloroform. 
The temperature was kept constant at $300~K$ using the weak coupling scheme \cite{Berendsen1984} and a coupling time of $0.1 ps^{–1}$. 
The pressure was kept at $1.031~bar$ ($1~atm$) with the same type of algorithm, a coupling time of $0.5~ps^{–1}$, and an isothermal compressibility of $0.001654~bar^{–1}$ for chloroform and $0.0004575~bar^{–1}$ for water. 
Translational motion of the center of mass of the simulation box was removed every $2~ps$. Energies and coordinates were written every $5~ps$.

Trajectory analysis was performed with PyEmma \cite{Scherer2015} and MDTraj \cite{McGibbon2015}. 
The selection of features for the clustering consisted of the distances between all pairs of polar atoms and the backbone torsional angles, which resulted in $57$ features. 
This selection was reduced to three to five dimensions (depending on the peptide) with TICA \cite{Molgedey1994} using a cumulative variance of $0.9$ as criterium and a TICA correlation lag time of $50~ps$. 
Based on these TICs, the frames were clustered with a common nearest neighbor (CNN) algorithm \cite{Keller2010, Weiß2021} using a cutoff of $0.2$ and a similarity of $20$. 
Comparison of selected clusters with NMR experiments was performed with the GROMOS++ package of programs. \cite{Eichenberger2011}
The coefficients for the Karplus curve were taken from Vögeli et al. \cite{Voegeli2015}
Analysis of hydrogen bonds and torsional angles was performed with MDTraj. 
The 3D polar surface area (3D-PSA) was calculated with our implementation \cite{Witek2019} of the workflow in ref \cite{Tyagi2018} using PyMol \cite{DeLano2020}.
Statistical analysis of all results was carried out using the Python packages pandas, NumPy and SciPy.\cite{Virtanen2020}

\subsection{NMR Measurements}
The selected peptides Nleu-5R, Nleu-5S, Nleu-2R, and Nleu-2S were characterized by NMR in chloroform-d (Cambridge Isotope Laboratories).
The following spectra were recorded if not stated otherwise: 1H NMR, total correlation spectroscopy (TOCSY), double-quantum filtered correlation spectroscopy (COSY), multiplicity edited 13C heteronuclear single quantum coherence (HSQC), 13C heteronuclear multiple bond correlation (HMBC), and NOESY.
All spectra were measured at 25 °C on a Bruker Avance III HD 600 MHz spectrometer equipped with a N2-cooled Prodigy triple resonance probe.
13C HSQC and TOCSY spectra were recorded with sensitivity enhancement. TOCSY was run with an 80 ms DIPSI2 isotropic mixing time.
The mixing time for the NOESY experiments was set to 300 ms if not stated otherwise. 
For compound Nleu-5R, an EASY-ROESY (78) spectrum with 100 ms mixing time was recorded instead of a NOESY. 
For all spectra, the time domain in both dimensions was extended to twice its size by zero filling, apodized with a cos2 function, and the baseline of the resulting spectra was corrected with a polynomial of fifth order or using the Whittacker smoother algorithm. (79)
Processing was done with Bruker TopSpin version 4.0 (Bruker Biospin AG) and MestReNova 12.0 (Mestrelab Research). Resonance assignment and volume integration of the ROESY cross-peaks were performed with SPARKY 3.115. (80)
The assignments are summarized in Table S1 in the SI.
3JHN–Hα coupling constants for compounds Nleu-5R, Nleu-5S, Nleu-2R, and Nleu-2S were extracted directly from the 1H spectrum with MestReNova and are summarized in Table S2 in the SI.
Volumes were extracted from NOESY and ROESY spectra by integration of the cross-peaks with a Gaussian function (eq 1).(1)
V1,2 is the volume of the cross-peak between proton 1 and 2, a and b are fitting parameters, and r1,2 is the corresponding interatomic distance.
A two-point calibration was done with the averaged interatomic distance (e.g., on both sides of the diagonal) between the diastereotopic protons NLeu Hα1–Hα2 (1.8 Å) and the distance between Hα and Hβ* in the alanine residue (2.65 Å). (81)
In the second case, the volume was previously divided by 3 to account for the three protons in the methyl group. 
Cross-peaks integrated on both sides of the diagonal were averaged, and error bonds of ±20\% were added to the calculated distance. 
Since the GROMOS++ programs can do multiplicity correction and averaging over indistinguishable protons automatically, the reported data do not account for that. 
The volumes and the corresponding distances can be found in Tables S3–S6 in the SI.