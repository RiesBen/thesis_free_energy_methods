\chapter{Summary}
Computational methods to calculate free-energy differences are of high interest in computer-aided drug discovery. Rigorous free-energy methods are based on molecular dynamics (MD) simulations. Chapter \ref{ch:intro} provides a short overview of the fundamental concepts of MD.   
Chapters \ref{ch:feens} - \ref{ch:fereeds} describe developments in the area of free-energy calculation. 
%Ensembler
In Chapter \ref{ch:feens}, the software package Ensembler for toy model simulations is introduced. Ensembler can be used to rapidly develop prototypes for method development or teach theoretical backgrounds of simulation techniques.  An application example showcases how Ensembler can be used to execute and compare different free-energy methods with one-dimensional models. 
%Restraints
Chapter \ref{ch:feres} provides a categorization of the existing approaches on how the end-states can be represented in relative free-energy calculations. Next, an algorithm is introduced for selecting (locally) optimal distance restraints to link molecules in a linked dual topology approach. The introduced algorithm can handle scaffold hopping-like transformations and is extended for multi-state methods. The performance of the approach is demonstrated in relative hydration free-energy calculations. 
%REEDS
In Chapter \ref{ch:fereeds}, the refined automated RE-EDS pipeline is presented, containing multiple methodological changes and tricks to ensure sufficient sampling of all end-states. The improved pipeline is applied to scaffold hopping-like transformations to showcase the capabilities of the methodology.

%PyGromos
In Chapter \ref{ch:pyGrom}, the Python API PyGromosTools is introduced. The API provides functionality to modify GROMOS files, to set up and perform MD simulations, and to analyze the resulting data. Considerations regarding the code style and the code structure are discussed. Finally, usage examples are provided to show how simulations can be set up and executed with PyGromosTools. 
%Cyclic peptides
Chapter \ref{ch:cycPep} explores the conformational behavior of pairs of semipeptidic macrocycles with a single stereo center change and its relationship with the observed passive membrane permeability. A particular interest is to rationalize a ``permeability cliff'' related to the stereocenter. We construct a hypothesis based on extensive MD simulations supported by NMR studies.

Finally, we will conclude the thesis with an outlook in Chapter \ref{ch:outlook} for the different topics covered in this work.