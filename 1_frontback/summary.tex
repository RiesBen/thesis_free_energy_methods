\chapter{Summary}
Computational methods calculating free energy differences are of high interest for drug discovery. Accurate free energy estimations require conformational ensembles of the target states, often generated by molecular dynamics simulations. Chapter \ref{ch:intro} gives a short overview of the concepts of molecular dynamics.   
The Chapters \ref{ch:feens} - \ref{ch:fereeds} stand in the context of free energy calculations. 
%Ensembler
In Chapter \ref{ch:feens} the software package Ensembler for toy model simulations is introduced. Ensembler can be used to rapidly develop prototypes for method development or teach theoretical backgrounds of simulation techniques.  An application example showcases how Ensembler can be used to present and compare multiple free energy methods with 1D models. 
%EREstraints
Chapter \ref{ch:feres} will provide a short categorization of the existing approaches on how a system can be represented in relative free energy calculations. Next, an algorithm for selecting distance restraints linking molecules is introduced. This algorithm is applied for building restraints for a linked dual topology approach for pairwise relative hydration-free energy calculations. The introduced algorithm is capable of dealing with scaffold hopping transformations and is expanded to handle multi-state systems.
%REEDS
In Chapter \ref{ch:fereeds}, the recently automatized RE-EDS pipeline is presented, containing multiple methodological changes and tricks to guarantee sufficient sampling of all end-state. The newly shaped pipeline is applied to scaffold hopping-like transformations to show the capabilities of the methodology.

%Pygromos
In Chapter \ref{ch:cycPep}, a study of the change of conformational behavior of semi-peptidic macrocycles depending on the change of a single stereo-center is presented. The particular interest was based on a ''permeability cliff'' related to the stereocenter change. We construct a hypothesis based on MD simulations validated by NMR studies.

Finally, we will conclude the thesis in Chapter \ref{ch:outlook} with an Outlook for the different topics covered in this thesis.
