Die computergestützte Berechnung von Freien Energiedifferenzen ist grundlegender Bestandteil  der \textit{in silico} Wirkstoffentwicklung. Um akkurate Freie Energien zu ermitteln, werden häufig konformationsgenerierende Methoden wie Molekulardynamik-Simulationen eingesetzt. In Kapitel \ref{ch:intro} wird eine Zusammenfassung der grundlegenden Konzepte von Molekulardynamik-Simulationen gegeben. 
Die darauf folgenden Kapitel \ref{ch:feens} - \ref{ch:fereeds} stehen im Kontext computergestützter Freier Energieberechnungen. 
In Kapitel \ref{ch:feens} wird die Software Ensembler vorgestellt. Sie kann verwendet werden, um schnell einfache Modelle für die Methodenentwicklung oder für Lehrzwecke zu generieren. In einem eindimensionalen Modellansatz werden verschiedene Freie Energieberechnungsmethoden der computergestützten Chemie vorgestellt. 
In Kapitel \ref{ch:feres} wird eine Einordnung für Systemrepräsentationen  in relativen Freien Energieansätzen vorgestellt. Außerdem wird ein Algorithmus, der mehrere Moleküle mittels optimaler Distanzdefinitionen verbindet, vorgestellt.  Der Algorithmus wird für den Ansatz der verknüpften dualen Topologien mit mehreren Molekülpaaren angewandt und daraus folgend werden Hydratisierungsenergie-Differenzen berechnet.
In Kapitel \ref{ch:fereeds} werden Fortschritte in der RE-EDS Methodik präsentiert, welche sich auf Modifizierungen im automatischen Optimierungsprozess stützt. Die Modifizierungen stellen in erster Linie die aktive Repräsentierung jedes Endzustandes der Simulation sicher. Die neue Pipeline wird auf ein komplexes Liganden-Protein-Komplex System angewandt, um das Potential der Methode aufzuzeigen.
In Kapitel \ref{ch:pyGrom} wird die Python-Programmierschnittstelle (API) PyGromosTools vorgestellt. Die API erlaubt Zugriff auf GROMOS-Dateien, die Generierung von GROMOS Systemen, die Durchführung von Simulationen sowie die Analyse der generierten Daten. Die Funktionalität der API wird anhand mehrerer Anwendungsbeispiele verdeutlicht.
In Kapitel \ref{ch:cycPep} wird eine Molekulardynamik-Simulationsstudie über semi-Peptidmakrozyklen  und den Einfluss eines chiralen Zentrums im Makrozyklus wird vorgestellt. Die Veränderung des chiralen Zentrums führte in einem Makrozyklenpaar zu einer starken  Reduktion der Membrangängigkeit, welche mittels NMR-Messungen experimentell validiert wurden.
Das letzte Kapitel rekapituliert die Dissertation und gibt einen Ausblick auf die mögliche weitere Entwicklung der Dissertationsthemen.


