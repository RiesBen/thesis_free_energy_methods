\chapter{Zusammenfassung}
Die computergestützte Berechnung von Freie Energiedifferenzen ist ein grundlegender Bestandteil der \textit{in silico} Wirkstoffentwicklung. Rigorose Methoden zur Freie Energieberechnung basieren auf Molekulardynamik-Computersimulationen (MD-Simulationen). Kapitel \ref{ch:intro} beinhaltet eine Zusammenfassung der grundlegenden Konzepte von MD-Simulationen. 
Die darauf folgenden Kapitel \ref{ch:feens} - \ref{ch:fereeds} beschreiben Entwicklungen im Bereich computergestützter Freie Energieberechnung. 
In Kapitel \ref{ch:feens} wird das Softwarepaket Ensembler vorgestellt. Dieses kann verwendet werden, um schnell einfache Modelle für die Methodenentwicklung oder für Lehrzwecke zu generieren. In einem eindimensionalen Modellansatz werden verschiedene computergestützte Freie Energieberechnungsmethoden verwendet und miteinander verglichen. 
Kapitel \ref{ch:feres} beinhaltet eine Einordnung für Systemrepräsentationen in relativen Freie Energieberechnungsmethoden. Zudem wird ein Algorithmus vorgestellt, der mehrere Moleküle mittels (lokal) optimaler Distanzdefinitionen für verknüpfte duale Topologien verbindet. Der entwickelte Algorithmus kann mit komplexen Ligandtransformationen, wie sie bei ``Scaffold hopping'' vorkommen, umgehen. Die Mächtigkeit des Algorithmus' wird am Beispiel der Berechnung von Hydratisierungsenergie-Differenzen aufgezeigt.
In Kapitel \ref{ch:fereeds} werden Fortschritte in der RE-EDS Methodik präsentiert, welche auf Modifizierungen im automatischen Optimierungsprozess beruhen. Diese Modifizierungen stellen in erster Linie eine ausreichende Repräsentation jedes Endzustandes in der Simulation sicher. Die verbesserte Pipeline wird auf ein komplexes Liganden-Protein System angewandt, um das Potential der Methode aufzuzeigen.

In Kapitel \ref{ch:pyGrom} wird die Python-Programmierschnittstelle (API) PyGromosTools vorgestellt. Die API ermöglicht den Zugriff auf GROMOS-Dateien, die Generierung von GROMOS-Systemen, die Durchführung von Simulationen sowie die Analyse der generierten Daten. Die Funktionalität der API wird anhand mehrerer Anwendungsbeispiele verdeutlicht.
Kapitel \ref{ch:cycPep} untersucht das konformationelle Verhalten von Paaren von semipeptidischen Makrozyklen, die sich in einem chiralen Zentrum unterschreiben, und verknüpft dieses mit der gemessenen passiven Membranpermeabilität. Die Änderung des chiralen Zentrums führte in einem Makrozyklenpaar zu einer starken  Ver{\"a}nderung der Membrangängigkeit. Die Resultate der Computersimulationen werden mittels NMR-Messungen experimentell validiert.
Das letzte Kapitel rekapituliert die Dissertation und gibt einen Ausblick auf  mögliche Weiterentwicklung der Dissertationsthemen.


